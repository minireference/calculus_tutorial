

	\subsection{Set operations}

		We use set operations like union $\cup$, intersection $\cap$, and set difference~$\setminus$ to define composite sets.

		\begin{itemize}

			\item $S\cup T$: the \emph{union} of two sets.						\index{set!union|textit}
				The union of $S$ and $T$ corresponds to the elements in either $S$ or $T$.

			\item $S \cap T$: the \emph{intersection} of two sets.				\index{set!intersection|textit}
				The intersection of $S$ and $T$ corresponds to the elements that are in both $S$ and $T$.

			\item $S \, \setminus \, T$: \emph{set difference} or \emph{set minus}.		\index{set!difference|textit}
				The set difference $S \setminus T$ corresponds to the elements of $S$ that are not in $T$.

		\end{itemize}

		\noindent
		Consider the overlapping intervals $A = [a,b]$ and $B = [c,d]$ illustrated in Figure~\ref{fig:sets_A_B_and_set_operations}.
		The union of these two intervals is the set of numbers that are \emph{either} between $a$ and $b$ \emph{or} between $c$ and $d$,
		which corresponds to the interval $[a,d]$.
		The intersection of $A$ and $B$ is the set of numbers that are in \emph{both} $A$ and $B$,
		and corresponds to the interval $[c,b]$.
		The figure also illustrates the two set differences,
		$A\;\setminus\;B$ and $B\;\setminus\;A$ which correspond to numbers that are in one set,
		but not in the other.

		\begin{figure}[htb]
			\centering
			\includegraphics[width=0.3\textwidth]{figures/cut_material/sets_A_B_and_set_operations.pdf}
			\caption{	Various intervals that can be obtained using set operations of the intervals $A$ and $B$.}
			\label{fig:sets_A_B_and_set_operations}
		\end{figure}
