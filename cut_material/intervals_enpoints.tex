	Here are some more examples of various intervals:
	\begin{itemize}

		\item	$[a,b]$: the interval from $a$ to $b$.
			This corresponds to the set of real numbers between $a$ and~$b$,
			including the endpoints $a$ and~$b$.
			The interval $[a,b]$ corresponds to the set $\{ x\in \mathbb{R}\ | \ a \leq x \leq b \}$.

		\item	$[a,\infty)$: the interval from $a$ until infinity,
			which corresponds to the set $\{ x\in \mathbb{R}\ | \ a \leq x \}$.

		\item	$(-\infty,b]$: the interval from negative infinity until $b$,
			which corresponds to the set $\{ x\in \mathbb{R}\ | \ x \leq b \}$.

	\end{itemize}

	\noindent
	The notation $[a,b]$ describes the \emph{closed} interval from $a$ to $b$,
	which means the endpoints $a$ and $b$ are included in the interval.
	The notation $(a,b)$ describes the \emph{open} interval from $a$ to $b$,
	defined as the set $\{ x\in \mathbb{R}\ | \ a < x < b \}$,
	which doesn't include the endpoints $a$ and $b$.
	In other words,
	intervals defined using square brackets ``$[$'' include the endpoints
	(defined using less-than-or-equal conditions)
	while intervals defined with round brackets ``$($''
	do not include their endpoints (defined using strictly-less-than conditions).


The distinction between open and closed intervals is important in general,
but makes no difference in the context of probability theory,
so you don't need to worry about the difference between $[a,b]$ and $(a,b)$ in this book.
 \SIDENOTE{ @Ivan: If not relevant to the book, why are we explaining it? }
 
 
 
 	
		\noindent
		I hope these definitions and examples made you feel more comfortable with sets,
		and the weird-looking curly bracket notation that mathematicians use to define sets.
		It might look a little complicated at first,
		but you'll get used to it in the rest of the book.
		% Robyn said: 	I love these reassuring paragraphs,
		%			though it's a bit repetative in this chapter. Just something to be aware of.
	
	
