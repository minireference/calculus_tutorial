\documentclass[10pt,oneside]{article}
%\documentclass[oneside,10pt]{book}

\title{ Problems for the Calculus Tutorial}
\author{Ivan Savov, Minireference Co.}


%%%   LATEX PACKAGES    %%%%%%%%%%%%%%%%%%%%%%%%%%%%%%%%%%%%%%%%%%%%%%%%%%%%%%%%
\usepackage[pdftex,
  pdfversion=1.5,
  hyperfootnotes=false,
  pdfstartview=Fit,   % default value
  pdfstartpage=1,     % default value
  pdfpagemode=UseNone,  % default value
  bookmarks=true,   % default value
  linktocpage=false,    % default value
  pdfpagelayout=SinglePage,
  pdfdisplaydoctitle,
  pdfpagelabels=true,
  bookmarksopen=true,
  bookmarksopenlevel=0,
  colorlinks=false,
  pdfborder={0 0 0},
  linkcolor=black]{hyperref}        % hyperref needs to be loaded after imakeidx for hyperlinks to work
\usepackage{amsthm}
\usepackage{amsfonts}
\usepackage{amsmath}
\usepackage{amssymb}
\usepackage[nosolutionfiles]{answers}           % inline answers for use when proofreading
%\usepackage{answers}
\usepackage{color}

%%%    HEADERS AND MACROS     %%%%%%%%%%%%%%%%%%%%%%%%%%%%%%%%%%%%%%%%%%%%%%%%%%%%

%%%% SHARED   %%%%%%%%%%%%%%%%%%%%%%%%%%%%%  
\newenvironment{hint}{\vspace{1.5mm}\par\noindent Hint:}{}		% more airy version used in LA book
%
% for preventing newlines in answers (appear on a single line)
\makeatletter
\newcommand\gobblepars{%
    \@ifnextchar\par%
        {\expandafter\gobblepars\@gobble}%
        {}}
\makeatother
%%%%%%%%%%%%%%%%%%%%%%%%%%%%%%%%%%%%%%%%%%%%%%%

%%%%   STANDARDIZE ON REFS TO PROBLEMS      %%%%%%%%%%%%%%%%%%%%%%%%%%%%%%%
\newcommand{\probref}[1]{P\ref{#1}}
%%%%%%%%%%%%%%%%%%%%%%%%%%%%%%%%%%%%%%%%%%%%%%%%%%%%%%%%%%%%%%%%%%%%%


%%%% END-OF-CHAPTER  PROBLEMS   %%%%%%%%%%%%%%%%%%%%%%%%%%%%%  
\Newassociation{answer}{Answer}{answers}			% optional replacement for solution (if no space)
\Newassociation{solution}{Solution}{solutions}			% defining the three saved "streams", solutions --> back of 

% A special theorem style for end-of-chapter problems  --- P4.1, P4.2, P4.3, ...
\newtheoremstyle{problemstyle}
  {0.3\topsep}   % Space above
  {0.7\topsep}   % Space below
  {\normalfont}  % Body font
  {0pt}       % Indent amount (empty value is the same as 0pt)
  {\bfseries} % Theorem head font
  {}          % Punctuation after theorem head
  {6pt plus 1pt minus 1pt} % Space after theorem head
  {P\thmnumber{#2} \thmnote{ (#3)}} % Theorem head spec
%
%
\theoremstyle{problemstyle}
\newtheorem{problem}{}[]

% a container for end-of-chapter exercises and end-of-chapter problems 
\newenvironment{problems}[1]{
	\setcounter{problem}{0}
	\Opensolutionfile{answers}[99anssol/answers_#1]
	\Opensolutionfile{solutions}[99anssol/solutions_#1] 
}{
	\Closesolutionfile{answers}
	\Closesolutionfile{solutions}
}
% Problem answers and solution display
%
\newcommand{\showProblemAnswers}[1]{	
	%	\renewcommand{\Answerlabel}[1]{E##1.}
	\renewenvironment{Answer}[1]{\textbf{P##1}}{\gobblepars}
	\input{99anssol/answers_#1}
}
%
\newcommand{\showProblemSolutions}[1]{	
	\renewcommand{\Solutionlabel}[1]{\textbf{P##1}}
	\input{99anssol/solutions_#1}
}
%%%%%%%%%%%%%%%%%%%%%%%%%%%%%%%%%%%%%%%%%%%%%%%




%%%   FONTS    %%%%%%%%%%%%%%%%%%%%%%%%%%%%%%%%%%%%%%%%%%%%%%%%%%%%%%%%%%%%%%
\usepackage{mathpazo}   % PALATINO (chosen Sep 10 2017)
\usepackage{mathabx}


\hypersetup{
	pdftitle={Problems for the Calculus Tutorial},
	pdfauthor={Ivan Savov},
	pdfkeywords={calculus, limits, derivatives, integrals},
}

    
\begin{document}


\maketitle 



%\setcounter{tocdepth}{1}
%\setcounter{secnumdepth}{1}
%\tableofcontents





%\mainmatter

\begin{problems}{tut}





	%%%%  LIMITS %%%%%%%%%%%%%%%%%%%%%%%%%%%%%%%%
	\section*{Limits}


	\begin{problem}
		Calculate the following limit expressions:
		\[
			\text{(a)}~\lim_{x\to \infty} \tfrac{7}{x+4}
			\qquad
			\text{(b)}~\lim_{x\to\infty}\tfrac{4x^2-7x+1}{x^2}
			\qquad
			\text{(c)}~\lim_{x\to 0^-} \tfrac{1}{x}
		\]
		\begin{answer}(a)~$0$. (b)~$4$. (c)~$-\infty$.\end{answer}
		\begin{solution}
			a) As $x$ goes to infinity, the denominator goes to infinity,
			so the fraction goes to zero.

			b) If we divide each term by $x^2$,
			we get the expression $\frac{4x^2-7x+1}{x^2}=4-\frac{7}{x}+\frac{1}{x^2}$.
			The limit of the first term is $4$.
			The limits of the second and third terms are zero as $x$ goes to infinity.

			c) As $x$ approaches $0$ from the left ($x\to 0^-$),
			the fraction $\frac{1}{x}$ takes on larger and larger negative numbers.
			Therefore $\lim_{x\to 0^-} \frac{1}{x}=-\infty$.			
		\end{solution}
	\end{problem}


	\begin{problem}
		Assuming $\lim_{x\to\infty} f(x)= 2$ and $\lim_{x\to\infty} g(x)=3$,
		compute
		\[
			\text{(a)}~\lim_{x\to\infty}\bigl(2f(x)-g(x)\bigr)
			\qquad
			\text{(b)}~\lim_{x\to\infty} f(x)\,g(x)
			\qquad
			\text{(c)}~\lim_{x\to\infty}\frac{4f(x)}{g(x)+1}.
		\]
		\begin{answer}(a)~$1$. (b)~$6$. (c)~$2$.\end{answer}
		\begin{solution}
			We're given $\lim_{x\to\infty} f(x)= 2$ and $\lim_{x\to\infty} g(x)=3$.
			For a) we use the sum rule:
			\[
				\lim_{x\to\infty}\bigl(2f(x)-g(x)\bigr)
				=2\lim_{x\to\infty}f(x)-\lim_{x\to\infty}g(x)
				=2\cdot 2-3
				=1.
			\]
			
			\noindent
			To solve b), we use the product rule for limits:
			\[
				\lim_{x\to\infty} f(x)\,g(x)
				=\left(\lim_{x\to\infty}f(x)\right)\left(\lim_{x\to\infty}g(x)\right)
				=2\cdot 3
				=6.
			\]
			
			\noindent
			To solve c) we use the quotient rule for limits:
			\[
				\lim_{x\to\infty}\frac{4f(x)}{g(x)+1}
				=\frac{4\lim_{x\to\infty}f(x)}{\lim_{x\to\infty}(g(x)+1)}
				=\frac{8}{\lim_{x\to\infty}g(x)+1}
				=\frac{8}{3+1}
				=2.
			\]
		\end{solution}
	\end{problem}





\vspace{1cm}

	%%%%  DERIVATIVES     %%%%%%%%%%%%%%%%%%%%%%%%%%%%%%%%
	\section*{Derivatives}


	\begin{problem} %P3
		Find the derivative with respect to $x$ of the functions:
		\[
			\text{(a)}~f(x) = x^{13}
			\qquad
			\text{(b)}~g(x) = \sqrt[3]{x}
			\qquad
			\text{(c)}~h(x) =  ax^2 + bx + c.
		\]
		\begin{answer}$f^{\prime\!}(x) = 13x^{12}$.
			$g^{\prime\!}(x) = \frac{1}{3} x^{-\frac{2}{3}}$.
			$h^{\prime\!}(x) = 2ax + b$.\end{answer}
		\begin{solution} 
			\noindent
			a) Use the power rule $\frac{d}{dx}x^n = nx^{n-1}$: $f'(x)=\frac{d}{dx}x^{13}=13x^{12}$.
		
			\noindent
			b) Rewrite $\sqrt[3]{x}=x^{1/3}$ and use the power rule: $g'(x)=\frac{d}{dx}x^{1/3}=\frac{1}{3}x^{-2/3}$.
		
			\noindent
			c) Differentiate term-by-term: $h'(x)=\frac{d}{dx}(ax^2+bx+c)=2ax+b$.
		\end{solution}
	\end{problem}


	\begin{problem} %P4
		Calculate the derivatives of the following functions:
		\[
			\text{(a)}~p(x) = \dfrac{2x + 3}{3x + 2}
			\qquad
			\text{(b)}~q(x) = \sqrt{x^2 + 1}
			\qquad
			\text{(c)}~r(\theta) =  \sin^3 \theta.
		\]
		\begin{answer}$p^{\prime\!}(x) = \frac{-5}{(3x + 2)^2}$.
			$q^{\prime\!}(x) = \frac{x}{\sqrt{ x^2 + 1}}$.
			$r^{\prime\!}(\theta) = 3\sin^2 \theta \cos\theta$.\end{answer}
		\begin{solution}
			\noindent
			a) Use the quotient rule $\left(\frac{u}{v}\right)'=\frac{u'v-uv'}{v^2}$
			with $u=2x+3$ and $v=3x+2$:
			\[
				p'(x)=\frac{2(3x+2)-(2x+3)\cdot 3}{(3x+2)^2}
				     =\frac{6x+4-6x-9}{(3x+2)^2}
				     =\frac{-5}{(3x+2)^2}.
			\]
		
			\noindent
			b) Write $q(x)=(x^2+1)^{\frac{1}{2}}$ and use the chain rule:
			\[
				q'(x)=\frac{1}{2}(x^2+1)^{-\frac{1}{2}}\cdot 2x=\frac{x}{\sqrt{x^2+1}}.
			\]
		
			\noindent
			c) Using the chain rule fives us $r'(\theta)=3\sin^2\theta\cdot \cos\theta$.
		\end{solution}
	\end{problem}





	\begin{problem} %P5
		Find the maximum and the minimum of $f(x) = x^5-5x$.
		\begin{answer}Max at $x = -1$; min at $x = 1$.\end{answer}
		\begin{solution} 
			First we compute the derivative $f'(x)=5x^4-5=5(x^4-1)$.
			The critical points are $x=-1$ and $x=1$.
			The second derivative is $f''(x)=20x^3$.
			We apply the second derivative test to each critical point:
			\[
				f''(-1)=-20<0 \ \Rightarrow\ \text{local maximum at }x=-1,
			\]
			\[
				f''(1)=20>0 \ \Rightarrow\ \text{local minimum at }x=1.
			\]
		\end{solution}
		
	\end{problem}




\vspace{1cm}

	%%%%  INTEGRALS     %%%%%%%%%%%%%%%%%%%%%%%%%%%%%%%%
	\section*{Integrals}


	\begin{problem} %P6 
		Calculate the integral function $F_0(b) = \int_0^b f(x)\,dx$
		for the polynomial $f(x) = 4x^3 + 3x^2 + 2x + 1$.
		\begin{answer}$F_0(b) = b^4 + b^3 + b^2 + b$.\end{answer}
		\begin{solution}
			Integrate term-by-term using the formula $\int_0^b x^n \,dx = \frac{1}{n+1}b^{n+1}$.
		\end{solution}
	\end{problem}

	\begin{problem} %P7
		Find the area under $f(x)=8-x^3$ between $x=0$ and $x=2$.
		\begin{answer}$A_f(0,2) = 12$.\end{answer}
		\begin{solution} 
			We compute the area using the definite integral:
			\[
				A_f(0,2)=\int_0^2 (8-x^3)\,dx
				=\left[8x-\frac{x^4}{4}\right]_0^2
				= \left(16-\frac{16}{4}\right)-0
				=12.
			\]
		\end{solution}

	\end{problem}

	\begin{problem} %P8
		Find the area under the graph of $g(x)=\sin(x)$ from $x=0$ to $x=\pi$.
		%\begin{hint}
		%\end{hint}
		\begin{answer}$A_g(0,\pi)=2$.\end{answer}
		\begin{solution} 
			The area is given by the following integral:
			\[
				A_g(0,\pi)=\int_0^\pi \sin(x)\,dx
				% =\left[-\cos(x)\right]_0^\pi
				= \bigl(-\cos\pi\bigr)-\bigl(-\cos 0\bigr)
				=1-(-1)=2.
			\]
		\end{solution}
	\end{problem}

	\begin{problem}	 %P9	% integration using substitution
		Compute $\int_{0}^{1} \frac{4x}{(1+x^2)^3}\,dx$
		using the substitution $u = 1 + x^2$.
		Check your answer numerically using the SciPy function \texttt{quad}.
		%\begin{hint}
		%\end{hint}
		\begin{answer}$\frac{3}{4}$.\end{answer}
		\begin{solution} 
			When using the change of variable $u = 1 + x^2$,
			we must also change the differential $du = 2x\,dx$,
			which conveniently contains $x$ that appears in the numerator,
			which allows us to write:
			\[
				\int_{x=0}^{x=1} \frac{4x}{(1+x^2)^3}\,dx
				=  \int_{x=0}^{x=1}  \frac{2}{u^3}\,du
				= \int_{x=0}^{x=1} 2  u^{-3}\,du.
			\]
			Next we must change the $x$-limits of integration
			to $u$-limits of integration:
			The lower limit $x = 0$ becomes $u = 1 + 0^2 = 1$,
			and the upper limit $x = 1$ becomes $u = 1 + 1^2 = 2$,
			which the complete substitution:
			\[
				\int_{x=0}^{x=1} \frac{4x}{(1+x^2)^3}\,dx
				=
				2 \int_{u=1}^{u=2} u^{-3}\,du.
			\]
			We can now proceed using the integral rule
			$\int x^n \,dx = \frac{1}{n+1}x^{n+1} + C$
			to obtain 
			\begin{align*}
				2 \int_{u=1}^{u=2} u^{-3}\,du
					&=	2 \left[ \frac{u^{-2}}{-2} \right]_{1}^{2}	
						= - \left[ u^{-2} \right]_{1}^{2}
						= - \left[ \frac{1}{u^2} \right]_{1}^{2}		\\
					&=	-\left( \frac{1}{2^2} - \frac{1}{1^2} \right)
						= -\left( \frac{1}{4} - 1 \right)				
						= \frac{3}{4}
			\end{align*}
		\end{solution}
	\end{problem}





\vspace{1cm}

	%%%%  SEQUENCES AND SERIES     %%%%%%%%%%%%%%%%%%%%%%%%%%%%%%%%
	\section*{Sequences and series}

	\begin{problem} %P10
		Calculate the value of the infinite series $\sum_{k=0}^\infty  \left(\frac{2}{3}\right)^{\!k}$.
		\begin{answer}$3$.\end{answer}
		\begin{solution}
			We can use the formula for the geometric series
			$\sum_{k=0}^{\infty} r^k=\frac{1}{1-r}$
			with $r=\frac{2}{3}$,
			which gives us $\sum_{k=0}^\infty  \left(\frac{2}{3}\right)^{\!k} = \frac{1}{1- \frac{2}{3}} =  3$.
		\end{solution}
	\end{problem}


	\begin{problem}
		Find the Taylor series for the function $f(x) = e^{-x}$.
		\begin{hint}
			Use algebraic manipulations starting from a Taylor series that you know.
		\end{hint}
		\begin{answer}$f(x) = \sum_{k=0}^\infty \frac{(-1)^k x^k }{ k! }$.\end{answer}
		\begin{solution}
			Start from the known Taylor series $e^x=\sum_{k=0}^{\infty}\frac{x^k}{k!}$.
			Substitute $-x$ for $x$ to get 
			\[
				f(x) =\sum_{k=0}^{\infty}\frac{(-x)^k}{k!}
				     =\sum_{k=0}^{\infty}\frac{(-1)^k x^k}{k!}.
			\]
		\end{solution}
	\end{problem}












\end{problems}




\clearpage

\section*{Answers}


\showProblemAnswers{tut}

\end{document}


