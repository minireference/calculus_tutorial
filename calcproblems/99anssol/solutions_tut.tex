\begin{Solution}{1}
			a) As $x$ goes to infinity, the denominator goes to infinity,
			so the fraction goes to zero.

			b) If we divide each term by $x^2$,
			we get the expression $\frac{4x^2-7x+1}{x^2}=4-\frac{7}{x}+\frac{1}{x^2}$.
			The limit of the first term is $4$.
			The limits of the second and third terms are zero as $x$ goes to infinity.

			c) As $x$ approaches $0$ from the left ($x\to 0^-$),
			the fraction $\frac{1}{x}$ takes on larger and larger negative numbers.
			Therefore $\lim_{x\to 0^-} \frac{1}{x}=-\infty$.			
		
\end{Solution}
\begin{Solution}{2}
			We're given $\lim_{x\to\infty} f(x)= 2$ and $\lim_{x\to\infty} g(x)=3$.
			For a) we use the sum rule:
			\[
				\lim_{x\to\infty}\bigl(2f(x)-g(x)\bigr)
				=2\lim_{x\to\infty}f(x)-\lim_{x\to\infty}g(x)
				=2\cdot 2-3
				=1.
			\]
			
			\noindent
			To solve b), we use the product rule for limits:
			\[
				\lim_{x\to\infty} f(x)\,g(x)
				=\left(\lim_{x\to\infty}f(x)\right)\left(\lim_{x\to\infty}g(x)\right)
				=2\cdot 3
				=6.
			\]
			
			\noindent
			To solve c) we use the quotient rule for limits:
			\[
				\lim_{x\to\infty}\frac{4f(x)}{g(x)+1}
				=\frac{4\lim_{x\to\infty}f(x)}{\lim_{x\to\infty}(g(x)+1)}
				=\frac{8}{\lim_{x\to\infty}g(x)+1}
				=\frac{8}{3+1}
				=2.
			\]
		
\end{Solution}
\begin{Solution}{3}
			\noindent
			a) Use the power rule $\frac{d}{dx}x^n = nx^{n-1}$: $f'(x)=\frac{d}{dx}x^{13}=13x^{12}$.
		
			\noindent
			b) Rewrite $\sqrt[3]{x}=x^{1/3}$ and use the power rule: $g'(x)=\frac{d}{dx}x^{1/3}=\frac{1}{3}x^{-2/3}$.
		
			\noindent
			c) Differentiate term-by-term: $h'(x)=\frac{d}{dx}(ax^2+bx+c)=2ax+b$.
		
\end{Solution}
\begin{Solution}{4}
			\noindent
			a) Use the quotient rule $\left(\frac{u}{v}\right)'=\frac{u'v-uv'}{v^2}$
			with $u=2x+3$ and $v=3x+2$:
			\[
				p'(x)=\frac{2(3x+2)-(2x+3)\cdot 3}{(3x+2)^2}
				     =\frac{6x+4-6x-9}{(3x+2)^2}
				     =\frac{-5}{(3x+2)^2}.
			\]
		
			\noindent
			b) Write $q(x)=(x^2+1)^{\frac{1}{2}}$ and use the chain rule:
			\[
				q'(x)=\frac{1}{2}(x^2+1)^{-\frac{1}{2}}\cdot 2x=\frac{x}{\sqrt{x^2+1}}.
			\]
		
			\noindent
			c) Using the chain rule fives us $r'(\theta)=3\sin^2\theta\cdot \cos\theta$.
		
\end{Solution}
\begin{Solution}{5}
			First we compute the derivative $f'(x)=5x^4-5=5(x^4-1)$.
			The critical points are $x=-1$ and $x=1$.
			The second derivative is $f''(x)=20x^3$.
			We apply the second derivative test to each critical point:
			\[
				f''(-1)=-20<0 \ \Rightarrow\ \text{local maximum at }x=-1,
			\]
			\[
				f''(1)=20>0 \ \Rightarrow\ \text{local minimum at }x=1.
			\]
		
\end{Solution}
\begin{Solution}{6}
			Integrate term-by-term using the formula $\int_0^b x^n \,dx = \frac{1}{n+1}b^{n+1}$.
		
\end{Solution}
\begin{Solution}{7}
			We compute the area using the definite integral:
			\[
				A_f(0,2)=\int_0^2 (8-x^3)\,dx
				=\left[8x-\frac{x^4}{4}\right]_0^2
				= \left(16-\frac{16}{4}\right)-0
				=12.
			\]
		
\end{Solution}
\begin{Solution}{8}
			The area is given by the following integral:
			\[
				A_g(0,\pi)=\int_0^\pi \sin(x)\,dx
				% =\left[-\cos(x)\right]_0^\pi
				= \bigl(-\cos\pi\bigr)-\bigl(-\cos 0\bigr)
				=1-(-1)=2.
			\]
		
\end{Solution}
\begin{Solution}{9}
			When using the change of variable $u = 1 + x^2$,
			we must also change the differential $du = 2x\,dx$,
			which conveniently contains $x$ that appears in the numerator,
			which allows us to write:
			\[
				\int_{x=0}^{x=1} \frac{4x}{(1+x^2)^3}\,dx
				=  \int_{x=0}^{x=1}  \frac{2}{u^3}\,du
				= \int_{x=0}^{x=1} 2  u^{-3}\,du.
			\]
			Next we must change the $x$-limits of integration
			to $u$-limits of integration:
			The lower limit $x = 0$ becomes $u = 1 + 0^2 = 1$,
			and the upper limit $x = 1$ becomes $u = 1 + 1^2 = 2$,
			which the complete substitution:
			\[
				\int_{x=0}^{x=1} \frac{4x}{(1+x^2)^3}\,dx
				=
				2 \int_{u=1}^{u=2} u^{-3}\,du.
			\]
			We can now proceed using the integral rule
			$\int x^n \,dx = \frac{1}{n+1}x^{n+1} + C$
			to obtain
			\begin{align*}
				2 \int_{u=1}^{u=2} u^{-3}\,du
					&=	2 \left[ \frac{u^{-2}}{-2} \right]_{1}^{2}	
						= - \left[ u^{-2} \right]_{1}^{2}
						= - \left[ \frac{1}{u^2} \right]_{1}^{2}		\\
					&=	-\left( \frac{1}{2^2} - \frac{1}{1^2} \right)
						= -\left( \frac{1}{4} - 1 \right)				
						= \frac{3}{4}
			\end{align*}
		
\end{Solution}
\begin{Solution}{10}
			We can use the formula for the geometric series
			$\sum_{k=0}^{\infty} r^k=\frac{1}{1-r}$
			with $r=\frac{2}{3}$,
			which gives us $\sum_{k=0}^\infty  \left(\frac{2}{3}\right)^{\!k} = \frac{1}{1- \frac{2}{3}} =  3$.
		
\end{Solution}
\begin{Solution}{11}
			Start from the known Taylor series $e^x=\sum_{k=0}^{\infty}\frac{x^k}{k!}$.
			Substitute $-x$ for $x$ to get
			\[
				f(x) =\sum_{k=0}^{\infty}\frac{(-x)^k}{k!}
				     =\sum_{k=0}^{\infty}\frac{(-1)^k x^k}{k!}.
			\]
		
\end{Solution}
