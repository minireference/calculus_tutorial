%!TEX root = ../../noBSstats.tex


\section{Sets}
\label{mathematical_preliminiaries:sets}

We'll start by some definitions of \emph{sets} and set operations.
Sets are important because they're used to represent random events.
Many rules of probability theory are direct consequences of the underlying rules for sets,
so if you understand set theory you automatically understand 80\% of probability theory.


	A \emphindexdef{set} is the mathematical notion that describes a group of objects.

	\subsection{Definitions}
	
		\begin{itemize}
		    \item   \emph{set}: a collection of mathematical objects
		    \item   $S,T$: the usual variable names for sets
		    \item   $\emptyset$: the empty set. The empty set is a set that contains no elements.
		    \item   $s \in S$: this statement is read ``$s$ is an element of $S$'' or ``$s$ is in $S$''
		    \item   $\mathbb{N}, \mathbb{Z}, \mathbb{Q}, \mathbb{R}$: some important number sets:
		    		the naturals, the integers,  the rationals, and the real numbers, respectively.
		    \item   $\emptyset$: the \emph{empty set} is a set that contains no elements.
		    		Mathematicians adopted the symbol $\emptyset$ because the notation $\{ \, \}$ is confusing.
		    \item   $\{ \textrm{  definition  } \}$: the curly brackets surround the definition of a set,
		    		and the expression inside the curly brackets describes what the set contains.
		    \item   $S\cup T$: the \emph{union} of two sets.													\index{set!union|textit}
		    		The union of $S$ and $T$ corresponds to the elements in either $S$ or $T$.
		    \item   $S \cap T$: the \emph{intersection} of the two sets.											\index{set!intersection|textit}
		    		The intersection of $S$ and $T$ corresponds to the elements that are in both $S$ and $T$.
		    \item   $S \setminus T$: \emph{set difference} or \emph{set minus}.									\index{set!difference|textit}
		    		The set difference $S \setminus T$ corresponds to the elements of $S$ that are not in $T$.
		    \item	$A^c$: the \emph{complement} of set $A$
		\end{itemize}
		Sets and set operations are used in probability theory to define random events.

		An \emphindexdef{interval} is a subset of the real line.
		We denote an interval by specifying its endpoints and surrounding them with
		either square brackets ``$[$'' or round brackets ``$($'' to indicate whether or not the corresponding endpoint is included in the interval.
		\begin{itemize}
		    \item   	$[a,b]$: the \emph{closed} interval from $a$ to $b$.
		    		This corresponds to the set of numbers between $a$ and $b$ on the real line,
				including the endpoints $a$ and $b$. $[a,b] \equiv \{ x\in \mathbb{R}\ | \ a \leq x \leq b \}$.
		    \item   	$(a,b)$: the \emph{open} interval from $a$ to $b$. 
		    		This corresponds to the set of numbers between $a$ and $b$ on the real line, 
				\emph{not} including the endpoints $a$ and $b$. $(a,b) \equiv \{ x\in \mathbb{R}\ | \ a < x < b \}$.
		    \item   	$[a,b)$: the half-open interval that includes the left endpoint $a$ but not the right endpoint $b$.
		    		$[a,b) \equiv \{ x\in \mathbb{R}\; | \; a \leq x < b \}$.
		    \item   	$[a,b] \cup [c,d]$: the union of the two intervals,
		    		which is the set of numbers \emph{either} between $a$ and $b$ (inclusive) \emph{or} between $c$ and $d$ (inclusive).
		\end{itemize}
		Intervals are used to describe the probability events for continuous random variables.

		
		The \emph{union} of two sets is the set defined as the set
		\[
		   A\cup B = \{x \ | \ x\in A \text{ or } x\in B\}.
		\]
		% The union operation corresponds to the logical \code{OR} operator.
	
		The \emph{intersection} of two sets is the set defined as
		\[
		   A\cap B = \{x \ | \ x\in A \text{ and } x\in B\}.
		\]
		The set intersection $A \cap B$ describes the elements that are part of both sets.
		% The intersection operation corresponds to the logical \code{AND} operator.

		The \emph{set difference} of two sets is defined as
		\[
		   A \setminus B = \{x \ | \ x\in A \text{ and } x\notin B\}.
		\]
		The set difference $A \setminus B$ describes all the elements that are in $A$ but not in $B$.


		Although sets are purely mathematical constructs and they have no ``shape,''
		we can draw \emph{Venn diagrams} to visualize relationships between sets and different subsets.
		Venn diagrams are useful for visualizing the subsets obtained from set operations.
		Figure~\ref{fig:set_operations} illustrates the set union $A \cup B$,
		the set intersection $A \cap B$, and the set difference $A \setminus B$,
		for two sets $A$ and $B$.

		\begin{figure}[htb]
		\vspace{-2mm}
		\centering
		\includegraphics[width=0.9\textwidth]{figures/prob/set_operations_union_intersection_difference.pdf}
		\vspace{-2mm}
		\caption{Venn diagrams showing different subsets obtained using the set operations:
				set union $A \cup B$, set intersection $A \cap B$, and set difference $A \setminus B$.}
		\label{fig:set_operations_union_intersection_difference}
		\end{figure}
		
		

		Consider now some set $A$ contained in another set $\Omega$, denoted $A \subseteq \Omega$.
		The \emph{complement} of a set $A$ in $\Omega$ is defined as
		\[
			A^c = \{x\in \Omega \ | \ x \notin A\}.
		\]
		
		By definition, $\emptyset^c=\Omega$ and $\Omega^c=\emptyset$.
		The union of any set with the empty set is the set itself, $A\cup \emptyset=A$.
		The intersection of any set with the empty set is the empty set, $A \cap \emptyset=\emptyset$.
		


		\begin{figure}[htb]
		\vspace{-2mm}
		\centering
		\includegraphics[width=0.33\textwidth]{figures/prob/set_operations_complement.pdf}
		\vspace{-2mm}
		\caption{The complement of the set $A$ is denoted $A^c$ and includes all elements of $\Omega$
				that are not in $A$.}
		\label{fig:set_operations_complement}
		\end{figure}







\emph{DeMorgan's Law} describes two important equations that apply to the complements of unions and intersections of sets:		
\[
	(A\cup B)^c = A^c \cap B^c
	\qquad
	\textrm{and}
	\qquad
	(A\cap B)^c = A^c \cup B^c.
\]
See Figure~\ref{fig:set_operations_deMorgans_law} for an illustration.

\begin{figure}[htb]
\vspace{-2mm}
\centering
\includegraphics[width=0.7\textwidth]{figures/prob/set_operations_deMorgans_law.pdf}
\vspace{-2mm}
\caption{DeMorgan's laws provide an alternative ways for computing the complements of unions and intersections of sets.
	Try to picture how each of the shaded ways can be constructed in two different ways.}
\label{fig:set_operations_deMorgans_law}
\end{figure}

%1. A∪∅=A, A∩∅=∅ 2.A∪S=S, A∩S=A
%3. A∪Ac =S,A∩Ac =∅
%4. (Ac)c = A
%5. The Commutative Property:
%A∪B=B∪A, A∩B=B∩A
%6. The Associative Property:
%(A∪B)∪C=A∪(B∪C), (A∩B)∩C=A∩(B∩C)
%7. The Distributive Property:
%A∪(B∩C)=(A∪B)∩(A∪C), A∩(B∪C)=(A∩B)∪(A∩C)

		% TODO: more rules and tricks from https://en.wikipedia.org/wiki/Algebra_of_sets
		% TODO: mention duality https://en.wikipedia.org/wiki/Algebra_of_sets#The_principle_of_duality







		\paragraph{Example}
			Consider the three sets $A=\{a,b,c\}$, $B=\{b,c,d\}$, and $C=\{c,d,e\}$.
			Using set operations, we can define new sets, such as
			\[
				A \cup B = 	\{a,b,c,d\},
				\quad 
				A \cap B =  	\{b,c\},
				\quad
				\textrm{and} 
				\quad
				A \setminus B =  \{a\},
			\]
			which correspond to elements in either $A$ or $B$,
			the set of elements in $A$ and $B$,
			and the set of elements in $A$ but not in $B$, respectively.
			
			We can also construct expressions involving three sets:
			\[
				A \cup B \cup C = \{a,b,c,d,e\},
				\qquad
				\quad
				A \cap B \cap C = \{c\}.
			\]
			And we can write more elaborate set expressions, like $(A \cup B ) \setminus C = \{a,b\}$,
			which is the set of elements that are in $A$ or $B$ but not in $C$.
			Another example of a complicated set expression is $(A \cap B) \cup (B\cap C) = \{b,c,d\}$,
			which describes the set of elements in both $A$ and $B$ or in both $B$ and $C$.


	\bigskip
		
	\noindent
	In probability theory we define the outcomes of random experiments in terms of sets.
	All the set formulas shown above will prove to be very useful when defining probability events and doing probability calculations.

	If this is the first time you're learning about sets,
	I'll need you to work through the following exercises to get some practice with these concepts.
	Solving this set of exercises about sets and set operations will set you up for success!

		TODO: import exercises
		TODO: add exercises with complements



