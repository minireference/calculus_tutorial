%!TEX root = ../../noBSstats.tex


\section{Functions}
\label{mathematical_preliminiaries:functions}

	The probability distribution of the random variable $X$ is represented by a mathematical functions $f_X$.
	Any knowledge and prior experience with manipulating math functions will therefore prove to be very useful for studying probability distributions.
	The math vocabulary and notation for math functions applies equally well to probability distribution functions.


	\subsection{Definitions}
	\label{functions:definitions}
	
		A \emphindexdef{function} is a mathematical object that takes numbers as inputs and produces numbers as outputs.
		We use the notation
		\[
		  f \colon A \to B
		\]
		to denote a function from the input set $A$ to the output set $B$.
		For every input $x$, the output value of $f$ for that input is denoted $f(x)$.

		We'll now define some fancy technical terms used to describe the input and output sets of functions.
		\begin{itemize}
		    \item   	The \emphindexdef{domain} of a function is the set of input values for which the function is defined.
		    \item   	The \emphindexdef{image} or \emph{range} of the function $f$ is the set of all possible output values of the function.
		\end{itemize}

		\begin{figure}[htb]
		\centering
		\includegraphics[width=0.4667\textwidth]{figures/math/functions-definition.pdf}
		\caption{	An abstract representation of a function $f$ from the set $A$ to the set $B$.
				The function $f$ is the arrow which \emph{maps} each input 
				$x$ in $A$ to an output $f(x)$ in $B$. 
				The output of the function $f(x)$ is also denoted $y$.}
		%\label{fig:}
		\end{figure}

		\noindent
		A function is a \emph{mapping} from numbers to numbers.
		We say ``$f$ maps $x$ to $f(x)$,''
		and use the following terminology to classify the type of mapping that a function performs: 
		\begin{itemize}
		    \item   	A function is \emph{one-to-one} or \emphindexdef{injective} if it maps different inputs to different outputs. 
		    \item   	A function is \emph{onto} or \emphindexdef{surjective} if it covers the entire output set 
		    		(in other words, if the image of the function is equal to the function's codomain).
		    \item   	A function is \emphindexdef{bijective} if it is both injective and surjective. 
		    		In this case, $f$ is a \emph{one-to-one correspondence} between the input set and the output set: 
				for each of the possible outputs $y \in Y$ (surjective part), 
		    		there exists exactly one input $x \in X$, such that $f(x)=y$ (injective part).
		\end{itemize}

		\noindent
		The term \emph{injective} is an allusion from the 1940s inviting us to picture the actions of
		injective functions as pipes through which numbers flow like fluids. 
		Since a fluid cannot be compressed, the output space must be at least as large as the input space. 
		A modern synonym for injective functions is to
		say they are \href{http://gowers.wordpress.com/2011/10/11/injections-surjections-and-all-that/}{\emph{two-to-two}}. 
		If we imagine two specks of paint floating around in the ``input fluid,'' 
		an injective function will contain two distinct specks of paint in the ``output fluid.''
		In contrast, non-injective functions can map several different inputs to the same output. 
		For example $f(x)=x^2$ is not injective since the inputs $2$ and $-2$ are both mapped to the output value $4$.
	
	
		\subsubsection{Function composition}
		
			We can combine two simple functions by chaining them together to build a more complicated function.
			This act of applying one function after another is called \emph{function composition}.
			Consider for example the composition:
			\[
			  f\!\circ\!g \, (x) \equiv  f\!\left( \: g(x) \: \right) = z.
			\]
	
			\begin{figure}[htb]
			\centering
			\includegraphics[width=0.4\textwidth]{figures/math/functions-composition.pdf}
			\vspace{-2mm}
			\caption{The function composition $f\!\circ\!g$ describes the combination of first applying the function $g$,
					followed by the function $f$: $f\circ g\,(x) \equiv f(g(x))$.}
			\label{fig:functions-composition}
			\end{figure}
	
			Figure~\ref{fig:functions-composition} illustrates this concept.
			First, the function $g:A\to B$ acts on some input $x$ to produce an intermediary value $y=g(x)$ in the set $B$.
			The intermediary value $y$ is then passed through the function $f:B \to C$ to produce
			the final output value $z=f(y) = f(g(x))$ in the set $C$.
			We can think of the \emph{composite function} $f \circ g$ as a function in its own right.
			The function $f\circ g: A \to C$ is defined through the formula $f\circ g\,(x) \equiv f(g(x))$.
	
			Don't worry too much about the $\circ$ symbol---it's just a convenient math notation I wanted you to know about.
			Writing $f\circ g$ is just as good as writing $f(g(x))$.
			The important takeaway from Figure~\ref{fig:functions-composition} is that
			functions can be combined by using the outputs of one function as the inputs to the next.
			This is a very useful idea for building math models.
			You can understand many complicated input-output transformations by describing them as compositions of simple functions.
	
	
		\subsubsection{Inverse function}
	
			Recall that a \emph{bijective} function is a one-to-one correspondence between
			a set of input values and a set of output values.
			For every input value $x$,
			there is exactly one corresponding output value $y$.
			This means that we can start from any output value $y$ and find the corresponding input value $x$ that produced it.
	
			\begin{figure}[htb]
			\centering
			\includegraphics[width=0.4333\textwidth]{figures/math/functions-inverse.pdf}
			\caption{The inverse $f^{-1}$ undoes the operation of the function $f$.}
			\label{fig:functions-inverse}
			\end{figure}
			
			Given a bijective function $f:A \to B$,
			there exists an inverse function $f^{-1}:B \to A$,
			which performs the \emph{inverse mapping} of $f$.
			If you start from some $x$, apply $f$, and then apply $f^{-1}$,
			you'll arrive---full circle---back to the original input $x$:
			\[
			 f^{-1}\!\big( \; f(x) \; \big) \equiv f^{-1}\!\circ\! f \;(x) = x.
			\]
			In Figure~\ref{fig:functions-inverse} the function $f$ is represented as a forward arrow,
			and the inverse function $f^{-1}$ is represented as a backward arrow
			that puts the value $f(x)$ back to the $x$ it came from.
	


	\subsection{Functions reference}
	
		Your \emph{function vocabulary} determines how well you can express yourself mathematically					\index{function}
		in the same way that your English vocabulary determines how well you can express yourself in English.
		The following sections aims to embiggen your function vocabulary.

		\subsubsection{Line}
		\label{function_reference:line}
		
			The equation of a line describes an input-output relationship
			where the change in the output is \emph{proportional} to the change in the input.
			\[
				f(x) = mx+b.
			\]	
			The constant $m$ describes the slope of the line.
			The constant $b$ is called the $y$-intercept and it corresponds to the value of the function when $x=0$.

			\begin{itemize}
			    \item 	Domain: $x\in \mathbb{R}$. \\
			    		The function $f(x)=mx+b$ is defined for all inputs $x \in \mathbb{R}$.
			    \item 	Image: $x\in\mathbb{R}$ if $m\neq 0$. 
			    		If $m=0$ the function is constant $f(x)=b$, so the image set contains only a single number $\{ b\}$.
			    \item 	$x=-b/m$: the $x$-intercept of $f(x)=mx+b$. The $x$-intercept is obtained by solving $f(x)=0$.
			    \item	A unique line passes through any two points $(x_1,y_1)$ and $(x_2,y_2)$ if $x_1 \neq x_2$.
			    \item	The inverse to the function $f(x)=mx+b$ is $f^{-1}(x)=\frac{1}{m}(x-b)$, which is also a line.
			    \item 	$Ax + By = C$: the \emph{general} equation of a line.
			    		Given the general equation of a line $Ax + By = C$,
					you can obtain the function form $y=f(x)=mx+b$ by setting $b=\frac{C}{B}$ and $m=\frac{-A}{B}$.
			\end{itemize}

			\begin{figure}[H]
			\centering
			\includegraphics[width=\textwidth]{figures/math/graph_of_2x_plus_3.pdf}
			\caption{	The graph of the function $f(x)=2x-3$.
					The slope is $m=2$.
					The $y$-intercept of this line is at $y=-3$. The $x$-intercept is at $x=\frac{3}{2}$.}
			\label{fig:graph_of_2x_plus_3}
			\end{figure}
	
	
		\subsubsection{Square}
		\label{function_reference:square}
	
			The function $x$ \emph{squared},
			is also called the \emph{quadratic} function, or \emphindexdef{parabola}.		\index{quadratic|textit}
			The formula for the quadratic function is
			\[
			  f(x)=x^2.
			\]
			The name ``quadratic'' comes from the Latin \emph{quadratus} for square,
			since the expression for the area of a square with side length $x$ is $x^2$.
	
			\begin{figure}[htb]
			\centering
			\includegraphics[width=\textwidth]{figures/math/quadratic_func.pdf}
			\caption{Plot of the quadratic function $f(x)=x^2$.
				The graph of the function passes through the following $(x,y)$ coordinates:
				$(-2,4)$, $(-1,1)$, $(0,0)$, $(1,1)$, $(2,4)$, $(3,9)$, etc.
				}
			\label{fig:quadratic_func}
			\end{figure}
	
			\begin{itemize}
				    \item 	Domain: $x\in\mathbb{R}$. \\
					    	The function $f(x)=x^2$ is defined for all input values $x \in \mathbb{R}$.
				    \item 	Image: $f(x)\in[0,\infty)$. \\
						The outputs are never negative: $x^2 \geq 0$, for all $x\in \mathbb{R}$.
				    \item 	The function $x^2$ is the inverse of the square root function $\sqrt{x}$.	     
				    		% for positive numbers $x$
						%	    $f(x)=x^2$ is the inverse function of $f(x)=\sqrt{x}$.		
				    \item 	$f(x)=x^2$ is \emph{two-to-one}: 
				    		it sends both $x$ and $-x$ to the same output value $x^2=(-x)^2$.
				    \item 	The quadratic function is \emph{convex}, meaning it curves upward.
				    \item	A general quadratic function has the form $f(x) = A(x-h)^2 + k$,
				    		where $A$ is a vertical scaling factor,
						$h$ is a horizontal displacement,
						and $k$ is a vertical displacement of the function.
			\end{itemize}



		\subsubsection{Square root}
		\label{function_reference:square_root}
		
			The square root function is denoted
			\[
			 f(x) = \sqrt{x} \equiv x^{\frac{1}{2}} .
			\]
			The square root $\sqrt{x}$ is the inverse function of the square function $x^2$ for $x\geq 0$.
			The symbol $\sqrt{c}$ refers to the \emph{positive} solution of $x^2=c$.
			Note that $-\sqrt{c}$ is also a solution of $x^2=c$.

			\begin{figure}[htb]
			\centering
			\includegraphics[width=0.9333\textwidth]{figures/math/sqrt_of_x.pdf}
			\caption{	The graph of the function $f(x)=\sqrt{x}$.
					The domain of the function is $x \in [0,\infty)$.
					You can't take the square root of a negative number.
				    }
			\label{fig:sqrt_of_x}
			\end{figure}
	
	
			\begin{itemize}
				    \item Domain: $x \in [0,\infty)$\,. \\
						The function $f(x)=\sqrt{x}$ is only defined for nonnegative inputs  $x \geq 0$.					\index{nonnegative}
						There is no real number $y$ such that  $y^2$ is negative, hence the function $f(x)=\sqrt{x}$ 
						is not  defined for negative inputs $x$.		    
				    \item Image: $f(x) \in [0,\infty)$\,.	\\ 
						The outputs of the function $f(x)=\sqrt{x}$ are never negative: 
						$\sqrt{x} \geq 0$, for all $x\in [0,\infty)$\,.
				    \item The \emph{cube} root function $f(x) = \sqrt[3]{x}  \equiv x^{\frac{1}{3}}$,
				    		is the inverse function for the cubic function $f(x)=x^3$.
				    \item The $n$\textsuperscript{th}-root function $\sqrt[n]{x} \equiv x^{\frac{1}{n}}$ is the inverse function of $x^n$.
			\end{itemize}



	\subsubsection{Absolute value}
	\label{function_reference:absolute_value}
		
		The \emphindexdef{absolute value} function tells us the size of numbers
		without paying attention to whether the number is positive or negative.
		\[
		  f(x)=|x|=  \left\{  \begin{array}{rl}  x &\text{ if } x\geq 0, \\  -x &\text{ if } x<0. \end{array} \right.
		\]
		We can compute a number's absolute value by \emph{ignoring the sign} of the number.
		A number's absolute value corresponds to its distance from the origin of the number line.							\index{origin}

		\begin{figure}[htb]
		\centering
		\includegraphics[width=\textwidth]{figures/math/absolute_value_of_x.pdf}
		\caption{The graph of the absolute value function $f(x)=|x|$.}
		%\label{fig:}
		\end{figure}

		\begin{itemize}
		    \item   	The absolute value $|x|$ is always a nonnegative number											\index{nonnegative}
		    \item   	The combination of squaring followed by square-root
		    		is equivalent to the absolute value function $\sqrt{x^2} \equiv |x|$,
				since squaring destroys the sign.
		\end{itemize}





	\subsubsection{Polynomials}
	\label{function_reference:polynomials}
																							\index{polynomial|textit}
		The general equation for a polynomial function of degree $n$ is written,
		\[
		  f(x)=a_0 + a_1x + a_2x^2 + a_3x^3 + \cdots + a_nx^n.
		\]
		The constants $a_i$ are known as the \emph{coefficients} of the polynomial.
		We call $a_0$ the constant term.														\index{term}
		We call the number $a_1$ the \emph{linear} coefficient, or \emph{first-order} coefficient.						\index{linearity}
		The coefficient $a_2$ is the \emph{quadratic} coefficient,
		$a_3$ the \emph{cubic} coefficient,
		and $a_n$ the $n$\textsuperscript{th} order coefficient.
		The largest power of $x$ that appears in the polynomial is called the \emph{degree} of the polynomial.
			
		 \begin{itemize}
			    \item 	Domain: $x\in\mathbb{R}$. 
				    	Polynomials are defined for all inputs $x \in \mathbb{R}$.
			    \item	The roots of $f(x)$ are the values of $x$ for which $f(x)=0$.
			    \item 	The image of a polynomial function depends on the coefficients.
			    \item 	A polynomial of degree $n$ has  $n+1$ coefficients: $a_0,a_1,a_2,\ldots, a_n$.
			    \item 	The sum of two polynomials is also a polynomial.
			    \item	To solve the polynomials equation $A(x)=B(x)$,
			    		you can rewrite it as $A(x)-B(x)=0$,
					and use one of the standard formulas.
					\begin{itemize}
						\item		The solution to $mx + b = 0$ is $x=\frac{-b}{m}$ (just move $b$ to the other side and divide by~$m$).
						\item		For a second-degree polynomial,														\index{quadratic}
								$ax^2 + bx + c = 0$,
								the solutions are $x_1=\frac{-b + \sqrt{ b^2 -4ac}}{2a}$ and $x_2=\frac{-b - \sqrt{b^2-4ac}}{2a}$.
								If $b^2-4ac < 0$,
								the solutions will involve taking the square root of a negative number.
								In those cases, we say no real solutions exist.
						\item		To solve polynomial equations of degree $3$ or higher we recommend using a computer algebra system
								like \texttt{SymPy}: \texttt{\href{http://live.sympy.org}{http://live.sympy.org}}.
								To solve the equation $C(x)=0$,
								call the function \texttt{solve(expr,var)},
								where the expression \texttt{expr} corresponds to $C(x)$,
								and \texttt{var} is the variable you want to solve for.
								For example,
								to solve $x^3-3x^2-4x=0$,
								use the command \code{solve(x**3-3*x**2-4*x, x)}
								and you'll obtain the output \code{[-1, 0, 4]},
								which tells you roots of this polynomial are $x=-1$, $x=0$, and $x=4$.					
					\end{itemize}
		\end{itemize}

		Depending on the choice of degree $n$ and coefficients $a_0$, $a_1$, $\ldots$, $a_n$,
		Consider the following observations about the symmetries of polynomials:
		\begin{itemize}
			\item
			If a polynomial contains only even powers of $x$,
			like $f(x)=1+x^2-x^4$ for example,
			we call this polynomial \emph{even}.														\index{function!even|textit}
			Even polynomials have the property $f(x)=f(-x)$.
			The sign of the input doesn't matter.
			\item
			If a polynomial contains only odd powers of $x$,  for example $g(x)=x+x^3-x^9$,
			we call this polynomial \emph{odd}.														\index{function!odd|textit}
			Odd polynomials have the property $g(x)=-g(-x)$.
			\item
			If a polynomial has both even and odd terms then it is neither even nor odd.
		\end{itemize}
		
		\noindent
		The terminology of \emph{odd} and \emph{even} applies to functions in general and not just to polynomials.
		All functions that satisfy $f(x)=f(-x)$ are called \emph{even functions},
		and all functions that satisfy $f(x)=-f(-x)$ are called \emph{odd functions}.



	\subsubsection{Exponential}
	\label{function_reference:exponential}
		
		The exponential function base $e=2.7182818\ldots$ is denoted										\index{exponential|textit}
		\[
		  f(x)=e^{x} \equiv \exp(x).
		\]

		\begin{figure}[htb]
		\centering
		\includegraphics[width=0.833\textwidth]{figures/math/exp_x.pdf}
		\caption{The graph of the exponential function $f(x)=e^x$
				passes through the following $(x,y)$ coordinates:
				$(-2,\frac{1}{e^2})$, 
				$(-1,\frac{1}{e})$, 
				$(0,1)$, 
				$(1,e)$, 
				$(2,e^2)$, 
				$(3,e^3=20.08\ldots)$, 
				$(5,148.41\ldots)$, and 
				$(10,22026.46\ldots)$.
				%			$(20,485165195.41\ldots)$, and 
				%			$(27,532048240601.80\ldots)$.
			}
		\label{fig:exp_x}
		\end{figure}

	
		\begin{itemize}
		    \item 	Domain: $x\in \mathbb{R}$
		    \item 	Image: $e^x \in (0, \infty)$  
		    \item	$f(a)f(b)=f(a+b)$ since $e^ae^b=e^{a+b}$
		    \item   	The derivative (the slope of the graph) of the exponential function 
		    		is the exponential function: $f(x) = e^x  \ \ \Rightarrow \ \ f'(x)=e^x$
				%	The function $e^x$ is the only function which is equal to its own derivative: $f(x)=f'(x)$.
		    \item   	A more general exponential function would be $f(x)=Ae^{\gamma x}$,
				where $A$ is the initial value
				and $\gamma$ (the Greek letter \emph{gamma}) is the \emph{rate} of the exponential.
				For $\gamma > 0$, the function $f(x)$ is increasing, as in Figure~\ref{fig:exp_x}.
				For $\gamma < 0$, the function is decreasing and tends to zero for large values of $x$.
		\end{itemize}




	\subsubsection{Natural logarithm}
	\label{function_reference:natural_logarithm}

		The natural logarithm function is denoted														\index{logarithm|textit}
		\[
		  f(x)=\ln(x) = \log_e(x).
		\]
		The function $\ln(x)$ is the inverse function of the exponential $e^x$.

		\begin{figure}[htb]
		\centering
		\vspace{-4mm}
		\includegraphics[width=0.9\textwidth]{figures/math/ln_of_x.pdf}
		\vspace{-1mm}		
		\caption{The graph of the function $\ln(x)$
				passes through the following $(x,y)$ coordinates:
				$(\frac{1}{e^2},-2)$, 
				$(\frac{1}{e},-1)$, 
				$(1,0)$, 
				$(e,1)$, 
				$(e^2,2)$, 
				$(e^3,3)$, 
				$(148.41\ldots,5)$, and 
				$(22026.46\ldots,10)$.
		}
		\label{fig:ln_of_x}
		\end{figure}

		\begin{itemize}
		    \item 	Domain: $x\in(0, \infty)$  
		    \item 	Image: $\ln(x) \in \mathbb{R}$.
		    \item	$\ln(x) + \ln(y) = \ln(xy)$ since $e^{a+b}=e^ae^b$.
		\end{itemize}
	


	\subsection{Function transformations}
	\label{function_reference:function_transformations}
		
		We can adjust the shape of a function by scaling it or moving it,
		so that it passes through certain points.
		Here are four useful transformations you can perform on \emph{any} function $f$ to obtain a transformed function $g$:

		\begin{itemize}[topsep=2pt,itemsep=0.5pt]
		    \item   Vertical translation: $g(x) = f(x)+k$
		    \item   Horizontal translation: $g(x) = f(x-h)$
		    \item   Vertical scaling: $g(x) = Af(x)$ 
		    \item   Horizontal scaling: $g(x) = f(ax)$
		\end{itemize}
		By applying these transformations,
		we can \emph{move} and \emph{stretch} a generic function to give it any desired shape.






