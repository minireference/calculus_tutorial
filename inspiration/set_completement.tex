		Consider now some set $A$ contained in another set $\Omega$, denoted $A \subseteq \Omega$.
		The \emph{complement} of a set $A$ in $\Omega$ is defined as
		\[
			A^c = \{x\in \Omega \ | \ x \notin A\}.
		\]
		
		By definition, $\emptyset^c=\Omega$ and $\Omega^c=\emptyset$.
		The union of any set with the empty set is the set itself, $A\cup \emptyset=A$.
		The intersection of any set with the empty set is the empty set, $A \cap \emptyset=\emptyset$.
		


		\begin{figure}[htb]
		\vspace{-2mm}
		\centering
		\includegraphics[width=0.33\textwidth]{figures/prob/set_operations_complement.pdf}
		\vspace{-2mm}
		\caption{The complement of the set $A$ is denoted $A^c$ and includes all elements of $\Omega$
				that are not in $A$.}
		\label{fig:set_operations_complement}
		\end{figure}







\emph{DeMorgan's Law} describes two important equations that apply to the complements of unions and intersections of sets:		
\[
	(A\cup B)^c = A^c \cap B^c
	\qquad
	\textrm{and}
	\qquad
	(A\cap B)^c = A^c \cup B^c.
\]
See Figure~\ref{fig:set_operations_deMorgans_law} for an illustration.

\begin{figure}[htb]
\vspace{-2mm}
\centering
\includegraphics[width=0.7\textwidth]{figures/prob/set_operations_deMorgans_law.pdf}
\vspace{-2mm}
\caption{DeMorgan's laws provide an alternative ways for computing the complements of unions and intersections of sets.
	Try to picture how each of the shaded ways can be constructed in two different ways.}
\label{fig:set_operations_deMorgans_law}
\end{figure}

%1. A∪∅=A, A∩∅=∅ 2.A∪S=S, A∩S=A
%3. A∪Ac =S,A∩Ac =∅
%4. (Ac)c = A
%5. The Commutative Property:
%A∪B=B∪A, A∩B=B∩A
%6. The Associative Property:
%(A∪B)∪C=A∪(B∪C), (A∩B)∩C=A∩(B∩C)
%7. The Distributive Property:
%A∪(B∩C)=(A∪B)∩(A∪C), A∩(B∪C)=(A∩B)∪(A∩C)

		% TODO: more rules and tricks from https://en.wikipedia.org/wiki/Algebra_of_sets
		% TODO: mention duality https://en.wikipedia.org/wiki/Algebra_of_sets#The_principle_of_duality


