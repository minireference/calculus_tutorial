%!TEX root = ../../noBSstats.tex


\section{Calculus}
\label{mathematical_preliminiaries:calculus}

	Calculus allows you to describe how functions change over time (derivatives),
	and to calculate the total amount of a quantity that accumulates over a time period (integrals).
	The language of calculus will allow you to speak precisely about the properties of probability density functions
	(Section~\ref{sec:continuous_prob_distr}) and better understand their behaviour.

	This section presents a bird's-eye view of the core concepts of calculus.
	We'll talk about derivatives, integrals, and also introduce some formulas for computing sums of sequences.


	\subsection{Definitions}
	
		\begin{itemize}

		    \item   	$\mathbb{R}$: the set of real numbers

		    \item   	$f(x)$:  a function of the form $f: \mathbb{R} \to \mathbb{R}$,
		    		which means $f$ takes real numbers as inputs and produces real numbers as outputs

		    \item   $\lim_{\delta \to 0}$: a limit expression in which the number $\delta$ tends to zero

		    \item   $f'(x)$: the derivative of $f(x)$ is the rate of change of $f$ at $x$:
		    		\[
				  f'(x) = \lim_{\delta \to 0} \frac{f(x+\delta)\ - \ f(x)}{\delta}\,.
			   	\] 
				The derivative is a function of the form $f': \mathbb{R} \to \mathbb{R}$.


		    \item   	$A(a,b)$: the value of the \emph{area} under the curve $f(x)$ from $x=a$ until $x=b$.
		    		The area $A(a,b)$ is computed as the following integral
		    		\[
			        	   A(a,b) = \int_a^b f(x)\;dx.
			    	\] 
			    	The $\int$ sign stands for \emph{sum}.
			    	Indeed, the integral is the ``sum'' of $f(x)$ for all values of $x$ between $a$ and $b$.
				
		    \item   	$F(c) \equiv A(0,c)$: the \emph{integral function} of $f(x)$.
		    		The integral function corresponds to the computation of the area under $f(x)$
				as a function of the upper limit of integration:
			       	\[
			        	   F(c) \equiv A(0,c) = \int_0^{c}\! f(u)\:du.
			      	\]
				The choice of $x=0$ as the lower limit of integration is arbitrary.

		\item		$F(x)+C$: The \emph{antiderivative} function of the function $f(x)$.								\index{antiderivative}
				An antiderivative function is defined as a function whose derivative equals to $f(x)$.
				The antiderivative function always includes an additive constant $C$.
				If the function $F(x)$ is an antiderivative (obeys $F'(x)=f(x)$)
				then the function $F(x)+C$ is also an antiderivative since  
				\[ \frac{d}{dx}\!\left[   F(x) + C \right] = f(x), \]
				for any constant $C$.
				
				
		\item		The fundamental theorem of calculus (FTC) states that
				the integral function $F(c)$ is the same as the antiderivative function $F(x)$
				up to an additive constant $C$:
				\[
				   A(0,x) \overset{\scriptscriptstyle {\textrm{FTC}}}{=} F(x)+C.
				\]
				The fundamental theorem leads us to the following formula for computing the area $A(a,b)$:
				\[
				   A(a,b) 
				 	= A(0,b) - A(0,a) 
				 	= F(b) - F(a).
				\]
				The area under the curve, $A(a,b)$, is equal to the change in the antiderivative function $F(x)$ between $x=a$ and $x=b$.

		    \item   	$\mathbb{N} \equiv \{0, 1, 2, 3, 4, 5, 6, \ldots \}$: the set of natural numbers

		    \item   	$\mathbb{N}_+ \equiv \mathbb{N} \setminus \{0\}  \equiv \{1, 2, 3, 4, 5, 6, \ldots \}$: 
		    		the set of positive natural numbers

		    \item   	$a_k$: a sequence of numbers $(a_0, a_1, a_2, a_3, a_4, \ldots)$

		    \item   	$\sum$: sum. This symbol indicates taking the sum of several objects grouped together. 
				The summation sign is the short way to express certain long expressions: 								\index{summation}
				\[
			      	    a_3 + a_4 + a_5 + a_6 + a_7
					= \sum_{3 \leq k \leq 7} a_k
					= \sum_{k=3}^7 a_k.
				\]

		    \item	$\sum a_n$: the \emph{series} $a_n$ is the sum of all terms in the sequence $a_n$:						\index{term}
			    	\[
				    S_\infty = \sum_{k=1}^\infty a_n  = a_1 + a_2 + a_3 + a_4 + a_5 + a_6 + \cdots.
			    	\]

		    \item   	$n!$: the \emph{factorial} function $n!=n(n-1)(n-2)\cdots 3\cdot2\cdot1$,
		    		if $n\geq1$. We define $0!=1$.

		    \item   	$f(x)=\sum_{k=0}^\infty c_k x^k$: the \emph{Taylor series} approximation of the function $f(x)$.				\index{approximation}
				It has the form of an infinitely long polynomial $c_0 + c_1x^1 + c_2x^2 + c_3x^3 + \ldots$
				where the coefficients $c_k$ are chosen so as to encode the properties of the function $f(x)$.

		\end{itemize}



		\subsection{Download example}
		
			Suppose you're downloading a large file to your computer.
			At $t=0$ you click ``save as'' in your browser and the download starts.
			Let $f(t)$ represent the size of the downloaded data.
			At any time $t$,
			the function $f(t)$ tells you the amount of disk space taken by the partially-downloaded file.
			You are downloading a $720$[MB] file, so the download progress at time $t$
			corresponds to the fraction $\frac{f(t)}{720 \text{[MB]}}$.

			\subsubsection{Download rate}
			\label{introduction:download_rate}

				The derivative function $f^\prime(t)$,
				pronounced ``\:\!\!$f$ prime,''
				describes how the function $f(t)$ changes over time.
				In our example $f^\prime(t)$ is the download speed.
				If your downloading speed is  $f'(t)=100$[kB/s],
				then the file size $f(t)$ must increase by 100[kB] each second.
				If you maintain this download speed, the file size will grow at a constant rate:
				$f(0)=0$[kB], $f(1)=100$[kB], $f(2)=200$[kB], $\ldots$, $f(100)=10$[MB].

				To calculate the ``estimated time remaining'' until the download's completion,
				we divide the amount of data that remains to be downloaded by the current download speed:
				\[
				  \text{time remaining } = \frac{ 720 - f(t) }{ f^\prime(t) } \quad [\textrm{s}].
				\]
				The bigger the derivative, the faster the download will finish.
				If your internet connection were 10 times faster, the download would finish 10 times more quickly.

			\subsubsection{Inverse problem}
			\label{introduction:inverse_problem}
	
				Let's consider this situation from the point of view of the modem that connects your computer to the internet.
				Any data you download comes through the modem.
				The modem knows the download rate $f^\prime(t)$[kB/s] at all times during the download.

				However, since the modem is separate from your computer,
				it does not know the file size $f(t)$ as the download progresses.
				Nevertheless,
				the modem can infer the file size at time $t$ from knowing the transmission rate $f^\prime(t)$.
				The integral of the download rate between $t=0$ and $t=\tau$
				corresponds to the total amount of downloaded data stored on your computer.
				During this download period, the change in file size is described by the integral
				\[
				   \Delta f = f(\tau)-f(0) = \int_0^\tau \! f'(t)\: dt\,.
				\]
				Assuming the file size starts from zero $f(0)=0$[kB] at $t=0$,
				the modem can use the integration procedure to find $f(\tau)$,
				the file size on your computer at $t=\tau$:
				\[
				  f(\tau) = \int_0^\tau\! f^\prime(t) dt\,.
				\]
				The download rate $f'(t)$ is measured in [kB/s],
				and each time step $dt$ is $1$[s] long,
				so the data downloaded during one second is $f'(t)dt$[kB].
				The file size at time $t=\tau$ is equal to the sum of the data
				downloaded during each second from $t=0$ until $t=\tau$.			

				The integral $\int_a^b q(t)\:dt$ is the calculation of the \emph{total}
				of some quantity $q(t)$ that accumulates during the time period from $t=a$ to $t=b$.
				Integrals are necessary any time you want to calculate the total of a quantity that changes over time.


	\subsection{Derivatives}
	\label{mathematical_preliminiaries:derivatives}

		Consider the function $f(x)$,
		which takes real numbers as inputs and produces real numbers as outputs, $f: \mathbb{R} \to \mathbb{R}$.
		The input variable for the function $f$ is usually denoted $x$.
		The function's output is denoted $f(x)$ and is identified with the $y$-coordinate in graphs.

		The \emph{derivative} function																		\index{derivative}
		describes the \emph{rate of change} of the function $f(x)$.
		For example,
		the constant function $f(x)=c$ has derivative $f'(x)=0$ since the function $f(x)$ does not change at all.

		Graphically speaking,
		the derivative function describes the \emph{slope} of the graph of the function $f(x)$.
		The derivative of a line $f(x)=mx+b$ is $f'(x)=m$ since the slope of this line is equal to $m$.
		In general, the slope of a function is different at different values of~$x$.
		For a given choice of input $x=x_0$, the value of the derivative function $f'(x_0)$ is equal 
		to the slope of $f(x)$ as it passes through the point $(x_0,f(x_0))$.




%\input{02_prob/prereqs/derivative_impl.tex}



















	\subsection{Integrals}
	\label{mathematical_preliminiaries:integrals}

		The integral of the function $f(x)$ between $x=a$ and $x=b$, denoted $\int_a^b f(x)\, dx$,
		corresponds to the concept of ``total amount of $f$ accumulated between $x=a$ and $x=b$.''
		The notion of an integral in foundational for understanding continuous random variables
		since computing probabilities of random events are defined as integral calculations.

		Graphically speaking,
		this integral corresponds to the area under the curve of the graph of $f(x)$ between the vertical lines at $x=a$ and $x=b$:
		\[
		A(a,b) = \int_a^b f(x) \: dx.
		\]
		We refer to the numbers $a$ and $b$ as the \emph{limits of integration},
		and the notation $\int_a^b f(x)\:dx$ is shorthand for $\int_{x=a}^{x=b} f(x)\: dx$.
		The $\int$ sign comes from the Latin word \emph{summa} for sum.
		Indeed,
		the integral is the ``sum'' of the values of $f(x)$ between the two limits of integration.

		\begin{figure}[htb]
		\centering
		\includegraphics[width=0.35\textwidth]{figures/calculus/integral_as_region_under_curve_Aab.pdf}
		\vspace{-2mm}
		\caption{The integral of the function $f(x)$ between $x=a$ and $x=b$ corresponds to the area $A(a,b)$.}
		\label{fig:integral_as_region_under_curve_Aab_repeat}
		\end{figure}



		\subsubsection{Integral function}
		
			The \emph{integral function} $F(c)$ corresponds to the area calculation $A(0,c)$ as a function of the upper limit of integration:
			\[
			   F(c) \equiv A(0,c) = \int_0^c \! f(x)\:dx\,.
			\]
			There are two variables and one constant in this formula.
			The input variable $c$ describes the upper limit of integration.
			The \emph{integration variable} $x$ performs a sweep from $x=0$ until $x=c$.
			The constant $0$ describes the lower limit of integration.
			Note that choosing $x=0$ for the starting point of the integral function was an arbitrary choice,
			and we obtain another integral function $F_a(c)=\int_a^c \! f(x)\:dx$  if we use $x=a$ as the starting point.

			The integral function $F(c)$ contains the ``precomputed'' information about the area under the graph of $f(x)$.
			The derivative function $f'(x)$ tells us the "slope of the graph" property of the function $f(x)$ for all values of $x$.
			Similarly,
			the integral function $F(c)$ tells us the "area under the graph" property of the function $f(x)$ for \emph{all} possible limits of integration.

			The area under $f(x)$ between $x=a$ and $x=b$ is obtained by calculating the \emph{change} in the integral function as follows:
			\[
			A(a,b) = \int_a^b \! f(x)\:dx
			 =  F(b)-F(a).
			\]

			\begin{figure}[H]
			\centering
			\includegraphics[width=0.966\textwidth]{figures/calculus/integral_as_difference_off.pdf}
			\caption{  The integral function $F(x)=A(0,x)$ computes the area under the curve $f(x)$ starting from $x=0$.
					The area under $f(x)$ between $x=a$ and $x=b$ is computed using the formula $A(a,b)=F(b)-F(a)$,
					which describes the change in $F(x)$ between $x=a$ to $x=b$.
					}
			\label{fig:integral_as_difference_off}
			\end{figure}






\input{02_prob/prereqs/integral_impl.tex}








		\subsubsection{Properties of integrals}
		
			In this section we'll state some important properties of integrals that you can use to simplify and solve integral calculations.
	
			\paragraph{Integrals are signed area calculations}
	
				The value of an integral can be either positive or negative.
				If the limits of integration $a$ and $b$ satisfy $a < b$ ($b$ is to the right of $a$ on the number line),
				and if $f(x) > 0$ (meaning $f(x)$ is a positive function),
				then the area under the curve will be positive:
				\[
				 A(a,b) = \int_{a}^{b} f(x) \; dx 	\ \ > \ \  0.
				\]
				For a function $g(x) < 0$,
				the integral from $a$ to $b$ corresponds to a negative area.
				In general, if $f(x)$ is above the $x$-axis in some places,
				these zones contribute positively to the total area under the curve;
				places where $f(x)$ is below the $x$-axis contribute negatively to the total area $A(a,b)$.
	
				We can also obtain a negative area if we swap the limits of integration.
				Suppose we have $f(x)>0$, and limits of integration $a$ and $b$ such that $a < b$.
				If we start integrating at $x=b$ and stop integrating at $x=a$,
				the area under the curve will be negative:
				\[
				    A(b,a) = \int_{b}^{a} f(x) \; dx 	\ \ < \ \  0.
				\]
				The function $f(x)$ is positive but each integration step $dx$ is \emph{negative},
				since we're moving from right to left.
		
				Integrals are \emph{signed} areas.
				Changing the direction of integration changes the sign of the integral:
				\[
				 A(b,a)
				 = \int_{b}^{a} f(x) \ dx 
				 = - \int_{a}^{b} f(x) \ dx 
				 = - A(a,b).
				\]
	
	
			\paragraph{Additivity}
	
				The integral from $a$ to $b$ plus the integral from $b$ to $c$
				is equal to the integral from $a$ to $c$:
				\begin{shadebox}
				\vspace{-2mm}
				\[
				  A(a,b) + A(b,c) = 
				 \int_a^b f(x) \; dx + \int_b^c f(x) \; dx  =
				  \int_a^c f(x) \; dx =
				  A(a,c).
				\]
				\vspace{-2mm}
				\end{shadebox}	
	
	
			\paragraph{Linearity}
			
				Integration is a \emph{linear} operation, meaning													\index{linearity}
				\begin{shadebox}
				\vspace{1mm}
				\[
				 \int [\alpha f(x) + \beta g(x)]\; dx 
				 =
				 \alpha  \int f(x)\; dx
				 + 
				 \beta \int g(x)\; dx,
				\]
				\vspace{-2mm}
				\end{shadebox}	
				for arbitrary constants $\alpha, \beta$.
	
				Recall the derivative is also a linear operation:
				\[
				[\alpha f(x) + \beta g(x)]'
				 =
				 \alpha  f'(x)
				  + 
				 \beta g'(x).
				\]
				Thus, we can say the operations of calculus as a whole are \emph{linear} operations.
				This property is really cool, 
				because it allows us to break down complicated problems into smaller chunks.
	

	


% %!TEX root = ../calculus_tutorial.tex

\section{Multivariable calculus}
\label{mathematical_preliminiaries:multivariable_calculus}

		Multivariable calculus is the extension of the ideas of differential and integral calculus
		to functions like $f(x,y)$ that depend on multiple input variables.
		You can plot a function $f: \mathbb{R} \times \mathbb{R}  \to \mathbb{R}$ as a \emph{surface},
		where the height $z$ of the surface above the point $(x,y)$ is function output $z=f(x,y)$.

		If you know single-variable calculus (derivatives and integrals),
		then then you won't have much new math to learn in multivariable calculus:
		it's essentially the same concepts but with more variables.

			
		\subsection{Partial derivatives}
			
			For a function of two variables $f(x,y)$,
			there is an ``$x$-derivative'' operator $\frac{\partial}{\partial x}$
			and a ``$y$-derivative'' operator $\frac{\partial}{\partial y}$.
			The operation $\frac{\partial}{\partial x}f(x,y)$ describes taking the derivative of $f(x,y)$ with respect to the input variable $x$,
			while keeping the input  variable $y$ constant.
			Taking the derivative of a multivariable function with respect to one of its input variables is called a \emph{partial derivative}
			and denoted with the symbol $\partial$.

			The partial derivative of $f(x,y)$ with respect to $x$ is
			\[
				\frac{\partial}{\partial x}f(x,y)
				\equiv
				\frac{\partial f}{\partial x}
				\equiv
				\lim_{ \delta \rightarrow 0}	\frac{f(x+\delta, y)-f(x,y)}{\delta}.
			\]
			Similarly the partial derivative of with respect to $y$ is
			\[
				\frac{\partial}{\partial y}f(x,y)
				\equiv
				\frac{\partial f}{\partial y}
				\equiv
				\lim_{ \delta \rightarrow 0}	\frac{f(x, y+\delta)-f(x,y)}{\delta}.
			\]
			Note that both $\frac{\partial f}{\partial x}$  and $\frac{\partial f}{\partial y}$ are function of $x$ and $y$.
			Indeed, we can ask the questions ``what is the slope in the $x$-direction''
			and ``what is the slope in the $y$-direction'' at any point $(x,y)$ on the surface of the function.
			That's precisely the information returned by the functions $\frac{\partial f}{\partial x}(x,y)$ and $\frac{\partial f}{\partial y}(x,y)$.

			TODO: example
			
			
		\subsection{Gradient}
		\label{mathematical_preliminiaries:gradient}


			The operator $\nabla$ is a combination of both the $x$ and $y$ derivatives:
			\[
				\nabla f(x,y)
				\equiv
				\left(
					\frac{\partial f}{\partial x},
					\frac{\partial f}{\partial y}
				\right).
			\]
			Note that $\nabla$ acts on a function $f(x,y)$ to produce a vector output.
			This vector is called the \emph{gradient} vector and it tells you the combined $x$- and $y$-slopes of the surface.
			More specifically,
			the gradient vector tells you the direction of the function's maximum increase---the
			``uphill'' direction at the surface of graph of $f(x,y)$ at the point $(x,y)$.
			
			
			\paragraph{Mountain map}
				Suppose the height of a mountain is described by the function $h(x,y)$.
				The coordinates $(x,y)$ tell us the horizontal position point in the $xy$-plane
				and the value of the function $h(x,y)$ represents the height of the mountain at those coordinate.

				We identify the $z$ coordinate with the hight of the mountain $z=h(x,y)$
				and graph the function $h(x,y)$ is as a surface in 3D as illustrated in Figure~\ref{fig:multivariable_caclulus_3d_surface_plot}.

				\begin{figure}[htb]
				\centering
				\includegraphics[width=0.3\textwidth]{figures/calculus/multivariable_caclulus_3d_surface_plot.pdf}
				\vspace{-2mm}
				\caption{The 3D surface plot of the the function $h(x,y)$.}
				\label{fig:multivariable_caclulus_3d_surface_plot}
				\end{figure}

				\noindent			
				Three dimensional surface plots are very good for visualizing multivariable functions,
				but they can be difficult to draw by hand.

				Another approach for representing function of the form $h(x,y)$ is to use a two-dimensional plot that shows the ``view from above'' of the surface $h(x,y)$.
				We can use colour to represent the height of the function through different sharing:
				darker-shading to represent large values of $h(x,y)$ and lighter-shading to represent small values of $h(x,y)$.			% TODO: update plot to show shading
				We can also trace \emph{level curves} in the plot,
				which is the same approach used for topographic maps:
				each level curve show the points that are at a certain height.
				
				\begin{figure}[htb]
				\centering
				\includegraphics[width=0.3\textwidth]{figures/calculus/multivariable_caclulus_topographic_map.png}
				\vspace{-4mm}
				\caption{The topographic map that shows the height of the function $h(x,y)$ using sharing to represent height.
						The level curves at each 10m intervals are also shown.}
				\label{fig:multivariable_caclulus_topographic_map}
				\end{figure}
				
				\noindent
				The curve labeled 30m line you see in Figure~\ref{fig:multivariable_caclulus_topographic_map} represents
				the solution to the equation $30= h(x,y)$,
				where $h(x,y)$ is the height of this hill for all coordinates $(x,y)$ on the map.
				
				Recall that the gradient vector $\nabla f(x,y)$ at any point $(x,y)$ tells you which way is ``uphill'' on the surface,
				and by extension,
				the negative of the gradient vector points ``downhill.''
				The gradient vector is always perpendicular to the \emph{level curve} at that point.

				The notion of an uphill or downhill direction for the surface $h(x,y)$ turns out to be very useful for optimization.
				If you want to find the local maximum of a function,
				you can start at some point and keep moving uphill (in the direction of $\nabla f(x,y)$ and you'll arrive at a local peak of the mountain.
				Similarly,
				to find the lowest point on the surface (minimum value of $h(x,y)$),
				you can start at some point and keep moving in the opposite direction to the gradient $-\nabla f(x,y)$.

				Figure~\ref{fig:multivariable_caclulus_topographic_map_waterflow_to_bottom} illustrates this process.
				Consider the path of a water stream whose source in some arbitrary point on the surface of the mountain.
				The water stream will naturally move downhill
				and descend the slope of the mountain until it reaches the minimum at the bottom of a valley.
				This intuitive notion of ``keep moving downhill until you get to a local minimum'' is
				the general idea behind the \emph{gradient descent} optimization algorithm which is very important for machine learning applications.
							
				\begin{figure}[htb]
				\centering
				\includegraphics[width=0.3\textwidth]{figures/calculus/multivariable_caclulus_topographic_map_waterflow_to_bottom.png}
				\vspace{-4mm}
				\caption{This graph shows the path taken by a hypothetical water as it flows to the bottom of the valley $h(x,y)$.}
				\label{fig:multivariable_caclulus_topographic_map_waterflow_to_bottom}
				\end{figure}

				We know we've reached the bottom of the valley,
				since the the gradient vector will be zero at the minimum of the function $h(x,y)$,
				since surface is locally flat there.



		\subsection{Multivariable integrals}
		
			The multivariable generalization of the integral $\int_{x \in I} f(x) \, dx$
			that computes the ``total'' amount of $f(x)$ on some interval $I=[a,b]$
			is the multivariable integral of the form:
			\[
				\int \! \int_{(x,y) \in R} f(x,y) \, dxdy,
			\]
			where $R$ is called the \emph{region of integration} and corresponds to some subset of the cartesian plane $\mathbb{R} \times \mathbb{R}$.
			The idea behind multivariable integrals is the same as for single variable integrals---to compute the total amount of some function for some range of input values.
			For single-variable integrals,
			we split the region into thing rectangular strips of width $dx$.
			For double integrals we split the two-dimensional region of integration into small squares of area $dxdy$,
			and compute the total volume of a many vertical columns whose base area is $dxdy$
			and whose height is given by the function $f(x,y)$.
			
			TODO: insert graphic of 3D integral split into vertical columns
			
			TODO: explain "sweep along x then sweep along y" idea + hint at change-of-variables techniques
			\vspace{1in}









