%!TEX root = ../calculus_tutorial.tex


	\subsection{Act 2: Integrals as functions}

		The \emph{integral function} $F_0(b)$
		corresponds to the area calculation with a variable upper limit of integration $A_f(0,b)$:
		\[
			F_0(b) \; \eqdef \; A_f(0,b) = \int_{x=0}^{x=b} \! f(x)\:dx\,.
		\]
		%	Depending on the choice of the variable $b$,
		%	the integral function $F_0(b)$ describes the results of the integral calculation
		%	starting at $x=0$ and going until $x=b$.
		As a matter of convention,
		we denote the integral function using the capital of the letter used to denote the original function.
		% The \emph{integration variable} $x$ performs a sweep from $x=0$ until $x=b$
		%
		Choosing $x=0$ for the starting point of the integral function is an arbitrary choice.
		We can obtain another integral function if we use $x=a$ as the starting point,
		$F_a(b) \eqdef \int_a^b \! f(x)\,dx$.
		The integral functions $F_a$ and $F_0$ differ only by a constant term:
		$F_0(b) = F_a(b) + C$, where $C = \int_{x=0}^{x=a} f(x)\,dx$.

		The integral function $F_0(b)$
		contains the ``precomputed'' information about the area under the graph of $f(x)$.
		Knowing $F_0$ allows us to compute the area under $f(x)$ between $x=a$ and $x=b$
		as the \emph{change} in the integral function:
		\[
			A_f(a,b) = \int_a^b \! f(x)\,dx	=  F_0(b) - F_0(a).
		\]
		Intuitively,
		this formula computes the area $A_f(a,b)$ as the difference between the areas of two regions:
		the area until $x=b$ minus the area until $x=a$,
		as illustrated in Figure~\ref{fig:integral_as_difference_off}.

		\begin{figure}[htb]
			\centering
			\includegraphics[width=0.95\columnwidth]{figures/calculus/integral_as_difference_off.pdf}%
			\vspace{-3mm}
			\caption{	The area under $f(x)$ between $x=a$ and $x=b$
					is computed using the formula $A_f(a,b)=F_0(b)-F_0(a)$,
					which describes the change in the output of $F_0(x)$ between $x=a$ and $x=b$.}
			\label{fig:integral_as_difference_off}
		\end{figure}


%TODO: warn there is no general F for any f
%only for certain special cases
%have exact symbolic formula
%for all other cases
%we're forced to do the split-into-vertical-strips --- i.e. there is no analytical shortcut.



		\subsubsection{Example 1 revisited}

			We can easily find the integral function for the constant function $f(x)=~3$
			because the region under the curve is rectangular.
			Choosing $x=0$ as the starting point,
			we obtain the integral function $F_0(b)$
			that corresponds to the area under $f(x)$ between $x=0$ and $x=b$
			as follows:
			\[ 
				F_0(b) = A_f(0,b) = \int_0^b \! f(x)\,dx	= 3 b.
			\]
			The integral function corresponds to the area of a rectangle of height $3$ and with width $b$,
			as shown in Figure~\ref{fig:simple_integral_function_fx_eq_3}.

			\begin{figure}[htb]
				\centering
				\includegraphics[width=0.7\columnwidth]{figures/calculus/simple_integral_function_fx_eq_3.pdf}
				\vspace{-3mm}
				\caption{	The integral function of the function $f(x) = 3$ is $F_0(b)=3b$.}
				\label{fig:simple_integral_function_fx_eq_3}
			\end{figure}

			Knowing the function $F_0(b)$ allows us to compute the area under the graph of $f(x)$
			between $x=0$ and $x=5$ as the difference
			$A_f(0,5) = F_0(5) - F_0(0) = 3\cdot 5 - 3\cdot 0 = 15$.


		\subsubsection{Example 2 revisited}

			Consider now the area under the graph of the line $g(x)=x$,
			starting from $x=0$.
			Since the region is triangular,
			we can compute its area using the formula for the area of a triangle:
			base times height divided by two.
			The integral function of $g(x)$ is:
			\[
				G_0(b) = A_g(0,b) = \int_0^b g(x) \, dx = \tfrac{1}{2} ( b \cdot b ) = \tfrac{1}{2}b^2.
			\]
			% The area of a triangle with base $b$ and height $b$ is equal to $\frac{1}{2}b^2$.

			\begin{figure}[htb]
				\centering
				\includegraphics[width=0.63212\columnwidth]{figures/calculus/simple_integral_function_gx_eq_x.pdf}
				\vspace{-3mm}

				\caption{	The integral function of the function $g(x) = x$ is $G_0(b)=\tfrac{1}{2}b^2$.}
				\label{fig:simple_integral_function_gx_eq_x}
			\end{figure}

			Knowing the function $G_0(b)$ allows us to compute the area under the graph of $g(x)$
			between $x=0$ and $x=5$ as the difference
			$A_g(0,5) = G_0(5) - G_0(0) = \frac{1}{2}5^2 - \frac{1}{2}0^2 = 12.5$.

	
		%	But don't worry,
		%	you don't need to take an integral calculus to learn statistics.
		%	What is important right now is that you understand the concept of integration.
		%	The integral of a function is the area under the graph of the function,
		%	which is in some sense the total amount of the function accumulated during some time period.


		\subsubsection{Example 3 revisited}
		
			The the area under $h(x) = 4 - x^2$ from $x=0$ until $x=b$
			is described by the following integral calculation:
			\[
				H_{0}(b) = A_h(0,b) = \int_{0}^b h(x) \, dx = \; ???.
			\]
			We were able to compute the integral functions $F_0(b)$ and $G_0(b)$
			thanks to the simple geometries of the areas under the graphs,
			but $h(x)$ is a curve so it requires some new integration methods.
			In the next few pages,
			we'll learn about symbolic integration techniques
			that will allow us to find the integral function $H_{0}(b)$.

			\begin{figure}[htb]
				\centering
				\includegraphics[width=0.9\columnwidth]{figures/calculus/integral_function_hx_eq_x.pdf}
				% TODO: redo figure using usetex=True
				\vspace{-3mm}
				\caption{	The integral of the function $h(x) = 4 - x^2$ from $x=0$ to $x=b$.}
				\label{fig:integral_function_hx_eq_x}
			\end{figure}







	\subsection{Intermission}
	
		The integral in Example~3 was chosen
		to motivate the need for more advanced methods for integration.
		Is there a math formula that describes the area $A_h(0,b)$
		of the region shown in Figure~\ref{fig:integral_function_hx_eq_x}?
		We previously used numerical methods to compute the particular area $A_h(0,2)$ for $b=2$,
		but now we're looking for a general math formula that computes the integral function $H_{0}(b) = A_h(0,b)$ for any $b$.		
		In the next section,
		we'll learn about the \emph{fundamental theorem of calculus},
		which will allow us to find the exact formula for $H_{0}(b)$.

		The section is like the intermission in the calculus show.
		%	I want you, dear reader,
		%	to have a break before the next pages where the level of ``mathyness'' will increase substantially.
		As in a real-world intermission,
		this is also your chance to skip the rest of the show.
		Perhaps you have better things to do right now than learning about advanced calculus concepts.
		I won't get offended---no worries!
		Feel free to skip ahead to Section~\ref{integrals:scipy}
		for more technical details about high-performance numerical integration,
		or jump straight to Section~\ref{sec:sequences_and_series} to learn about sequences and series.
		%which is the third foundational topics in calculus.

		As a teacher,
		I'm happy that you know
		that integrals compute areas under the graph of functions,
		and can be approximated numerically using Riemann sums.
		These are the two key ideas related to integrals,
		so I feel I've done my job!
		I think you should stay though---you might enjoy the \emph{knowledge buzz} moments
		that are coming your way in the next few pages.		
		My calculus teacher in college
		described the realization you get after understanding the fundamental theorem of calculus
		as similar to the feeling you get when smoking some of that ``funny stuff.''
		%	Differential calculus and integral calculus are two sides of the same coin.
		%	If you understand why the theorem is true, 
		%	you will understand something very deep about calculus. 
		%	Differentiation is the inverse operation of integration.

		Still with me?
		Okay,
		with your consent,
		let's continue with the calculus show.


	

