%!TEX root = ../calculus_tutorial.tex


	\subsection{Act 4: Techniques of integration}
	\label{integrals:techniques_of_integration}

		Okay we're getting into the fourth act of the calculus show,
		and I want you to remind you that you can ``tap out'' at any time.
		The material in this act is some of the most boring stuff.
		% that  only for students who are currently taking CALC~II class.
		If you're taking a integral calculus class, %  (a.k.a. CALC 102, Integral calculus),
		then you need to know this stuff because it is going to be your final exam.
		Everyone else,
		feel free to skip ahead to the next section.		% MAYBE: explicit FWD reference

		There are a bunch of tricks that extend the reach of analytical integration methods 
		(anti-differentiation) to more complicated functions.
		We don't have space to discuss all these tricks in this tutorial,
		but we'll show the two most important tricks.

		\subsubsection{Substitution trick}

			Suppose the function we want to integrate has the structure $f(u(x))u^{\prime\!}(x)$,
			which consists of inner function wrapped in an outer function multiplied by the derivative of the inner function.
			We can use the \emph{substitution trick} to rewrite this integral
			in terms of the function $f(u)$ using $u$ as the variable of integration:
			\[
				\int_{x = a}^{x = b} f(u(x)) \, u^{\prime\!}(x) \, dx
					\;\; = \;\; 	\int_{u = u_a}^{u = u_b} f(u) \, du.			
			\]

			\noindent
			The substitution trick is sometimes called \emph{change of variable},
			since we're replacing the variable $x$ with the variable $u$,
			just like the the search-and-replace operation in a text editor.
			Because we're doing the substitution ``inside'' an integral operation,
			we must change the limits integration (form $a$ and~$b$ to $u_a$ and $u_b$),
			and also change the ``step'' parameter (from $dx$ to $du$).

			\begin{shadebox}
			\vspace{1mm}
			\noindent
			Follow these three steps to apply the substitution trick:
			\begin{enumerate}
				\item	Replace all occurrences of $u(x)$ with $u$.
				\item	Replace $dx$ with $\frac{1}{u^{\prime\!}(x)}du$.
				\item	Replace the $x$-limits of integration $x=a$ and $x=b$
					with $u$-limits of integration: $u_a = u(a)$ and $u_b = u(b)$.
			\end{enumerate}
			\vspace{-1mm}
			\end{shadebox}	


			\paragraph{Example}

				Let's compute the integral $\int_a^b \frac{1}{x - \sqrt{x}} \, dx$.
				This looks like a scary formula,
				but we can use the substitution trick to compute this integral. 
				% ALT. becomes more manageable 
				We'll apply the substitution $u=\sqrt{x}$,
				which implies $u^{\prime\!}(x) =  \frac{1}{2\sqrt{x}}$,
				and $dx = 2\sqrt{x}\,du = 2u\,du$.
				The new limits of integration are $u_a = \sqrt{a}$ and $u_b = \sqrt{b}$.

				Performing the three steps of the substitution trick gives us:
				\[
					\int_{x=a}^{x=b} \!\! \frac{1}{x - \sqrt{x}} \, dx 
						=  \int_{u=u(a)}^{u=u(b)} \!\!\frac{1}{u^2 - u}  2u\,du
						=  \int_{u=\sqrt{a}}^{u=\sqrt{b}} \!\frac{1}{u^2 - u}  2u\,du.
				\]

				\noindent
				We're simply doing the search-and-replace on $u = \sqrt{x}$,
				but to do this right we need to also replace $dx$ with $du$,
				and use the new limits of integration.

				We can now simplify the expression inside the integral:
				\[
					\int_{\sqrt{a}}^{\sqrt{b}} \frac{1}{u^2 - u} \; 2u\,du
					= \int_{\sqrt{a}}^{\sqrt{b}}  \frac{2}{u - 1} \, du
					= 2 \int_{\sqrt{a}}^{\sqrt{b}}  \frac{1}{u - 1} \, du.
				\]
				The function inside the integral,
				$f(u) = \frac{1}{u-1}$,
				is similar to the inverse function $f(u) = \frac{1}{u}$
				whose integral function is $\ln(u)$.
				Accounting for the $-1$ horizontal shift inside the function,
				leads us to the integral formula $\int \frac{1}{u - 1} \, du = 2\ln(u-1)$,
				which leads us to the following final answer:
				\[
					2 \int_{\sqrt{a}}^{\sqrt{b}}  \frac{1}{u - 1} \, du
					\; = \; 2\ln(\sqrt{b}-1) - 2\ln(\sqrt{a}-1).
				\]

				\noindent
				I know this sequence of steps went quickly,
				but and there are a lot of integral symbols,
				but if you read each step carefully,
				you'll see we're just doing search-and-replace.



			The substitution trick for integrals comes from the chain rule for derivatives
			$\left[ f(u(x)) \right]^\prime = f^\prime(u(x))u^\prime(x)$.
			We can use substitution only when computing the integral
			of a function that has the special structure $f^{\prime\!}(u(x))u^{\prime\!}(x)$.

			% TODO: mention it has great power -- basis of multivariable calculus when using parameterization FWD REF


		\subsubsection{Integration by parts}

			The integration by parts trick can be used
			when the function we're integrating is the product of two factors,
			$\int f(x)g^{\prime\!}(x)\, dx$,
			where $f(x)$ is some arbitrary function,
			and $g^{\prime\!}(x)$ is the derivative of some other function.
			

			\begin{shadebox}
			\vspace{-1mm}
			\[
				\int_a^b \!\! f(x)g^{\prime\!}(x)\, dx
					\; = \;	
					\big[f(b)g(b) - f(a)g(a)\big] \; -  \; \int_a^b \!\! f^{\prime\!}(x)g(x) \, dx.
			\]
			\vspace{-2mm}
			\end{shadebox}	


			\paragraph{Example}

				Let's calculate $\int_0^5 x e^x \, dx$ using the integration by parts procedure.
				The expression $x e^x$ consists of two factors: $x$ and $e^x$.
				We'll identify $x$ with $f(x)$ and $e^x$ as $g^{\prime\!}(x)$.
				This means $f^{\prime\!}(x) = 1$
				and $g(x)  = \int g^{\prime\!}(x)\,dx = e^x$.
				We now know all the parts we need to apply the integration by parts formula:
				\begin{align*}
					\int_0^5 \underbrace{x}_{f(x)} \underbrace{e^x}_{g^{\prime\!}(x)} \, dx
						&= \big[f(5)g(5)  - f(0)g(0)\big]	- \int_0^5 \!\! f^{\prime\!}(x)g(x) \, dx			\\[-4mm]
						&= \big[ 5 e^5 \qquad \; - 0e^0 \big] \qquad \;\;\; - \int_0^5 1 \cdot e^x \, dx  	\\
						&= \big[ 5 e^5 - 0e^0 \big] - \big[  e^5 - e^0 \big] 						\\
						&= 5 e^5 - e^5 + 1  = 4e^5 + 1.
				\end{align*}

			%	\begin{align*}
			%		 &= \left( x e^x \right) \Big|_0^5  \ - \  \int_0^5 e^x \; dx  		\\
			%		 &= \left( x e^x \right) \Big|_0^5      \ - \  e^x\Big|_0^5			\\[1mm]
			%		 &=  \left[ 5 e^5 - 0e^0 \right]    \ - \  \left[  e^5 - e^0 \right] 	\\
			%		 &= 5 e^5 - e^5 + 1 								\\
			%		 &= 4e^5 + 1 .
			%	\end{align*}

			%	It is easier to remember the integration by parts formula in its shorthand notation,
			%	$\int u\; dv = uv - \int v\; du$.
			%	You can think of integration by parts as a form of ``double substitution,''
			%	where you simultaneously replace $u$ and $dv$.
			%	For definite integrals,
			%	integration by parts requires evaluating the product of the functions at the limits:
			%	\[
			%	 \int_a^b u\; dv = \left(uv\right)\Big|_a^b \ \  - \ \ \int_a^b v \; du.
			%	\]



		\subsubsection{Other tricks}

			Substitution and integration by parts
			are only two of the multitude of integration techniques.			
			%	Mathematicians and physicist have come up with hundreds of formulas and tricks
			%	for calculating integrals,
			%	since for centuries people were forced to use pen and paper calculations.			
			There are tricks for trigonometric functions, square roots,
			fractions that that involve $x^2 + a^2$, etc.
			There is an entire course,
			called integral calculus,
			that is dedicated to learning integration tricks.
			If you want to pursue advanced studies in physicis or engineering,
			you should definitely take this course to learn more integration tricks.
			Section~5.15 in \textbf{No Bullshit Guide to Math and Physics} is a good introduction to the material.

			Despite all the formulas and integration techniques,
			not all function are \emph{integrable}.
			There are many functions that don't have an antiderivative function,
			and hence no ``closed form'' integral function.
			For example,
			the function $f(x) = e^{-x^2}$ doesn't have an antiderivative:
			there is no function $F(x)$ such that $F^{\prime\!}(x) = e^{-x^2}$.
			For such functions,
			we can't use the ``reverse engineering'' analytical shortcut
			to find the integral,
			and we must use numerical integration (the split-the-area-into-thin-rectangular-strips procedure).
			Speaking of which...
			%	but there is no 
			%	that allows us to compute areas using the integral function.
			
%			it is possible that integral not to have 
%	MAYBE: 	warn there is no general F for any f
%			only for certain special cases have exact symbolic formula
%			for all other cases we're forced to do the 
%			i.e. 









