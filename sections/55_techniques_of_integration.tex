%!TEX root = ../calculus_tutorial.tex


	\subsection{Act 4: Techniques of integration}
	
% (A4)
Okay we're getting into the fourth act of the calculus show,
and I want you to remind you that you can ``tap out'' at any time.
The material we'll show % (briefly and superficially)
in the in this act is some of the most boring topics
that CALC~II students are forced to endure.
% that  only for students who are currently taking CALC~II class.
If you're taking a CALC~II class (a.k.a. CALC 102, Integral calculus),
then you need to know this stuff because it is going to be your final exam.
Everyone else,
feel free to skip ahead to the next section.		% MAYBE: explicit FWD reference





		There are a bunch of tricks that extend the reach of analytical integration methods 
		(anti-differentiation) to more complicated functions.
		We don't have space to discuss all these tricks in this tutorial,
		so we'll focus on the two most important ones.
		% MAYBE link to Sec 5.15 in the book 


		\subsubsection{Substitution trick}
		
			Suppose the function we want to integrate has the structure $f(u(x))u'(x)$,
			which consists of inner function wrapped in an outer function multiplied by the derivative of the inner function.
			We can use the \emph{substitution trick} to rewrite this integral
			in terms of the function $f(u)$ using $u$ as the variable of integration:
			\[
				\int_{x = a}^{x = b} f(u(x)) \, u'(x) dx
					\;\; = \;\; 	\int_{u = u_a}^{u = u_b} f(u) \; du.
					% \;\; \text{where} \;\; u = u(x).
				%	\int_{x=a}^{x=b} f(u(x)) \; u'(x) \; dx
				%		&=  \int_{u(a)}^{u(b)} f(u) \; du.					
			\]
			
			\noindent
			The substitution trick is also called a \emph{change of variable} operation:
			we're replacing the variable $x$ with the variable $u$,
			similar to a search-and-replace operation when editing a text file.
			Because we're doing the substitution ``inside'' an integral operation,
			we must also change the region of interval of integration (form $[a,b]$ to $[u_a, u_b]$)
			and change of the ``step'' parameter (from $dx$ to $du$).

			\begin{shadebox}
			\vspace{1mm}
			\noindent
			Follow these three steps to apply the substitution trick:

			\begin{enumerate}
				\item	Replace $dx$ with $\frac{1}{u'(x)}du$.
				\item	Replace all occurrences of $u(x)$ with $u$.
				\item	Replace the $x$-limits of integration $x=a$ and $x=b$
					with $u$-limits of integration: $u_a = u(a)$ and $u_b = u(b)$.
			\end{enumerate}
			\vspace{-1mm}
			\end{shadebox}	

			\paragraph{Example}

				Let's compute the integral $\int_a^b \frac{1}{x - \sqrt{x}} \, dx$.
				This looks like a scary integral,
				but we can use the substitution trick to compute this integral. 
				% ALT. becomes more manageable 
				We'll apply the substitution $u=\sqrt{x}$,
				which implies $u'(x) =  \frac{1}{2\sqrt{x}}$,
				and the new limits of integration $u_a = \sqrt{a}$ and $u_b = \sqrt{b}$.
				Performing the three steps of the substitution trick gives us:

				{\allowdisplaybreaks
				\begin{align*}
				     \int_{x=a}^{x=b} \frac{1}{x - \sqrt{x}} \, dx  
					&=  \int_{x=a}^{x=b}  \frac{1}{x - \sqrt{x}} \, \frac{1}{2\sqrt{x}} \, du			\\
					&=  \int_{x=a}^{x=b} \frac{1}{u^2 - u} \; 2u\,du							\\
					&=  \int_{u(a)}^{u(b)} \frac{1}{u^2 - u} \; 2u\,du
						= \int_{u(a)}^{u(b)} \frac{2u}{u^2 - u} \, du							\\
					&=  \int_{u(a)}^{u(b)} \frac{2}{u - 1} \, du
						=  2\ln(u-1) \bigg|_{u(a)}^{u(b)}									\\
					% &=  2\ln(\sqrt{x}-1) \bigg|_{x=a}^{x=b}								\\
					&=  2\ln(\sqrt{b}-1) - 2\ln(\sqrt{a}-1).
				\end{align*}
				}

				\noindent
				Read this sequence of equations slowly
				and try to identify the what changes in each step.
				I know it looks like alien scribbles,
				but if you read it carefully you'll see the scribbles make sense logically.
				We're simply doing the search-and-replace on $u = \sqrt{x}$,
				but to do this right we need to also replace $dx$ with $du$,
				and use the new limits of integration.

				In the fourth line,
				we recognized the general form of the function inside the integral, $f(u)=\frac{2}{u-1}$,
				to be similar to the function $f(u)=\frac{1}{u}$ whose integral function is $\ln(u)$.
				Accounting for the $-1$ horizontal shift and the factor of $2$ in the numerator,
				we obtain the answer $2\ln(u-1)$.
				In the last step,
				we changed back from $u$-variables to $x$-variables to compute the final answer.


			The substitution trick for integrals comes from the chain rule for derivatives
			$\left[ f(u(x)) \right]^\prime = f^\prime(u(x))u^\prime(x)$.
			The substitution rule only works for computing integrals of function
			that have the special structure $f'(u(x))u'(x)$.
			
% TODO: mention it has great power -- basis of multivariable calculus when using parameterization FWD REF


		\subsubsection{Integration by parts}

			Integration by parts is useful when the function you're integrating
			has the special structure $f(x)g'(x)$.

			\begin{shadebox}
			\vspace{1mm}
			\[
				\int f(x) \; g'(x)\:dx	\; \; = \; \;	f(x) g(x) \; \; -  \int f'(x)g(x) \; dx.
			\]
			\vspace{-2mm}
			\end{shadebox}	
			
			\noindent
			It is easier to remember the integration by parts formula in its shorthand notation,
			$\int u\; dv = uv - \int v\; du$.
			You can think of integration by parts as a form of ``double substitution,''
			where you simultaneously replace $u$ and $dv$.
			For definite integrals,
			integration by parts requires evaluating the product of the functions at the limits:
			\[
			 \int_a^b u\; dv = \left(uv\right)\Big|_a^b \ \  - \ \ \int_a^b v \; du.
			\]
			
			\paragraph{Example}
				Let's calculate $\int_0^5 x e^x \, dx$ using the integration by parts procedure.
				We apply the substitutions $u=x$ and $dv=e^x dx$,
				which means $du=dx$ and $v= e^x$.
				Applying the formula for integration by parts gives us
	
				\begin{align*} 
				\int_0^5 x \; e^x dx 
					 &= \left( x e^x \right) \Big|_0^5  \ - \  \int_0^5 e^x \; dx  		\\
					 &= \left( x e^x \right) \Big|_0^5      \ - \  e^x\Big|_0^5			\\[1mm]
					 &=  \left[ 5 e^5 - 0e^0 \right]    \ - \  \left[  e^5 - e^0 \right] 	\\
					 &= 5 e^5 - e^5 + 1 								\\
					 &= 4e^5 + 1 .
				\end{align*}
	

		\subsubsection{Other tricks}

			The two tricks we showed are only the tip of the iceberg.
			Mathematicians and physicist have come up with hundreds of formulas and tricks
			for calculating integrals,
			since for centuries people were forced to use pen and paper calculations.			
			There are tricks for trigonometric functions,
			square roots,
			fractions that that involve $x^2 + a^2$,
			etc.
			There is an entire course called integral calculus (CALC II),
			which is dedicated to learning integration tricks.
			CALC II is a required course for physicists and engineers.
% CLEANUP
%	but for the purpose of learning probability and statistics,
%	you're not required to learn all these integration techniques.
%	Instead,
%	you can rely on computers to do integration for you.
%	Specifically,
%	you can use the Python modules SciPy and SymPy to compute all the integrals you need,
%	as we'll show in the next two sections.

% Robyn said: 	Confusing that before you said that integrals is a topic that can't be skipped,
%			and here you say that you can rely on computers.
% 			If they don't need to know the math, then perhaps this section could be made even shorter.

			
			
