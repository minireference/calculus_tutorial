%!TEX root = ../calculus_tutorial.tex


\subsection{Computing integrals using SymPy}

	We can also use Python to do \emph{symbolic} integration using variables (symbols) instead of numbers.
	Symbolic integration allows us to obtain exact formulas for integrals
	that are valid for \emph{any} limits of integration $x=a$ and $x=b$.
	The Python module \tt{sympy} provides the functionality for doing symbolic math calculations
	similar to the calculation you could do using pen and paper.

	The following code block imports the SymPy function \tt{symbols},
	which is used to define new symbolic variables,
	and the function \tt{integrate} that we'll use for computing integrals.

	\begin{codeblock}[import-sympy-symbols-and-integrate]
	>>> from sympy import symbols, integrate
	\end{codeblock}
	
	\noindent
	Next we define four symbols \tt{x}, \tt{a}, \tt{b}, and \tt{c},
	which we'll use to denote mathematical variables and constants in the following code examples.

	\begin{codeblock}[define-symbols-xabcm]
	>>> x, a, b, c = symbols('x a b c')
	\end{codeblock}
	% TODO: explain symbol variables like x, a, b, c is symbol can be used in expressions

	\subsubsection{Example 1 revisited again}

		Consider the constant function $f(x) = c$.
		The symbolic expression that represents the value of this function is simply the constant $c$,
		which we can define as follows:

		\begin{codeblock}[sympy-define-fx]
		>>> fx = c
		>>> fx
		c
		\end{codeblock}

		\noindent
		The variable \tt{fx} is defined as the constant \tt{c},
		one of the SymPy symbols we defined earlier,
		which we assume corresponds to some unspecified constant value.

		To compute the integral $\int_a^b f(x) dx$,
		we call the SymPy function \tt{integrate},
		passing in the expression we want to integrate as the first argument.
		The second argument is a triple $(x,a,b)$,
		which specifies the variable of integration $x$,
		the lower limit of integration $a$,
		and the upper limit of integration $b$.

		\begin{codeblock}[sympy-integrate-fx-a-b]
		>>> integrate(fx, (x,a,b))  # = A_f(a,b)
		c*(b-a)
		\end{codeblock}

		\noindent
		Since $a$, $b$, and $c$ are arbitrary constants,
		the expression we obtain for $A_f(a,b) = \int_a^b f(x) dx$ is a general purpose formula
		that works for all functions $f(x) = c$ and all possible integration intervals $[a,b]$.
		Geometrically speaking,
		this is just the height-times-width formula for the area of a rectangle.

		To compute the specific integral between $a=0$ and $b=5$ under the graph of $f(x)=3$,
		we can use the method \tt{subs} (short for substitute) on the SymPy expression we obtained as a result of the integration.
		The \tt{subs} method expects as inputs a Python dictionary whose keys are symbols,
		and whose values represent the numbers we want to ``plug'' into the expression.
		In our case,
		we want to make the substitutions $c=3$, $a=0$, and $b=5$.

		\begin{codeblock}[integrate-fx-subs-vals]
		>>> integrate(fx, (x,a,b)).subs({c:3, a:0, b:5})
		15
		\end{codeblock}

		\noindent
		This result matches the value we obtained using the intuitive geometrical calculation (see Figure~\ref{fig:simple_integral_fx_eq_3})
		and the value we obtained using numerical integration, \tt{quad(f,0,5) = 15}.

		We can also use SymPy to compute the integral function $F_0(b)$,
		which is defined as  $F_0(b) \eqdef \int_0^b f(x) dx$,
		for the function $f(x) = \tt{fx}$.

		\begin{codeblock}[sympy-integral-function-F]
		>>> integrate(fx, (x,0,b))  # = F_0(b)
		b*c
		\end{codeblock}

		\noindent
		Recall that the integral function $F_0$ is simply the area-under-the-graph calculation
		with a variable upper limit of integration $b$.
		See Figure~\ref{fig:simple_integral_function_fx_eq_3} for an illustration of the integral function $F_0(b)$.



	\subsubsection{Example 2 revisited again}

		Let's now compute some integrals of the function $g(x) = x$.
		First we'll define a SymPy expression that corresponds to the function.

		\begin{codeblock}[sympy-define-gx]
		>>> gx = 1*x
		>>> gx
		x
		\end{codeblock}

		\noindent
		We can now compute the integral $\int_a^b g(x) dx$
		by calling the function \tt{integrate} with arguments \tt{gx},
		followed by the triple specifying the variable of the integration and the limits of integration.

		\begin{codeblock}[sympy-integrate-gx-a-b]
		>>> integrate(gx, (x,a,b))  # = A_g(a,b)
		b**2/2 - a**2/2
		\end{codeblock}

		\noindent
		To obtain the numerical value for the integral  $\int_0^5 g(x) dx$,
		we call the method \tt{subs} on the result of the integration.

		\begin{codeblock}[integrate-gx-subs-vals]
		>>> integrate(gx, (x,a,b)).subs({a:0, b:5})
		25/2
		\end{codeblock}

		\noindent
		SymPy computed the exact answer for us as a fraction $\frac{25}{2}$,
		but we sometimes want to force the answer to be computed as a floating-point number (a Python \tt{float}),
		which we can do by calling the \tt{.evalf()} method on the SymPy expression.				

		\begin{codeblock}[integrate-gx-subs-vals-evalf]
		>>> integrate(gx, (x,a,b)).subs({a:0, b:5}).evalf()
		12.5
		\end{codeblock}

		\noindent
		This result matches the value we obtained earlier using numerical integration,
		\tt{quad(g,0,5) = 12.5}.

		If we use the symbol \tt{b} for the upper limit of integration,
		we can obtain an expression for the integral function $G_0(b) \eqdef \int_0^b g(x) dx$.

		\begin{codeblock}[sympy-integral-function-G]
		>>> integrate(gx, (x,0,b))  # = G_0(b)
		b**2 / 2
		\end{codeblock}

		\noindent
		Note the expression for $G_0(b)$ we obtain from SymPy is identical
		to the formula we obtained earlier using a geometrical calculation (the area of a triangle with base $b$ and height $b$).
		See Figure~\ref{fig:simple_integral_function_gx_eq_x}.




	\subsubsection{Example 3 revisited again}

		\begin{codeblock}[sympy-define-gx]
		>>> hx = x**3 - 5*x**2 + x + 10
		>>> hx
		x**3 - 5*x**2 + x + 10
		\end{codeblock}
	
		TODO code
		
		\vspace{4cm}




	\bigskip
	\noindent
	Unfortunately,
	it's not always possible to use symbolic manipulations to find integrals.
	We can only use \tt{sympy.integrate} for certain simple examples
	where it is possible to obtain exact expressions for integral functions.
	For most practical calculations in probability and statistics,
	we'll need to rely on the \tt{scipy.integrate} function \tt{quad(f,a,b)},
	which computes the integral $\int_a^b f(x)dx$ for \emph{any} function $f(x)$ expressed as a Python function \tt{f}.









\subsection{Properties of integrals}

	We'll now state some properties of integrals that follow from their interpretation as area calculations.

	\begin{itemize}
	
		\item \textbf{Additivity.}
			The integral from $a$ to $b$ plus the integral from $b$ to $c$ is equal to the integral from $a$ to $c$:
			\[
				\int_a^b f(x) \; dx + \int_b^c f(x) \; dx		=	\int_a^c f(x) \; dx.
			\]

		% TODO: add backward steps giving negative?

%		\item \textbf{Constant multiple of a function.}
%			The integral of the function $cf(x)$ is equal to $c$ times the integral of $f(x)$,
%			for any constant $c$:
%			\[
%				\int cf(x)\; dx	=	c\int f(x)\; dx.
%			\]
%
%		\item \textbf{Sum of two functions.}
%			The integral of a sum of two functions is equal to the sum of the integrals of the individual functions:
%			\[
%				\int [f(x) + g(x)]\; dx	=	\int f(x)\; dx +  \int g(x)\; dx.
%			\]

		\item \textbf{Linearity.}
			% The combination of the above two properties tells us that
			Integration is a \emph{linear} operation: it preserves linear combinations.
			The integral of the linear combination of two functions $\alpha f(x) + \beta g(x)$,
			is equal to the same linear combination of the integrals of the two functions:
			\[
				\int [\alpha f(x) + \beta g(x)]\; dx 
				= \alpha  \int f(x)\; dx  \; \; + \; \; \beta \int g(x)\; dx,
			\]
			where $\alpha$ and $\beta$ are two arbitrary constants.

		\item \textbf{Integral at a single point.}
			Integrals over intervals with zero length have zero value for any function $f(x)$:
			\[
				\int_a^a f(x)\; dx	=	0.
			\]
			Thinking geometrically,
			this integral defines a region with height $f(x)$ and width~$0$,
			so it corresponds to zero area.
			% see https://www.khanacademy.org/math/ap-calculus-ab/ab-integration-new/ab-6-6/v/same-integration-bounds

	\end{itemize}

	% exercise https://www.khanacademy.org/math/ap-calculus-ab/ab-integration-new/ab-6-6/a/definite-integrals-properties-review



