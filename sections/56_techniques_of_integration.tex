%!TEX root = ../calculus_tutorial.tex


	\subsection{Techniques of integration}
	
		There are a bunch of tricks that extend the reach of analytical integration methods 
		(anti-differentiation) to more complicated functions.
		There many such tricks an we don't have room to discuss all of them here,
		but I'll show you the two most important ones.
		% MAYBE link to Sec 5.15 in the book 


		\subsubsection{Substitution trick}
		
			Suppose the function we want to integrate has the structure $f(u(x))u'(x)$,
			which consists of inner function wrapped in an outer function multiplied by the derivative of the inner function.
			We can use the \emph{substitution trick} to rewrite this integral in terms of the function $f(u)$ using $u$ as the variable of integration:
			\begin{align*}
				\int_{x \in \calX} f(u(x)) \, u'(x) dx
					&= \int_{u \in \calU} f(u) \; du.
				%	\int_{x=a}^{x=b} f(u(x)) \; u'(x) \; dx
				%		&=  \int_{u(a)}^{u(b)} f(u) \; du.					
			\end{align*}
			The substitution trick is ``change of variable'' operation from the variable $x$ to the variable $u$,
			similar to a search-and-replace operation when editing text.
			Because we're doing the substitution ``inside'' an integral operation,
			we must also change the region of integration ($\calX$ to $\calU$) and change of the ``step'' parameter ($dx$ to $du$).

			\begin{shadebox}
			\vspace{1mm}
			\noindent
			Follow these three steps to apply the substitution trick:
			\begin{enumerate}
			    \item   Replace $dx$ with $\frac{1}{u'(x)}du$.
			    \item   Replace all occurrences of $u(x)$ with $u$.
			    \item   Replace the $x$ limits of integration with $u$ limits of integration.
			\end{enumerate}
			\vspace{-1mm}
			\end{shadebox}	

			\noindent
			For example,
			let's compute the integral $\int_a^b \frac{1}{x - \sqrt{x}} \; dx$
			by applying the substitution $u=\sqrt{x}$,
			which implies $u'(x) =  \frac{1}{2\sqrt{x}}$.

			Performing the three steps of the substitution trick gives
			{\allowdisplaybreaks
			\begin{align*}
			     \int_{x=a}^{x=b} \frac{1}{x - \sqrt{x}} \; dx  
				&=  \int_{x=a}^{x=b}  \frac{1}{x - \sqrt{x}} \; \frac{1}{2\sqrt{x}} du				\\
				&=  \int_{x=a}^{x=b} \frac{1}{u^2 - u} \; 2u\,du							\\
				&=  \int_{u(a)}^{u(b)} \frac{1}{u^2 - u} \; 2u\,du
					= \int_{u(a)}^{u(b)} \frac{2u}{u^2 - u} \; du							\\
				&=  \int_{u(a)}^{u(b)} \frac{2}{u - 1} \; du
					=  2\ln(u-1) \bigg|_{u(a)}^{u(b)}									\\
				&=  2\ln(\sqrt{x}-1) \bigg|_{x=a}^{x=b}
					=  2\ln(\sqrt{b}-1) - 2\ln(\sqrt{a}-1).
			\end{align*}
			}
			
			\noindent
			In the fourth line,
			we recognized the general form of the function inside the integral, $f(u)=\frac{2}{u-1}$,
			to be similar to the function $f(u)=\frac{1}{u}$ whose integral function is $\ln(u)$.
			Accounting for the $-1$ horizontal shift and the factor of $2$ in the numerator,
			we obtain the answer $2\ln(u-1)$.
			In the last step,
			we changed back from $u$-variables to $x$-variables to compute the final answer.

			The substitution trick for integrals comes from the chain rule for derivatives
			$\left[ f(u(x)) \right]^\prime = f^\prime(u(x))u^\prime(x)$.
			Unlike the chain rule which you can apply to \emph{all} functions of the form $f(u(x))$,
			the substitution rule only works when you're computing integrals
			where the function you're integrating has the special structure $f^\prime(u(x))u^\prime(x)$.



		\subsubsection{Integration by parts}

			Integration by parts is useful whenever the function we're integrating has the special structure $f(x)g'(x)$.
			\begin{shadebox}
			\vspace{1mm}
			\[
			  \int f(x) \; g'(x)\:dx \ \ = \ \
				 f(x) g(x) \ \ 
				-  \int f'(x)g(x) \; dx.
			\]
			\vspace{-2mm}
			\end{shadebox}	
			
			\noindent
			It is easier to remember the integration by parts formula in its shorthand notation,
			$\int u\; dv = uv - \int v\; du$.
			You can think of integration by parts as a form of ``double substitution,''
			where you simultaneously replace $u$ and $dv$.
			For definite integrals,
			the integration by parts rule must account for evaluation at the function's limits:
			\[
			 \int_a^b u\; dv = \left(uv\right)\Big|_a^b \ \  - \ \ \int_a^b v \; du.
			\]
			Let's see how we can calculate $\int_0^5 x e^x \, dx$ using the integration by parts procedure.
			We apply the substitutions $u=x$ and $dv=e^x dx$,
			which means $du=dx$ and $v= e^x$.
			Applying the formula for integration by parts gives us
			\begin{align*} 
			\int_0^5 x \; e^x dx 
				 &= \left( x e^x \right) \Big|_0^5  \ - \  \int_0^5 e^x \; dx  		\\
				 &= \left( x e^x \right) \Big|_0^5      \ - \  e^x\Big|_0^5			\\[1mm]
				 &=  \left[ 5 e^5 - 0e^0 \right]    \ - \  \left[  e^5 - e^0 \right] 	\\
				 &= 5 e^5 - e^5 + 1 								\\
				 &= 4e^5 + 1 .
			\end{align*}


		\subsubsection{Other techniques}
		
			Special tricks for trigonometric functions,
			square roots  etc.
			
			
			
