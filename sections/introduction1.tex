%!TEX root = ../calculus_tutorial.tex


\section{Introduction}

	Calculus = study of functions + operations on functions

Calculus is the study of the properties of functions.
The operations of calculus are used to describe the limit behaviour of functions,
calculate their rates of change,
and calculate the areas under their graphs.
In this section we'll learn about the \texttt{SymPy} functions for calculating
limits, derivatives, integrals, and summations.



	% Main idea = calculations with infinite number of steps
	previously = numbers + operations that produce other numbers as outputs
	in calculus = functions + operators that produce other functions as outputs 

	In high school math,
	we learn all kinds of math procedures for solving problems using a finite number of steps of math operations.
	Whether you're manipulating expressions using algebra,
	or applying the inverse function to simplify an equation,
	all problems in high school math can be solved by using less than five steps,
	or if your teacher really doesn't like you 10 steps.
	% INFINITY
	In calculus,
	we learn a broader class of problem-solving strategies
	that include procedures with an infinite number of steps.


	\subsection{Definitions}


		\begin{itemize}
			\item   	$\mathbb{R}$: the set of real numbers
			\item   	$f(x)$:  a function of the form $f: \mathbb{R} \to \mathbb{R}$,
				which means $f$ takes real numbers as inputs and produces real numbers as outputs
			\item	$\lim_{\delta \to 0}$: a limit expression in which the number $\delta$ tends to zero
			\item	$f'(x)$: the derivative of $f(x)$ is the rate of change of $f$ at $x$: 
				\[
					f'(x) \eqdef \lim_{\delta \to 0} \frac{f(x+\delta)\; - \; f(x)}{\delta}\,.
				\] 
				The derivative is a function of the form $f': \mathbb{R} \to \mathbb{R}$.

			\item	$A(a,b)$: the value of the \emph{area} under the curve $f(x)$ from $x=a$ until $x=b$.
		    		The area $A(a,b)$ is computed as the following integral
		    		\[
			        	   A(a,b) = \int_a^b f(x)\;dx.
			    	\] 
			    	The $\int$ sign stands for \emph{sum}.
			    	Indeed, the integral is the ``sum'' of $f(x)$ for all values of $x$ between $a$ and $b$.

				\vspace{2mm}
				
			\item	$A_0(x)$: the \emph{integral function} of $f(x)$.
		    		The integral function corresponds to the computation of the area under $f(x)$
				as a function of the upper limit of integration:
			       	\[
			        	   A_0(x) \eqdef A(0,x) = \int_0^{x}\! f(u)\:du.
			      	\]
				The choice of $x=0$ as the lower limit of integration is arbitrary.

				\vspace{2mm}

			\item sequence 

			\item series

		\end{itemize}

% cf. \input{../../../MATHPHYSbook/05_calc/02.overview.tex}
% cf. \input{../../../MATHPHYSbook/02_intro2phys/03.introduction_to_calculus.tex}




	\subsection{Example 1: file download}
	\label{introduction:download_example}
	
		Suppose you're downloading a large file to your computer.
		At $t=0$ you click ``save as'' in your browser and the download starts.
		Let $f(t)$ represent the size of the downloaded data.
		At any time $t$,
		the function $f(t)$ tells you the amount of disk space taken by the partially-downloaded file.
		You are downloading a $720$[MB] file, so the download progress
		at time $t$ corresponds to the fraction $\frac{f(t)}{720 \text{[MB]}}$.

		\subsubsection{Download rate}
		\label{introduction:download_rate}

			The derivative function $f^\prime(t)$, pronounced ``\:\!\!$f$ prime,''
			describes how the function $f(t)$ changes over time.
			In our example $f^\prime(t)$ is the download speed.
			If your downloading speed is  $f'(t)=100$[kB/s],
			then the file size $f(t)$ must increase by 100[kB] each second.
			If you maintain this download speed, the file size will grow at a constant rate:
			$f(0)=0$[kB], $f(1)=100$[kB], $f(2)=200$[kB], $\ldots$, $f(100)=10$[MB].

			To calculate the ``estimated time remaining'' until the download's completion,
			we divide the amount of data that remains to be downloaded by the current download speed:
			\[
				\text{time remaining } = \frac{ 720 - f(t) }{ f^\prime(t) } \quad [\textrm{s}].
			\]
			The bigger the derivative, the faster the download will finish.
			If your internet connection were 10 times faster,
			the download would finish 10 times more quickly.


		\subsubsection{Inverse problem}
		\label{introduction:inverse_problem}

			Let's consider this situation from the point of view of the modem that connects your computer to the internet.			\index{inverse!operation}
			Any data you download comes through the modem.
			The modem knows the download rate $f^\prime(t)$[kB/s] at all times during the download.

			However, since the modem is separate from your computer,
			it does not know the file size $f(t)$ as the download progresses.
			Nevertheless,
			the modem can infer the file size at time $t$ from knowing the transmission rate $f^\prime(t)$.
			The integral of the download rate between $t=0$ and $t=\tau$
			corresponds to the total amount of downloaded data stored on your computer.
			During this download period, the change in file size is described by the integral
			\[
				\Delta f = f(\tau)-f(0) = \int_0^\tau \! f'(t)\: dt\,.
			\]
			Assuming the file size starts from zero $f(0)=0$[kB] at $t=0$,
			the modem can use the integration procedure to find $f(\tau)$,
			the file size on your computer at $t=\tau$:
			\[
				f(\tau) = \int_0^\tau\! f^\prime(t) dt\,.
			\]
			The download rate $f'(t)$ is measured in [kB/s],
			and each time step $dt$ is $1$[s] long,
			so the data downloaded during one second is $f'(t)dt$[kB].
			The file size at time $t=\tau$ is equal to the sum of the data
			downloaded during each second from $t=0$ until $t=\tau$.			

			The integral $\int_a^b q(t)\:dt$ is the calculation of the \emph{total}
			of some quantity $q(t)$ that accumulates during the time period from $t=a$ to $t=b$.
			Integrals are necessary any time you want to calculate the total of a quantity that changes over time.
			
			% TODOv6: remind how tau is pronounced and why we need a variable different from t

		\bigskip
		
		As demonstrated above,
		calculus is much more than the theoretical activity reserved for math specialists.
		Calculus relates to everyday notions you're already familiar with.
		Indeed, we carry out calculus-like operations in our head every day---we just
		don't necessarily use calculus terminology when we do so.

		%	Usually, differential calculus and integral calculus are taught as two separate subjects.
		%	Perhaps teachers and university administrators are worried the undergraduates' little heads will explode
		%	from sudden exposure to \emph{all} of calculus.
		%	However, this separation actually makes calculus more difficult,
		%	and prevents students from discovering the connections between differential and integral calculus.
		%	We'll have no such split in this book, because I believe you can handle the material in one go.
		%	Understanding calculus involves figuring out new mathematical concepts like infinity, limits, and summations,
		%	but these ideas are not \emph{that} complicated.
		%	By getting this far, you've proven you're more than ready to learn the theory,
		%	techniques, and applications of derivatives, integrals, sequences, and series.
		
		% Let's begin with an overview of the material.


	Derivatives and integrals are two main actors in calculus.
	The other supporting actors are limits and series,
	which we'll briefly introduce in the following example.



	\subsection{Example 2: Euler's number}
	\label{introduction:eulers_number}
	
		Suppose you take out a loan with a 100\% annual interest rate.
		If the interest is calculated once per year,
		this means at the end of the year you'll owe double the amount you borrowed.
		However,
		if the interest rate is calculated more often,
		you'll owe more at the end of the year.
		For example,
		if the interest rate is computed twice per year,
		you'll owe $(1+\frac{1}{2})(1+\frac{1}{2}) = 2.25$.
		If computed three times per year,
		$(1+\frac{1}{3})(1+\frac{1}{3})(1+\frac{1}{3}) = 2.370$.
		In general,
		when the compounding is performed $n$ times per year,
		the amount owed at the end of the year will be
		\[
			\underbrace{
			\left(1 + \tfrac{1}{n} \right)
			\left(1 + \tfrac{1}{n} \right)
			\cdots
			\left(1 + \tfrac{1}{n} \right)
			}_{n \text{ times}}
			= 
			\left(1 + \tfrac{1}{n} \right)^{\!n}.
		\]
		Euler's number is defined as the \emph{limit} of the expression $\left(1 + \frac{1}{n} \right)^{\!n}$
		as $n$ grows to infinity,
		which we write as
		\[
			e 	\eqdef 	\lim_{n\to \infty} \left( 1 + \frac{1}{n}\right)^{n}
				= 		2.718281828\ldots.
		\]
		This limit expression describes the annual growth rate of a loan
		with infinitely frequent compounding.
		If we borrow $\$1000$,
		we'll owe $\$2718.28$ at the end of one~year.

		Euler's number $e$ can also be obtained from another expression:
		\[
			e	\eqdef	1  + 1 + \frac{1}{2!} + \frac{1}{3!} + \frac{1}{4!} + \frac{1}{5!} + \cdots
				= 		\sum_{k=0}^\infty \frac{1}{k!}
				= 		2.718281828\ldots.
		\]
		Note this is a math expression with an infinite number of terms.
		Each term is comes from a common ``pattern'' $\frac{1}{k!}$,
		where $k! = k\cdot (k-1) \cdot (k-2) \cdots 3\cdot 2 \cdot 1$
		is the factorial function.
		This kind of expression is called a \emph{series},
		and provides a powerful way to compute quantities by summing together a bunch of terms.


	\subsection{Applications}

		Learning the language of calculus will help you think more clearly about certain types of problems.
		Understanding the language of calculus is \emph{essential} for learning science
		because many laws of nature are best described in terms of derivatives and integrals.
		
		Using concepts to describe phenomena in:
		physics,
		engineering,
		finance,
		probability calculations (continuous random variables),
		optimization (used in business, statistics, machine learning, etc.).



	\subsection{Doing calculus}

		Symbolic calculations using pen and paper

		Symbolic calculations using SymPy

		Numerical calculations using NumPy and SciPy



