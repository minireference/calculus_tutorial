%!TEX root = ../calculus_tutorial.tex

\section{Math prerequisites}

	Before we get started with the new calculus topics,
	let's do a quick review of key concepts from high school math.
	These are the basic building blocks I assume you've seen before.

	\subsection{Set notation}

		Sets are collections of math objects.
		Many math ideas are expressed in the language of sets,
		so it's worth knowing the notation for sets.	
		\begin{itemize}

			% \item $S,T$: the usual variable names for sets

			\item $\{ \textrm{  definition  } \}$:
				we use curly brackets to define sets.
				The definition in the curly brackets is either a math description of the set's contents,
				or a list of elements in the set.
				% The math symbol ``$|$'' is shorthand for the phrase ``such that'' and it is often used in definitions.

			\item $\mathbb{N}$: the set of natural numbers $\mathbb{N}\eqdef \{0,1,2,3,4,5,\ldots\}$
	
			\item $\mathbb{N}_+$: the positive natural numbers $\mathbb{N}_+ \eqdef \{1,2,3,4,5,\ldots\}$.

			\item	$\mathbb{R}$: the set of real numbers.
	
			\item	$\mathbb{R}_+$: the set of nonnegative real numbers.

			\item $x \in S$: this statement is read ``$x$ is an element of $S$.'' % or ``$x$ is in $S$.''
				We use this notation to indicate the ``type'' of the variable~$x$.
				For example,
				writing ``$x \in \mathbb{R}$'' tells us $x$ is a real number.
	
			% seems not needed; put back if needed somewhere in the book (grep for it)
			%	\item $\emptyset$: the \emph{empty set} is a set that contains no elements.
			%		Mathematicians adopted the symbol $\emptyset$ because the notation $\{ \, \}$ is confusing.
	
	

			%	\item $\mathbb{Z}$: the set of integers $\mathbb{Z}\eqdef\{\ldots,-2,-1,0,1,2,3,\ldots\}$
			%
			%	\item $\mathbb{Q}$: the set of rational numbers,
			%		$\mathbb{Q} \eqdef \left\{  \frac{m}{n} \; \Big| \; m \in \mathbb{Z}, \; n \in \mathbb{N}, \; n \neq 0  \right\}$.
			%		The set $\mathbb{Q}$ consists of all numbers that can be expressed as \emph{fractions} of the form $\frac{m}{n}$,
			%		where $m$ is an integer, $n$ is a natural number, and $n \neq 0$.
	
	
		\end{itemize}
	
		\noindent
		We use the math symbol $\eqdef$ to define new concepts.
		We often use the \emph{set-builder} notation $\{\, \cdot \; | \;  \cdot \, \}$ to define sets.
		Inside the curly brackets,
		we first describe the general kind of mathematical objects we are talking about,
		followed by the symbol ``$|$'' (which stands for ``such that''),
		followed by the conditions that identifies the elements of the set.
		For example,
		the nonnegative real numbers $\mathbb{R}_+$
		are defined as ``all real numbers $x$ such that $x \geq 0$,''
		which is expressed more compactly as 
		$\mathbb{R}_+ \eqdef \{ x \in \mathbb{R} \; | \; x \geq 0 \}$
		using the set-builder notation.


		\subsubsection{The number line}
	
			The \emph{number line} is a visual representation of the set of real numbers $\mathbb{R}$,
			as shown in Figure~\ref{fig:number_line_rationals_and_reals}.
			The real numbers correspond to all the points on the number line,
			from $-\infty$ to $\infty$.
		
			\begin{figure}[htb]
				\centering
				\includegraphics[width=0.98\columnwidth]{figures/calculus/number_line_rationals_and_reals.pdf}
				\vspace{-2mm}
				\caption{The real numbers $\mathbb{R}$ cover the entire number line.}
				\label{fig:number_line_rationals_and_reals}
			\end{figure}
		
			\noindent
			The set of real numbers includes
			the natural numbers $\{0,1,2,3,\ldots\}$,
			the integers $\{\ldots, -3,-2,-1,0,1,2,3,\ldots\}$,
			rational numbers like $-\frac{3}{2}$, $0.5$, and $\frac{9}{2}$,
			as well as irrational numbers like $\sqrt{2}$, $e$, and $\pi$.
			All the number you will run into when doing math
			can be visualized as a point on the number line.



		\subsubsection{Infinity}

			The math symbol $\infty$ describes the concept of \emph{infinity}.
			We use the symbol $\infty$ to represent an infinitely large quantity,
			that is greater than any number you can think of.
			Geometrically speaking,
			we can imagine the number line extends to the right forever towards infinity,
			as illustrated in Figure~\ref{fig:number_line_rationals_and_reals}.
			The number line also extends forever to the left,
			which we denote as negative infinity $-\infty$.
			% to describe larger and larger negative numbers,

			Infinity is not a number but a \emph{process}.
			% ALT. not a destination you can get to
			When we use the symbol $+\infty$,
			we're describing the process of moving to the right on the number line \emph{forever}.
			We go past larger and larger positive numbers and never stop.

			Infinity is a key concepts in calculus,
			so it's important that you develop the right way to think about
			infinitely large numbers,
			infinitely small distances,
			and procedures with infinite number of steps.
			We'll continue the discussion of infinity-related topics in Section~\ref{sec:limits}.
			%  we'll learn about \emph{limits}, which allow us to describe 

			%	Infinity is the main new concept in calculus.
			%	Everything else we'll talk about (numbers, variables, expressions, algebra, equations, functions, etc.)
			%	are standard topics from high school math, which you're familiar with.
			%	Indeed, calculus can be described as the ``infinity upgrade'' to the high school math
			%	that gives you a language for describing and solving a new class of problems.

			% COUNTABLE INFINITY
			%	Consider the set of natural numbers $\mathbb{N} \eqdef \{0, 1, 2, 3, 4, 5, 6, \ldots \}$.
			%	The natural numbers describe the process of counting starting at $0$.
			%	The natural number $n$
			%	is obtained by starting at $0$ and performing the $+1$ operation $n$ times.
			%	% Look ahead to the number line shown in Figure~XX.
			%	Geometrically,
			%	you can think of the $+1$ operation as taking one step to the right on the number line
			%	shown in Figure~\ref{fig:number_line_rationals_and_reals}
			%	(see page~\pageref{fig:number_line_rationals_and_reals}).
			%	In this context,
			%	you can think of infinity $\infty$ as performing the $+1$ operation forever.
			%	Infinity is greater than any natural number $n$.
			%	Indeed,
			%	getting to $n$ takes a finite number of steps,
			%	but $\infty$ describes taking an infinite number of steps
			%	so $\infty$ must be to the right of $n$.


	% CUT BECAUSE NOT ESSENTIAL
	%	\subsubsection{Number intervals}
	%
	%		The number line can be used to represent subsets of the real numbers,
	%		which we call \emph{intervals}.
	%		Figure~\ref{fig:interval_2closed_to_4closed}
	%		shows an illustration of the interval $[2,4] \eqdef \{ x \in \mathbb{R} \;|\; 2 \leq x \leq 4 \}$,
	%		which is the subset of the real numbers between $2$ and $4$.
	%	
	%		\begin{figure}[htb]
	%			\centering	
	%			\includegraphics[width=0.9\columnwidth]{figures/calculus/interval_2closed_to_4closed.pdf}
	%			\vspace{-3mm}
	%			\caption{	The interval $[2,4] \protect\eqdef \{ x \in \mathbb{R} \; | \; 2 \leq x \leq 4 \}$
	%					is a subset of $\mathbb{R}$.} % the real numbers.
	%			\label{fig:interval_2closed_to_4closed}
	%		\end{figure}






	\subsection{Functions}
	
		A \emph{function} is a mathematical object that takes numbers as inputs and produces numbers as outputs.
		The output of the function $f$ for the input $x$ is denoted $f(x)$.
		For example,
		the function $f(x) \eqdef x^2$
		takes any number $x$ as input,
		squares it and divides the result by two to produce the output.
		For example,
		$f(3) = 3^2 = 9$.
		
		In this tutorial,
		we'll often show code examples that mirror the math calculations.
		For example, 
		here is the Python code that defines the function \tt{f}
		and evaluates it for the input $x=3$.

		\begin{codeblock}[def-fx-one-half-x-squared]
		>>> def f(x):
		        return x**2
		>>> f(3)
		9.0
		\end{codeblock}
		
		\noindent
		Note the Python syntax for evaluating the function \tt{f} on the input \tt{3}
		is the same as the math syntax $f(3)$.
	
		\subsubsection{Function graphs}

			The \emph{graph} of a function is a curve that passes through all input-output pairs of a function.
			%	Imagine we take out a piece of paper and draw a coordinate system
			%	with a horizontal axis and a vertical axis.
			%	The horizontal axis describes the different input values $x$,
			%	while the vertical axis describes the output values $f(x)$.
			Each input-output pair corresponds to a point $(x,f(x))$ in a Cartesian coordinate system.
			We obtain the graph of the function by varying the input coordinate $x$ and plotting all the points $(x, f(x))$,
			as illustrated in Figure~\ref{fig:graph_of_function_f_eq_x2}.
			%
			The graph of the function $f$ allows us to see at a glance the behaviour of the function for many inputs.
			Function graphs are an essential visualization tool for calculus calculations.

			Let's see how we can use the Python modules \tt{numpy} and \tt{seaborn}
			to plot the graph of the function $f(x) \eqdef x^2$ that we defined in the previous code block.
			We start by importing the module \tt{numpy} under the alias \tt{np}.
			Next,
			we use the function \tt{np.linspace} to create an array (a list of numbers) \tt{xs}
			that contains 1000 input $x$-values that range between $x=-3$ and $x=3$.
			We then evaluate the function for all inputs \tt{xs}
			and store the outputs of the function in an array called \tt{fxs}.

			\begin{codeblock}[plot-fx-minus3-to-plus3]
			>>> import numpy as np
			>>> xs = np.linspace(-3, 3, 1000)
			>>> fxs = f(xs)
			\end{codeblock}

			\noindent
			At this point,
			the arrays \tt{xs} and \tt{fxs} contain $1000$ input-output coordinate pairs of the form $(x, f(x))$.
			To generate the graph of $f(x)$,
			we just need to trace a line passing through these coordinate pairs.
			We can do this by importing the \tt{seaborn} module (alias \tt{sns})
			and calling the function \tt{sns.lineplot}.
	
			\begin{codeblock}[plot-fx-minus3-to-plus3]
			>>> import seaborn as sns
			>>> sns.lineplot(x=xs, y=fxs)
			See Figure ¡\ref{fig:graph_of_function_f_eq_x2}¡ for the output.
			\end{codeblock}
	
			\begin{figure}[htb]
				\centering
				\includegraphics[width=0.9\columnwidth]{figures/calculus/graph_of_function_f_eq_x2.pdf}%
				\vspace{-3mm}
				\caption{	Graph of the function $f(x) = x^2$ from $x=-3$ until $x=+3$.
						The graph of the function $f$ passes through the coordinates pairs $(x,f(x))$
						for all $x$-values between $x=-3$ and $x=3$.}
				\label{fig:graph_of_function_f_eq_x2}
			\end{figure}

			\noindent
			We can use this combination of \tt{np.linspace}, function evaluation, and \tt{sns.lineplot}
			whenever we need to plot the graph of any function.
			% The function \tt{plot\_func(f, xlim=[-3,3])} produces the same output.


		\subsubsection{Inverse functions}
	
			The inverse function $f^{-1}$ % $f^{-1} \colon B \to A$
			performs the \emph{inverse operation} of the function $f$. % $f \colon A \to B$.
			If you start from some $x$,
			apply $f$,
			then apply $f^{-1}$,
			you'll arrive---full circle---back to the original input $x$:
			\[
				f^{-1}\!\big( \; f(x) \; \big) = x.
			\]
			In Figure~\ref{fig:functions-inverse},
			the function $f$ is represented as a forward arrow,
			and the inverse function $f^{-1}$ is represented as a backward arrow
			that puts the value $f(x)$ back to the $x$ it came from.
	
			\begin{figure}[htb]
				\centering
				\includegraphics[width=0.4\columnwidth]{figures/calculus/functions-inverse.pdf}
				\caption{The inverse $f^{-1}$ undoes the operation of the function $f$.}
				\label{fig:functions-inverse}
			\end{figure}
	
			For example,
			when $x \geq 0$,
			the inverse of the function $f(x) = x^2$
			is the function $f^{-1}(x) = \sqrt{x}$.
			Earlier we computed $f(3) = 9$.
			If we apply the inverse function $f^{-1}(x) = \sqrt{x}$ to $9$,
			we get back to the number $3$ that we started from
			$f^{-1}(9) = \sqrt{9} = 3$.

			\begin{codeblock}[fun-inv-fun-combo-log]
			>>> from math import sqrt
			>>> sqrt(9)
			3
			\end{codeblock}
	


		\subsubsection{Function properties}

			We often think about the possible inputs and outputs of functions.
			We use the notation $f \colon A \to B$
			to denote a function from the input set $A$ to the output set $B$.
			The set of allowed inputs is called the \emph{domain} of the function,
			while the set of possible outputs is called the \emph{image} % or \emph{range}
			of the function.
			For example,
			the domain of the function $f(x) = x^2$
			is $\mathbb{R}$ (any real number)
			and it's image is $\mathbb{R}_+$ (nonnegative real numbers),
			so we write it as $f \colon \mathbb{R} \to \mathbb{R}_+$.
			
			%	Another important property is called \emph{continuity},
			%	which roughly corresponds to ability to draw a function without lifting the pen.
			%	We'll give a formal definition of continuity later in Section~\ref{limits:continuity}.


	\subsection{Function inventory}
	
		Your function ``vocabulary'' determines which math expressions you'll be able to read and understand
		in the same way your English vocabulary
		determines determines which English sentences you'll be able to read and understand.
		Figure~\ref{fig:panel_function_graphs1}
		shows the graphs of six important functions
		that are used in many areas of mathematical modelling.

		\begin{figure}[htb]
			\centering
			\includegraphics[width=0.99\columnwidth]{figures/calculus/panel_function_graphs1.pdf}%
			\vspace{-2mm}
			\caption{	Graph of six math functions that you should know about.}
			\label{fig:panel_function_graphs1}
		\end{figure}



		\subsubsection{Constant function}
		
			The constant function $f(x) \eqdef c$ produces the same output $c$ for all inputs $x$.


		\subsubsection{Linear function}
		
			The linear function $f(x) \eqdef mx$ describes an input-output relationship
			where the output value $f(x)$ are \emph{proportional} to the input value $x$,
			and the constant of proportionality is $m$.
			Geometrically,
			$m$ is the slope in the graph of $f(x)$.
			Figure~\ref{fig:panel_function_graphs1}
			show the graph of $f(x) \eqdef x$ which is the linear function with $m=1$
			for which the outputs $f(x)$ is equal to the input $x$.
			%
			More generally,
			we can define the line $f(x) \eqdef mx+b$,
			where $m$ describes the slope of the line,
			and $b$ describes the value of the function when $x=0$.
			% The inverse to the line $f(x)=mx+b$ is $f^{-1}(x)=\frac{1}{m}(x-b)$, which is also a line.



		\subsubsection{Quadratic function}
			
			The quadratic function $f(x) \eqdef x^2$ calculates the square of the input $x$.
			The name ``quadratic'' comes from the Latin \emph{quadratus} for square.
			Geometrically,
			$x^2$ is the area of a square with side length $x$.
			See Figure~\ref{fig:panel_function_graphs1}~(b) for the graph.
			The outputs the quadratic function are always nonnegative numbers
			since $x^2 \geq 0$, for all real numbers $x$.
			%	$f(x)=x^2$ is \emph{two-to-one}: 
			%	it sends both $x$ and $-x$ to the same output value $x^2=(-x)^2$.



		\subsubsection{Polynomial functions}
		
			We can combine the constant, linear, and quadratic functions
			to obtain the polynomial function $f(x) \eqdef a_2x^2 + a_1x +a_0$,
			where $a_2$, $a_1$, $a_0$ are arbitrary constants,
			which are called the \emph{coefficients} of the polynomial.
			This is called a second degree polynomial,
			since the highest power of $x$ it contains is $x^2$.
			%
			The general equation for a polynomial function of degree $n$ is
			\[
				P_n(x) = a_0 + a_1x + a_2x^2 + a_3x^3 + \cdots + a_nx^n.
			\]
			% A polynomial of degree $n$ has  $n+1$ coefficients: $a_0,a_1,a_2,\ldots, a_n$.
			Polynomials are a very useful family of functions.
			%	We call $a_0$ the constant term.
			%	$a_1$: the \emph{linear} coefficient, or \emph{first-order} coefficient
			%	$a_2$: the \emph{quadratic} coefficient
			%	$a_3$: the \emph{cubic} coefficient
			%	$a_n$: the $n$\textsuperscript{th} order coefficient
			%	$n$: the \emph{degree} of the polynomial.
			%	The degree of $f(x)$ is the largest power of $x$ that appears in the polynomial.

			% The roots of $f(x)$ are the values of $x$ for which $f(x)=0$.







		\subsubsection{Exponential function}

			The exponential function base $e=2.7182818\ldots$ is defined
			as $f(x) \eqdef e^{x} = \exp(x)$.
			Figure~\ref{fig:panel_function_graphs1}~(c)
			shows the graph of the exponential function $f(x)=e^x$,
			which passes through the following points:
			$(-2,\frac{1}{e^2})$, 
			$(-1,\frac{1}{e})$, 
			$(0,1)$, 
			$(1,e)$, 
			and $(2,e^2)$.
			%	
			%	For larger inputs 
			%	$(3,e^3)$, 
			%	$(4,e^4)$, etc.
			
			% $f(a)f(b)=f(a+b)$ since $e^ae^b=e^{a+b}$

			% The derivative (the slope of the graph) of the exponential function 
			%    		is the exponential function: $f(x) = e^x  \; \; \Rightarrow \; \; f'(x)=e^x$
			%	The function $e^x$ is the only function which is equal to its own derivative: $f(x)=f'(x)$.







		\subsubsection{Absolute value function}
		
			The \emph{absolute value} function tells us the size of numbers
			without paying attention to whether the number is positive or negative.
			We compute the absolute value of the number $x$ by \emph{forgetting} the sign of $x$.
			Geometrically,
			$|x|$ corresponds to its distance between $x$ and the origin of the number line.
			%
			We see the absolute values whenever we apply
			the combination of squaring followed by square-root on some number,
			$\sqrt{ x^2 }  = |x|$,
			since squaring destroys the sign.




		\subsubsection{Square root function}
		
			The square root function is denoted $f(x) \eqdef \sqrt{x}$.
			The square root $\sqrt{x}$ is the inverse function of the square function $x^2$,
			when $x \geq 0$.
			% when the two functions are defined as $f : \mathbb{R}_+ \to \mathbb{R}_+$.
			The symbol $\sqrt{c}$ refers to the \emph{positive} solution to the equation $x^2=c$.
			Note that $-\sqrt{c}$ is also a solution of $x^2=c$.
			%
			Another notation for the square root function is $f(x) \eqdef x^{\frac{1}{2}}$,
			where the fractional exponent $\frac{1}{2}$ makes sense
			since if we square $x^{\frac{1}{2}}$,
			we get back to $x$:
			$\left(x^{\frac{1}{2}}\right)^2 = x^{\frac{2}{2}} = x^{1} = x$.
			%
			In addition to \emph{square} root,
			there is also \emph{cube} root $f(x) = \sqrt[3]{x}  = x^{\frac{1}{3}}$,
			which is the inverse function for the cubic function $f(x)=x^3$.
			We have $\sqrt[3]{8}=2$ since $2\times2\times2=8$.




		\subsubsection{Logarithmic function}
		
			The natural logarithm function is denoted
			$f(x) \eqdef \ln(x) = \log_e(x)$.
			The function $\ln(x)$ is the inverse function of the exponential $e^x$.
			The graph of the function $\ln(x)$
			passes through the following coordinates:
			$(\frac{1}{e^2},-2)$, 
			$(\frac{1}{e},-1)$, 
			$(1,0)$, 
			$(e, 1)$, 
			$(e^2, 2)$, 
			$(e^3, 3)$, 
			$(e^4, 4)$, etc.


		% MAYBE OTHER FUNCTIONS
		%gaussian-like erf $e^{-x^2}$
		%sigmoid $\frac{1}{1-e^{-x}}$




	\subsection{Functions with discrete inputs}

		Later in this tutorial,
		we'll study functions with discrete inputs,
		$a_k : \mathbb{N} \to \mathbb{R}$,
		which are called \emph{sequences}.
		We often express sequences by writing explicitly the first few values
		the sequence $[a_0, a_1, a_2, a_3, \ldots]$,
		which correspond to evaluating $a_k$ for $k=0$, $k=1$, $k=2$, $k=3$, etc.




	\subsection{Geometry of rectangles and triangles}
	
		% 	We'll now briefly review some geometry formulas.
		%	\subsubsection{Circle}
		%		The area enclosed by a circle of radius $r$ is given by $A = \pi r^2$,
		%		where $\pi = 3.14159\ldots$.
		%		A circle of radius $r=1$ has area $\pi$.
		%		The circumference of a circle of radius $r$ is $C = 2 \pi r$.
		%		A circle of radius $r=1$ has circumference $2\pi$.

		The area of a rectangle of base $b$ and height $h$ is $A = bh$,
		as illustrated in Figure~\ref{fig:geometry_areas_circle_rect_triangle}~(a).
		The area of a triangle
		is equal to $\frac{1}{2}$ times the length of its base $b$ times its height $h$:
		$A = \tfrac{1}{2} b h$,
		as shown in Figure~\ref{fig:geometry_areas_circle_rect_triangle}~(b).
		% Note that $h_a$ is the height of the triangle \emph{relative to} the side $a$.

		\begin{figure}[htb]
			\centering
			\includegraphics[width=0.8\columnwidth]{figures/calculus/geometry_areas_circle_rect_triangle.pdf}
			\vspace*{-5mm}
			\caption{	Formulas for calculating the area of a rectangle and a triangle.}
			\label{fig:geometry_areas_circle_rect_triangle}
		\end{figure}





	\subsection{Trigonometric functions}
	
		The \emph{unit circle} is a circle of radius one centred at the origin,
		as illustrated in Figure~\ref{fig:trigonometric_functions_as_point_or_as_triangle}.
		% 
		The unit circle consists of all points $(x,y)$ that satisfy the equation $x^2 + y^2 = 1$.
		A point on the unit circle has coordinates $(\cos\theta,\sin\theta)$,
		where $\theta$ is the angle the point makes with the $x$-axis.

		
		%FIGURE  right-angle triangle with hypotenuse r, adj = rcosθ, opp = rsinθ
		\begin{figure}[htb]
			\centering
			\includegraphics[width=0.99\columnwidth]{figures/calculus/trigonometric_functions_as_point_or_as_triangle.pdf}%
			\vspace{-2mm}
			\caption{	The coordinates of the point on the unit circle are $\cos\theta$  and $\sin\theta$.}
			\label{fig:trigonometric_functions_as_point_or_as_triangle}
		\end{figure}

		% RADIANS
		In math,
		we use \emph{radians} to measure angles instead of degrees~$^\circ$.
		One full circle is $360^\circ$ or $2\pi$ radians.
		Some common angle measures are $30^\circ = \frac{\pi}{6}$,
		$45^\circ\!=\!\frac{\pi}{4}$, $60^\circ = \frac{\pi}{3}$, and $90^\circ = \frac{\pi}{2}$.
		The trigonometric functions $\sin$, $\cos$, and $\tan$ expect inputs in radians,
		so we often convert angles from degrees to radians. % and back.


		\subsubsection{Sine function}

			The graph of the sine function $f(\theta) \eqdef \sin(\theta)$
			\emph{oscillates} up and down and crosses the $x$-axis multiple times,
			as shown in Figure~\ref{fig:panel_function_graphs2}~(a).
			This graph corresponds to the vertical position of the point turning around on the unit circle,
			as illustrated in Figure~\ref{fig:trigonometric_functions_as_point_or_as_triangle}~(a).
			We also use the sine function to find the $y$-component of unit length,
			as shown in Figure~\ref{fig:trigonometric_functions_as_point_or_as_triangle}~(b).
			
			%	The graph of the function $\sin(\theta)$  passes through the following $(x,y)$ coordinates:
			%	$(0,0)$, $(\frac{\pi}{6},\frac{1}{2})$, $(\frac{\pi}{4},\frac{\sqrt{2}}{2})$, $(\frac{\pi}{3},\frac{\sqrt{3}}{2})$, 
			%	$(\frac{\pi}{2},1)$, $(\frac{2\pi}{3},\frac{\sqrt{3}}{2})$, $(\frac{3\pi}{4},\frac{\sqrt{2}}{2})$, $(\frac{5\pi}{6},\frac{1}{2})$, and $(\pi,0)$.
			%	For $\theta$ between $\pi$ and $2\pi$,
			%	the function's graph has the same shape it has for $\theta$ between $0$ and $\pi$, but with negative values.
			%	
			%		Let's start at $\theta=0$ and follow the graph of the function $\sin(x)$
			%		as it goes up and down.
			%		The graph starts from $(0,0)$ and smoothly increases until it reaches 
			%		the maximum value at \mbox{$\theta=\frac{\pi}{2}$}. 
			%		Afterward, the function comes back down to cross the $x$-axis at $x=\pi$.
			%		After $\pi$, the function drops below the $x$-axis and reaches
			%		its minimum value of $-1$ at $x=\frac{3\pi}{2}$.
			%		It then travels up again to cross the $x$-axis at $x=2\pi$.
			%		This $2\pi$-long cycle repeats after $x=2\pi$.
			%		This is why we call the function \emph{periodic}---the shape of the graph repeats.
			
			%	    \item 	Domain: $\mathbb{R}$.
			%	    		The function $f(x)=\sin(x)$ is defined for all input values.
			%	    \item	Image: $\{ y \in \mathbb{R} \; | \;  -1 \leq y \leq 1 \}$.
			%	    		The outputs of the sine function are always between $-1$ and $1$.
			%	    \item	Roots: $\{\,\ldots, -3\pi, -2\pi,-\pi,0,\pi,2\pi,3\pi, \ldots\,\}$.\\
			%	    		The function $\sin(x)$ has roots at all multiples of $\pi$.
			%	    \item   	The function is periodic, with period $2\pi$: $\sin(x) = \sin(x+2\pi)$.
			%	    \item   	The $\sin$ function is \emph{odd}: $\sin(x)=-\sin(-x)$
			%	    \item   	Relation to $\cos$: $\sin^2 x + \cos^2 x = 1$
			%	    \item   	Relation to $\csc$: $\csc(x) = \frac{1}{\sin x}$ ($\csc$ is read \emph{cosecant})
			%	    \item   	The inverse function of $\sin(x)$ is denoted as $\sin^{-1}(x)$ or $\arcsin(x)$,
			%	    		not  to be confused with $(\sin(x))^{-1}=\frac{1}{\sin(x)} = \csc(x)$.
			%	    \item	The number $\sin(\theta)$ is the length-ratio of the
			%	    		vertical side and the hypotenuse in a right-angle triangle with angle $\theta$ at the base.


		\subsubsection{Cosine function}

			The cosine function is the same as the sine function \emph{shifted} by $\frac{\pi}{2}$ to the left:
			$f(\theta) = \cos(\theta) = \sin(\theta+\frac{\pi}{2})$,
			as shown in Figure~\ref{fig:panel_function_graphs2}~(b).
			The cosine function represents the horizontal position of the unit circle,
			and the $x$-component of unit length,
			as illustrated in Figure~\ref{fig:trigonometric_functions_as_point_or_as_triangle}.


			%	The graph of the function $y=\cos(x)$  passes through the following $(x,y)$ coordinates:
			%	$(0,1)$, $(\frac{\pi}{6},\frac{\sqrt{3}}{2})$, $(\frac{\pi}{4},\frac{\sqrt{2}}{2})$, $(\frac{\pi}{3},\frac{1}{2})$, 
			%	$(\frac{\pi}{2},0)$, $(\frac{2\pi}{3},-\frac{1}{2})$, 
			%	$(\frac{3\pi}{4}, -\frac{\sqrt{2}}{2})$, $(\frac{5\pi}{6}, -\frac{\sqrt{3}}{2})$, and $(\pi,-1)$.
			%		
			%		The cos function starts at $\cos(0)=1$, then drops down to 
			%		cross the $x$-axis at $x=\frac{\pi}{2}$.
			%		Cos continues until it reaches its minimum value at $x=\pi$. 
			%		The function then moves upward, crossing the $x$-axis again at 
			%		$x=\frac{3\pi}{2}$, and reaching its maximum value again at $x=2\pi$.

			%	    \item	Domain: $\mathbb{R}$
			%	    \item	Image: $\{ y \in \mathbb{R} \; | \;  -1 \leq y \leq 1 \}$
			%	    \item	Roots: $\{\, \ldots, \, -\frac{3\pi}{2}, \, -\frac{\pi}{2}, \frac{\pi}{2}, \frac{3\pi}{2},  \frac{5\pi}{2}, \, \ldots\,\}$
			%	    \item	Relation to $\sin$: $\sin^2 x + \cos^2 x = 1$
			%	    \item	Relation to $\sec$: $\sec(x) = \frac{1}{\cos x}$ ($\sec$ is read \emph{secant})
			%	    \item	The inverse function of $\cos(x)$ is denoted $\cos^{-1}(x)$ or $\arccos(x)$.
			%	    \item	The $\cos$ function is \emph{even}: $\cos(x) = \cos(-x)$
			%    	    \item	The number $\cos(\theta)$ is the length-ratio of the
			%    			horizontal side and the hypotenuse in a right-angle triangle with angle $\theta$ at the base



		\subsubsection{Tangent function}

			The tangent function is defined as the ratio of the sine and cosine functions:
			$f(\theta) = \tan(\theta) \eqdef \frac{ \sin(\theta) } { \cos(\theta) }$.
			The graph of the tangent functionis shown in Figure~\ref{fig:panel_function_graphs2}~(c).

			%    \item	Domain: $\{ x\in \mathbb{R} \;|\; x\neq \frac{(2n+1)\pi}{2} \textrm{ for any } n \in \mathbb{Z} \} $
			%    \item	Image: $\mathbb{R}$
			%    \item	The function $\tan$ is periodic with period $\pi$.
			%    \item	The $\tan$ function ``blows up'' at values of $x$ where $\cos x=0$.
			%    		These are called \emph{asymptotes} of the function
			%		and their locations are $x=\ldots, \frac{-3\pi}{2},\frac{-\pi}{2},\frac{\pi}{2},\frac{3\pi}{2},\ldots$.
			%		% The values of $\tan$ approaches $\infty$ from the left, and $-\infty$ from the right.
			%    \item   	Value at $x=0$: $\tan(0)=\frac{0}{1}=0$, because $\sin(0)=0$.
			%    \item   	Value at $x=\frac{\pi}{4}$: $\tan\left(\frac{\pi}{4} \right) 
			%		       = \frac{ \sin(\frac{\pi}{4}) }{ \cos(\frac{\pi}{4}) }
			%		       = \frac{ \frac{\sqrt{2}}{2}  }{ \frac{\sqrt{2}}{2}  }
			%		       = 1$.
			%    			    \item	The number $\tan(\theta)$ is the length-ratio of the
			%    		vertical and the horizontal sides in a right-angle triangle with angle $\theta$.
			%    \item	The inverse function of $\tan(x)$ is denoted $\tan^{-1}(x)$ or $\arctan(x)$.
			%    \item 	The inverse tangent function is used to compute the angle at the base 
			%    		in a right-angle triangle with horizontal side length~$\ell_h$ and vertical side length $\ell_v$: 
			%		$\theta=\tan^{-1}\!\left(\frac{\ell_v}{\ell_h}\right)$.



		\begin{figure}[htb]
			\centering
			\includegraphics[width=0.99\columnwidth]{figures/calculus/panel_function_graphs2.pdf}%
			\vspace{-2mm}
			\caption{	Graphs of the trigonometric functions $\sin(\theta)$, $\cos(\theta)$, and $\tan(\theta)$.}
			\label{fig:panel_function_graphs2}
		\end{figure}


		% APPLICATIONS
		We can use the trigonometric functions $\sin(\theta)$, $\cos(\theta)$, and $\tan(\theta)$
		to calculate the \emph{components} of vectors.
		Sines and cosines are also describe waves and periodic motion in physics.


