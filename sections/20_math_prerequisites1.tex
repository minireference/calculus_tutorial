%!TEX root = ../calculus_tutorial.tex

\section{Math prerequisites}

	Before we dig into the new calculus topics,
	let's do a quick review of some key concepts from high school math.
	These are the basic building blocks I assume you've seen before,
	or at least heard about them.

	\subsection{Notation for sets and intervals}
	
		Sets are collections of math objects.
		Many math ideas are expressed in the language of sets,
		so it's worth knowing the notation conventions for sets.
	
		\begin{itemize}

			% \item $S,T$: the usual variable names for sets

			\item $\{ \textrm{  definition  } \}$: the curly brackets surround the definition of a set,
				and the expression inside the curly brackets describes what the set contains.
				% The math symbol ``$|$'' is shorthand for the phrase ``such that'' and it is often used in definitions.

			\item $s \in S$: this statement is read ``$s$ is an element of $S$'' or ``$s$ is in $S$''
	
			% seems not needed; put back if needed somewhere in the book (grep for it)
			%	\item $\emptyset$: the \emph{empty set} is a set that contains no elements.
			%		Mathematicians adopted the symbol $\emptyset$ because the notation $\{ \, \}$ is confusing.
	
	
			\item $\mathbb{N}$: the set of natural numbers $\mathbb{N}\eqdef \{0,1,2,3,4,5,\ldots\}$
	
			\item $\mathbb{N}_+$: the set of positive natural numbers $\mathbb{N}_+ \eqdef \{1,2,3,4,5,\ldots\}$.

			%	\item $\mathbb{Z}$: the set of integers $\mathbb{Z}\eqdef\{\ldots,-2,-1,0,1,2,3,\ldots\}$
			%
			%	\item $\mathbb{Q}$: the set of rational numbers,
			%		$\mathbb{Q} \eqdef \left\{  \frac{m}{n} \; \Big| \; m \in \mathbb{Z}, \; n \in \mathbb{N}, \; n \neq 0  \right\}$.
			%		The set $\mathbb{Q}$ consists of all numbers that can be expressed as \emph{fractions} of the form $\frac{m}{n}$,
			%		where $m$ is an integer, $n$ is a natural number, and $n \neq 0$.
	
			\item	$\mathbb{R}$: the set of real numbers.
	
			\item	$\mathbb{R}_+$: the set of nonnegative real numbers.
	
		\end{itemize}
	
		\noindent
		We often use the \emph{set-builder} notation $\{\, \cdot \; | \;  \cdot \, \}$ to define sets.
		Inside the curly brackets,
		we first describe the general kind of mathematical objects we are talking about,
		followed by the symbol ``$|$'' (which stands for ``such that''),
		followed by the conditions that identifies the elements of the set.
		For example,
		the nonnegative real numbers $\mathbb{R}_+$
		are defined as ``all real numbers $x$ such that $x \geq 0$,''
		which can be expressed more compactly as 
		$\mathbb{R}_+ \eqdef \{ x \in \mathbb{R} \; | \; x \geq 0 \}$
		using the set-builder notation.


		\subsubsection{The number line}
	
			The \emph{number line} is a visual representation of the set of real numbers $\mathbb{R}$,
			as shown in Figure~\ref{fig:number_line_rationals_and_reals}.
			The real numbers correspond to all the points on the number line,
			from $-\infty$ to $\infty$.
		
			\begin{figure}[htb]
				\centering
				\includegraphics[width=0.9\columnwidth]{figures/calculus/number_line_rationals_and_reals.pdf}
				\vspace{-2mm}
				\caption{The real numbers $\mathbb{R}$ cover the entire number line.}
				\label{fig:number_line_rationals_and_reals}
			\end{figure}
		
			\noindent
			The set of real numbers includes
			the natural numbers $\{0,1,2,3,\ldots\}$,
			the integers $\{\ldots, -3,-2,-1,0,1,2,3,\ldots\}$,
			the rational numbers like $\frac{5}{3}$, $\frac{22}{7}$, $1.5$,
			as well as irrational numbers like $\sqrt{2}$, $e$, and $\pi$.
			This means any number you run into when solving a math problem
			can be visualized as a point on the number line.
	
			The number line extends forever to the left and to the right.
			We use the notation $-\infty$ (negative infinity)
			to describe larger and larger negative numbers,
			and $+\infty$ to describe larger and larger positive numbers.
			Remember what we said in the introduction,
			$\infty$ is not a number but a process.
			% ALT. not a destination you can get to 



		\subsubsection{Number intervals}
	
			The number line can be used to represent subsets of the real numbers,
			which we call \emph{intervals}.
			Figure~\ref{fig:interval_2closed_to_4closed}
			shows an illustration of the interval $[2,4] \eqdef \{ x \in \mathbb{R} \;|\; 2 \leq x \leq 4 \}$,
			which is the subset of the real numbers between $2$ and $4$.
		
			\begin{figure}[htb]
				\centering	
				\includegraphics[width=0.9\columnwidth]{figures/calculus/interval_2closed_to_4closed.pdf}
				\vspace{-3mm}
				\caption{	The interval $[2,4] \protect\eqdef \{ x \in \mathbb{R} \; | \; 2 \leq x \leq 4 \}$
						is a subset of $\mathbb{R}$.} % the real numbers.
				\label{fig:interval_2closed_to_4closed}
			\end{figure}
	
	
	
	\subsection{Functions}
	
		A \emphindexdef{function} is a mathematical object that takes numbers as inputs and produces numbers as outputs.
		The output of the function $f$ for the input $x$ is denoted $f(x)$.
		For example,
		the function $f(x) \eqdef \frac{1}{2}x^2$
		takes any number $x$ as input,
		squares it and divides the result by two to produce the output.
		For example,
		$f(3) = \frac{1}{2} 3^2 = \frac{9}{2} = 4.5$.
		Here is the Python code that defines the function \tt{f}
		and evaluates it for the input $x=3$.

		\begin{codeblock}[def-fx-one-half-x-squared]
		>>> def f(x):
		        return 0.5 * x**2
		>>> f(3)
		4.5
		\end{codeblock}
		
		\noindent
		Not the Python syntax for evaluating the function \tt{f} on the input \tt{3}
		is the same as the math syntax $f(3)$.
	
		\subsubsection{Function graphs}

			The \emph{graph} of a function is a line that passes through all input-output pairs of a function.
			%	Imagine we take out a piece of paper and draw a coordinate system
			%	with a horizontal axis and a vertical axis.
			%	The horizontal axis describes the different input values $x$,
			%	while the vertical axis describes the output values $f(x)$.
			Each input-output pair of the function $f$ corresponds to the point $(x,f(x))$ in a coordinate system.
			We obtain the graph of the function by varying the input coordinate $x$ and plotting all the points $(x, f(x))$,
			as illustrated in Figure~\ref{fig:graph_of_f_nonnegative}.
			%
			The graph of the function $f$ allows us to see at a glance the behaviour of the function for all possible inputs,
			and forms an essential visualization tool.
			Calculus calculations can be understood geometrically as operations based on the graph of the function.

			We can use the Python modules \tt{numpy} and \tt{seaborn} to plot the graph of any function.
			For example,
			consider the function $f(x) \eqdef \frac{1}{2}x^2$ that we defined earlier.
			We start by importing the module \tt{numpy} under the alias \tt{np},
			and evaluating the function for all inputs $x$ in the interval $[-3,3]$.

			\begin{codeblock}[plot-fx-minus3-to-plus3]
			>>> import numpy as np
			>>> xs = np.linspace(-3, 3, 1000)
			>>> fxs = f(xs)
			\end{codeblock}

			We used the function \tt{np.linspace} to create an array (a list of numbers) \tt{xs},
			which contains 1000 input values that range from $x=-3$ until $x=3$.
			Next we applied the function $f$ to the array of inputs \tt{xs}
			and stored the outputs of the function in the array \tt{fxs}.
			% which contains all the output values of the function for the input values \tt{xs}.
			At this point,
			the arrays \tt{xs} and \tt{fxs} contain $1000$ input-output pairs of the form $(x, f(x))$,
			which is exactly what we need to plot the graph of the function.
	
			\begin{codeblock}[plot-fx-minus3-to-plus3]
			>>> import seaborn as sns
			>>> sns.lineplot(x=xs, y=fxs)
			See Figure ¡\ref{fig:graph_of_function_f_eq_halfx2}¡ for the output.
			\end{codeblock}
	
			\noindent
			We imported the \tt{seaborn} module under the alias \tt{sns}
			then called the function \tt{sns.lineplot} to produce the graph of $f(x)$
			shown in Figure~\ref{fig:graph_of_function_f_eq_halfx2}.

			\begin{figure}[htb]
				\centering
				\includegraphics[width=0.9\columnwidth]{figures/calculus/graph_of_function_f_eq_halfx2.pdf}%
				\vspace{-3mm}
				\caption{	Graph of the function $f(x)=\frac{1}{2}x^2$ from $x=-3$ until $x=+3$.
						The graph of the function $f$
						consists of all the coordinate pairs $(x,f(x))$
						over some interval of $x$ values.	}
				\label{fig:graph_of_function_f_eq_halfx2}
			\end{figure}
	
	
	
		\subsubsection{Inverse functions}
	
			The inverse function $f^{-1}$ % $f^{-1} \colon B \to A$
			performs the \emph{inverse operation} of the function $f$. % $f \colon A \to B$.
			If you start from some $x$,
			apply $f$,
			then apply $f^{-1}$,
			you'll arrive---full circle---back to the original input $x$:
			\[
				f^{-1}\!\big( \; f(x) \; \big) = x.
			\]
			In Figure~\ref{fig:functions-inverse},
			the function $f$ is represented as a forward arrow,
			and the inverse function $f^{-1}$ is represented as a backward arrow
			that puts the value $f(x)$ back to the $x$ it came from.
	
			\begin{figure}[htb]
				\centering
				\includegraphics[width=0.4\columnwidth]{figures/calculus/functions-inverse.pdf}
				\caption{The inverse $f^{-1}$ undoes the operation of the function $f$.}
				\label{fig:functions-inverse}
			\end{figure}
	
			For example,
			when $x \geq 0$,
			the inverse of the function $f(x) = \frac{1}{2}x^2$
			is the function $f^{-1}(x) = \sqrt{2x}$.
			Earlier we computed $f(3) = 4.5$.
			If we apply the inverse function to $4.5$,
			we get $f^{-1}(4.5) = \sqrt{2 \cdot 4.5} = \sqrt{9} = 3$.

	
			\begin{codeblock}[fun-inv-fun-combo-log]
			>>> from math import exp, log
			>>> log(exp(5))
			5.0
			\end{codeblock}
	


		\subsubsection{Function properties}

			We often think about the possible inputs and outputs of functions.
			We use the notation $f \colon A \to B$
			to denote a function from the input set $A$ to the output set $B$.
			The set of allowed inputs is called the \emph{domain} of the function,
			while the set of possible outputs is called the \emph{image} or \emph{range} of the function.
			For example,
			the domain of the function $f(x)= \frac{1}{2}x^2$
			is $\mathbb{R}$ (any real number)
			and it's image is $\mathbb{R}_+$ (nonnegative real numbers),
			so we write it as $f \colon \mathbb{R} \to \mathbb{R}_+$.
			
			%	Another important property is called \emph{continuity},
			%	which roughly corresponds to ability to draw a function without lifting the pen.
			%	We'll give a formal definition of continuity later in Section~\ref{limits:continuity}.


	\subsection{Function inventory}
	
		% (10 essential functions)
		%|x|
		%line
		%quadratic
		%square root
		%polynomial
		%exp
		%log
		%gaussian-like erf $e^{-x^2}$
		%sigmoid $\frac{1}{1-e^{-x}}$

			\begin{figure}[htb]
				\centering
				\includegraphics[width=0.99\columnwidth]{figures/calculus/panel_function_graphs1.pdf}%
				\vspace{-2mm}
				\caption{	Graph of six math functions you should know about.}
				\label{fig:panel_function_graphs1}
			\end{figure}



%		\vspace{2in}



	\subsection{Functions with discrete inputs}

		Later in this tutorial,
		we'll study functions with discrete inputs,
		$a_k : \mathbb{N} \to \mathbb{R}$,
		which are called sequences.
		We often express sequences by writing explicitly the first possible value
		$[a_0, a_1, a_2, a_3, \ldots]$,
		which correspond to evaluating $a_k$ for $k=0$, $k=1$, $k=2$, $k=3$, etc.



	\subsection{Geometry}
	
		We'll now briefly review some geometry formulas.
		
		\subsubsection{Circle}
		
			The area enclosed by a circle of radius $r$ is given by $A = \pi r^2$,
			where $\pi = 3.14159\ldots$.
			A circle of radius $r=1$ has area $\pi$.
			The circumference of a circle of radius $r$ is $C = 2 \pi r$.
			A circle of radius $r=1$ has circumference $2\pi$.

		
		\subsubsection{Rectangle}

			The area of a rectangle of base $b$ and height $h$ is $A = bh$.


		\subsubsection{Triangle}

			The area of a triangle
			is equal to $\frac{1}{2}$ times the length of its base $b$ times its height $h$
			$A = \tfrac{1}{2} b h$.
			% Note that $h_a$ is the height of the triangle \emph{relative to} the side $a$.

		\noindent
		The three area formulas are illustrated in Figure~\ref{fig:geometry_areas_circle_rect_triangle}.

		\begin{figure}[htb]
			\centering
			\includegraphics[width=0.99\columnwidth]{figures/calculus/geometry_areas_circle_rect_triangle.pdf}
			% hexagon-octagon-dodecagon.png}
			\vspace*{-6mm}
			\caption{	Area formulas for a circle, a rectangle, and a triangle.}
			\label{fig:geometry_areas_circle_rect_triangle}
		\end{figure}






	\subsection{Trigonometry}
	
		

		
		\begin{figure}[htb]
			\centering
			\includegraphics[width=0.99\columnwidth]{figures/calculus/panel_function_graphs2.pdf}%
			\vspace{-2mm}
			\caption{	Graph of trigonometric functions $\sin\theta$, $\cos\theta$, and $\tan\theta$.}
			\label{fig:panel_function_graphs2}
		\end{figure}


		%FIGURE  right-angle triangle with hypotenuse r, adj = rcosθ, opp = rsinθ
		%define functions: sin, cos
		%FIGURE  plot of cos and sin functions
		%applications: vector

%		\vspace{2in}
