%!TEX root = ../calculus_tutorial.tex

\section{Vector calculus}

	A discussion on vector calculus is out of scope for an introductory tutorial.
	Like, \emph{waaaaay} out of scope.
	However,
	I want to show you one picture and explain
	where you might encounter vector calculus concepts in your future studies.
	%	Vector calculus is \emph{waaaaay} out of scope for an introductory calculus tutorial,
	%	so I will just show you some simple definitions of the building blocks.	
	%
	Vector calculus is the study of \emph{vector fields} and their properties.
	In a three dimensional space,
	vector fields are functions of the form $\mathbf{F}: \mathbb{R}^3 \to \mathbb{R}^3$.
	The vector field $\mathbf{F}$ assigns a three-dimensional output vector $\mathbf{F} = (F_x, F_y, F_z)$
	for each point $\mathbf{r} = (x,y,z)$ in $\mathbb{R}^3$.
	Note we use boldface to denote vectors.
	
	\subsection{Example: electric field around a positive charge}
	
		Figure~\ref{fig:vector_calc_E_field_positive_charge}
		shows the \emph{electric field} $\mathbf{E}$
		around a positive charge~$q$ Coulombs
		located at the origin of the three-dimensional coordinate system. % $(0,0,0)$.
		The electric field is strongest close to the charge,
		and gets weaker as you move away from the charge.
	
	
		\begin{figure}[htb]
			\centering
			\includegraphics[width=0.6\columnwidth]{figures/calculus/vector_calc_E_field_positive_charge.pdf}
			\vspace{-2mm}
			\caption{	The electric field $\mathbf{E}(x,y,z) = \frac{kq}{r^2} \hat{ \mathbf{r} }$ around a positive charge $q$.}
			\label{fig:vector_calc_E_field_positive_charge}
		\end{figure}
	
		The strength of the electric field $\mathbf{E}$ 
		at the point $\mathbf{r} = (x,y,z)$ is
		\[
			\mathbf{E}(x,y,z)
				= \frac{kq}{(x^2 + y^2 + z^2)^{3/2}}(x,y,z)
				= \frac{kq}{r^3}  \mathbf{r}
				= \frac{kq}{r^2} \hat{ \mathbf{r} },
		\]
		where $k$ is Coulomb's constant,
		$r \eqdef |\mathbf{r}| = \sqrt{x^2+y^2+z^2}$ is the distance from the origin,
		and $\hat{\mathbf{r}} \eqdef \frac{\mathbf{r}}{r}$
		is a \emph{unit vector} in the same direction as $\mathbf{r}$.
		%
		Electric fields are used in the study of electromagnetism.
		Specifically,
		the electric field $\mathbf{E}(x,y,z)$ describes the strength and the direction
		of the \emph{electric force} that a charged particle would experience if placed at $(x,y,z)$.




	% DERIVATIVES
	\subsection{Vector calculus derivatives}

		There are two derivative operations for vector fields,
		and these are written in terms of the vector derivative operator $\nabla$ (\emph{nabla}),
		which is defined as $\nabla \eqdef \big( \frac{\partial}{\partial x}, \frac{\partial}{\partial y}, \frac{\partial}{\partial z} \big)$.
		The \emph{divergence} of the vector field $\mathbf{F}$
		is computed by taking the ``dot product'' of $\nabla$ and the vector field $\mathbf{F} = (F_x, F_y, F_z)$:
		\[
			\nabla \cdot \mathbf{F}(x,y,z)
				= \frac{\partial F_x}{\partial x} 	+ \frac{\partial F_y}{\partial y}	+ \frac{\partial F_z}{\partial z}.
		\]
		The divergence tells us if the field $\mathbf{F}$
		is acting as a ``source'' or a ``sink'' at the point $(x,y,z)$.
		% For example, positive charges are sources of electric field lines
		% while negative charges are sinks (arrows go into them).
		% Example: electric charge
		% Positive charges have positive divergence---we say they are the ``sources'' of the electric field.
		% In contrast, negative charges have negative divergence and we say they are the ``sinks'' of the electric field.

		% ALT. measures the local volumetric outflow rate per unit volume of the vector field.

		The \emph{curl} of the vector field $\mathbf{F}$ is defined as the ``cross product''
		of $\nabla$ and the vector field $\mathbf{F} = (F_x, F_y, F_z)$:
		\[
			\nabla \times \mathbf{F}(x,y,z)
				= 	\left(
						\frac{\partial F_z}{\partial y} - \frac{\partial F_y}{\partial z}, \;\;
						\frac{\partial F_x}{\partial z} - \frac{\partial F_z}{\partial x}, \;\;
						\frac{\partial F_y}{\partial x} - \frac{\partial F_x}{\partial y}
					\right)\!.
		\]
		The curl tells us the rotational tendency of the vector field $\mathbf{F}$.

		% INTUITION DIV and CURL
		%	In electromagnetic phenomena, 
		%	we have the electric field and the magnetic field.
		%	You can think of this these as two separate fields,
		%	but it's better to understand the two fields as ``components'' of the electromagnetic field,
		%	in analogy with the classical motion which splits into linear motion and rotational motion.
		%
		%	We need two different derivatives to capture different aspects of the electromagnetic field.
		%	Electric field calculations depend on the divergence, while magnetic calculations involve the cross product and the right-hand rule.







	% INTEGRALS
	\subsection{Vector calculus integrals}

		There are several different kinds of integral operations you can use with vector fields,
		depending on the type of ``total'' you want to compute,
		and the region of integration.
		%
		% PATH INTEGRALS
		The concept of a vector path integral is denoted
		$\int_C \mathbf{F}(\mathbf{r}) \cdot d\mathbf{r}$,
		where $C$ is some curve in three dimensional space,
		and $d\mathbf{r}$ describes a short step along this curve.
		This integral computes the total action of the vector field $\mathbf{F}$
		in the direction along the curve $C$.
		%
		% SURFACE INTEGRALS
		The vector surface integral is denoted
		$\iint_S \mathbf{F}(\mathbf{r}) \cdot d\mathbf{S}$,
		where $S$ is some surface in three dimensional space,
		and $d\mathbf{S}$ %  = \hat{\mathbf{n}}dS$
		is a small ``piece of the surface.''
		This integral computes the total \emph{flux} of the vector field $\mathbf{F}$
		flowing through the surface $S$.
		% To obtain the component of $\mathbf{F}$ in the direction of the surface normal,
		% we take the dot product with  $\eqdef  \eqdef (\mathbf{r}_u^\prime \times \mathbf{r}_v^\prime) dudv$,


		% VECTOR CALCULUS THEOREMS
		The two main results in vector calculus
		are \href{https://en.wikipedia.org/wiki/Divergence_theorem}{\emph{Gauss' divergence theorem}}
		and \href{https://en.wikipedia.org/wiki/Stokes's_theorem}{\emph{Stokes' theorem}}.
		Both theorems can be understood as extensions of the fundamental theorem of calculus (FTC),
		since they show equivalences between
		certain vector derivative and integral operations.

	%	% PARAMETRIZATION
	%	The main practical skill you'll learn if you take a vector calculus course
	%	is how to \emph{parametrize} regions of space.
	%	This skill is closely related to the substitution trick (change of variables)
	%	that we saw in Section~\ref{integrals:techniques_of_integration},
	%	but working with vectors.
	%		thing we'll have to learn is how to parametrize .
	%		In fact, we could even say that the main purpose of this course is to get you comfortable
	%		with parameterizations of curves, surfaces, and volumes.
	%		Once you have a parametrization for a region you can perform any integral calculation over this region.
	%	A big part of the day today activities in vector calculus involves manual labour like defining regions of integration, 
	%	changing variables, setting up boundaries. 

	% TODO: mention parametirzation ~= substitution trick it has great power -- basis of multivariable calculus when using parameterization BACKREF




	\subsection{Applications of vector calculus}

		Vector calculus is the math machinery used in physics
		(electricity and magnetism, mechanics, thermodynamics)
		and electrical engineering.
		If you're not planning to work in these fields, % become a physicist or an engineer,
		you can probably skip vector calculus:
		it's just derivatives and integrals applied to vector quantities.
		
		