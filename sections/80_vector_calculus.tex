%!TEX root = ../calculus_tutorial.tex

\section{Vector calculus}

	Vector calculus is the study of vector fields $\mathbf{F}$,
	which are functions of the form $\mathbf{F}:\mathbb{R}^3 \to \mathbb{R}^3$,
	which defines $3$-dimensional output vector
	at each point $(x,y,z)$ in space.
	For example,
	the electric field $\mathbf{E}(x,y,z)$ describes the strength and the direction of the
	electric force that a changed particle would experience if placed at $(x,y,z)$.

	Vector calculus is \emph{waaaaay} out of scope for an introductory calculus tutorial,
	so I will just show you some simple definitions of the building blocks.	


		\begin{figure}[htb]
			\centering
			\includegraphics[width=0.7\columnwidth]{figures/calculus/vector_calc_E_field_positive_charge.pdf}
			\vspace{-3mm}
			\caption{	Visualization of the vector field $\vec{E}(x,y,z)$ around a positive charge.}
			\label{fig:multivar_combined_slices_through_paraboloid}
		\end{figure}

For a point charge (q) located at the origin, the electric field at position
$\vec{r} = (x,y,z)$ is

\[
	\vec{E}(x,y,z) = \frac{kq}{(x^2 + y^2 + z^2)^{3/2}} (x,y,z)
\]

Then the field can be expressed compactly as:

\[
	\vec{E}(x,y,z)  = \frac{kq}{r^2} \hat{r} =  \frac{k q}{r^3}(x,y,z).
\]
where $\vec{r} = (x,y,z)$,
$r = |\vec{r}| = \sqrt{x^2+y^2+z^2}$,
and $\hat{r} = \frac{\vec{r}}{r}$.


\[
	\vec{E}(x,y,z) = 
\]


Let me know if you want the same expression in spherical coordinates or the field of a dipole or multipole.



% TODO: mention parametirzation ~= substitution trick it has great power -- basis of multivariable calculus when using parameterization BACKREF
		
			
	\subsection{Definitions}

		\begin{itemize}

			\item $\mathbf{r} = (x,y,z)$: position in space with coordinates $(x,y,z)$.
			\item $\nabla \eqdef \big( \frac{\partial}{\partial x}, \frac{\partial}{\partial y}, \frac{\partial}{\partial z} \big)$:
				the vector derivative operator (\emph{nabla}).

			\item	$\nabla \cdot \vec{F}(x,y,z)$: the \emph{divergence} of the field $\vec{F}$
				tells us if field $\vec{F}$ is acting as a ``source'' or a ``sink'' at the point $(x,y,z)$.
				% For example, positive charges are sources of electric field lines
				% while negative charges are sinks (arrows go into them).
				% Example: electric charge
				% Positive charges have positive divergence---we say they are the ``sources'' of the electric field.
				% In contrast, negative charges have negative divergence and we say they are the ``sinks'' of the electric field.

			\item	$\nabla \times  \vec{F}(x,y,z)$: the \emph{curl} of the field $\vec{F}$
				tells us the ``rotational tendency'' of the vector field $\vec{F}$ at $(x,y,z)$.
				% Example: current carrying wire
				%	A current carrying wire causes a magnetic field everywhere around it.
				%	The magnetic field lines will circle around the wire.
				%	We say $\mathbf{B}$ has a circulation, or curl.

		\end{itemize}


		% INTUITION DIV and CURL
		%	In electromagnetic phenomena, 
		%	we have the electric field and the magnetic field.
		%	You can think of this these as two separate fields,
		%	but it's better to understand the two fields as ``components'' of the electromagnetic field,
		%	in analogy with the classical motion which splits into linear motion and rotational motion.
		%
		%	We need two different derivatives to capture different aspects of the electromagnetic field.
		%	Electric field calculations depend on the divergence, while magnetic calculations involve the cross product and the right-hand rule.



	\subsection{Path integrals}

		Consider the curve $C$ traced by $r(t)$ as $t$ varies from $t_i$ and $t_f$.

		 path integrals of vectors fields,

		Scalar path integral.
			\[
				\int_C f(\mathbf{r})\, dr 
				\eqdef
				\int_{t_i}^{t_f} f(\mathbf{r}) \, \|\mathbf{r}'(t)\| \, dt.
			\]
			Here the curve $C \in \mathbb{R}^3$ is described by the parametrization $\mathbf{r}: \mathbb{R} \to \mathbb{R}^3$,
			which assigns a coordinate $\mathbf{r}=(x,y,z)$ for each value of the parameter (denoted $t$ in the above).
			Note $dr \eqdef \|\mathbf{r}'(t)\| \, dt$,
			which involves computing the derivative of $\mathbf{r}(t)$ then computing the length.

		Vector path integral. 
			\[
				\int_C \mathbf{F}(\mathbf{r}) \cdot d\mathbf{r} 
				\eqdef 
				\int_{t_i}^{t_f} \mathbf{F}(\mathbf{r})  \cdot \mathbf{r}'(t) \, dt.
			\]
			This integral computes the total of the vector field $\mathbf{F}$ in the direction of the tangent line to the curve $C$ describe by $\mathbf{r}(t)$.
			To obtain the component of $\mathbf{F}$ in the direction of the tangent line,
			we take the dot product with $d\mathbf{r}  \eqdef \mathbf{r}'(t) dt$ during each step.



	\subsection{Surface integrals}

		flux integrals of vectors fields through surfaces,


		Scalar surface integral.
			\[
				\iint_S f(\mathbf{r})\, dS
				\eqdef
				\int_{v_i}^{v_f} \int_{u_i}^{u_f} f(\mathbf{r}) \, \| \mathbf{r}_u^\prime \times \mathbf{r}_v^\prime \| \, dudv.
			\]
			Here the surface $S \in \mathbb{R}^3$ is described by the parametrization $\mathbf{r}: \mathbb{R}^2  \to \mathbb{R}^3$,
			which assigns a coordinate $\mathbf{r}=(x,y,z)$ for each pair of the parameter $u$ and $v$.
			Note $dS 	= \| \mathbf{r}^\prime_u \times \mathbf{r}^\prime_v \| \, dudv$,
			which involves computing the partial derivatives of $\mathbf{r}(u,v)$ with respect to the two parameters,
			taking the cross product, then computing the length.

		Vector surface integral. 
			\[
				\iint\limits_S \mathbf{F}(\mathbf{r}) \cdot d\mathbf{S}
				\eqdef 
				\int_{v_i}^{v_f} \int_{u_i}^{u_f} \mathbf{F}(\mathbf{r})  \cdot (\mathbf{r}_u^\prime \times \mathbf{r}_v^\prime) \, dudv.
			\]
			This integral computes the total \emph{flux} of the vector field $\mathbf{F}$ flowing
			perpendicularly through the surface $S$.
			To obtain the component of $\mathbf{F}$ in the direction of the surface normal,
			we take the dot product with $d\mathbf{S}  \eqdef \hat{\mathbf{n}}dS \eqdef (\mathbf{r}_u^\prime \times \mathbf{r}_v^\prime) dudv$,
			for each piece of the surface.




	% PARAMETRIZATION
	The main thing we'll have to learn is how to parametrize regions of space.
	In fact, we could even say that the main purpose of this course is to get you comfortable with parameterizations of curves, surfaces, and volumes.
	Once you have a parametrization for a region you can perform any integral calculation over this region.
	%	A big part of the day today activities in vector calculus involves manual labour like defining regions of integration, 
	%	changing variables, setting up boundaries. 


	\subsection{Vector calculus theorems}

		The main results in vector calculus are two theorems:
		\emph{Gauss' divergence theorem} and \emph{Stokes theorem}.
		Both theorems can be understood as extensions of the fundamental theorem of calculus (FTC),
		which relates the integral of the differential of some quantity over a region $R$ to the value of this quantity on the boundary of a region,
		denoted $\partial R$.
		In the case of the fundamental theorem of calculus,
		the region is the interval $I = [a,b] \subseteq \mathbb{R}$ whose boundary $\partial I$ consists of the two points $a$ and $b$.
		The fundamental theorem of calculus is
		\[
		  \int_a^b f^\prime(x) \; dx = \int_I f^\prime(x) \; dx  = f_{\partial I}  = f(b) - f(a),
		\]

		\noindent
		\textbf{Gauss' Divergence Theorem} relates the volume integral of the quantity $\nabla \cdot \vec{F}$, 
		which is called the divergence of $\vec{F}$,
		to the total flux of the vector field through the surface  $\partial\mathrm{V}$,
		which is the boundary of the volume $V$.
		%which states the integral of the divergence $\textrm{div}(\vec{F}) \eqdef \nabla \cdot \vec{F}$ 
		%of the field $\vec{F}$ over the volume $\mathrm{V}$ 
		%is equal to the flux of $\vec{F}$ through the volume's boundary $\partial\mathrm{V}$.
		Gauss' divergence theorem is:
		\[
		    \iiint_{\mathrm{V}} \nabla \cdot \vec{F} \ d\mathrm{V} 
		    = \int\!\!\!\int_{\partial    \mathrm{V}} \vec{F} \cdot d \vec{S}
		\] 
		Intuitively, the \emph{divergence} of a vector field describes how much of the vector field emanates from a given point in space.
		The \emph{flux} of a vector field over a surface $S$ accounts for the strength of the vector field flowing through the surface.
		%
		In the above example, we saw Gauss' divergence theorem applied to the electric field,
		but the vector field $\vec{F}$ could also represent thermal flows, or fluid flows.

		\noindent
		\textbf{Stokes' Theorem} uses the ``other'' vector derivative $\nabla \times \vec{F}$, which is called the \emph{curl} of $\vec{F}$.
		The curl of a vector field, denoted $(\nabla \times \vec{F})(x,y,z)$
		describes the local rotational tendency of the vector field $\vec{F}$ at the point $(x,y,z)$.	
		%
		Given any surface $S$ in space,
		we can cut up the surface into tiny little rectangles and calculate the total surface area as a double integral $S = \int dS$.
		Stokes' theorem is the application of this ``splitting up into little squares'' idea and the fundamental theorem of calculus, 
		which leads us to the following equation.
		\[
		    \iint_{\Sigma} \nabla \times \vec{F} \cdot d\vec{S} 
		    = \int_{\partial\Sigma} \vec{F} \cdot d \vec{r},
		\] 
		The surface integral of the curl $\nabla \times \vec{F}$ over any surface $\Sigma$ 
		is equal to the circulation of $\vec{F}$ along the boundary of the surface  $\partial\Sigma$.
		%
		Both the left and right sides of this equation correspond to scalar numbers.
		The left side is the vector surface integral of a vector quantity (the curl of $\vec{F}$),
		the right side corresponds to a vector path integral of a vector quantity over an oriented curve $\partial \Sigma$.


		%\begin{itemize}
		%    \item   	Stokes' Theorem: 
		%    		\[
		%		    \int\!\!\!\int_{\Sigma} \nabla \times \vec{F} \cdot d\vec{S} 
		%		    = \int_{\partial\Sigma} \vec{F} \cdot d \vec{r},
		%		\] 
		%		which states that the integral of the curl, $\textrm{curl}(\vec{F}) \eqdef \nabla \times \vec{F}$, 
		%		of the field $\vec{F}$ over the surface $\Sigma$ 
		%		is equal to the circulation of $\vec{F}$ along the boundary of the surface  $\partial\Sigma$.
		%
		%    \item   	Gauss' Divergence Theorem: 
		%    		\[
		%		    \int\!\!\!\int\!\!\!\int_{\mathrm{V}}
		%		    		\nabla \cdot \vec{F} \; dV
		%		    = 	\int\!\!\!\int_{\partial    \mathrm{V}}
		%		    		\vec{F} \cdot d \vec{S},
		%		\]
		%    		which states the integral of the divergence, $\textrm{div}(\vec{F}) \eqdef \nabla \cdot \vec{F}$, 
		%		of the field $\vec{F}$ over the volume $\mathrm{V}$ 
		%		is equal to the flux of $\vec{F}$ through the volume's boundary $\partial\mathrm{V}$.
		%\end{itemize}
		%	Both theorems relate the total of the derivative of a quantity over some region $R$ 
		%	to the value of that quantity on the boundary of the region,
		%	which we denote $\partial R$. 
		%	The fundamental theorem of calculus can also be interpreted in this manner:
		%	\[
		%	  \int_a^b F^\prime(x) \, dx
		%	  	= \int_I F^\prime(x) \, dx
		%		= F_{\partial I}
		%		= F(b) - F(a),
		%	\]
		%	where $I=[a,b]$ is the \emph{interval} from $a$ to $b$ on the real line and the
		%	two points $a$ and $b$ form its boundary $\partial I$.



	\subsection{Applications of vector calculus}

		% Vector calculus is of interest mainly for physicists and engineers.

		Vector calculus is the math machinery used for electricity and magnetism,
		which is the study electric field $\mathbf{E}(x,y,z)$,
		the magnetic field $\mathbf{B}(x,y,z)$,
		and the interactions between them.

