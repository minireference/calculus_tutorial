%!TEX root = ../calculus_tutorial.tex

\section{Integrals}
\label{sec:integrals}

	Integration is process of computing the ``total'' of some quantity that varies over time.
	The integral sign $\int$ used to denote integrals is an elongated letter $S$,
	as a reminder that we're summing together some quantity.

	There are actually two different tasks that are both called integration.
	The \emph{definite integral} of $f(x)$ between $x=a$ and $x=b$ 
	is denoted $\int_{x=a}^{x=b} f(x)\,dx=A(a,b)$
	and correspond to the computation of the area under graph of $f(x)$ between $a$ and $b$.
	The definite integral is a number $A(a,b) \in \mathbb{R}$.
	In contrast,
	the \emph{indefinite integral} of $f(x)$ is denoted $\int_{x=0}^{x=b} f(x)\,dx = F_0(b)$
	is a \emph{function} that describes the are-under-the-graph-of-$f(x)$
	with a variable upper limit of integraiton.
	The two integration operations are related.
	The area under the curve $A(a,b)$ can be computed as the \emph{change}
	integral function: $A(a,b)=F_0(b)-F_0(a)$.
	Both integration tasks are important,
	and we'll discuss each of them in turn.

