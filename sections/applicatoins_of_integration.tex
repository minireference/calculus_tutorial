%!TEX root = ../calculus_tutorial.tex


\subsection{Applications of integration}
\label{mathematical_preliminiaries:applications-of-integration}
	
			One of the key applications of integration to computing probabilities for continuous random variables.
			A continuous random variable $X$ is described by its probability density function $f_X$
			and the probability of the event $\{ a \leq X \leq b\}$ is defined as the following integral:
			\[
				\Pr( \{ a \leq X \leq b\} )
				\equiv
				\int_a^b f_X(x)\:dx.
			\]
			The probability density $f_X$ varies for different values of $x$,
			so if we want to compute the total probability of $X$ falling between $x=a$ and $x=b$,
			we must compute the integral of $f_X$ between $x=a$ and $x=b$.
			
			We also use integration to compute \emph{expectations} for quantities that depend on continuous random variables.
			The expected value of the quantity $G=g(X)$ under the randomness of a continuous random variable $X$
			is defined as the following integral calculation:
			\[
				\EE_X [G]
				\equiv
				\EE_X[g(X)]
				\equiv
				\int_{x \in \calX} g(x)f_X(x)\:dx.
			\]
			The expected value is computed by ``weighing'' each value of $g(x)$ by the corresponding probability density for the event $\{X=x\}$,
			summed over all possible values for the random variable $X$.
	
			The mean $\mu = \EE_X[X]$ and the variance $\sigma^2 = \EE_X[(X-\mu)^2]$
			are two central concepts in probability theory and statistics that are computed as expectation integrals.
			Every time we use the $\EE_X$ notation in Section~\ref{sec:continuous_prob_distr},
			there will be some integral calculation going on behind the scenes,
			so if you want know what's going on you need to know a thing or two about integrals.
	
