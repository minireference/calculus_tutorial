%!TEX root = ../calculus_tutorial.tex

\section{Math prerequisites}

	\subsection{Notation for sets and intervals}
	
		Sets are arbitrary collections of math objects.
		Many math ideas are expressed using the language of sets,
		so it's worth going over the basic definitions and notation conventions.
	
		\begin{itemize}
	
			\item $S,T$: the usual variable names for sets
	
			\item $s \in S$: this statement is read ``$s$ is an element of $S$'' or ``$s$ is in $S$''
	
			% seems not needed; put back if needed somewhere in the book (grep for it)
			%	\item $\emptyset$: the \emph{empty set} is a set that contains no elements.
			%		Mathematicians adopted the symbol $\emptyset$ because the notation $\{ \, \}$ is confusing.
	
			\item $\{ \textrm{  definition  } \}$: the curly brackets surround the definition of a set,
				and the expression inside the curly brackets describes what the set contains.
				% The math symbol ``$|$'' is shorthand for the phrase ``such that'' and it is often used in definitions.
	
			\item $\mathbb{N}$: the set of natural numbers $\mathbb{N}\eqdef\{0,1,2,\ldots\}$
	
			\item $\mathbb{N}_+$: the set of positive natural numbers $\mathbb{N}_+ \eqdef \{1,2,3\ldots\}$.

			%	\item $\mathbb{Z}$: the set of integers $\mathbb{Z}\eqdef\{\ldots,-2,-1,0,1,2,3,\ldots\}$
			%
			%	\item $\mathbb{Q}$: the set of rational numbers,
			%		$\mathbb{Q} \eqdef \left\{  \frac{m}{n} \; \Big| \; m \in \mathbb{Z}, \; n \in \mathbb{N}, \; n \neq 0  \right\}$.
			%		The set $\mathbb{Q}$ consists of all numbers that can be expressed as \emph{fractions} of the form $\frac{m}{n}$,
			%		where $m$ is an integer, $n$ is a natural number, and $n \neq 0$.
	
			\item	$\mathbb{R}$: the set of real numbers.
	
			\item	$\mathbb{R}_+$: the set of nonnegative real numbers.
				The definition of the nonnegative is written as
				$\mathbb{R}_+ \eqdef  \{ \text{all } x \text{ in } \mathbb{R} \text{ such that } x \geq 0 \}$,
				or it can be expressed more compactly as $\mathbb{R}_+ \eqdef \{ x \in \mathbb{R} \; | \; x \geq 0 \}$.
	
		\end{itemize}
	
		\noindent
		We often define sets using the \emph{set-builder} notation $\{\, \cdot \; | \;  \cdot \, \}$.
		Inside the curly brackets we first describe the general kind of mathematical objects we are talking about,
		followed by the symbol ``$|$'' (read ``such that''),
		followed by the conditions that must be satisfied by all elements of the set.
		The definitions of nonnegative real numbers $\mathbb{R}_+$ above
		is an example of the set-builder notation.
	
		The \emph{number line} is a visual representation of the set of real numbers $\mathbb{R}$,
		as shown in Figure~\ref{fig:number_line_rationals_and_reals}.
		The real numbers correspond to all the points on the number line,
		from $-\infty$ to $\infty$.
	
		\begin{figure}[htb]
			\centering
			\includegraphics[width=0.4\textwidth]{figures/calculus/number_line_rationals_and_reals.pdf}
			\vspace{-2mm}
			\caption{The real numbers $\mathbb{R}$ cover the entire number line.}
			\label{fig:number_line_rationals_and_reals}
		\end{figure}
	
		\noindent
		The set of real numbers includes all the rational numbers like $-\frac{3}{2}$, $\frac{1}{2}$, and $\frac{9}{2}$,
		as well as irrational numbers like $\sqrt{2}$, $e$, and $\pi$.
		This means any number you are likely to run into when solving math problems
		can be visualized as a point on the number line.
	
		\subsubsection{Number intervals}
	
			The number line can also be used to represent subsets of the real numbers,
			which we call \emph{intervals}.
			Figure~\ref{fig:interval_2closed_to_4closed} shows an illustration of the interval $[2,4] = \{ x \in \mathbb{R} \;|\; 2 \leq x \leq 4 \}$,
			which is a subset of the real numbers.
	
			Here are some more examples of various intervals:
			\begin{itemize}
	
				\item	$[a,b]$: the interval from $a$ to $b$.
					This corresponds to the set of real numbers between $a$ and~$b$,
					including the endpoints $a$ and~$b$.
					The interval $[a,b]$ corresponds to the set $\{ x\in \mathbb{R}\ | \ a \leq x \leq b \}$.
	
				\item	$[a,\infty)$: the interval from $a$ until infinity,
					which corresponds to the set $\{ x\in \mathbb{R}\ | \ a \leq x \}$.
	
				\item	$(-\infty,b]$: the interval from negative infinity until $b$,
					which corresponds to the set $\{ x\in \mathbb{R}\ | \ x \leq b \}$.
	
			\end{itemize}
	
			\noindent
			The notation $[a,b]$ describes the \emph{closed} interval from $a$ to $b$,
			which means the endpoints $a$ and $b$ are included in the interval.
			The notation $(a,b)$ describes the \emph{open} interval from $a$ to $b$,
			defined as the set $\{ x\in \mathbb{R}\ | \ a < x < b \}$,
			which doesn't include the endpoints $a$ and $b$.
			In other words,
			intervals defined using square brackets ``$[$'' include the endpoints (defined using less-than-or-equal conditions)
			while intervals defined with round brackets ``$($'' do not include their endpoints (defined using strictly-less-than conditions).
			The distinction between open and closed intervals is important in general,
			but makes no difference in the context of probability theory,
			so you don't need to worry about the difference between $[a,b]$ and $(a,b)$ in this book.
	
			% \SIDENOTE{ @Ivan: If not relevant to the book, why are we explaining it? }
	
			\begin{figure}[htb]
				\centering	
				\includegraphics[width=0.4\textwidth]{figures/calculus/interval_2closed_to_4closed.pdf}
				\vspace{-3mm}
				\caption{The interval $[2,4] \protect\eqdef \{ x \in \mathbb{R} \; | \; 2 \leq x \leq 4 \}$. }
				\label{fig:interval_2closed_to_4closed}
			\end{figure}
	
	
	
		\noindent
		I hope these definitions and examples made you feel more comfortable with sets,
		and the weird-looking curly bracket notation that mathematicians use to define sets.
		It might look a little complicated at first,
		but you'll get used to it in the rest of the book.
		% Robyn said: I love these reassuring paragraphs, though it's a bit repetative in this chapter. Just something to be aware of.
	
	
	
	\subsection{Functions}
	
		A \emphindexdef{function} is a mathematical object that takes numbers as inputs and produces numbers as outputs.
		For every input $x$, the output value of $f$ for that input is denoted $f(x)$.
		For example,
		the function $f(x) = \frac{1}{2}x^2$
		takes any number $x$ as input,
		squares it and divides the result by two to produce the output.
		For example,
		$f(3) = \frac{1}{2} 3^2 = \frac{9}{2} = 4.5$.
		Here is the Python code that defines the function \tt{f}
		and evaluates it for the input $x=3$.

		\begin{codeblock}[plot-fx-minus3-to-plus3]
		>>> def f(x):
		        return x**2 / 2
		>>> f(3)
		4.5
		\end{codeblock}
		
		\noindent
		We'll 
	
		\subsubsection{Function graph}
	
			The \emph{graph} of a function is a line that passes through all input-output pairs of a function.
			Imagine we take out a piece of paper and draw a coordinate system with a horizontal axis and a vertical axis.
			The horizontal axis describes the different input values $x$,
			while the vertical axis describes the output values $f(x)$.
			Each input-output pair of the function $f$ corresponds to the point $(x,f(x))$ in the coordinate system.
			We obtain the graph of the function by varying the input coordinate $x$ and plotting all the points $(x, f(x))$,
			as illustrated in Figure~\ref{fig:graph_of_f_nonnegative}.
	
			The graph of the function $f$ allows us to see at a glance the behaviour of the function for all possible inputs,
			and forms an essential visualization tool.
			Indeed,
			many phenomena and calculations related to functions
			can be understood geometrically as operations based on the graph of the function.

	
		\subsubsection{Plotting function graphs}
		%  using NumPy and Seaborn
	
			We can use a combination of the \tt{numpy} and \tt{seaborn} modules
			to plot the graph of the function $f(x) = \frac{1}{2}x^2$,
			as shown in the code example below.
	
			\begin{codeblock}[plot-fx-minus3-to-plus3]
			>>> import numpy as np
			>>> import seaborn as sns
			>>> xs = np.linspace(-3, 3, 1000)
			>>> fxs = f(xs)
			>>> sns.lineplot(x=xs, y=fxs)
			See Figure ¡\ref{fig:graph_of_function_f_eq_halfx2}¡ for the output.
			\end{codeblock}
	
			\noindent
			We import the module \tt{numpy} under the alias \tt{np}.
			We use the function \tt{np.linspace} to create an array (a list of numbers) \tt{xs},
			which contains 1000 input values that range from $x=-3$ until $x=3$.
			Next we apply the function $f$ to the array of inputs \tt{xs} and store the result in the array \tt{fxs},
			which contains all the output values of the function for the input values \tt{xs}.
			At this point,
			the arrays \tt{xs} and \tt{fxs} contain $1000$ input-output pairs of the form $(x, f(x))$,
			which is exactly what we need to plot the graph of the function.
			On the last line,
			we call the function \tt{lineplot} to create the graph of $f(x)$,
			which produces the plot shown in Figure~\ref{fig:graph_of_function_f_eq_halfx2}.
	
			\begin{figure}[htb]
				\centering
				\includegraphics[width=0.49\textwidth]{figures/calculus/graph_of_function_f_eq_halfx2.pdf}%
				\vspace{-2mm}
				\caption{	Graph of the function $f(x)=\frac{1}{2}x^2$ from $x=-3$ until $x=+3$.
						The graph of the function $f$
						consists of all the coordinate pairs $(x,f(x))$
						over some interval of $x$ values.
				}
				\label{fig:graph_of_function_f_eq_halfx2}
			\end{figure}
	
	%			Note the steps we used to obtain the function graph in code~\ref{plot-fx-minus3-to-plus3}
	%			correspond exactly to the mathematical procedure for drawing the graph of $f(x)$:
	%			draw the line that passes through all $(x, f(x))$ input-output pairs.
	
	
	
	
	
	
	
		\subsubsection{Inverse functions}
	
			The inverse function $f^{-1} \colon B \to A$ performs the \emph{inverse operation} of the function $f \colon A \to B$.
			If you start from some $x$, apply $f$, and then apply $f^{-1}$,
			you'll arrive---full circle---back to the original input $x$:
			\[
				f^{-1}\!\big( \; f(x) \; \big) = x.
			\]
			In Figure~\ref{fig:functions-inverse} the function $f$ is represented as a forward arrow,
			and the inverse function $f^{-1}$ is represented as a backward arrow
			that puts the value $f(x)$ back to the $x$ it came from.
	
			\begin{figure}[htb]
				\centering
				\includegraphics[width=0.25\textwidth]{figures/calculus/functions-inverse.pdf}
				\caption{The inverse $f^{-1}$ undoes the operation of the function $f$.}
				\label{fig:functions-inverse}
			\end{figure}
	
			For example,
			when $x \geq 0$,
			the inverse of the function $f(x) = \frac{1}{2}x^2$
			is the function $f^{-1}(x) = \sqrt{2x}$.
			Earlier we computed $f(3) = 4.5$,
			and if we apply the inverse operation
			we get $f^{-1}(4.5) = \sqrt{2 \cdot 4.5} = \sqrt{9} = 3$.

			\begin{codeblock}[fun-inv-applied to input]
			>>> from math import sqrt
			>>> sqrt(2 * 4.5)
			3.0
			\end{codeblock}

			The logarithmic function $\log_e(x)$ is the inverse of the exponential function $e^x$.
			If if we compute the exponential function of a number,
			then apply the logarithmic function,
			we get back the original input.
	
			\begin{codeblock}[fun-inv-fun-combo-log]
			>>> from math import exp, log
			>>> log(exp(5))
			5.0
			\end{codeblock}
	

		\subsubsection{Function properties}

			It's important to keep track of the possible inputs and outputs for each function.
			We use the notation $f \colon A \to B$
			to denote a function from the input set $A$ to the output set $B$.
			The set of allowed inputs is called the \emph{domain} of the function,
			while the set of possible outputs is called the \emph{image} or \emph{range} of the function.
			For example,
			the domain of the function $f(x)= \frac{1}{2}x^2$
			is $\mathbb{R}$ (any real number)
			and it's image is $\mathbb{R}_+$ (nonnegative real numbers),
			so we write it as $f \colon \mathbb{R} \to \mathbb{R}_+$.
			
			Another important property is called \emph{continuity},
			which roughly corresponds to ability to draw a function without lifting the pen.
			We'll give a formal definition of continuity later in Section~\ref{limits:continuity}.


	\subsection{Function inventory}
	
		% (10 essential functions)
		%|x|
		%line
		%quadratic
		%square root
		%polynomial
		%exp
		%log
		%gaussian-like erf $e^{-x^2}$
		%sigmoid $\frac{1}{1-e^{-x}}$

			\begin{figure}[htb]
				\centering
				\includegraphics[width=0.49\textwidth]{figures/calculus/panel_function_graphs1.pdf}%
				\vspace{-2mm}
				\caption{	Graph of common functions.
				}
				\label{fig:panel_function_graphs1}
			\end{figure}
			


		\vspace{2in}



	\subsection{Functions with discrete inputs}

		We're sometimes interested in functions with discrete inputs
		
$a_k : \mathbb{N} \to \mathbb{R}$

Sequences 
We can write ...





	\subsection{Geometry}

		The area enclosed by a circle of radius $r$ is given by $A = \pi r^2$.
		A circle of radius $r=1$ has area $\pi$.

		The circumference of a circle of radius $r$ is
		\[
		 C = 2 \pi r.
		\]
		A circle of radius $r=1$ has circumference $2\pi$.
		%	For example,
		%	the circumference of a circle of radius $3$\,m is $C=2\pi(3) =18.85$\,m.
		%	This is how far you'll need to walk to complete a full turn around a circle of radius $r=3$\,m.
		
		The area of a rectangle of base $b$ and height $h$ is $A = bh$.

		The area of a triangle is equal to $\frac{1}{2}$ times the length of its base times its height:	
		\[
			A = \tfrac{1}{2} a h_a.
		\]
		Note that $h_a$ is the height of the triangle \emph{relative to} the side $a$.


		\begin{figure}[htb]
			\centering
			\includegraphics[width=0.5\textwidth]{figures/calculus/geometry_areas_circle_rect_triangle.pdf}
			% hexagon-octagon-dodecagon.png}
			\vspace*{-8mm}
			\caption{	Area formulas for circles, rectangles, and triangles.}
			\label{fig:geometry_areas_circle_rect_triangle}
		\end{figure}


	\subsection{Trigonometry}


		%FIGURE  right-angle triangle with hypotenuse r, adj = rcosθ, opp = rsinθ
		%define functions: sin, cos
		%FIGURE  plot of cos and sin functions
		%applications: vector

		\vspace{2in}
