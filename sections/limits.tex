%!TEX root = ../calculus_tutorial.tex

	\subsection{Limits}

		In high school math,
		we learn all kinds of math procedures for solving problems using a finite number of steps of math operations.
		Whether you're manipulating expressions using algebra,
		or applying the inverse function to simplify an equation,
		all problems in high school math can be solved by using less than five steps,
		or if your teacher really doesn't like you 10 steps.
		% INFINITY
		In calculus,
		we learn a broader class of problem-solving strategies
		that include procedures with an infinite number of steps.

		% LIMITS
		Limit expressions provide a precise mathematical language
		for talking about infinitely large numbers, infinitely small steps,
		and mathematical procedures with an infinite number of steps.
		Here are three representative examples of limit expressions:

		\begin{itemize}

			% ASYMPTOTICS
			\item	$\lim_{x \to \infty} f(x)$: limit expression that describes what happens to $f(x)$
				when the input to the function $x$ tends to infinity (gets larger and larger).
				In words,
				this limit expression is read as ``limit of $f(x)$ as $n$ goes to infinity.''

			% INFINITELY MANY
			\item	$\lim_{n \to \infty} \textrm{proc}(n)$: limit expression that describes the value of $\textrm{proc}(n)$ as the integer $n$ tends to infinity.
				The integer $n$ usually describes the number of steps in a given procedure,
				and $\textrm{proc}(n)$ describes the output of this procedure when $n$ steps are used.

			% INFINITELY SMALL
			\item	$\lim_{\delta \to 0} h(\delta)$: limit expression that describes what happens to
				$h(\delta)$ as the real number $\delta$ tends to zero.
				The number $\delta$ (the Greek letter delta) usually describes a small distance,
				and the limit as delta goes to zero ($\delta \to 0$) describes the behaviour of the expression $h(\delta)$
				for an infinitely short distance $\delta$.

		\end{itemize}

		%	Using limits allows us to obtain answers computed by mathematical procedures with an infinite number of steps!

		\noindent
		The SymPy function \tt{limit} allows us to compute limit expressions.
		For example,
		if we want to see if the exponential function $e^x$ or the polynomial function $x^{100}$ grows faster
		in the limit as $x$ goes to infinity,
		The code for computing the limit of the ratio between these two expressions is

		\begin{codeblock}[sympy-limit-exp-over-x-100]
		>>> from sympy import limit, exp, oo
		>>> limit(exp(x)/x**100, x, oo) 
		oo
		\end{codeblock}

		\noindent
		The answer $\infty$,
		written as \tt{oo} (two lowercase letters ``o''),
		tells us exponential functions grow faster than polynomial functions.
		%	This result has implications in computer science,
		%	where algorithms whose running time grows exponentially with the size of their input are considered bad

		% EXAMPLE 2: splitting up an interval into $n$ segments, then making $n$ go to infinity
		%	splitting with an infinite number of segments
		%	\begin{codeblock}[sympy-limit-sement-zero-length]
		%	>>> from sympy import limit, oo, summation
		%	>>> delta = (b - a)/n
		%	>>> limit(delta, n, oo)
		%	0
		%	\end{codeblock}
		%
		%	\begin{codeblock}[sympy-limit-sements-add-to-interval]
		%	>>> summation(delta, (i, 0, n-1))
		%	b - a				
		%	\end{codeblock}

		Limits are important in calculus because they are used in the formal definitions of the derivative and integral operations.
		The derivative is defined as a rise-over-run calculation for an infinitely short run.
		The integral is defined as a Riemann sum with infinitely narrow rectangles.
		We'll explain both of these in the next sections.



		\subsection{Applications of limits}
		
			Definition of the derivative
			
			Definition of the Riemann sum for computing integrals

