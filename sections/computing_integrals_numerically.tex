%!TEX root = ../calculus_tutorial.tex

\subsection{Computing integrals numerically}

	There are numerous ways to compute integrals using Python.
	Computing integrals ``numerically'' means we're splitting the region of integration into thousands or millions of subregions,
	computing the areas of these subregions,
	then adding up the areas of the subregions to obtain the total area.

	The Python function \tt{quad} in the module \tt{scipy.integrate} allows us to compute the integral of any function.
	The name \tt{quad} is short for ``quadrature'' which is the historical math term used for find-the-area procedures.
	Let's start by importing the \tt{quad} function.

	\begin{codeblock}[import-quad-from-scipy]
	>>> from scipy.integrate import quad
	\end{codeblock}

	\noindent
	Now let's define a Python function \tt{f} that corresponds to the constant function $f(x) = 3$.

	\begin{codeblock}[deffun-f-eq-3-and-call]
	>>> def f(x):
	        return 3
	>>> f(333)
	3
	\end{codeblock}
	
	\noindent
	No matter what input $x$ we choose,
	the output will always be the same $f(x)=3$.

	To compute the integral $\int_0^5 f(x) dx$ we call the function \tt{quad}
	with inputs \tt{f} as the first argument,
	and the limits of integration $a=0$ and $b=5$ as the second and third arguments.

	\begin{codeblock}[quad-f-0-5-tuple]
	>>> quad(f, 0, 5)
	(15.0, 1.1102230246251565e-13)
	\end{codeblock}

	\noindent
	The function \tt{quad} returns a tuple (a pair of numbers) as output: $(A,\epsilon)$.
	The first number in the tuple is the value of the area calculation.
	The second number $\epsilon$ tells us the accuracy of the procedure used to calculate the area.
	In the above calculation,
	the output tells us the integral $\int_0^5 f(x) dx$ is equal to $15.0$ up to a precision on the order of $10^{-13}$.

	Since we're usually only interested in the value of the area $A$ and not the precision $\epsilon$,
	we often select the first number in the output of \tt{quad}.
	This is why you'll often see the expression \tt{quad(...)[0]} in code examples.

	\begin{codeblock}[quad-f-0-5]
	>>> quad(f, 0, 5)[0]  # extract A
	15.0
	\end{codeblock}

	\noindent
	As a second example,
	let's calculate the area under the graph of the function $g(x)=x$ between $x=0$ and $x=5$.

	\begin{codeblock}[deffun-g-eq-x-and-quad-g-0-5]
	>>> def g(x):
	        return x
	>>> quad(g, 0, 5)[0]
	12.5
	\end{codeblock}

	\noindent
	The answer we obtained matches the results of the general formula we obtained above,
	$A_g(0,5) = \frac{1}{2}b^2$,
	when the upper limit of integration is $b=5$.

	We'll use the function \tt{quad} hundreds of times in the remainder of the book to compute various integrals
	as part of probability and statistics calculations,
	so make sure you understand what is going on in the above code examples.
	The main takeaway message is that the \tt{quad} function is your friend whenever you need to compute integrals.
	All the scary-looking math equations that contain the $\int$ symbol can be computed using one or two lines of Python code.



