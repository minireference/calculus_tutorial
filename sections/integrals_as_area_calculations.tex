%!TEX root = ../calculus_tutorial.tex

	\subsection{Integrals as area calculations}
	
		Figure~\ref{fig:integral_as_region_under_curve_Aab} 
		illustrates the calculation of the \emph{area} enclosed between
		the graph of $f(x)$ and the $x$-axis between the vertical lines at $x=a$ and $x=b$.
		In calculus,
		we refer to this area calculation as definite integral
		and denote it:
		\[
			A_f(a,b) = \int_{x=a}^{x=b} f(x) \, dx.
		\]
		The numbers $a$ and $b$ are called the \emph{limits of integration}.
		We often use the notation $\int_a^b f(x)\,dx$ as shorthand for $\int_{x=a}^{x=b} f(x)\,dx$.
		We read this expression as ``the integral of $f(x)$ between $a$ and $b$.''

		\begin{figure}[htb]	% LAYOUT
			\centering
			\includegraphics[width=0.2\textwidth]{figures/calculus/integral_as_region_under_curve_Aab.pdf}
			\vspace{-2mm}
			\caption{	The integral of the function $f(x)$ between $x=a$ and $x=b$
					corresponds to the shaded area $A_f(a,b)$.}
			\label{fig:integral_as_region_under_curve_Aab}
		\end{figure}
		
		\ifthenelse{\boolean{FORSTATSBOOK}}{
			The notion of an integral is foundational for understanding continuous random variables.
			Every time we compute the probability of some outcome of a continuous random variable,
			there is an integral calculation going on under the hood,
			so integrals is not a topic you can skip.	%, if you want to represent.
		}{}
	
		\noindent
		If this is the first time you're seeing the notation $\int_a^b f(x) \,dx$,
		you're probably freaking out,
		but bear with me for two more pages,
		and you'll see this intimidating-looking math notation
		is nothing to worry about!
		This's just a fancy way to denote a particular calculation
		that involves some function $f(x)$.
		% TODO: explain template
		You can think of $\int_a^b \tt{<f>} \,dx$
		as a ``template'' that you can fill in by replacing $\tt{<f>}$
		with any function $f(x)$ to denote the area-under-the-graph-of-$f(x)$ calculation,
		which is also denoted $A_f(a,b)$.

%		Remember that is a calculus tutorial,
%		so it's normal there will be calculations,
%		and it is a math tutorial,
%		so you should also expect there will be some 

%	it's understandable if you feel intimidated by the complicated math notation,
%	but you have to trust me on this one:
%	except for the notation,
%	there is 
%	In the next few pages,
%	I'll do my best to introduce you to the topic of integrals,
%	and you'll learn three different ways to do compute integrals.
	
		Let's look at some examples.

	
		\subsubsection{Example 1: integral of a constant function}
	
			Consider the constant function $f(x)=~3$.
			No matter what the input $x$ is,
			the output is always $3$.
			We can easily find the area under the graph of this function between any two points,
			since the region under the graph has a rectangular shape.
			See Figure~\ref{fig:simple_integral_fx_eq_3} for an illustration.
	
			The area under $f(x)$ between $x=0$ and $x=5$
			corresponds to the following integral calculation:
			\[
				A_f(0,5) = \int_0^5 f(x)\,dx = 3\cdot 5 = 15.
			\]
			The area under the graph of $f(x)$ is a rectangle with height $3$ and width $5$,
			so its area is $3 \cdot 5 = 15$.
	
			\begin{figure}[htb]
				\centering
				\includegraphics[width=0.3\textwidth]{figures/calculus/simple_integral_fx_eq_3.pdf}
				\caption{The area of a rectangle of height $3$ and width $5$ is equal to $15$.}
				\label{fig:simple_integral_fx_eq_3}
			\end{figure}
	
	
		\subsubsection{Example 2: integral of a linear function}
		
			Consider now the area under the graph of the line $g(x)=x$
			between $x=0$ and $x=5$,
			as shown in Figure~\ref{fig:simple_integral_gx_eq_x}.
			This area is described by the following integral calculation:			
			\[
				A_g(0,5) = \int_0^5 g(x) \, dx = \tfrac{1}{2} 5 \cdot 5 = \tfrac{1}{2}5^2 = \frac{25}{2} = 12.5.
			\]
			The region under the graph of $g(x)$ has a triangular shape,
			so we can compute its area using the formula for the area of a triangle,
			which is ``base times height divided by 2.''

			\begin{figure}[htb]
				\centering
				\includegraphics[width=0.3\textwidth]{figures/calculus/simple_integral_gx_eq_x.pdf}
				\caption{The area of a triangle with base $5$ and height $5$ is  $\frac{1}{2}5^2=\frac{25}{2}=12.5$.}
				\label{fig:simple_integral_gx_eq_x}
			\end{figure}



		\bigskip
		\noindent
		I hope these two examples are starting to convince you
		that the scary-looking integral sign is not that complicated after all.
		It's just a fancy way to describe ``area under the graph of a function'' calculations.


		\subsubsection{Example 3: integral of a polynomial}
		
Consider now the function $h(x) = x^3 - 5x^22 + x + 10$.
Suppose we want to know the area under the graph of $h(x)$
between $x=-1$ and $x=4$,
as illustrated in Figure~XX.
We need to calculate the following integral:
\[
	A_h(0,5) \eqdef \int_{-1}^4 h(x)\,dx \; = \; ?.
\]
Looking at the graph of $h(x)$,
we see it doesn't have a recognizable geometric shape with a known area formula.
How could we compute the area then?

One way to approximate the area under $h(x)$
is to split it up into bunch of thin vertical rectangular strips of some fixed width,
which we'll denote $\Delta x$.
The height of each rectangular strip will vary depending on $f(x)$.
Look ahead to Figure~\ref{fig:riemannsum-25-50} to see where we're going with this.
We can calculate the area of the $k$\textsuperscript{th}
strip using the ``base times height'' formula for the area of a rectangle $\Delta x f(x_k)$,
where $x_k$ is the left endpoint of the $k$\textsuperscript{th} strip.
For example,
we splitting up the area $A_h(0,5)$ into $n=25$ strips,
calculating the area of the each strip,
and summing them together produces the approximation $A_h(0,5) \approx ...$.
If we split the area $A_h(0,5)$ into $n=50$ strips,
we obtain the more accurate approximation $A_h(0,5) \approx ...$.
%
The integral $\int_{-1}^4 h(x)\,dx$ is defined as the limit
where the number of vertical strips $n$ goes to infinity.


		Relax,
		we won't be doing the $n=25$ and $n=50$ calculation by hand,
		let alone the exact version as $n$ goes to infinity!
		Instead,
		we'll write a computer program and that performs the integration procedure for us.
		Computer are really proficient at this stuff,
		and this is not a coincidence---computers % like there ENIAC
		were originally invented precisely for computer integrals.
		




