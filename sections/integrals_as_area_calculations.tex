%!TEX root = ../calculus_tutorial.tex


\subsection{Integrals as area calculations}

	An integral corresponds to the computation of the \emph{area} enclosed between
	the curve $f(x)$ and the $x$-axis over some interval of $x$ values:
	\[
		A_f(a,b) = \int_{x=a}^{x=b} f(x) \: dx.
	\]
	We refer to the numbers $a$ and $b$ as the \emph{limits of integration},
	and the notation $\int_a^b f(x)\:dx$ is shorthand for $\int_{x=a}^{x=b} f(x)\: dx$.

	\begin{figure}[htb]	% LAYOUT
		\centering
		\includegraphics[width=0.3\textwidth]{figures/calculus/integral_as_region_under_curve_Aab.pdf}
		\vspace{-2mm}
		\caption{	The integral of the function $f(x)$ between $x=a$ and $x=b$ corresponds to the shaded area.}
		\label{fig:integral_as_region_under_curve_Aab_repeat}
	\end{figure}

	\noindent
	
	\ifthenelse{\boolean{FORSTATSBOOK}}{
		The notion of an integral is foundational for understanding continuous random variables.
		Every time we compute the probability of some outcome of a continuous random variable,
		there is an integral calculation going on under the hood,
		so integrals is not a topic you can skip.	%, if you want to represent.
	}{}

	If this is the first time you're learning about integrals,
	it's understandable if you feel intimidated by the complicated math notation,
	but you have to trust me on this one:
	except for the notation,
	there is nothing to worry about!
	In the next few pages,
	I'll do my best to introduce you to the topic of integrals,
	and you'll learn three different ways to do compute integrals.

	Let's start with some examples.


	\subsubsection{Example 1: integral of a constant function}

		Consider the constant function $f(x)=~3$.
		No matter what the input $x$ is,
		the output is always $3$.
		We can easily find the area under the graph of the function $f(x)$ between any two points,
		since the region under the graph has a rectangular shape.
		See Figure~\ref{fig:simple_integral_fx_eq_3} for an illustration.

		The area under $f(x)$ between $x=0$ and $x=5$ corresponds to the following calculation:
		\[
			A_f(0,5) = \int_0^5 f(x)\;dx = 3\cdot 5 = 15.
		\]
		The area under the graph of $f(x)$ is a rectangle with height $3$ and width $5$,
		so its area is $3 \cdot 5 = 15$.

		\begin{figure}[htb]
			\centering
			\includegraphics[width=0.3\textwidth]{figures/calculus/simple_integral_fx_eq_3.pdf}
			\caption{The area of a rectangle of height $3$ and width $5$ is equal to $15$.}
			\label{fig:simple_integral_fx_eq_3}
		\end{figure}


	\subsubsection{Example 2: integral of a linear function}
	
		Consider now the area under the graph of the line $g(x)=x$ between $x=0$ and $x=5$,
		as shown in Figure~\ref{fig:simple_integral_gx_eq_x}.
		Since the region under the curve is triangular,
		we can compute its area using the formula for the area of a triangle,
		which is ``base times height divided by 2.''
	
		The integral of $g(x)$ from $x=0$ until $x=5$ is described by the following calculation:
		\[
			A_g(0,5) = \int_0^5 g(x) \; dx = \tfrac{1}{2} 5 \cdot 5 = \tfrac{1}{2}5^2 = \frac{25}{2} = 12.5.
		\]
	
		\begin{figure}[htb]
			\centering
			\includegraphics[width=0.3\textwidth]{figures/calculus/simple_integral_gx_eq_x.pdf}
			\caption{The area of a triangle with base $5$ and height $5$ is equal to $\frac{1}{2}5^2=\frac{25}{2}=12.5$.}
			\label{fig:simple_integral_gx_eq_x}
		\end{figure}



	\subsubsection{Example 3: integral a polynomial}
	
		problem statement

		The area under $h(x)$ between $x=-1$ and $x=4$ corresponds to the following integral:
		\[
			A_h(0,5) = \int_{-1}^4 h(x)\;dx = ?
		\]
		
		Split up into rectangles
		
		
		FORMULA for n=25


		FORMULA for n=50
		
	
		
	

	\bigskip
	\noindent
	I hope these examples helped you see that the scary-looking integral sign is not that complicated after all.
	It's just a fancy way to describe ``area under the graph of a function'' calculations.









