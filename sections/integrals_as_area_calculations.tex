%!TEX root = ../calculus_tutorial.tex


\subsection{Integrals as area calculations}

	An integral corresponds to the computation of the \emph{area} enclosed between
	the curve $f(x)$ and the $x$-axis over some interval of $x$ values:
	\[
		A_f(a,b) = \int_{x=a}^{x=b} f(x) \: dx.
	\]
	We refer to the numbers $a$ and $b$ as the \emph{limits of integration},
	and the notation $\int_a^b f(x)\:dx$ is shorthand for $\int_{x=a}^{x=b} f(x)\: dx$.

	\begin{figure}[htb]	% LAYOUT
		\centering
		\includegraphics[width=0.3\textwidth]{figures/calculus/integral_as_region_under_curve_Aab.pdf}
		\vspace{-2mm}
		\caption{	The integral of the function $f(x)$ between $x=a$ and $x=b$ corresponds to the shaded area.}
		\label{fig:integral_as_region_under_curve_Aab_repeat}
	\end{figure}

	\noindent
	The notion of an integral is foundational for understanding continuous random variables.
	Every time we compute the probability of some outcome of a continuous random variable,
	there is an integral calculation going on under the hood,
	so integrals is not a topic you can skip.	%, if you want to represent.

	If this is the first time you're learning about integrals,
	it's understandable if you feel intimidated by the complicated math notation,
	but you have to trust me on this one:
	except for the notation,
	there is nothing to worry about!
	In the next few pages,
	I'll do my best to introduce you to the topic of integrals,
	and you'll learn three different ways to do compute integrals.

	Let's start with some examples.


	\subsubsection{Example 1: integral of a constant function}

		Consider the constant function $f(x)=~3$.
		No matter what the input $x$ is,
		the output is always $3$.
		We can easily find the area under the graph of the function $f(x)$ between any two points,
		since the region under the graph has a rectangular shape.
		See Figure~\ref{fig:simple_integral_fx_eq_3} for an illustration.

		The area under $f(x)$ between $x=0$ and $x=5$ corresponds to the following calculation:
		\[
			A_f(0,5) = \int_0^5 f(x)\;dx = 3\cdot 5 = 15.
		\]
		The area under the graph of $f(x)$ is a rectangle with height $3$ and width $5$,
		so its area is $3 \cdot 5 = 15$.

		\begin{figure}[htb]
			\centering
			\includegraphics[width=0.3\textwidth]{figures/calculus/simple_integral_fx_eq_3.pdf}
			\caption{The area of a rectangle of height $3$ and width $5$ is equal to $15$.}
			\label{fig:simple_integral_fx_eq_3}
		\end{figure}


	\subsubsection{Example 2: integral of a linear function}
	
		Consider now the area under the graph of the line $g(x)=x$ between $x=0$ and $x=5$,
		as shown in Figure~\ref{fig:simple_integral_gx_eq_x}.
		Since the region under the curve is triangular,
		we can compute its area using the formula for the area of a triangle,
		which is ``base times height divided by 2.''
	
		The integral of $g(x)$ from $x=0$ until $x=5$ is described by the following calculation:
		\[
			A_g(0,5) = \int_0^5 g(x) \; dx = \tfrac{1}{2} 5 \cdot 5 = \tfrac{1}{2}5^2 = \frac{25}{2} = 12.5.
		\]
	
		\begin{figure}[htb]
			\centering
			\includegraphics[width=0.3\textwidth]{figures/calculus/simple_integral_gx_eq_x.pdf}
			\caption{The area of a triangle with base $5$ and height $5$ is equal to $\frac{1}{2}5^2=\frac{25}{2}=12.5$.}
			\label{fig:simple_integral_gx_eq_x}
		\end{figure}


	\bigskip
	\noindent
	I hope these examples helped you see that the scary-looking integral sign is not that complicated after all.
	It's just a fancy way to describe ``area under the graph of a function'' calculations.



\subsection{Properties of integrals}

	We'll now state some properties of integrals that follow from their interpretation as area calculations.

	\begin{itemize}
	
		\item \textbf{Additivity.}
			The integral from $a$ to $b$ plus the integral from $b$ to $c$ is equal to the integral from $a$ to $c$:
			\[
				\int_a^b f(x) \; dx + \int_b^c f(x) \; dx		=	\int_a^c f(x) \; dx.
			\]

		% TODO: add backward steps giving negative?

		\item \textbf{Constant multiple of a function.}
			The integral of the function $cf(x)$ is equal to $c$ times the integral of $f(x)$,
			for any constant $c$:
			\[
				\int cf(x)\; dx	=	c\int f(x)\; dx.
			\]

		\item \textbf{Sum of two functions.}
			The integral of a sum of two functions is equal to the sum of the integrals of the individual functions:
			\[
				\int [f(x) + g(x)]\; dx	=	\int f(x)\; dx +  \int g(x)\; dx.
			\]

		\item \textbf{Linearity.}
			The combination of the above two properties tells us that integration is a \emph{linear} operation,
			meaning it preserves linear combinations.
			The integral of the linear combination of two functions $\alpha f(x) + \beta g(x)$,
			is equal to the same linear combination of the integrals of the two functions:
			\[
				\int [\alpha f(x) + \beta g(x)]\; dx 
				= \alpha  \int f(x)\; dx  \; \; + \; \; \beta \int g(x)\; dx,
			\]
			where $\alpha$ and $\beta$ are two arbitrary constants.

		\item \textbf{Integral at a single point.}
			Integrals over intervals with zero length have zero value for any function $f(x)$:
			\[
				\int_a^a f(x)\; dx	=	0.
			\]
			Thinking geometrically,
			this integral defines a region with height $f(x)$ and width~$0$,
			so it corresponds to zero area.
			% see https://www.khanacademy.org/math/ap-calculus-ab/ab-integration-new/ab-6-6/v/same-integration-bounds

	\end{itemize}

	% exercise https://www.khanacademy.org/math/ap-calculus-ab/ab-integration-new/ab-6-6/a/definite-integrals-properties-review




LETS SEE SOME CODE



			Relax, we won't be doing the calculation by hand.
			We can write a computer program and make a computer performs the integration procedure for us.
			Here is a sample code that takes an arbitrary function \tt{func}
			and performs the $n$-rectangle area approximation calculation:

			\begin{codeblock}[]
			def integrate(func, a, b, n):
			    """
			    Compute the area under `func` between x=`a` and x=`b`
			    using an approximation with `n` rectangles.
			    """
			    dx = (b-a)/n               # width of each rectangle
			    total = 0.0                # accumulator variable for S_n
			    k = 1                      # counter variable
			    x = a + dx                 # start at first right endpoint
			    while k <= n:              # repeat n times:
			        total += func(x)*dx    #   s_k = height * width
			        x += dx                #   move one step to the right
			        k += 1                 #   increment counter
			    return total
			\end{codeblock}

			\noindent
			The logic of the of the sample code follows closely follows procedure we defined in the equations above.
			We variable \code{dx} holds the information about the width of the rectangles used in the approximation $\Delta x$,
			and we use the counter variable \code{k} to step through the interval $[a,b]$ using $n$ steps of width $\Delta x$.
			
			We can then use this code to compute the integral of any function.
			To do this,
			we must first define the function we want to integrate:

			\begin{codeblock} 
			def f(x):
			    return x**3 - 5*x**2 + x + 10
    			\end{codeblock}

			\noindent
			Then you can compute $S_{25}$ by calling \code{integrate(f, -1, 4, 25)},
			which returns $S_{25}=12.4$.
			Calling \code{integrate(f, -1, 4, 50)} you'll obtain $S_{50}=12.6625$.
			The approximations $S_n$ get better and better as the number of rectangles used in the approximations grows.
			For $n=100$,
			the sum of the rectangles' areas is $S_{100} =  12.7906$,
			for $n=1000$ the approximation gives us $S_{1000} = 12.9041562$,
			which is accurate to the first decimal.


%		We can approximate the total area under the function $f(x)$ between $x=a$ and $x=b$ by splitting the region into $n$ tiny vertical strips of width $\Delta x$,
%		then adding up the areas of the rectangular strips.
%		This is known as a \emph{Riemann sum} approximation for an area.
%		Figure~\ref{fig:riemannsum-25-50} shows the Riemann sum approximations for the area under the function
%		$f(x)=x^3-5x^2+x+10$ between $x=-1$ and $x=4$,
%		obtained by using $n=25$ and $n=50$ vertical rectangular strips.

		\begin{figure}[htb]
			\subfigure[$n=25$]{				
				\includegraphics[width=0.24\textwidth]{figures/calculus/riemannsum-25.png}
			}
			\subfigure[$n=50$]{				
				\includegraphics[width=0.24\textwidth]{figures/calculus/riemannsum-50.png}
			}
			\vspace{-2mm}
			\caption{An approximation to the area under the graph of the function $f(x)=x^3-5x^2+x+10$ 
					using $n=25$ and $n=50$ rectangles.}
			\label{fig:riemannsum-25-50}
		\end{figure}
		
%		As you can see,
%		the approximations get better and better as we increase the number of rectangles.
%		Let's come up with some math expression to describe the $n$-rectangle approximate area calculation.
%		The width of each rectangle is $\Delta x = \frac{4-(-1)}{n}=\frac{5}{n}$.
%		The left endpoint of the first rectangle is at $x=a$ and its right endpoint is at $x=x_1 \equiv a+\Delta x$.
%		Since we're using choosing the height of the rectangles according to their right endpoints,
%		the area of the first rectangle is
%		\[
%			s_1 = f(x_1)\Delta x = f(a + \Delta x)\Delta x,
%		\]
%		which corresponds to the height of the function $f$ at $a+\Delta x$ times the width $\Delta x$.
%		To find the $x$ coordinate of the right endpoint of the second rectangle,
%		we take a step of width $\Delta x$ to the right: $x_2 = x_1 + \Delta x = a + 2 \Delta x$.
%		The area of the second rectangle is $s_2 = f(a+2\Delta x)\Delta x$.
%		
%		We can iterate this one-step-to-the-right procedure to obtain all the right endpoints
%		\[
%			x_{k+1} = x_k + \Delta x, % , \quad \textrm{for all}  \ \ k < n.
%		\]
%		and compute the area of each rectangle using
%		\[
%			s_k = f(x_k)\Delta x =  (x_k^3-5x_k^2+x_k+10)\frac{5}{n}.
%		\]
%		The total area of the $n$-rectangle approximation is the sum of the rectangles' areas:
%		\begin{align*}
%		  S_{n}(a,b) \equiv \sum_{k=1}^n s_k
%			&= \sum_{k=1}^{n} f(a + k\Delta x)\Delta x 				\\
%		  	&= \sum_{k=1}^{n} (x_k^3-5x_k^2+x_k+10)\frac{5}{n}.
%		\end{align*}
%		Wow that looks like a mean math expression!
%		Indeed if you had to do all these calculations by hand,
%		it would take you forever.
%		Computing an approximation with $n=1000$ rectangles requires computing $1000$ rectangle areas
%		and the sum of $1000$ terms!


		In the limit as the number of rectangles $n$ approaches $\infty$, 							\index{infinity}
		the approximation to the area under the curve becomes \emph{arbitrarily close} to the true area.
		The notion of applying the a rectangular-strip approximation to the area of a function,
		where the number of rectangles grows to infinity is known as the \emph{Riemann sum}
		and is the basis for the definitions of the integral:

		The definite integral between $x=a$ and $x=b$ is \emph{defined} as the limit of a 			\index{integral}
		Riemann sum as $n$ goes to infinity:											\index{Riemann sum|textit}
		\[
			\int_{a}^{b}\!f(x)\:dx 
				\equiv \lim_{n\to\infty} \sum_{k=1}^{n} f(a + k\Delta x)\Delta x \equiv A(a,b).
		\]
	




