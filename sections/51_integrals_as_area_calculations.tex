%!TEX root = ../calculus_tutorial.tex

	\subsection{Act 1: Integrals as area calculations}
	
		Figure~\ref{fig:integral_as_region_under_curve_Aab} 
		shows a shaded region enclosed between the graph of $f(x)$ from above,
		the $x$-axis from below,
		and vertical lines at $x=a$ and $x=b$.
		The calculation of the \emph{area} of this region
		is described by the following integral calculation:
		\[
			A_f(a,b) = \int_{x=a}^{x=b} f(x) \, dx.
		\]
		The numbers $a$ and $b$ and called the \emph{limits of integration}.
		We refer to this type of integral as a \emph{definite integral}
		since both limits of integration are defined.
		% so its value is fully determined.


		\begin{figure}[htb]	% LAYOUT
			\centering
			\includegraphics[width=0.4\columnwidth]{figures/calculus/integral_as_region_under_curve_Aab.pdf}
			\vspace{-2mm}
			\caption{	The integral of the function $f(x)$ between $x=a$ and $x=b$
					corresponds to the area of the shaded region $A_f(a,b)=\int_a^b f(x)\,dx$.}
			\label{fig:integral_as_region_under_curve_Aab}
		\end{figure}
		
		\ifthenelse{\boolean{FORSTATSBOOK}}{
			The notion of an integral is foundational for understanding continuous random variables.
			Every time we compute the probability of some outcome of a continuous random variable,
			there is an integral calculation going on under the hood,
			so integrals is not a topic you can skip.	%, if you want to represent.
		}{}
	

		\noindent
		We often use the simplified notation $\int_a^b f(x)\,dx$ as shorthand for $\int_{x=a}^{x=b} f(x)\,dx$
		and read this expression as ``the integral of $f(x)$ between $a$ and $b$.''
		If this is the first time you're seeing the notation for integrals,
		it might seem very intimidating and complicated to you,
		but don't freak out and bear with me for two more pages.
		You'll see this fancy-looking math notation is nothing to worry about!
		It's just the calculus way to denote a particular calculation that involves the function $f(x)$.
		You can think of $\int_a^b \tt{<f>} \,dx$
		as a ``template'' that you fill in by replacing $\tt{<f>}$
		with the function $f(x)$ you're interested in
		whenever you need to compute the area $A_f(a,b)$.
		% to denote the area-under-the-graph-of-$f(x)$ calculation,

%		Remember that is a calculus tutorial,
%		so it's normal there will be calculations,
%		and it is a math tutorial,
%		so you should also expect there will be some 

%	it's understandable if you feel intimidated by the complicated math notation,
%	but you have to trust me on this one:
%	except for the notation,
%	there is 
%	In the next few pages,
%	I'll do my best to introduce you to the topic of integrals,
%	and you'll learn three different ways to do compute integrals.




	\subsection{Properties of integrals}

		We'll now state some properties of integrals that follow from their interpretation as area calculations.
	
		\begin{itemize}
		
			% \textbf{Additivity.}
			\item The sum of the integral from $a$ to $b$
				and the integral from $b$ to $c$
				is equal to the integral starting from $a$ going all the way to $c$:
				$\int_a^b f(x) \, dx + \int_b^c f(x) \, dx = \int_a^c f(x) \, dx$.
	
			% TODO: add backward steps giving negative?

			% \textbf{Constant multiple of a function.}
			\item The integral of $k$ times the function $f(x)$
				is equal to $k$ times the integral of $f(x)$:
				% for any constant $k$:
				$\int_a^b kf(x)\, dx = k\int_a^b f(x)\, dx$.
	
			% \textbf{Sum of two functions.}
			\item The integral of the sum of two functions
				is the sum of their integrals:
				$\int_a^b [f(x) + g(x)]\, dx = \int_a^b f(x)\, dx +  \int_a^b g(x)\, dx$.

			%	\item \textbf{Linearity.}
			%		% The combination of the above two properties tells us that
			%		Integration is a \emph{linear} operation: it preserves linear combinations.
			%		The integral of the linear combination of two functions $\alpha f(x) + \beta g(x)$,
			%		is equal to the same linear combination of the integrals of the two functions:
			%		\[
			%			\int [\alpha f(x) + \beta g(x)]\, dx 
			%			= \alpha  \int f(x)\, dx  \, \, + \, \, \beta \int g(x)\, dx,
			%		\]
			%		where $\alpha$ and $\beta$ are two arbitrary constants.
	
			% \textbf{Integral at a single point.}
			\item Integrals over intervals with zero length have zero value:
				$\int_a^a f(x)\, dx = 0$.
				Geometrically,
				this integral defines a region with height $f(x)$ and width~$0$,
				so it has zero area.
				% see https://www.khanacademy.org/math/ap-calculus-ab/ab-integration-new/ab-6-6/v/same-integration-bounds
	
		\end{itemize}

	% exercise https://www.khanacademy.org/math/ap-calculus-ab/ab-integration-new/ab-6-6/a/definite-integrals-properties-review





		Let's look at some examples.

	
		\subsubsection{Example 1. Integral of a constant function}
	
			Consider the constant function $f(x) = 3$.
			We can easily find the area under the graph of this function between any two points,
			since the region under the graph has a rectangular shape.	
			The area under $f(x)$ between $x=0$ and $x=5$
			is described by the following integral calculation:
			\[
				A_f(0,5) = \int_0^5 f(x)\,dx = 3\cdot 5 = 15.
			\]
			The area under the graph of $f(x)$ is a rectangle with height $3$ and width $5$,
			so its area is $3 \cdot 5 = 15$,
			as shown in Figure~\ref{fig:simple_integral_fx_eq_3}.
	
			\begin{figure}[htb]
				\centering
				\includegraphics[width=0.63212\columnwidth]{figures/calculus/simple_integral_fx_eq_3.pdf}
				\vspace{-3mm}
				\caption{The area of a rectangle of height $3$ and width $5$ equals $15$.}
				\label{fig:simple_integral_fx_eq_3}
			\end{figure}
	
	
		\subsubsection{Example 2. Integral of a linear function}
		
			Consider now the area under the graph of the line $g(x)=x$
			between $x=0$ and $x=5$,
			as shown in Figure~\ref{fig:simple_integral_gx_eq_x}.
			This area is described by the following integral calculation:			
			\[
				A_g(0,5) = \int_0^5 g(x) \, dx = \tfrac{1}{2} 5 \cdot 5 = \tfrac{1}{2}5^2 = \tfrac{25}{2} = 12.5.
			\]
			The region under the graph of $g(x)$ has a triangular shape,
			so we can compute its area using the formula for the area of a triangle:
			base times height divided by 2.

			\begin{figure}[htb]
				\centering
				\includegraphics[width=0.63212\columnwidth]{figures/calculus/simple_integral_gx_eq_x.pdf}
				\vspace{-3mm}
				\caption{The area of a triangle with base $5$ and height $5$ is  $\frac{1}{2}5^2=\frac{25}{2}=12.5$.}
				\label{fig:simple_integral_gx_eq_x}
			\end{figure}

		\medskip
		\noindent
		I hope these two examples are starting to convince you
		that the scary-looking integral notation is not that complicated after all.
		It's just a fancy way to describe the ``area under the graph of the function'' calculation.


		\subsubsection{Example 3. Integral of a polynomial}
		
			Consider now the function $h(x) = 4 - x^2$.
			We want to know the area under the graph of $h(x)$
			between $x=0$ and $x=2$,
			as illustrated in Figure~\ref{fig:simple_integral_hx_eq_x}.
			We need to calculate the following integral:
			\[
				A_h(0,2) = \int_{0}^2 h(x)\,dx \, = \; ???.
			\]
			The area under the graph of $h(x)$ is a curved region
			and not a simple recognizable geometric shape with a known area formula.
			How could we compute the area in this case?

			\begin{figure}[htb]
				\centering
				\includegraphics[width=0.7\columnwidth]{figures/calculus/simple_integral_hx_eq_x.pdf}
				\vspace{-3mm}
				\caption{The area under the graph of $h(x)$ between $x=0$ and $x=2$.}
				\label{fig:simple_integral_hx_eq_x}
			\end{figure}

			One way to approximate the area under $h(x)$
			is to split it up into bunch of vertical rectangular strips of some fixed width,
			which we'll denote $\Delta x$.
% TODO: mention this is called a \emph{Riemann sum} approximation
			The height of each rectangular strip will vary depending on $h(x)$.
			Look ahead to figures \ref{fig:riemann_sum_n10_n20} and \ref{fig:riemann_sum_n50_n100}
			to see where we're going with this.
			Splitting up the area $A_h(0,2)$ into $n=10$ strips,
			calculating the area of the individual strips,
			and summing them together produces the approximation $A_h(0,2) \approx 4.92$.
			If we split the area $A_h(0,2)$ into $n=20$ strips,
			we obtain the more accurate approximation $A_h(0,2) \approx 5.13$.
			The approximation with $n=1000$ rectangular strips gives us $A_h(0,2) \approx 5.329332$,
			and with $n=1\,000\,000$ rectangles is $A_h(0,2) \approx 5.333329333332$.
			The more finely we chop up the region into rectangles,
			the closer we get to the \emph{exact} value,
			which is $\int_{0}^2 h(x)\,dx=5\frac{1}{3} = 5.\overline{3} = 5.333333333333333\ldots$.
			%  computed as the limit where the number of vertical strips $n$ goes to infinity.


		In the next section,
		we'll learn more about split-into-rectangles are area calculations (a.k.a. \emph{integration}).
		Don't worry,
		I won't make you calculate sums with $n=10$ or $n=20$ terms by hand,
		let alone the sum with $n=1\,000\,000$ terms!
		Instead,
		we'll write a computer program that performs the integration procedure for us.
		Modern computer are really good at this stuff.
		Indeed early computers were often called ``numerical integrator''
		since they were built primarily to evaluate integrals.
		% required for military and science applications.
		%	and this is not a coincidence---computers % like there ENIAC
		%	were originally invented precisely for computer integrals.

%Early electronic computers were built primarily to perform numerical integration and other scientific calculations, so “computer” and “numerical integrator” were functionally synonymous for several decades.
%
%A clear example is the ENIAC (1945), whose original purpose was solving artillery-trajectory differential equations using large arrays of accumulators functioning as electronic integrators; see the ENIAC article on Wikipedia. The earlier Vannevar Bush differential analyzer (1931) was explicitly an analog integrator built from mechanical integrator wheels and gears; see Differential analyzer on Wikipedia. Even the term “computer” originally meant a human performing numerical calculations, often by hand or with desk calculators; see Computer (occupation) on Wikipedia.
%
%In short: early computing machines were built as automatic numerical integrators for differential equations, and only later became general-purpose symbolic, data-processing, or logic-oriented devices.
