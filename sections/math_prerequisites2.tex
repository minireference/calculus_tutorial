%!TEX root = ../calc_tutorial.tex


\section{Math prerequisites}


	Let's start by defining all the concepts from university-level math you need to know about.
	Don't worry if you're seeing some of these concepts for the first time,
	you'll see plenty of examples using these concepts,
	so you'll get to know them very well by the end of this section.

	\subsection{Sets and intervals}

		\begin{itemize}
	
	%		\item \emph{set}: a collection of math objects.
	%			Sets are denoted using curly brackets $\{ \ldots \}$.
	%% Patrick said: you've already covered sets
	%			A set can be defined as a finite list of elements like $\{ \texttt{heads}, \texttt{tails} \}$,
	%			by specifying a pattern $\{ 0, 1, 2, 3, \ldots \}$,
	%			or through some other math expression $\{ \textrm{<def'n>} \}$.
	
			\item $\mathbb{N}$: the set of natural numbers $\mathbb{N} \eqdef \{0,1,2,3,\ldots\}$.
	
			\item $\mathbb{N}_+$: the set of positive natural numbers $\mathbb{N}_+ \eqdef \{1,2,3\ldots\}$.
	
			\item	$\mathbb{R}$: the set of real numbers.
	
			\item	$\mathbb{R}_+$: the set of nonnegative real numbers.
				The definition of the nonnegative is written as
				$\mathbb{R}_+ \eqdef  \{ \text{all } x \text{ in } \mathbb{R} \text{ such that } x \geq 0 \}$,
				or it can be expressed more compactly as $\mathbb{R}_+ \eqdef \{ x \in \mathbb{R} \; | \; x \geq 0 \}$.

		\end{itemize}

		
		%	$S^c$: the \emph{complement} of the set $S$,
		%	is defined as all elements that are not in the set $S$.
		%	% TODO: explain assumption of "universe" and complement as all elements of the universe that are not in $S$.

% (inP)
	In probability theory,
	we use finite sets and countably infinite sets like the natural numbers to represent the sample spaces of discrete random variables.
	We also use intervals to describe outcomes in the sample space of continuous random variables.


	

%	\noindent
%	In the next few pages,
%	we'll go into some details about each of these math concepts.
%	Don't be intimidated by all the fancy-looking math notation---it's just a bunch of language mathematicians
%	invented in order to describe concepts precisely and concisely.
%	It looks weird to everyone who sees this specialized math notation for the first time (a.k.a. alien symbols),
%	but you'll quickly get used to it.


	\subsection{Functions}

		In probability theory,
		we use functions to describe the probability distributions of random variables.
		Discrete random variables are described by
		a probability mass function of the form $f\colon \mathcal{X} \to \mathbb{R}$,
		where the sample space $\mathcal{X}$
		is either a finite set or a countably infinite set like the natural numbers $\mathbb{N}$.
		Continuous random variables are described by probability density functions of the form 
		$f\colon \mathcal{X} \to \mathbb{R}$,
		where the sample space $\mathcal{X}$ is some subset of the real numbers $\mathbb{R}$.

		\noindent
		In probability theory,
		we often do calculations using the cumulative distribution function (CDF) $F_X \colon \calX \to [0,1]$,
		and also use the inverse of the cumulative distribution function $F_X^{-1}\colon [0,1] \to \calX$.
		Knowing about inverse functions (and the weird superscript $^{-1}$ notation used to describe them)
		is useful for your conceptual understanding of these concepts:
		instead of thinking about the inverse-CDF $F_X^{-1}$ as some new complicated concept you have to memorize,
		you can think of $F_X^{-1}$ as the ``undo operation'' for $F_X$.
		In other words,
		$F_X$ and $F_X^{-1}$ describe the same mapping,
		but used in opposite directions.

		%		$F_X(b) = q_b$
		%		vs. $F_X^{-1}(q) = b_q$  how far in the sample space do I have to go to encompass proportion $q$ of the total probability.


	\subsection{Function inventory}
	
		line
		
		quadratic
		
		square root
		
		exp
		
		log
		
		gaussian $e^{-x^2}$
		
		sigmoid $\frac{1}{1-e^{-x}}$
		
