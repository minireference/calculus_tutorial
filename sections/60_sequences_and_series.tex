%!TEX root = ../calculus_tutorial.tex

\section{Sequences and series}
\label{sec:sequences_and_series}

	A \emph{sequence} $a_k$ is a function that take whole numbers as inputs
	and produce real numbers as outputs: $a_k : \mathbb{N} \to \mathbb{R}$.
	The \emph{series} $\sum a_k$ describes the sum of the terms in the sequence $a_k$.
	Series are a powerful computational tool
	that allow us to describe procedures with infinite number of steps
	and use these procedures to approximate irrational numbers like $e$,
	and transcendental functions like the exponential function $f(x) = e^x$.	
	Sequences and series are the third pillar of the basic calculus knowledge I want you to have.

	\subsection{Sequences are functions with discrete inputs}
	\label{sequences_and_series:sequences}

		We use the notation $f: \mathbb{R} \to \mathbb{R}$
		to describe the functions that take real numbers $x \in \mathbb{R}$ as inputs
		and produce real numbers as outputs $f(x) \in \mathbb{R}$.
		When studying functions that take natural numbers $k \in \mathbb{N}$ as inputs,
		we use a different notation: $a_k: \mathbb{N} \to \mathbb{R}$,
		where $a_k$ describes the $k$\textsuperscript{th} term in the sequence.
		The sequence's input variable is usually denoted $k$.
		and corresponds to the \emph{index} within the sequence.
		Usually $k$ is a natural number $k \in \mathbb{N} \eqdef \{0,1,2,3,4,\ldots\}$
		but some sequences are only defined nor positive natural numbers $k \in \mathbb{N}_+ \eqdef \{1,2,3,4,\ldots\}$.
		Note the chance in notation:
		we use $a_k$ for sequences instead of the usual function notation $a(k)$.
		% TODO: MERGE
		%	Note the new notation for the input variable as a subscript.
		%	This is the standard notation for describing sequences,
		%	and is used instead of the standard function notation $a(k)$.

		
		We can define a sequence by specifying the formula for the $k$\textsuperscript{th} term in the sequence $a_k$.
		For example,
		the \emph{harmonic sequence} is defined by the formula $a_k = \frac{1}{n}$.
		Another way to define a sequence is by listing the first few values in the sequence:
		$[a_0, a_1, a_2, a_3, \ldots]$,
		which correspond to evaluating formula $a_k$ for $k=0$, $k=1$, $k=2$, $k=3$, etc.
		We'll now look at some common examples sequences
		specified both as functions of $k$ and show the the first few values in each sequence.


		\paragraph{The natural numbers}
		
			The simplest possible example of a sequence is the identity function,
			that returns the index input $k$ as output:
			\[
				n_k\eqdef k, \text{ for } k \in \mathbb{N}
				  \ \ \Leftrightarrow \ \ 
				  \left(0, 1, 2, 3, 4, 5, 6, 7, \ldots \, \right),
			\]
			This is the fundamental counting sequence that describes
			the process of taking ``unit step'' to the right on the number line,
			starting at the origin.
		

		\paragraph{Squares of natural numbers}
			
			The sequence-equivalent of the quadratic function $f(x)=x^2$
			is the sequence of squares of the integers:
			\[
				q_k\eqdef k^2, \text{ for } k \in \mathbb{N}
				  \ \ \Leftrightarrow \ \ 
				  \left(0, 1, 4, 9, 16, 25, 36, 49, \ldots \, \right),
			\]


		\paragraph{Harmonic sequence}
			
			Another ...
			\[
				h_k\eqdef \frac{1}{k}, \text{ for } k \in \mathbb{N}_+
				  \ \ \Leftrightarrow \ \ 
				  \left(1, \tfrac{1}{2}, \tfrac{1}{3}, \tfrac{1}{4}, \tfrac{1}{5}, \tfrac{1}{6}, \tfrac{1}{7}, \ldots \, \right),
			\]
			The \emph{harmonic sequence} appears in music.
			For most music instruments,
			when we want to play the note that corresponds to the frequency $f$,
			we also hear notes with frequencies that are integer multiple of the ``dominant'' frequency:
			$2f$, $3f$, $4f$, etc.,
			which are called the harmonics.
			The harmonic sequence describes the wavelengths of the harmonics frequencies.
			On a string instrument,
			the harmonic sequence tells you where to place your fingers if you want to play higher harmonics.
			%	The harmonic series is a really old concept,
			%	since it is a common 			
			%	that was well known by ancient civi Greeks knew about it
			%	played by the instrument consists of a ``main vibration'' with 
			%	Most music instrument that produce a given note and higher harmonics,
			%	which are frequencies 
			%	each note played by a musical instrument has a ``dominant'' frequency


		\paragraph{The alternating harmonic sequence}

			Consider now a harmonic sequence with alternating positive an negative outputs:
			\[
				a_k\eqdef \frac{(-1)^{k+1}}{k}, \text{ for } k \in \mathbb{N}_+
				  \ \ \Leftrightarrow \ \ 
				  \left(1, -\tfrac{1}{2}, \tfrac{1}{3}, -\tfrac{1}{4}, \tfrac{1}{5}, -\tfrac{1}{6}, \tfrac{1}{7}, \ldots \, \right),
			\]
			The factor $(-1)^{k+1}$ is positive for all odd inputs $k \in \{1,3,5,7,\ldots \}$
			since $(-1)^m=+1$ for any even number $m$.
			The factor $(-1)^{k+1}$ is negative
			for all even inputs $k \in \{2,4,6,8,\ldots \}$,
			hence the values in the sequence oscillate between positive and negative.


		\paragraph{Inverse factorial sequence}

			The factorial function is denoted $k!$
			and describes the product of the first $k$ positive natural numbers;
			$k! \;\; \eqdef \;\; k \cdot (k-1) \cdots 3 \cdot 2 \cdot 1$.
			We'll see factorials in several formulas in this section.
			In particular,
			the following sequence is of particular interest:
			\[
				f_k\eqdef \tfrac{1}{k!}, \text{ for } k \in \mathbb{N}_+
				  \ \ \Leftrightarrow \ \ 
				  \left(1, \tfrac{1}{2!}, \tfrac{1}{3!}, \tfrac{1}{4!}, \tfrac{1}{5!}, \tfrac{1}{6!}, \tfrac{1}{7!}, \ldots \, \right),
			\]
			The values in the \emph{inverse factorial sequence} quickly become very small,
			because the factorial function grows very quickly:
			$2!=2$,
			$3!=6$,
			$4!=24$,
			$5!=120$,
			$6!=720$,
			$7! = 5040$, \ldots,
			$10! = 3628800$, \ldots, 
			$13! \approx 6.2 \times 10^9$, \ldots, 
			$70! \approx 1.2 \times 10^{100}$,
			etc.
			% This sequence is a will be ingredient in power-series expansions such as the exponential function.


		\paragraph{Geometric sequence}

			The sequences-equivalent of the exponential function $f(x) = e^x$
			is the \emph{geometric sequence}
			where the $k$\textsuperscript{th} value in the sequence
			is the $k$\textsuperscript{th} powers of some number $r$:
			\[
				g_k\eqdef r^k, \text{ for } k \in \mathbb{N}
				  \ \ \Leftrightarrow \ \ 
				  \left(1, r, r^2, r^3, r^4, r^5, r^6, r^7, \ldots \, \right),
			\]
			Each term in the sequence equals $r$ times the previous term,
			which describes \emph{geometric process} that repeatedly grows/shrinks by the amount $r$.
			When $r < 1$ the values in sequence $g_k$ quickly go to zero,
			similar to how exponential function $e^{-x}$ goes to zero for large value of $x$.
			When $r > 1$ the sequence $g_k$ increases quickly,
			similar to how exponential function $e^{x}$ increases for large value of $x$.


		\paragraph{Powers of two}

			The special case of the geometric series with $r=2$ is of particular interest,
			so we'll alias-define it as the sequence $b_k$:
			\[
				b_k \eqdef 2^k, \text{ for } k \in \mathbb{N}
				  \ \ \Leftrightarrow \ \ 
				  \left(1, 2, 4, 8, 16, 32, 64, 128, \ldots \, \right),
			\]
			This sequence comes up all over the place in computer science
			because it is describes the number of different numbers we can store in $k$ bits of memory.






	\subsection{Summation notation}
	\label{sequences_and_series:summation_notation}
	
		We're often interested in computing the sum of all the values in this given a sequence $a_k$.
		To describe the sum of 3\textsuperscript{rd}, 4\textsuperscript{th}, and 5\textsuperscript{th}
		elements of the sequence $a_k$,
		we turn to summation notation:
	    	\[
		      a_3 + a_4 + a_5 
			     = \sum_{3 \leq k \leq 5}\!! a_k 
			      = \sum_{k=3}^{5} a_k \,.      
		\] 
		The capital Greek letter \emph{sigma} stands in for the word \emph{sum}, 
		and the range of index values included in this sum is denoted below and above the summation sign.
	
		The sum of the values in the sequence $a_k$ from $k=0$ until $k=n$
		is denoted as
		\[
			\sum_{k=0}^n a_k =  a_0 + a_1 + a_2 + \cdots + a_{n-1} + a_n.
		\]
		% This is called \emph{finite series} 

		\noindent
		Since this is a calculus tutorial,
		you should expect that an infinity of some kind will show up soon,
		and indeed we'll learn about \emph{infinite series},
		that describe the sum of \emph{all} the values in the sequence $a_k$:
		$\sum a_k  \eqdef 	\lim_{n \to \infty} \sum_{k=0}^n a_k$.
		% =		\sum_{k=0}^\infty a_k = a_0+ a_1 + a_2 + a_3 + a_4 + \cdots .
		%	In calculus,
		%	the notion of a \emph{series} describes the sum of \emph{all} the values in the sequence $a_k$:
		%	\[
		%	   \sum a_k 
		%	    	\eqdef 	\lim_{n \to \infty} \sum_{k=0}^n a_k
		%		=		\sum_{k=0}^\infty a_k = a_0+ a_1 + a_2 + a_3 + a_4 + \cdots .
		%	\]
		%	Note if the sequence $a_k$ continues indefinitely,
		%	computing the sum requires an infinite number of addition operations.
		%
		%
		But before we get to infinite sums,
		let's start at some finite sums to get a feel of this thing.	



	\subsection{Exact sums}

		We'll now show some useful formulas for calculating sum of the terms in certain sequences.
		For example,
		here is a formula for the sum of the first $n$ terms in the geometric sequence:
		\[
			% \sum_{k=0}^n g_k=
			G_n = 
			\sum_{k=0}^n r^k 
				= 1 + r + r^2 + \cdots + r^n 
				= \frac{1-r^{n+1}}{1-r}.
		\]
		We can use this formula to find the sum of the powers of $2$:
		\[
		   \sum_{k=0}^n 2^k = 1 + 2 + 4 + 8 + \cdots + 2^n = \frac{1-2^{n+1}}{1-2} = 2^{n+1} -1.
		\]


		The sum of the first $n$ positive integers and the sum of their squares are:
		\[
			\sum_{k=1}^n k = \frac{n(n+1)}{2}
			\qquad \text{and} \qquad
			\sum_{k=1}^n k^2=\frac{n(n+1)(2n+1)}{6}.
		\]
		% MAYBE add k^3 formula too?
		% 	See problem \textbf{P\ref{problem:infinite_sum_formulas_derivation}}
		%	for the derivations of these formulas.


TODO explain: 
The Binomial series
\[
	\sum_{k=0}^n {n \choose k} a^{n-k} b^k=(a+b)^n
\]
special case when one of the terms is 1:
\[
	\sum_{k=0}^n {n \choose k} x^k=(1+x)^n
\]

\vspace{2in}


% NOTEBOOK ONLY
%	We define a sequence by specifying an expression for its $n$\textsuperscript{th} term:
%	
%	
%	\begin{codeblock}[]
%	>>> k = sp.symbols("k")
%	>>> a_k = 1 / k
%	>>> b_k = 1 / sp.factorial(k)
%	\end{codeblock}
%	
%	\noindent
%	Substitute the desired value of $n$ to see the value of the $n$\textsuperscript{th} term:
%	
%	\begin{codeblock}[]
%	>>> a_k.subs({k:5})
%	1/5
%	\end{codeblock}
%	
%	\noindent
%	%We can use 
%	The Python list comprehension syntax \texttt{[item for item in list]}
%	can be used to print the sequence values for some range of indices:
%	
%	
%	
%	\begin{codeblock}[]
%	>>> [ a_k.subs({k:i}) for i in range(1,8) ]
%	[1, 1/2, 1/3, 1/4,  1/5,   1/6,   1/7]  
%	>>> [ b_k.subs({k:i}) for i in range(0,8) ]
%	[1,  1, 1/2, 1/6, 1/24, 1/120, 1/720, 1/5040]
%	\end{codeblock}
%	
%	\noindent
%	Observe that $a_k$ is not defined for $k=0$
%	since $\frac{1}{0}$ is a division-by-zero error.
%	In other words,
%	the domain of $a_k$ is the nonnegative natural numbers $a_k:\mathbb{N}_+ \to \mathbb{R}$.
%	Observe how quickly the `factorial` function $k!=1\cdot2\cdot3\cdots(k-1)\cdot k$ grows:
%	$7!= 5040$, $10!=3628800$, $20! > 10^{18}$.
%	
%	We're often interested in calculating the limits of sequences as $k\to \infty$.
%	What happens to the terms in the sequence when $k$ becomes large?
%	
%	\begin{codeblock}[]
%	>>> sp.limit(a_k, k, sp.oo)
%	0
%	>>> sp.limit(b_k, k, sp.oo)
%	0
%	\end{codeblock}
%	
%	\noindent
%	Both $a_k=\frac{1}{k}$ and $b_k = \frac{1}{k!}$ \emph{converge} to $0$ as $k \to \infty$.








	\subsection{Series}
	
		% Series are sums of sequences.
		Series are defined as the sums computed from the terms in the sequence $a_k$.
		The \emph{finite series} $\sum_{k=1}^n a_k$ computes the first $n$ terms of the sequence:
		\[
			\sum_{k=1}^n a_k
				= a_1 + a_2 + a_3 + a_4 + a_5  + \cdots  + a_{n-1} + a_n.
		\]
		The \emph{infinite series} $\sum a_k$ computes \emph{all} the terms in the sequence:
		\[
			\sum a_k
				\;\; \eqdef \;\;	\lim_{n\to \infty}  \sum_{k=1}^n a_k
				= a_1 + a_2 + a_3 + a_4 + a_5 + a_6 + \cdots.
		\]
		%	and we want to compute the sum of all the values in this sequence $\sum_{k=-}^\infty a_k$.
		The computing the infinite series $\sum a_k$
		of the sequence $a_k : \mathbb{N} \to \mathbb{R}$
		is analogous to the integral $\int_0^\infty f(x) \,dx$
		of a function $f : \mathbb{R} \to \mathbb{R}$.

		For example,
		when $|r|<1$,
		the limit as $n \to \infty$ of the geometric series is described
		converges to the following value:
		\[
			G_\infty 
				= \lim_{n \to \infty} G_n 
				= \sum_{k=0}^\infty r^k
				= 1 + r + r^2 +  r^3 + \cdots 
				= \frac{1}{1-r}.
		\]
		This expression describes an infinite sum,
		which is not possible to compute in practice,
		but we can see the truth of this equation using our mind's eye.
		The term $r^{n+1}$ in the formula for $G_n = \frac{1-r^{n+1}}{1-r}$
		goes to zero as $n \to \infty$,
		so the only part of the formula that remains is $\frac{1}{1-r}$.
		

		\paragraph{Example: sum of a geometric series}

			Let's compute infinite series of the geometric sequence with $r=\frac{1}{2}$.
			Applying the above formula,
			we obtain 
			\[
				%G_\infty = 
				\sum_{k=0}^\infty \big(\tfrac{1}{2}\big)^k
					= 1 + \tfrac{1}{2} + \tfrac{1}{4} + \tfrac{1}{8} + \tfrac{1}{16} + \tfrac{1}{32} + \cdots 
					=\frac{1}{1-\frac{1}{2}} = 2.
			\]
			You can visualize this infinite summation graphically
			in Figure~\ref{fig:geometric_progression_of_one_half}.
	
			\begin{figure}[htb]
				\centering
				\includegraphics[width=0.69\columnwidth]{figures/calculus/geometric_progression_of_one_half.png}
				\caption{	A graphical representation of the infinite sum of the geometric series with $r=\frac{1}{2}$.
						The area of each region corresponds to one of the terms in the series.
						The total area is equal to $\sum_{k=0}^\infty (\frac{1}{2})^k=\frac{1}{1-\frac{1}{2}}=2$.}
				\label{fig:geometric_progression_of_one_half}
			\end{figure}


		\paragraph{Example 2: sum of the inverse factorial sequence}

			\[
				F_k = \sum f_k	
					= e
					= 2.71828\ldots.
			\]


Using standard Python commands,  
we can obtain an approximation to $e^5$ that is accurate to six decimals by summing 10 terms in the series: 


\begin{codeblock}[]
>>> import math
>>> def f_k(n): 
        return 1 / math.factorial(n)
>>> sum( [f_k(k) for k in range(0, 10)] )
2.7182815255731922
\end{codeblock}
% >>> sp.E.evalf()
% 2.718281 82845905       # true value


			The infinite sum $\sum_{k=0}^\infty \frac{1}{k!}$
			converges to the number $e=2.71828\ldots$
			% NOTEBOOK ONLY
			%	\begin{codeblock}[]
			%	>>> f_k = 1 / sp.factorial(k)
			%	>>> sp.summation(f_k, (k,0,sp.oo))
			%	E
			%	\end{codeblock}

%To work with series in \texttt{SymPy},
%use the \texttt{summation} function whose syntax is analogous to the \texttt{integrate} function: 	
%The \texttt{summation} command is useful because it allows us to compute \emph{infinite} sums,
%but for most practical applications we don't need to take an infinite number of terms in a series to obtain a good approximation. 
%This is why series are so neat: they represent a great way to obtain approximations.



	\subsubsection{Convergent and divergent series}

		The geometric series $G_\infty = \sum g_k$
		and the inverse factorial series $F_k = \sum f_k	$ converges (or \emph{is convergent}).
		As we sum together more and more terms of the sequences $g_k$ and $f_k$,
		the total becomes closer and finite numbers.
		In this case,
		of $G_\infty$ we conserve to the faction $\frac{1}{1-r}$,
		while the infinite sum $\sum_{k=0}^\infty \frac{1}{k!}$
		converges to the number $e=2.71828\ldots$.


		In contrast,
		the harmonic series $\sum h_k$ \emph{diverges} to infinity (or \emph{is divergent})



	\subsection{Power series}
	\label{series:power_series}

		The term \emph{power series} describes a series whose terms contain different powers of the variable $x$.
		The $k$\textsuperscript{th} term in a power series
		is a function of both the sequence index $k$ and the input variable $x$:
		\[
			a
		\]




	\subsection{Taylor series}
	\label{series:taylor_series}

		The \emphindexdef{Taylor series} of a function $f(x)$ approximates the function by an infinitely long polynomial:
		\[
		    f(x)
			= \sum_{k=0}^\infty c_k x^k
			=  c_0 + c_1x + c_2x^2 + c_3x^3 + c_4x^4 + \cdots \,.
		\]
		Each term in the series is of the form $a_k=c_k x^k$, 
		where the coefficient $c_k$ depends on the properties of the function $f(x)$.
		Specifically,
		$c_k = \frac{f^{(k)}(0)}{k!}$,
		where $f^{(k)}(0)$ is the $k$\textsuperscript{th} derivative of $f(x)$ and $k!$ is the factorial function:
		\begin{align*}
		  f(x)
		 	& =f(0)+f'(0)x+\frac{f^{\prime\prime}(0)}{2!}x^2+\frac{f^{\prime\prime\prime}(0)}{3!}x^3 +\frac{f^{(4)}(0)}{4!}x^4 + \cdots \\
		 	& = \sum_{k=0}^\infty \frac{f^{(k)}(0)}{k!}x^k.
		\end{align*}
		Using this formula and your knowledge of derivatives,
		you can compute the Taylor series of any function $f(x)$.

		For example,
		let's find the Taylor series of the function $f(x)=e^x$ at $x=0$.
		The first derivative of $f(x)=e^x$ is $f'(x)=e^x$.
		The second derivative of $f(x)=e^x$ is $f''(x)=e^x$.
		In fact,
		all the derivatives of $f(x)$ will be $e^x$ because the $e^x$ is a special function that is equal to its derivative!
		The $k$\textsuperscript{th} coefficient in the power series of $f(x)=e^x$ at the point $x=0$ 
		is equal to the value of the $k$\textsuperscript{th} derivative of $f(x)$ evaluated at $x=0$.
		In the case of $f(x)=e^x$ we have $f^{(k)}(0)=e^0=1$,
		so the coefficient of the $k$\textsuperscript{th} term is $c_k = \tfrac{f^{(k)}(0)}{k!}  = \tfrac{1}{k!}$.
		The Taylor series of $f(x)=e^x$ is
		\[
		 e^x      	= \sum_{k=0}^\infty \frac{1}{k!}x^k
		 	 	= 1 + x + \frac{x^2}{2} + \frac{x^3}{3!} + \frac{x^4}{4!} + \frac{x^5}{5!} + \cdots 
		 \]
		Series are a powerful computational tool for approximating numbers and functions.
		As we compute more terms from the above series,
		our the polynomial approximation to the function $f(x)=e^x$ becomes more accurate.
		The exact value of the function at $x=1$ is $f(1) = e^1 = e$.
		The partial sum of the first six terms (as shown above) gives us an approximation of $e^1$ that is accurate to three decimals. 
		The partial sum of the first 12 terms gives us $e$ to an accuracy of nine decimals.
		% http://bit.ly/12DrCZY


Table~\ref{table:taylor_series} shows 
$f(x) = \sum_{k=0}^\infty \frac{f^{(k)}(0)}{k!}x^k$

		\begin{table}[htb]
		\centering
		\caption{Taylor series formulas for commonly used functions}
		\begin{shadebox}%
		\vspace{-2mm}
		\begin{align*}
		% f(x)			&=	\sum_{k=0}^\infty \frac{f^{(k)}(0)}{k!}x^k		\\
		\frac{1}{1-x}	&=	\sum_{k=0}^\infty x^k
						= 1 + x + x^2 + x^3 + x^4 + x^5 + x^6 + \cdots \\
		\frac{1}{1+x}	&=	\sum_{k=0}^\infty (-x)^k
						= 1 - x + x^2 - x^3 + x^4 - x^5 + x^6 + \cdots \\
		e^x			&=	\sum_{k=0}^\infty \frac{x^k}{k!}
						= 1 + x + \frac{x^2}{2} + \frac{x^3}{3!} + \frac{x^4}{4!} + \frac{x^5}{5!} + \cdots \\
		 \sin(x)  		& = \sum_{n=0}^\infty \frac{(-1)^n}{(2n+1)!} \: x^{2n+1}
		 	  			= x - \frac{x^3}{3!}  + \frac{x^5}{5!} - \frac{x^7}{7!} % + \frac{x^9}{9!} - \frac{x^{11}}{11!} 
						+ \cdots \\
		 \cos(x)  		& = \sum_{n=0}^\infty \frac{(-1)^n }{(2n)!}\:x^{2n} 
		 				=  1 - \frac{x^2}{2} + \frac{x^4}{4!} - \frac{x^6}{6!} % +  \frac{x^8}{8!}  - \frac{x^{10}}{10!}  
						+  \cdots \\
		\!\!\ln(x+1) 		& = \sum_{n=1}^\infty \frac{(-1)^{n+1}}{n}\:x^n
		 				=  x - \frac{x^2}2 + \frac{x^3}{3} - \frac{x^4}{4} + \frac{x^5}{5}   % - \frac{x^6}{6} 
							+ \cdots 		 		
		\end{align*}%
		\end{shadebox}
		\label{table:taylor_series}
		\end{table}




For example, the power series of the function $\exp(x)=e^x$ is 
\[
	\exp(x)	\eqdef  	1 + x + \frac{x^2}{2} + \frac{x^3}{3!} + \frac{x^4}{4!} + \frac{x^5}{5!} + \cdots         
			=       	\sum_{k=0}^\infty \frac{x^k}{k!}.
\]
This is, IMHO, one of the most important ideas in calculus:
you can compute the value of $\exp(5)$ by taking the infinite sum of the terms in the power series with $x=5$:



\begin{codeblock}[]
>>> exp_xk = x**k / sp.factorial(k)
>>> sp.summation( exp_xk.subs({x:5}), (k,0,sp.oo)).evalf()
148.413159102577
>>> sp.exp(5).evalf()  # the true value
148.413159102577
\end{codeblock}

\noindent
Note that \texttt{SymPy} is actually smart enough to recognize that the infinite series
you're computing corresponds to the closed-form expression $e^5$:



\begin{codeblock}[]
>>> sp.summation( exp_xk.subs({x:5}), (k,0,sp.oo))
exp(5)
\end{codeblock}

%	Taking as few as 35 terms in the series is sufficient to obtain an approximation to $e^5$
%	that is accurate to $16$ decimals:
%	%so series are not some abstract thing for mathematicians but a practical trick you can when you code:
%	
%	\begin{codeblock}[]
%	>>> import math                    # redo using only python 
%	>>> def exp_xnf(x,n): 
%	        return x**n/math.factorial(n)
%	>>> sum( [exp_xnf(5.0,i) for i in range(0,35)] )
%	148.413159102577
%	\end{codeblock}

\noindent
The coefficients in the power series of a function (also known as the \emph{Taylor series})
depend on the value of the higher derivatives of the function. 
The formula for the $k$\textsuperscript{th} term in the Taylor series of $f(x)$ expanded at $x=c$
is $a_k(x) = \frac{f^{(n)}(c)}{k!}(x-c)^k$,
where $f^{(k)}(c)$ is the value of the $k$\textsuperscript{th} derivative of $f(x)$ evaluated at $x=c$.

The \texttt{SymPy} function \texttt{series} is a convenient way to obtain the Taylor series of any function.
Calling \texttt{series(expr,var,at,nmax)} 
will show you the series expansion of \texttt{expr} 
near \texttt{var}=\texttt{at} 
up to power \texttt{nmax}:

\begin{codeblock}[]
>>> x = sp.symbols("x")
>>> sp.series( sp.sin(x), x, x0=0, n=8)
x - x**3/6 + x**5/120 - x**7/5040 + O(x**8)
>>> sp.series( sp.cos(x), x, x0=0, n=8)
1 - x**2/2 + x**4/24 - x**6/720 + O(x**8)
\end{codeblock}




\ifthenelse{\boolean{FORSTATSBOOK}}{
	TODO: any extra series formulas or concepts required to solve the exercises and problems in noBSstats.
}{}


\vfill