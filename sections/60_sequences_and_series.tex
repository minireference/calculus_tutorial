%!TEX root = ../calculus_tutorial.tex

\section{Sequences and series}
\label{sec:sequences_and_series}



	\subsection{Sequences}

Sequences are functions that take whole numbers as inputs.
Instead of continuous inputs $x\in \mathbb{R}$,
sequences take natural numbers $k\in\mathbb{N}$ as inputs.
We denote sequences as $a_k$ instead of the usual function notation $a(k)$.

	A sequence is a function of the form $a: \mathbb{N} \to \mathbb{R}$.	
	The sequence's input variable is usually denoted $k$ or $n$,
	and it corresponds to the \emph{index} or number in the sequence.
	We describe sequences either by specifying the formula $a_k$ for the $k$\textsuperscript{th} term in the sequence
	or by listing all the values of the sequence:
	\[
		a_k, k \in \mathbb{N}  \ \ \Leftrightarrow \ \  \left(a_0, a_1, a_2, a_3, a_4, \ldots \, \right).
	\]
	Note the new notation for the input variable as a subscript.
	This is the standard notation for describing sequences,
	and is used instead of the standard function notation $a(k)$.


Examples




	We're often interested in computing the sum of all the values in this given a sequence $a_k$.
	To describe the sum of 3\textsuperscript{rd}, 4\textsuperscript{th}, and 5\textsuperscript{th} elements of the sequence $a_k$,
	we turn to summation notation:
    	\[
	      a_3 + a_4 + a_5 
	      \equiv \sum_{3 \leq k \leq 5}\!\! a_k 
	      \equiv \sum_{k=3}^{5} a_k \,.      
	\] 
	The capital Greek letter \emph{sigma} stands in for the word \emph{sum}, 
	and the range of index values included in this sum is denoted below and above the summation sign.

	The partial sum of the sequence values $a_k$ ranging from $k=0$ until $k=n$ is denoted as
	\[
		S_n = \sum_{k=0}^n a_k =  a_0 + a_1 + a_2 + \cdots + a_{n-1} + a_n.
	\]

	In calculus,
	the notion of a \emph{series} describes the sum of \emph{all} the values in the sequence $a_k$:
	\[
	   \sum a_k 
	    	\equiv 	S_\infty	
		= 		\lim_{n \to \infty} S_n
		=		\sum_{k=0}^\infty a_k = a_0+ a_1 + a_2 + a_3 + a_4 + \cdots .
	\]
	Note if the sequence $a_k$ continues indefinitely,
	computing the sum requires an infinite number of addition operations.













	\subsection{Exact sums}

		Formulas exist for calculating the sum of certain series, even series with infinite number of terms.

		The formulas for the sum of the first $n$ positive integers is
		\[
		   \sum_{k=1}^n k = \frac{n(n+1)}{2}.
		\]
		The the sum of the squares of the first $n$ positive integers is
		\[
		   \sum_{k=1}^n k^2=\frac{n(n+1)(2n+1)}{6}.
		\]
		% MAYBE add k^3 formula too?
		% See problem \textbf{P\ref{problem:infinite_sum_formulas_derivation}} for the derivations of these formulas.

		There is another nice series for powers of $2$:
		\[
		   \sum_{k=0}^n 2^k = 1 + 2 + 4 + 8 + \cdots + 2^n = 2^{n+1} -1.
		\]


%TODO explain
The Binomial series
\[
	\sum_{k=0}^n {n \choose k} a^{n-k} b^k=(a+b)^n
\]
special case when one of the terms is 1:
\[
	\sum_{k=0}^n {n \choose k} x^k=(1+x)^n
\]










We define a sequence by specifying an expression for its $n$\textsuperscript{th} term:


\begin{codeblock}[]
>>> k = sp.symbols("k")
>>> a_k = 1 / k
>>> b_k = 1 / sp.factorial(k)
\end{codeblock}

\noindent
Substitute the desired value of $n$ to see the value of the $n$\textsuperscript{th} term:

\begin{codeblock}[]
>>> a_k.subs({k:5})
1/5
\end{codeblock}

\noindent
%We can use 
The Python list comprehension syntax \texttt{[item for item in list]}
can be used to print the sequence values for some range of indices:



\begin{codeblock}[]
>>> [ a_k.subs({k:i}) for i in range(1,8) ]
[1, 1/2, 1/3, 1/4,  1/5,   1/6,   1/7]  
>>> [ b_k.subs({k:i}) for i in range(0,8) ]
[1,  1, 1/2, 1/6, 1/24, 1/120, 1/720, 1/5040]
\end{codeblock}

\noindent
Observe that $a_k$ is not defined for $k=0$
since $\frac{1}{0}$ is a division-by-zero error.
In other words,
the domain of $a_k$ is the nonnegative natural numbers $a_k:\mathbb{N}_+ \to \mathbb{R}$.
Observe how quickly the `factorial` function $k!=1\cdot2\cdot3\cdots(k-1)\cdot k$ grows:
$7!= 5040$, $10!=3628800$, $20! > 10^{18}$.

We're often interested in calculating the limits of sequences as $k\to \infty$.
What happens to the terms in the sequence when $k$ becomes large?

\begin{codeblock}[]
>>> sp.limit(a_k, k, sp.oo)
0
>>> sp.limit(b_k, k, sp.oo)
0
\end{codeblock}

\noindent
Both $a_k=\frac{1}{k}$ and $b_k = \frac{1}{k!}$ \emph{converge} to $0$ as $k \to \infty$.






	\subsection{Series}
	
		Suppose we're given a sequence $a_k$
		and we want to compute the sum of all the values in this sequence $\sum_{k=-}^\infty a_k$.
		Series are sums of sequences.
		Summing the values of a sequence $a_k:\mathbb{N}\to \mathbb{R}$
		is analogous to taking the integral of a function $f:\mathbb{R}\to \mathbb{R}$.




The formula for the geometric sequence is $a_k = r^k$.
The sum of the first $n$ terms in the geometric sequence is
\[
  S_n = \sum_{k=0}^n r^k
   = 1 + r + r^2 + \cdots + r^n 
   =\frac{1-r^{n+1}}{1-r}.
\]
If $|r|<1$, taking the limit $n\to \infty$ in the above expression leads to
\[ 
  S_\infty 
   = \lim_{n \to \infty} S_n 
   = \sum_{k=0}^\infty r^k
   = 1 + r + r^2 +  r^3 + \cdots 
   =\frac{1}{1-r}.
\]

		
		\paragraph{Example}

			Consider the geometric series with $r=\frac{1}{2}$.
			Applying the above formula, we obtain 
			\[
			 S_\infty  
			 	=  \sum_{k=0}^\infty \left(\frac{1}{2}\right)^k
				= 1 + \frac{1}{2} + \frac{1}{4} + \frac{1}{8} + \frac{1}{16} + \frac{1}{32} + \cdots 
				=\frac{1}{1-\frac{1}{2}} = 2.
			\]
			You can visualize this infinite summation graphically in Figure~\ref{fig:geometric_progression_of_one_half}.
	
			\begin{figure}[htb]
			\centering
			\includegraphics[width=0.56\columnwidth]{figures/calculus/geometric_progression_of_one_half.png}
			\caption{	A graphical representation of the infinite sum of the geometric series with $r=\frac{1}{2}$.
					The area of each region corresponds to one of the terms in the series.
					The total area is equal to $\sum_{k=0}^\infty (\frac{1}{2})^k=\frac{1}{1-\frac{1}{2}}=2$.}
			\label{fig:geometric_progression_of_one_half}
			\end{figure}







To work with series in \texttt{SymPy},
use the \texttt{summation} function whose syntax is analogous to the \texttt{integrate} function: 						\index{summation}
																							\index{sequence!harmonic}

\begin{codeblock}[]
>>> a_k = 1 / k
>>> sp.summation(a_k, (k,1,sp.oo))
oo
>>> b_k = 1 / sp.factorial(k)
>>> sp.summation(b_k, (k,0,sp.oo))
E
\end{codeblock}


We say the series $\sum a_k$ \emph{diverges} to infinity (or \emph{is divergent})
while the series $\sum b_k$ converges (or \emph{is convergent}).
As we sum together more and more terms of the sequence $b_k$,
the total becomes closer and closer to some finite number.
In this case,
the infinite sum $\sum_{k=0}^\infty \frac{1}{k!}$ converges to the number $e=2.71828\ldots$.


The \texttt{summation} command is useful because it allows us to compute \emph{infinite} sums,
but for most practical applications we don't need to take an infinite number of terms in a series to obtain a good approximation. 
This is why series are so neat: they represent a great way to obtain approximations.

Using standard Python commands,  
we can obtain an approximation to $e^5$ that is accurate to six decimals by summing 10 terms in the series: 


\begin{codeblock}[]
>>> import math
>>> def b_kf(n): 
        return 1.0/math.factorial(n)
>>> sum( [b_kf(k) for k in range(0, 10)] )
2.7182815255731922
>>> sp.E.evalf()
2.718281 82845905       # true value
\end{codeblock}






	\subsection{Taylor series}
	\label{series:taylor_series}

		The \emphindexdef{Taylor series} of a function $f(x)$ approximates the function by an infinitely long polynomial:
		\[
		    f(x)
			= \sum_{k=0}^\infty c_k x^k
			=  c_0 + c_1x + c_2x^2 + c_3x^3 + c_4x^4 + \cdots \,.
		\]
		Each term in the series is of the form $a_k=c_k x^k$, 
		where the coefficient $c_k$ depends on the properties of the function $f(x)$.
		Specifically,
		$c_k = \frac{f^{(k)}(0)}{k!}$,
		where $f^{(k)}(0)$ is the $k$\textsuperscript{th} derivative of $f(x)$ and $k!$ is the factorial function:
		\begin{align*}
		  f(x)
		 	& =f(0)+f'(0)x+\frac{f^{\prime\prime}(0)}{2!}x^2+\frac{f^{\prime\prime\prime}(0)}{3!}x^3 +\frac{f^{(4)}(0)}{4!}x^4 + \cdots \\
		 	& = \sum_{k=0}^\infty \frac{f^{(k)}(0)}{k!}x^k.
		\end{align*}
		Using this formula and your knowledge of derivatives,
		you can compute the Taylor series of any function $f(x)$.

		For example,
		let's find the Taylor series of the function $f(x)=e^x$ at $x=0$.
		The first derivative of $f(x)=e^x$ is $f'(x)=e^x$.
		The second derivative of $f(x)=e^x$ is $f''(x)=e^x$.
		In fact,
		all the derivatives of $f(x)$ will be $e^x$ because the $e^x$ is a special function that is equal to its derivative!
		The $k$\textsuperscript{th} coefficient in the power series of $f(x)=e^x$ at the point $x=0$ 
		is equal to the value of the $k$\textsuperscript{th} derivative of $f(x)$ evaluated at $x=0$.
		In the case of $f(x)=e^x$ we have $f^{(k)}(0)=e^0=1$,
		so the coefficient of the $k$\textsuperscript{th} term is $c_k = \tfrac{f^{(k)}(0)}{k!}  = \tfrac{1}{k!}$.
		The Taylor series of $f(x)=e^x$ is
		\[
		 e^x      	= \sum_{k=0}^\infty \frac{1}{k!}x^k
		 	 	= 1 + x + \frac{x^2}{2} + \frac{x^3}{3!} + \frac{x^4}{4!} + \frac{x^5}{5!} + \cdots 
		 \]
		Series are a powerful computational tool for approximating numbers and functions.
		As we compute more terms from the above series,
		our the polynomial approximation to the function $f(x)=e^x$ becomes more accurate.
		The exact value of the function at $x=1$ is $f(1) = e^1 = e$.
		The partial sum of the first six terms (as shown above) gives us an approximation of $e^1$ that is accurate to three decimals. 
		The partial sum of the first 12 terms gives us $e$ to an accuracy of nine decimals.
		% http://bit.ly/12DrCZY





A \emph{power series} is a series whose terms contain different powers of the variable $x$.
The $k$\textsuperscript{th} term in a power series
is a function of both the sequence index $k$ and the input variable $x$.

For example, the power series of the function $\exp(x)=e^x$ is 
\[
 \exp(x) \equiv  1 + x + \frac{x^2}{2} + \frac{x^3}{3!} + \frac{x^4}{4!} + \frac{x^5}{5!} + \cdots         
  =       \sum_{k=0}^\infty \frac{x^k}{k!}.
\]
This is, IMHO, one of the most important ideas in calculus:
you can compute the value of $\exp(5)$ by taking the infinite sum of the terms in the power series with $x=5$:



\begin{codeblock}[]
>>> exp_xk = x**k / sp.factorial(k)
>>> sp.summation( exp_xk.subs({x:5}), (k,0,sp.oo)).evalf()
148.413159102577
>>> sp.exp(5).evalf()  # the true value
148.413159102577
\end{codeblock}

\noindent
Note that \texttt{SymPy} is actually smart enough to recognize that the infinite series
you're computing corresponds to the closed-form expression $e^5$:



\begin{codeblock}[]
>>> sp.summation( exp_xk.subs({x:5}), (k,0,sp.oo))
exp(5)
\end{codeblock}

%	Taking as few as 35 terms in the series is sufficient to obtain an approximation to $e^5$
%	that is accurate to $16$ decimals:
%	%so series are not some abstract thing for mathematicians but a practical trick you can when you code:
%	
%	\begin{codeblock}[]
%	>>> import math                    # redo using only python 
%	>>> def exp_xnf(x,n): 
%	        return x**n/math.factorial(n)
%	>>> sum( [exp_xnf(5.0,i) for i in range(0,35)] )
%	148.413159102577
%	\end{codeblock}

\noindent
The coefficients in the power series of a function (also known as the \emph{Taylor series})
depend on the value of the higher derivatives of the function. 
The formula for the $k$\textsuperscript{th} term in the Taylor series of $f(x)$ expanded at $x=c$
is $a_k(x) = \frac{f^{(n)}(c)}{k!}(x-c)^k$,
where $f^{(k)}(c)$ is the value of the $k$\textsuperscript{th} derivative of $f(x)$ evaluated at $x=c$.

The \texttt{SymPy} function \texttt{series} is a convenient way to obtain the Taylor series of any function.
Calling \texttt{series(expr,var,at,nmax)} 
will show you the series expansion of \texttt{expr} 
near \texttt{var}=\texttt{at} 
up to power \texttt{nmax}:

\begin{codeblock}[]
>>> x = sp.symbols("x")
>>> sp.series( sp.sin(x), x, x0=0, n=8)
x - x**3/6 + x**5/120 - x**7/5040 + O(x**8)
>>> sp.series( sp.cos(x), x, x0=0, n=8)
1 - x**2/2 + x**4/24 - x**6/720 + O(x**8)
\end{codeblock}




\ifthenelse{\boolean{FORSTATSBOOK}}{
	TODO: any extra series formulas or concepts required to solve the exercises and problems in noBSstats.
}{}

