%!TEX root = ../calculus_tutorial.tex


\section{Definitions}

	Let's start by defining all the concepts from university-level math you need to know about.
	Don't worry if you're seeing some of these concepts for the first time,
	you'll see plenty of examples using these concepts,
	so you'll get to know them very well by the end of this section.

	\begin{itemize}

		\item \emph{set}: a collection of math objects.
			Sets are denoted using curly brackets $\{ \ldots \}$.
% Patrick said: you've already covered sets
			A set can be defined as a finite list of elements like $\{ \texttt{heads}, \texttt{tails} \}$,
			by specifying a pattern $\{ 0, 1, 2, 3, \ldots \}$,
			or through some other math expression $\{ \textrm{<def'n>} \}$.

		\item $f(x)$: a function of the form $f: \mathbb{R} \to \mathbb{R}$,
			which means $f$ takes real numbers as inputs and produces real numbers as outputs.
			Functions are usually defined through an analytical formula like $f(x) = x^2$,
			which tells us how to compute the output $f(x)$ for a given input $x$.
			Functions can also be represented visually as a function graph
			\includegraphics[width=2em]{figures/calculus/graph_of_f_nonnegative.pdf},
			which is a curve that passes through all the coordinates pairs $(x,f(x))$ in the Cartesian plane.

		\item	$A_f(a,b)$: the value of the \emph{area} under the graph of the function $f(x)$ from $x=a$ until $x=b$.
			The area $A_f(a,b)$ corresponds to the following integral
			\[
				A_f(a,b) \eqdef \int_a^b f(x)\;dx.
			\] 
			The $\int$ sign stands for \emph{sum}.
			Indeed,
			the integral is the ``sum'' of all the values of $f(x)$ for inputs $x$ between $x=a$ and $x=b$.

		\item	$F_0(b) \eqdef A_f(0,b)$: the \emph{integral function} of $f(x)$.
			The integral function corresponds to the computation of the area under $f(x)$
			as a function of the upper limit of integration:
			\[
				F(b) \eqdef A_f(0,b) = \int_{0}^{b}\! f(x)\:dx.
			\]
			The choice of $x=0$ as the lower limit of integration is arbitrary.
			We could define any number of other integral functions $F_a(b)$ for different starting points $x=a$.

	\end{itemize}


	\noindent
	In the next few pages,
	we'll go into some details about each of these math concepts.
	Don't be intimidated by all the fancy-looking math notation---it's just a bunch of language mathematicians
	invented in order to describe concepts precisely and concisely.
	It looks weird to everyone who sees this specialized math notation for the first time (a.k.a. alien symbols),
	but you'll quickly get used to it.


