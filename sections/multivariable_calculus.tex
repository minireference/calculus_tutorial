%!TEX root = ../calculus_tutorial.tex

\section{Multivariable calculus}
\label{mathematical_preliminiaries:multivariable_calculus}

		Multivariable calculus is the extension of the ideas of differential and integral calculus
		to functions like $f(x,y)$ that depend on multiple input variables.
		You can plot a function $f: \mathbb{R} \times \mathbb{R}  \to \mathbb{R}$ as a \emph{surface},
		where the height $z$ of the surface above the point $(x,y)$ is function output $z=f(x,y)$.

		If you know single-variable calculus (derivatives and integrals),
		then then you won't have much new math to learn in multivariable calculus:
		it's essentially the same concepts but with more variables.

			
		\subsection{Partial derivatives}
			
			For a function of two variables $f(x,y)$,
			there is an ``$x$-derivative'' operator $\frac{\partial}{\partial x}$
			and a ``$y$-derivative'' operator $\frac{\partial}{\partial y}$.
			The operation $\frac{\partial}{\partial x}f(x,y)$ describes taking the derivative of $f(x,y)$ with respect to the input variable $x$,
			while keeping the input  variable $y$ constant.
			Taking the derivative of a multivariable function with respect to one of its input variables is called a \emph{partial derivative}
			and denoted with the symbol $\partial$.

			The partial derivative of $f(x,y)$ with respect to $x$ is
			\[
				\frac{\partial}{\partial x}f(x,y)
				\equiv
				\frac{\partial f}{\partial x}
				\equiv
				\lim_{ \delta \rightarrow 0}	\frac{f(x+\delta, y)-f(x,y)}{\delta}.
			\]
			Similarly the partial derivative of with respect to $y$ is
			\[
				\frac{\partial}{\partial y}f(x,y)
				\equiv
				\frac{\partial f}{\partial y}
				\equiv
				\lim_{ \delta \rightarrow 0}	\frac{f(x, y+\delta)-f(x,y)}{\delta}.
			\]
			Note that both $\frac{\partial f}{\partial x}$  and $\frac{\partial f}{\partial y}$ are function of $x$ and $y$.
			Indeed, we can ask the questions ``what is the slope in the $x$-direction''
			and ``what is the slope in the $y$-direction'' at any point $(x,y)$ on the surface of the function.
			That's precisely the information returned by the functions $\frac{\partial f}{\partial x}(x,y)$ and $\frac{\partial f}{\partial y}(x,y)$.

			TODO: example
			
			
		\subsection{Gradient}
		\label{mathematical_preliminiaries:gradient}


			The operator $\nabla$ is a combination of both the $x$ and $y$ derivatives:
			\[
				\nabla f(x,y)
				\equiv
				\left(
					\frac{\partial f}{\partial x},
					\frac{\partial f}{\partial y}
				\right).
			\]
			Note that $\nabla$ acts on a function $f(x,y)$ to produce a vector output.
			This vector is called the \emph{gradient} vector and it tells you the combined $x$- and $y$-slopes of the surface.
			More specifically,
			the gradient vector tells you the direction of the function's maximum increase---the
			``uphill'' direction at the surface of graph of $f(x,y)$ at the point $(x,y)$.
			
			
			\paragraph{Mountain map}
				Suppose the height of a mountain is described by the function $h(x,y)$.
				The coordinates $(x,y)$ tell us the horizontal position point in the $xy$-plane
				and the value of the function $h(x,y)$ represents the height of the mountain at those coordinate.

				We identify the $z$ coordinate with the hight of the mountain $z=h(x,y)$
				and graph the function $h(x,y)$ is as a surface in 3D as illustrated in Figure~\ref{fig:multivariable_caclulus_3d_surface_plot}.

				\begin{figure}[htb]
				\centering
				\includegraphics[width=0.3\textwidth]{figures/calculus/multivariable_caclulus_3d_surface_plot.pdf}
				\vspace{-2mm}
				\caption{The 3D surface plot of the the function $h(x,y)$.}
				\label{fig:multivariable_caclulus_3d_surface_plot}
				\end{figure}

				\noindent			
				Three dimensional surface plots are very good for visualizing multivariable functions,
				but they can be difficult to draw by hand.

				Another approach for representing function of the form $h(x,y)$ is to use a two-dimensional plot that shows the ``view from above'' of the surface $h(x,y)$.
				We can use colour to represent the height of the function through different sharing:
				darker-shading to represent large values of $h(x,y)$ and lighter-shading to represent small values of $h(x,y)$.			% TODO: update plot to show shading
				We can also trace \emph{level curves} in the plot,
				which is the same approach used for topographic maps:
				each level curve show the points that are at a certain height.
				
				\begin{figure}[htb]
				\centering
				\includegraphics[width=0.3\textwidth]{figures/calculus/multivariable_caclulus_topographic_map.png}
				\vspace{-4mm}
				\caption{The topographic map that shows the height of the function $h(x,y)$ using sharing to represent height.
						The level curves at each 10m intervals are also shown.}
				\label{fig:multivariable_caclulus_topographic_map}
				\end{figure}
				
				\noindent
				The curve labeled 30m line you see in Figure~\ref{fig:multivariable_caclulus_topographic_map} represents
				the solution to the equation $30= h(x,y)$,
				where $h(x,y)$ is the height of this hill for all coordinates $(x,y)$ on the map.
				
				Recall that the gradient vector $\nabla f(x,y)$ at any point $(x,y)$ tells you which way is ``uphill'' on the surface,
				and by extension,
				the negative of the gradient vector points ``downhill.''
				The gradient vector is always perpendicular to the \emph{level curve} at that point.

				The notion of an uphill or downhill direction for the surface $h(x,y)$ turns out to be very useful for optimization.
				If you want to find the local maximum of a function,
				you can start at some point and keep moving uphill (in the direction of $\nabla f(x,y)$ and you'll arrive at a local peak of the mountain.
				Similarly,
				to find the lowest point on the surface (minimum value of $h(x,y)$),
				you can start at some point and keep moving in the opposite direction to the gradient $-\nabla f(x,y)$.

				Figure~\ref{fig:multivariable_caclulus_topographic_map_waterflow_to_bottom} illustrates this process.
				Consider the path of a water stream whose source in some arbitrary point on the surface of the mountain.
				The water stream will naturally move downhill
				and descend the slope of the mountain until it reaches the minimum at the bottom of a valley.
				This intuitive notion of ``keep moving downhill until you get to a local minimum'' is
				the general idea behind the \emph{gradient descent} optimization algorithm which is very important for machine learning applications.
							
				\begin{figure}[htb]
				\centering
				\includegraphics[width=0.3\textwidth]{figures/calculus/multivariable_caclulus_topographic_map_waterflow_to_bottom.png}
				\vspace{-4mm}
				\caption{This graph shows the path taken by a hypothetical water as it flows to the bottom of the valley $h(x,y)$.}
				\label{fig:multivariable_caclulus_topographic_map_waterflow_to_bottom}
				\end{figure}

				We know we've reached the bottom of the valley,
				since the the gradient vector will be zero at the minimum of the function $h(x,y)$,
				since surface is locally flat there.



		\subsection{Multivariable integrals}
		
			The multivariable generalization of the integral $\int_{x \in I} f(x) \, dx$
			that computes the ``total'' amount of $f(x)$ on some interval $I=[a,b]$
			is the multivariable integral of the form:
			\[
				\int \! \int_{(x,y) \in R} f(x,y) \, dxdy,
			\]
			where $R$ is called the \emph{region of integration} and corresponds to some subset of the cartesian plane $\mathbb{R} \times \mathbb{R}$.
			The idea behind multivariable integrals is the same as for single variable integrals---to compute the total amount of some function for some range of input values.
			For single-variable integrals,
			we split the region into thing rectangular strips of width $dx$.
			For double integrals we split the two-dimensional region of integration into small squares of area $dxdy$,
			and compute the total volume of a many vertical columns whose base area is $dxdy$
			and whose height is given by the function $f(x,y)$.
			
			TODO: insert graphic of 3D integral split into vertical columns
			
			TODO: explain "sweep along x then sweep along y" idea + hint at change-of-variables techniques
			\vspace{1in}

