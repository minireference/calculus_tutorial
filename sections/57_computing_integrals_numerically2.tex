%!TEX root = ../calculus_tutorial.tex


	\subsection{Computing integrals numerically using SciPy}

The symbolic math techniques for calculating integrals
are ...

\vspace{1in}

		The Python function \tt{integrate} is a useful teaching tool,
		but it would be much too slow to use for practical scientific computing tasks that require computing integrals thousands of times.
		The Python function \tt{quad} defined in the module \tt{scipy.integrate} is a much more powerful
		tool for compute the integral of any function.
		The name \tt{quad} is short for ``quadrature,''
		which is the old math term for the find-the-area-under-the-graph-of-a-function calculation by splitting up into small pieces (squares)
		and counting the total number of squares.

		We'll now show briefly revisit the Examples 1N, 2N, and 3N using the function \tt{quad}.
		To compute the integral $\int_0^5 f(x) dx$ we call the function \tt{quad}
		with inputs \tt{f} as the first argument,
		and the limits of integration $a=0$ and $b=5$ as the second and third arguments.
	
		\begin{codeblock}[quad-f-0-5-tuple]
		>>> from scipy.integrate import quad
		>>> quad(f, 0, 5)
		(15.0, 1.1102230246251565e-13)
		\end{codeblock}
	
		\noindent
		The function \tt{quad} returns to numbers as output: $(A,\epsilon)$.
		The first number is the value of the area we're interested in.
		The second number $\epsilon$ tells us the accuracy of the procedure used to calculate the area.
		In the above calculation,
		the output tells us the integral $\int_0^5 f(x) dx$ is equal to $15.0$
		up to a precision on the order of $10^{-13}$.
	
		Since we're usually only interested in the value of the area $A$ and not the precision $\epsilon$,
		we often select the first number in the output of \tt{quad}.
		This is why you'll often see the expression \tt{quad(...)[0]} in code examples.
	
		\begin{codeblock}[quad-f-0-5]
		>>> quad(f, 0, 5)[0]  # extract the value A_f(0,5)
		15.0
		\end{codeblock}

		\noindent
		We can also \tt{quad} to calculate the integrals $\int_{0}^5 g(x)\,dx$ and $\int_{-2}^2 h(x)\,dx$:
		% area under the graph of the function $g(x)$ and $h(x)$

		\begin{codeblock}[]
		>>> quad(g, 0, 5)[0]
		12.5
		>>> quad(h, -2, 2)[0]
		10.666666666666666
		\end{codeblock}
	
		\noindent
		The answer we obtained match the results we obtained earlier.
		% the general formula we obtained above, $A_g(0,5) = \frac{1}{2}b^2$,
		% when the upper limit of integration is $b=5$.
		The main takeaway message is that the \tt{quad} function is your friend whenever you need to compute integrals.
		All the scary-looking math equations that contain the $\int$ symbol can be computed using one or two lines of Python code.

	


		\ifthenelse{\boolean{FORSTATSBOOK}}{	
			We'll use the function \tt{quad} hundreds of times in the remainder of the book to compute various integrals
			as part of probability and statistics calculations,
			so make sure you understand what is going on in the above code examples.
		}{}
