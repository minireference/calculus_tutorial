%!TEX root = ../calculus_tutorial.tex


\section{Introduction}

	% STUDY OF FUNCTIONS + OPERATIONS ON FUNCTIONS
	Calculus is the study of functions and their properties.
	The two calculus techniques we'll learn in this tutorial
	are \emph{derivatives},
	which tell how functions \emph{change} over time,
	and \emph{integrals},
	which compute the total \emph{accumulation} of functions over time.
	Derivatives and integrals might sound like fancy math jargon,
	but actually they are common-sense concepts
	that you're already familiar with,
	as you'll see in the following example.

	% cf. MATHPHYSbook/05_calc/02.overview.tex
	% cf. MATHPHYSbook/02_intro2phys/03.introduction_to_calculus.tex


	\subsection{Example: file download}
	\label{introduction:download_example}

		Suppose you're downloading a $720$[MB] file from the internet to your computer.
		At $t=0$ you click ``Save as'' to start the download.
		Consider the function $f(t)$
		that describes the amount of disk space occupied by the partially-downloaded file.
		%	At time $t$,
		%	your browser reports the download progress as a percentage
		%	that corresponds to the fraction $\frac{f(t)}{720 \text{[MB]}}$.

		\subsubsection{Download rate}
		\label{introduction:download_rate}

			The \emph{derivative function} $f^{\prime\!}(t)$,
			pronounced ``\:\!\!$f$ prime,''
			describes how the function $f(t)$ changes over time.
			In our example $f^{\prime\!}(t)$ is the download speed.
			If your download speed is $f^{\prime\!}(t)=2$[MB/s],
			then the file size $f(t)$ will increase by 2[MB] each second.
			If you maintain this download speed,
			the file size will grow at a constant rate:
			$f(0)=0$[MB], $f(1)=2$[MB], $f(2)=4$[MB], $\ldots$, $f(100)=200$[MB],
			and so on until $t=360$[s] when the download will be done.

			To estimate the time that remains before the download completes,
			we can divide the amount of data that remains to be downloaded
			by the current download speed:
			\[
				\text{time remaining at } t = \tfrac{ 720 - f(t) }{ f^{\prime\!}(t) } \quad [\textrm{s}].
			\]
			The bigger the derivative $f^{\prime\!}(t)$,
			the faster the download will finish.
			If your internet connection were 10 times faster,
			the download would finish 10 times more quickly.


		\subsubsection{Inverse problem}
		\label{introduction:inverse_problem}

			Let's now consider the download scenario
			from the point of view of the modem that connects your computer to the internet.
			Any data you download comes through the modem,
			so the modem knows the download rate $f^{\prime\!}(t)$[MB/s]
			at all times during the download.
			%
			The modem is separate from your computer,
			so it doesn't know the file size $f(t)$.
			Nevertheless,
			the modem can infer the file size at time $t$
			from the transmission rate $f^{\prime\!}(t)$.
			Think about it---if the modem sees data flowing through at the rate of $f^{\prime\!}(t)=2$[MB/s],
			then it knows that the data accumulated on your computer
			is growing at the rate of $2$[MB] each second.
			In calculus,
			we describe the total file size accumulated until time $t=\tau$ (the Greek letter \emph{tau})
			as the \emph{integral} of the download rate $f^{\prime\!}(x)$ between $t=0$ and $t=\tau$:
			\[
				f(\tau) \; = \; \int_{t=0}^{t=\tau}  f^{\prime\!}(t)\, dt.
			\]
			The symbol $\int$ is an elongated $S$ that stands for \emph{sum}.
			Indeed,
			the ``integral of $f^{\prime\!}(t)$ between $0$ and $\tau$''
			is in some sense the sum of $f^{\prime\!}(t)$
			during each time instant $dt$ between $t=0$ and $t=\tau$.
			To calculate the total accumulated file size,
			we split the time interval between $t=0$ and $t=\tau$
			into many short time intervals $dt$ of length $1$[s].
			During each second,
			the file size grows by $f^{\prime\!}(t)\,dt$,
			where $f^{\prime\!}(t)$ is the the download rate measured in [MB/s],
			and $dt$ is a time interval of duration $1$[s].
			%	Note the units work out:
			%	the data downloaded during one second is $f^{\prime\!}(t)dt$[MB].
			%	The file size on your computer at $t=\tau$
			%	is the sum of these 1-second contributions $f^{\prime\!}(t)\,dt$
			%	as $t$ varies from $t=0$ to $t=\tau$.
			% during each second from $t=0$ until $t=\tau$.
			% The file size at time $t=\tau$ is equal to the sum of the data downloaded 

			%	corresponds to the total amount of downloaded data stored on your computer.
			%	During this download period,
			%	the change in file size is described by the integral


	\noindent
	The situation described in the above example
	shows that calculus concepts are not some theoretical constructs reserved for math specialists,
	but something you encounter everyday.
	The derivative $q^{\prime\!}(t)$ describe the rate of change of the quantity $q(t)$.
	The integral $\int_a^b q(t)dt$ measures the total accumulation of the quantity $q(t)$
	during the time period from $t=a$ to $t=b$.

	%	Indeed,
	%	we carry out calculus-like operations in our head every day---we just
	%	don't necessarily use calculus terminology when we do so.












% REDO COMPLETELY
% - words only; no code
% - explain why we need + how we use in class, self-study, industry

	\subsection{Doing calculus: then and now}

		The key ideas of calculus were developed by Isaac Newton and Gottfried W. Leibniz
		in the 17\textsuperscript{th} century
		using symbolic calculations performed with pen and paper.
		% We're not in the 17\textsuperscript{th} century anymore,
		% and we're no longer limited to using pen-and-paper calculations.
		Today we have computers at our disposal
		that are extremely good at doing numerical calculations.
		This tutorial combines both symbolic and numerical methods
		of doing calculus to give a complete picture of all the tools available to you
		and their use cases.


		\subsubsection{Symbolic calculations}

			The pen-and-paper approach
			is still a good way to learn calculus,
			because manipulating math symbols ``by hand'' 
			develops your intuitive understanding of calculus procedures.
			% EXACT ANSWERS
			Writing math on paper
			allows you to use high-level abstractions
			and arrive at exact symbolic answers.
			% INTUITION + PROCEDURE
			%	I encourage you to keep a notebook
			%	and reproduce the calculations presented in this tutorial on your own.
			%	The goal is for you to get used to manipulating the new calculus concepts and notation.

		\subsubsection{Symbolic calculations using SymPy}

			The Python library SymPy allows you to do symbolic math calculations on a computer.
			% similar to the calculation you could do using pen and paper, but much faster.
			% EXTEND REACH TO MORE ADVANCED
			Using a computer algebra system like SymPy extends
			the reach of symbolic calculus operations you can do
			by automating some of the tedious steps
			and jumping straight to answers.
			% CHECK ANSWERS
			You can also use SymPy to check the answers you obtain
			from pen-and-paper calculations.
			% DON'T CHEAT
			% like using ChatGPT
			% win the round by doing less effort today,
			% lose the fight by learning nothing from this course
			% CREATE PROBLEMS
			% You can also use SymPy to create new exercises for yourself.


			% NOTEBOOK ONLY
			%	To use SymPy,
			%	we define a symbols \tt{x}
			%	that works like the math variable $x$.
			%	We can then write arbitrary math expressions that involve $\tt{x}$
			%	and ask SymPy to \tt{factor}, \tt{expand}, and \tt{simplify} them,
			%	which are the standard algebra operations we normally perform using pen and paper.
			%
			%	\begin{codeblock}[]
			%	>>> import sympy as sp
			%	>>> x = sp.symbols("x")
			%	>>> expr = 4 - x**2
			%	>>> expr
			%	4 - x**2
			%	>>> sp.factor(expr)
			%	-(x - 2)*(x + 2)
			%	>>> sp.expand( (x-4)*(x+2) )
			%	x**2 - 2*x - 8
			%	>>> sp.simplify(5*x - 3*x + 42)
			%	2*x + 42
			%	\end{codeblock}

			% NOTEBOOK ONLY
			%	You can also \tt{subs}itue particular values for $\tt{x}$
			%	into the expression and evaluate the expression
			%	to obtain an exact symbolic value
			%	or a numerical approximation as a floating point number.
			%
			%	\begin{codeblock}[]
			%	>>> expr.subs({x:1})
			%	3
			%	\end{codeblock}
			%
			%	\noindent
			%	We can also also ask SymPy to \emph{solve} the equation \t{expr = 0},
			%	which means to find the values of $x$ that satisfy the equation $4 - x^2 = 0$.
			%
			%	\begin{codeblock}[]
			%	>>> sp.solve(expr, x)
			%	[-2, 2]
			%	\end{codeblock}
			%	%	>>> sp.solve(x**2 + 2*x - 8, x)
			%	%	[-4, 2]
			%
			%	\noindent
			%	We'll use SymPy code examples to illustrate some concepts.


		\subsubsection{Numerical computing using NumPy and SciPy}

			Most practical applications of calculus don't require exact symbolic answers.
			Engineers don't care about the exact value of the square root of two $\sqrt{2}$,
			% (the length of the diagonal of a square with side length 1)
			and instead represent $\sqrt{2}$ approximately the floating point number
			$1.4142135623730951$ on a computer.
			This numerical approximation is good enough for most engineering and scientific use cases.
			What we give up in mathematical exactitude,
			we gain manyfold in computational power:
			modern computers can perform trillions ($10^{12}$) of floating point operations per second!
			The Python libraries NumPy and SciPy make it easy to do numerical calculus operations on a computer,
			as we'll demonstrate in code examples throughout this tutorial.
			% examples of practical calculations

			%  access all this computational power for doing calculus operations.
			%REPEATS
			% The calculus answer we obtain using numerical computing
			% are not exact like the answer we obtain using pen-and-paper or SymPy,
			% but engineers are willing to accept that trade-off
			% for the ability to perform any derivative, integral, or summations almost instantly.
			%	like the ones the ancients only dreamed of in less than a second.
			%/REPEATS

			% Defining the specific data format for representing numbers (float32, float64, etc.)
			% allows computer engineers to build high-performance hardware for doing math calculations.

			% Outside the classroom setting,
			% the importance of exact symbolic manipulations 
			% is an irrational number. % (requires infinitely many digits after the decimal to describe exactly).

			%	Calculus also has an engineering lineage.
			%	From the first mechanical calculators to modern GPUs,
			%	there has been many computational developments in industry.
			%	Engineers don't care about exact analytical results like knowing that
			%	$\sqrt{2}$ (the length of the diagonal of a square with side length 1)
			%	is an irrational number. % (requires infinitely many digits after the decimal to describe exactly).
			%	For most engineering contexts,
			%	if we can represent $\sqrt{2}$ approximately as  $1.4142135623730951$
			%	then they're good.
			%	In fact probably $1.4143521$ would be enough for most use cases.


			% NOTEBOOK ONLY
			% \subsubsection{Scientific computing using SciPy}
			%	The Python module SciPy is a toolbox of scientific computing helper functions.
			%	For example,
			%	computing the integral of the function $h(x) = 4 - x^2$
			%	between $x=0$ and $x=2$ requires only a few lines of code:
			%	\begin{codeblock}[]
			%	>>> def h(x):
			%	        return 4 - x**2
			%	>>> from scipy.integrate import quad
			%	>>> quad(h, a=0, b=2)[0]
			%	5.333333333333333
			%	\end{codeblock}
			% The answer is $5.\overline{3}$
			%  (the first number in the output)
			% and the precision of this answer is $\pm 1.8\times 10^{-13}$,
			% which tells us the first $12$ digits of the answer are exact.

			%	The above code example shows the complete level of WIN
			%	humankind has achieved for practical math calculations.
			%	Calculus ideas started with Archimedes,
			%	then levelled up by Newton and Leibniz,
			%	and formalized as analysis (pure math) and numerical analysis (applied math).
			%	In parallel,
			%	compute hardware has improved its raw performance exponentially for many years.
			%	This means today you can perform the integrals
			%	like the ones the ancients only dreamed of in less than a second.

			%	One of the only exponential trends that has the longest track record,
			%	is the number of floating point operations you have access to 




	\subsection{Applications of calculus}

		We use calculus concepts
		to describe various quantities in physics, chemistry, biology, engineering,
		% machine learning,
		business, economics and other domains where quantitative analysis is used.
		Many laws of nature are expressed in terms of derivatives and integrals,
		so it's essential that you learn the language of calculus if you want to study science.
		%
		In all this areas,
		the quantities of interest are described by functions
		and we use derivatives and integrals to do various useful calculations based on these functions.
		For example,
		derivatives are used for optimization,
		and integrals are used to compute probabilities in statistics and machine learning.
		%
		This is the power of math abstraction:
		the calculus techniques you learn for analyzing the rates of change of functions
		apply to solving real-world problems in many different domains.
		% ALT. apply these techniques in any domain.
		% ALT. think more clearly about these types of problems
		% ALT. the techniques of calculus allow us to do all kinds of useful calculations based on the function $f(t)$.







% CUT MATERIAL


%	Usually, differential calculus and integral calculus are taught as two separate subjects.
%	Perhaps teachers and university administrators are worried the undergraduates' little heads will explode
%	from sudden exposure to \emph{all} of calculus.
%	However, this separation actually makes calculus more difficult,
%	and prevents students from discovering the connections between differential and integral calculus.
%	We'll have no such split in this book, because I believe you can handle the material in one go.
%	Understanding calculus involves figuring out new mathematical concepts like infinity, limits, and summations,
%	but these ideas are not \emph{that} complicated.
%	By getting this far, you've proven you're more than ready to learn the theory,
%	techniques, and applications of derivatives, integrals, sequences, and series.



%	\subsection{Definitions}
%
%		\begin{itemize}
%
%			\item	$\mathbb{N} \eqdef \{0, 1, 2, 3, \ldots \}$: the set of natural numbers.
%
%			\item	$\mathbb{R}$: the set of real numbers.
%
%			\item	$f: \mathbb{R} \to \mathbb{R}$:
%				a \emph{function} that takes real numbers as inputs
%				and produces real numbers as outputs.
%
%			\item	$\lim_{\delta \to 0}$: a limit expression in which the number $\delta$ tends to zero
%
%			\item	$f^{\prime\!}(x)$: the derivative of the function $f(x)$.
%				The derivative $f^{\prime\!}(x)$ describes the rate of change of the function $f(x)$
%				and it is defined using the following formula:
%				\[
%					f^{\prime\!}(x) \eqdef \lim_{\delta \to 0} \frac{f(x+\delta)\; - \; f(x)}{\delta}\,.
%				\] 
%				The derivative is a function of the form $f': \mathbb{R} \to \mathbb{R}$.
%
%			\item	$A_f(a,b)$: the \emph{area} under the graph of the function $f(x)$
%				between $x=a$ and $x=b$.
%				The area $A_f(a,b)$ is computed as the following integral
%				\[
%					A_f(a,b) = \int_a^b f(x)\;dx.
%				\] 
%				
%			\item	$A_0(x)$: the \emph{integral function} of $f(x)$.
%		    		The integral function describes the area calculation
%				with variable upper limit of integration:
%			       	\[
%			        	   A_0(x) \eqdef A_f(0,x) = \int_0^{x}\! f(u)\:du.
%			      	\]
%				The choice of $x=0$ as the lower limit of integration is arbitrary.
%
%			\item $a_k:\mathbb{N} \to \mathbb{R}$: a \emph{sequence} of 
%
%			\item series
%
%		\end{itemize}




%so we should probably say a few about this as well.


%we learn a broader class of problem-solving strategies
%that include procedures with an infinite number of steps.


%	Infinity is a powerful math concept 
%	The concept of infinity is a powerful building block that    



%	% HIGH SCHOOL MATH
%	In high school math,
%	we learn all kinds of math procedures for solving problems using a finite number of steps of math operations.
%	Whether you're manipulating expressions using algebra,
%	or applying the inverse function to simplify an equation,
%	all problems in high school math can be solved by using less than five steps,
%	or if your teacher really doesn't like you 10 steps.
%	% INFINITY
%	In calculus,
%	we learn a broader class of problem-solving strategies
%	that include procedures with an infinite number of steps.
%	
%	% Main idea = calculations with infinite number of steps
%	previously = numbers + operations that produce other numbers as outputs
%	in calculus = functions + operators that produce other functions as outputs 


%	Derivatives and integrals are two main actors in calculus.
%	The other supporting actors are limits and series,
%	which we'll briefly introduce in the following example.

%	The operations of calculus are used to describe the limit behaviour of functions,

% We can also use calculus to describe the long-term tendency of quantities over time (limits).



%	In the previous section I made a lot of promises about the usefulness of calculus,
%	as motivation talk
%	to motivate you to read the rest of this tutorial
%	so that you'll be interested in learning all the complicated-looking topics 
%	concepts, symbols, etc.
%	Before getting to this,
