%!TEX root = ../calculus_tutorial.tex


\section{Introduction}

	% STUDY OF FUNCTIONS + OPERATIONS ON FUNCTIONS
	Calculus is the study of functions that change over time.
	We use calculus concepts to describe
	various quantities in physics, chemistry, biology, engineering, business,
	and other fields where quantitative math models are used.
	If we know that some quantity of interest is described by the function $f(t)$ at time $t$,
	then the techniques of calculus allow us to do all kinds of useful calculations
	based on the function $f(t)$.
	%
	The two main techniques of calculus
	involve calculating how functions \emph{change} over time (derivatives),
	and how to compute the total \emph{accumulation} of functions over time (integrals).
	Derivatives and integrals might sound like fancy math jargon,
	but actually they are common-sense concepts
	that you're already familiar with,
	as you'll see in the following example.

	% cf. \input{../../../MATHPHYSbook/05_calc/02.overview.tex}
	% cf. \input{../../../MATHPHYSbook/02_intro2phys/03.introduction_to_calculus.tex}


	\subsection{Example 1: file download}
	\label{introduction:download_example}
	
		Suppose you're downloading a $720$[MB] file from the internet to your computer.
		At $t=0$ you click ``save as'' in your browser and the download starts.
		Consider the function $f(t)$
		that describes the amount of disk space taken by the partially-downloaded file at time $t$.
		At time $t$,
		your browser reports the download progress as a percentage
		that corresponds to the fraction $\frac{f(t)}{720 \text{[MB]}}$.

		\subsubsection{Download rate}
		\label{introduction:download_rate}

			The \emph{derivative function} $f^\prime(t)$,
			pronounced ``\:\!\!$f$ prime,''
			describes how the function $f(t)$ changes over time.
			In our example $f^\prime(t)$ is the download speed.
			If your downloading speed is $f'(t)=2$[MB/s],
			then the file size $f(t)$ will increase by 2[MB] each second.
			If you maintain this download speed,
			the file size will grow at a constant rate:
			$f(0)=0$[MB], $f(1)=2$[MB], $f(2)=4$[MB], $\ldots$, $f(100)=200$[MB],
			and so on until $t=360$[s] when we expect the download to complete.

			Let's look at how your browser calculates the ``estimated time remaining''
			for the download at time $t$.
			To calculate the time until the download completes,
			we divide the amount of data that remains to be downloaded
			by the current download speed:
			\[
				\text{time remaining at } t = \frac{ 720 - f(t) }{ f^\prime(t) } \quad [\textrm{s}].
			\]
			The bigger the derivative $f^\prime(t)$,
			the faster the download will finish.
			If your internet connection were 10 times faster,
			the download would finish 10 times more quickly.


		\subsubsection{Inverse problem}
		\label{introduction:inverse_problem}

			Let's now consider the download scenario
			from the point of view of the modem that connects your computer to the internet.
			Any data you download comes through the modem,
			so the modem knows the download rate $f^\prime(t)$[MB/s]
			at all times during the download.

			The modem is separate from your computer,
			so it doesn't know the file size $f(t)$ as the download progresses.
			Nevertheless,
			the modem can infer the file size at time $t$
			from the transmission rate $f^\prime(t)$.
			Think about it---if the modem sees data flowing through at the rate of $f'(t)=2$[MB/s],
			then it knows that the data accumulated on your computer
			is growing at the rate of $2$[MB] each second.
			In calculus,
			we describe the total file size accumulated until time $t=\tau$ (the Greek letter \emph{tau})
			as the \emph{integral} of the download rate $f'(x)$ between $t=0$ and $t=\tau$:
			\[
				f(\tau) \; = \; \int_{t=0}^{t=\tau}  f'(t)\, dt.
			\]
			The symbol $\int$ is an elongated $S$ that stands for \emph{sum}.
			Indeed,
			the ``integral of $f^\prime(t)$ between $0$ and $\tau$''
			is in some sense the sum of $f^\prime(t)$
			during each time instant $dt$ between $t=0$ and $t=\tau$.
			To calculate the total accumulated file size,
			we split the time interval between $t=0$ and $t=\tau$
			into many short time intervals $dt$ of length $1$[s].
			During each second,
			the file size grows by $f'(t)\,dt$,
			where $f'(t)$ is the the download rate measured in [MB/s],
			and $dt$ is $1$ [s].
			Note the units check out,
			the data downloaded during one second is $f'(t)dt$[MB].
			The file size on your computer at $t=\tau$
			is the sum of these 1-second contributions $f'(t)\,dt$
			as $t$ varies from $t=0$ to $t=\tau$.
			% during each second from $t=0$ until $t=\tau$.
			% The file size at time $t=\tau$ is equal to the sum of the data downloaded 

			%	corresponds to the total amount of downloaded data stored on your computer.
			%	During this download period,
			%	the change in file size is described by the integral


	\noindent
	The situation described in the above example
	shows that calculus concepts are not some theoretical constructs reserved for math specialists,
	but something you encounter everyday.
	The derivative $q'(t)$ describe the rate of change of the quantity $q(t)$.
	The integral $\int_a^b q(t)\:dt$ measures the total accumulation of the quantity $q(t)$
	 during the time period from $t=a$ to $t=b$.
	
%	Derivatives and integrals are particularly important
%	for quantities that change over time.

%	Indeed,
%	we carry out calculus-like operations in our head every day---we just
%	don't necessarily use calculus terminology when we do so.


	\subsection{Infinity}
	
		The math symbol $\infty$ describes the concept of \emph{infinity}.
		Infinity is the key building block for everything we do in calculus,
		so it's important that you develop the right way to think about infinity.
		% so I want you to have a good way to think about it.

		\textbf{Infinity is not a number but a process}.
		Consider the set of natural numbers $\mathbb{N} \eqdef \{0, 1, 2, 3, 4, 5, 6, \ldots \}$.
		The natural numbers describe the process of counting starting at $0$.
		The natural number $n$
		is obtained by starting at $0$ and performing the $+1$ operation $n$ times.
		% Look ahead to the number line shown in Figure~XX.
		Geometrically,
		you can think of the $+1$ operation as taking one step to the right on the number line
		shown in Figure~\ref{fig:number_line_rationals_and_reals}
		(page~\pageref{fig:number_line_rationals_and_reals}).
		In this context,
		you can think of infinity $\infty$ as performing the $+1$ operation forever.
		Infinity is greater than any natural number $n$.
		Indeed,
		getting to $n$ takes a finite number of steps,
		but $\infty$ describes taking an infinite number of steps
		so $\infty$ must be to the right of $n$.

		Infinity is the main new concept in calculus.
		Everything else we'll talk about (numbers, variables, expressions, algebra, equations, functions, etc.)
		are standard topics from high school math,
		which I assume you're familiar with.
		Indeed,
		calculus can be described as the ``infinity upgrade''
		to the high school math calculations you're familiar with
		that gives you a language for describing and solving a new class of problems.

		Let's look at another example.
		% an example where infinity comes up.


	\subsection{Example 2: Euler's number}
	\label{introduction:eulers_number}
	
		Suppose you take out a loan with 100\% nominal interest rate.
		This is a very bad loan that nobody would agree to the real world,
		but we'll use it for this example to make the math come out simpler.
		An interest of $100\%$ calculated yearly means at the end of one year,
		you'll owe $(1+100\%) = (1+1)=2$ times the amount you borrowed initially.

		However,
		most banks don't calculate the interest owed only once per year,
		but more often.
		If the bank calculates the interest twice per year,
		during the first six months you'll have accrued $\frac{100\%}{2} = 50\%$ of interest,
		so you'll owe them $(1+50\%) = (1+\frac{1}{2}) = 1.5$ times the initial amount.
		Then during the second six months,
		the amount owed will grow by an additional $(1+50\%) = (1+\frac{1}{2}) = 1.5$,
		so at the end of the year,
		you'll owe $(1+\frac{1}{2})(1+\frac{1}{2}) = 2.25$.

		If the bank computes the interest three times per year,
		the amounted owed after one year will be $(1+\frac{1}{3})(1+\frac{1}{3})(1+\frac{1}{3}) = 2.370$.
		If they compute the interest four times per year (quarterly),
		then you'll owe $(1+\frac{1}{4})(1+\frac{1}{4})(1+\frac{1}{4})(1+\frac{1}{4})  = 2.441$.
		Note the amount owed after one year keeps changing,
		as the compounding is performed more frequently.
		In general,
		when the compounding is performed $n$ times per year,
		the amount owed at the end of the year will be
		\[
			\underbrace{
			\left(1 + \tfrac{1}{n} \right)
			\left(1 + \tfrac{1}{n} \right)
			\cdots
			\left(1 + \tfrac{1}{n} \right)
			}_{n \text{ times}}
			= 
			\left(1 + \tfrac{1}{n} \right)^{\!n}.
		\]

		\noindent
		With monthly compounding ($n=12$),
		the amount owed will be $(1 + \tfrac{1}{12})^{12} = 2.613$
		at the end of one year.
		With daily compounding,
		the amount would be $(1 + \tfrac{1}{365})^{365} = 2.715$.
		If computing the interest $n=1000$ times per year,
		the amount ill be $(1 + \tfrac{1}{1000})^{1000} = 2.717$.
		The amount owed keeps increasing,
		but it seems to ``stabilize'' around the value $2.71$.

		What happens if we perform the compounding even more frequently?
		Specifically,
	 	we want to know what happens if the compound interest interest is calculated infinitely often.
		The infinitely-often calculation corresponds
		to computing the \emph{limit} of expression  $(1 + \frac{1}{n})^n$,
		as $n$ goes to infinity,
		which is written as follows using math notation:
		\[
			\lim_{n\to \infty} \left( 1 + \frac{1}{n}\right)^{\!n}
				\;\; = \;\; e
				\;\;= \;\; 2.718281828\ldots.
		\]
		This limit expression \emph{converges} to the value $e = 2.71828\ldots$,
		which is known as \emph{Euler's number}.
		% The number $e$ describes the limit of the annual growth rate of a loan
		% with 
		If we borrow $\$1000$,
		we'll owe $\$1000e = \$2718.28$ at the end of one~year.
		
		The definition of the number $e$ as a limit
		is a fascinating new concept that goes beyond the ``regular'' math operations
		that we learn in high school math.
		We're not talking about any particular large number $n$
		when calculating the expression $(1 + \frac{1}{n})^n$,
		but the \emph{process} of plugging in large and larger $n$s.
		This is what the limit notation $\displaystyle \lim_{n\to \infty} (1 + \frac{1}{n})^n$ means:
		it describes the behaviour of the expression $(1 + \frac{1}{n})^n$
		as $n$ goes to infinity.
		We'll learn more about limits in Section~\ref{sec:limits}.

		Euler's number $e$ can also be obtained from another limit expression:
		\[
			e	=	1  + 1 + \frac{1}{2!} + \frac{1}{3!} + \frac{1}{4!}  + \cdots
				= 	\lim_{n\to \infty} \sum_{k=0}^n \frac{1}{k!}
				= 	2.718281828\ldots.
		\]
		This alternative expression
		tells us we can compute $e$ as the sum ($\sum$)
		with an infinite number of terms.
		Each term comes from a common ``pattern'' $\frac{1}{k!}$,
		where $k! = k\cdot (k-1) \cdot (k-2) \cdots 3\cdot 2 \cdot 1$
		is the factorial function.
		The notation $\sum_{k=0}^n$
		describes the summation starting at $k=0$ and going all the way to $k=n$.
		The limit $\lim_{n\to \infty}$ tells us the summation has infinitely many terms.
		This kind of infinite sum expression are called a \emph{series},
		and provides a powerful way to compute quantities by summing together a bunch of terms.
		We'll learn more about sequences and series in Section~\ref{sec:sequences_and_series}.


	\subsection{Applications}

		Many laws of nature are expressed in terms of derivatives and integrals.
		It is therefore essential that you learn the language of calculus if you want
		to understand physics, chemistry, biology, ecology, and other sciences.
		Calculus is also heavily used in engineering, business, economics, 
		any many other subjects based on quantitative analysis.
		We also use calculus in probability, statistics, and machine learning.

		In all this areas,
		there are quantities described by functions,
		and we use derivatives and integrals to do useful calculations.
		For example,
		optimization,
		solving differential equations,
		computing probabilities involving continuous random variables,
		etc.
		
		The goal of this tutorial is to show you the basics of derivatives and integrals,
		so that you you can think more clearly about these types of problems.
		This is the power of math:
		we learn techniques to analyze functions in general,
		which means our technique apply to any domain.



	\subsection{Doing calculus}

		In the previous section I made a lot of promises about the usefulness of calculus,
		as motivation talk
		to motivate you to read the rest of this tutorial
		so that you'll be interested in learning all the complicated-looking topics 
		concepts, symbols, etc.
		Befogging getting to this,
		it's worth describing more specifically what doing calculus looks like.

		\subsubsection{Symbolic calculations using pen and paper}

			The key ideas of calculus and were developed by Isaac Newton
			and Wilhelm Leibnitz in the 17th century
			using mostly pen and paper calculations.
			The pen-and-paper approach continues to be the best way to learn about
			limits, derivatives, and integrals even to this day.

			I encourage you to keep a notebook or use printer paper
			to reproduce the calculations presented in this tutorial on your own.
			The goal is for you to get used to manipulating functions, variables,
			and get used to the new calculus notation.


		\subsubsection{Symbolic calculations using SymPy}

			We're no longer in the 17th century,
			so we don't \emph{have} to use pen and paper for symbolic math calculations.
%	The Python module \tt{sympy} provides the functionality for doing symbolic math calculations
%	similar to the calculation you could do using pen and paper.
			Using a computer algebra system like SymPy
			allows us to do symbolic math calculations very similar to
			what we could do on paper.
			SymPy is a Python module for 
			When using SymPy,
			we can define a symbols \tt{x}
			that works like the math variable $x$.
			We can then write arbitrary math expressions that involve $\tt{x}$
			and ask SymPy to \tt{simplify}, \tt{factor}, \tt{expand}, etc.

			\begin{codeblock}[]
			>>> import sympy as sp
			>>> TODO
			\end{codeblock}


			You can also \tt{subs}itue particular values for $\tt{x}$
			into the expression and evaluate the expression
			to obtain an exact symbolic value
			or a numerical approximation as a floating point number.

			\begin{codeblock}[]
			>>> import sympy as sp
			>>> TODO
			\end{codeblock}

			You can also solve equaiton

			\begin{codeblock}[]
			>>> import sympy as sp
			>>> TODO
			\end{codeblock}


		\subsubsection{Numerical computing using NumPy}
		
			Calculus also has an engineering lineage.
			From the first mechanical calculators to modern CPUs and GPUs,
			there has been many computational developments in industry too.
			An engineer doesn't care about exact analytical results like knowing that
			$\sqrt{2}$ (the length of the diagonal of a square with side length 1)
			is an irrational number (requires infinitely many digits after the decimal to describe exactly).
			For most engineering concerts,
			if we can represent $\sqrt{2}$ approximately as  $1.4......15$
			then they're good.
			In fact probably $1.4143521$ would be enough for most use cases.

			What engineers give up in mathematical exactitude,
			they gain manyfold in the form of computational power.
			Defining the specific data format for representing numbers (float32, float64, etc.)
			allows computer engineers to build high-performance hardware for doing math calculations.


			\begin{codeblock}[]
			>>> import sympy as sp
			>>> TODO
			\end{codeblock}
			

			
		\subsubsection{Scientific computing using SciPy}

			The Python module SciPy is a toolbox of scientific computing helper functions
			that greatly simplify our life.
			For example,
			computing integral of the function \tt{f} between $x=-2$ and $x=2$ requires only two lines of code:

			\begin{codeblock}[]
			>>> from scipy.integrate import quad
			>>> quad(h, -2, 2)
			(10.666666666666666, 1.1842378929335001e-13)
			\end{codeblock}
			
			\noindent
			The answer is $10.\overline{6}$ (the first number in the output)
			and the precision of this answer is $\pm 1.8\times 10^{-13}$,
			which tells us the first $12$ digits of the answer are exact.
			
These two lines of code represent the complete level of WIN
humankind has achieved over practical math calculations.
Calculus ideas started with Archimedes,
then levelled up by Newton and Leibniz,
and formalized as analysis (pure math) and numerical analysis (applied math).
In parallel,
compute hardware has improved its raw performance exponentially for many years.
This means today you can perform the integrals like the ones the ancients only dreamed of in less than a second.




			

			
			%	One of the only exponential trends that has the longest track record,
			%	is the number of floating point operations you have access to 
			
			



% CUT MATERIAL


%	Usually, differential calculus and integral calculus are taught as two separate subjects.
%	Perhaps teachers and university administrators are worried the undergraduates' little heads will explode
%	from sudden exposure to \emph{all} of calculus.
%	However, this separation actually makes calculus more difficult,
%	and prevents students from discovering the connections between differential and integral calculus.
%	We'll have no such split in this book, because I believe you can handle the material in one go.
%	Understanding calculus involves figuring out new mathematical concepts like infinity, limits, and summations,
%	but these ideas are not \emph{that} complicated.
%	By getting this far, you've proven you're more than ready to learn the theory,
%	techniques, and applications of derivatives, integrals, sequences, and series.
		


%	\subsection{Definitions}
%
%		\begin{itemize}
%
%			\item	$\mathbb{N} \eqdef \{0, 1, 2, 3, \ldots \}$: the set of natural numbers.
%
%			\item	$\mathbb{R}$: the set of real numbers.
%
%			\item	$f: \mathbb{R} \to \mathbb{R}$:
%				a \emph{function} that takes real numbers as inputs
%				and produces real numbers as outputs.
%
%			\item	$\lim_{\delta \to 0}$: a limit expression in which the number $\delta$ tends to zero
%
%			\item	$f'(x)$: the derivative of the function $f(x)$.
%				The derivative $f'(x)$ describes the rate of change of the function $f(x)$
%				and it is defined using the following formula:
%				\[
%					f'(x) \eqdef \lim_{\delta \to 0} \frac{f(x+\delta)\; - \; f(x)}{\delta}\,.
%				\] 
%				The derivative is a function of the form $f': \mathbb{R} \to \mathbb{R}$.
%
%			\item	$A_f(a,b)$: the \emph{area} under the graph of the function $f(x)$
%				between $x=a$ and $x=b$.
%				The area $A_f(a,b)$ is computed as the following integral
%				\[
%					A_f(a,b) = \int_a^b f(x)\;dx.
%				\] 
%				
%			\item	$A_0(x)$: the \emph{integral function} of $f(x)$.
%		    		The integral function describes the area calculation
%				with variable upper limit of integration:
%			       	\[
%			        	   A_0(x) \eqdef A_f(0,x) = \int_0^{x}\! f(u)\:du.
%			      	\]
%				The choice of $x=0$ as the lower limit of integration is arbitrary.
%
%			\item $a_k:\mathbb{N} \to \mathbb{R}$: a \emph{sequence} of 
%
%			\item series
%
%		\end{itemize}




%so we should probably say a few about this as well.

		
%we learn a broader class of problem-solving strategies
%that include procedures with an infinite number of steps.


%	Infinity is a powerful math concept 
%	The concept of infinity is a powerful building block that    

		
	
%	% HIGH SCHOOL MATH
%	In high school math,
%	we learn all kinds of math procedures for solving problems using a finite number of steps of math operations.
%	Whether you're manipulating expressions using algebra,
%	or applying the inverse function to simplify an equation,
%	all problems in high school math can be solved by using less than five steps,
%	or if your teacher really doesn't like you 10 steps.
%	% INFINITY
%	In calculus,
%	we learn a broader class of problem-solving strategies
%	that include procedures with an infinite number of steps.
%	
%	% Main idea = calculations with infinite number of steps
%	previously = numbers + operations that produce other numbers as outputs
%	in calculus = functions + operators that produce other functions as outputs 


%	Derivatives and integrals are two main actors in calculus.
%	The other supporting actors are limits and series,
%	which we'll briefly introduce in the following example.

%	The operations of calculus are used to describe the limit behaviour of functions,

% We can also use calculus to describe the long-term tendency of quantities over time (limits).

