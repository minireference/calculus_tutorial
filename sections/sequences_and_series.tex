%!TEX root = ../calculus_tutorial.tex

\section{Sequences and series}
\label{sec:sequences_and_series}

	A sequence is a function of the form $a: \mathbb{N} \to \mathbb{R}$.	
	The sequence's input variable is usually denoted $k$ or $n$,
	and it corresponds to the \emph{index} or number in the sequence.
	We describe sequences either by specifying the formula $a_k$ for the $k$\textsuperscript{th} term in the sequence
	or by listing all the values of the sequence:
	\[
		a_k, k \in \mathbb{N}  \ \ \Leftrightarrow \ \  \left(a_0, a_1, a_2, a_3, a_4, \ldots \, \right).
	\]
	Note the new notation for the input variable as a subscript.
	This is the standard notation for describing sequences,
	and is used instead of the standard function notation $a(k)$.

	We're often interested in computing the sum of all the values in this given a sequence $a_k$.
	To describe the sum of 3\textsuperscript{rd}, 4\textsuperscript{th}, and 5\textsuperscript{th} elements of the sequence $a_k$,
	we turn to summation notation:
    	\[
	      a_3 + a_4 + a_5 
	      \equiv \sum_{3 \leq k \leq 5}\!\! a_k 
	      \equiv \sum_{k=3}^{5} a_k \,.      
	\] 
	The capital Greek letter \emph{sigma} stands in for the word \emph{sum}, 
	and the range of index values included in this sum is denoted below and above the summation sign.

	The partial sum of the sequence values $a_k$ ranging from $k=0$ until $k=n$ is denoted as
	\[
		S_n = \sum_{k=0}^n a_k =  a_0 + a_1 + a_2 + \cdots + a_{n-1} + a_n.
	\]

	In calculus,
	the notion of a \emph{series} describes the sum of \emph{all} the values in the sequence $a_k$:
	\[
	   \sum a_k 
	    	\equiv 	S_\infty	
		= 		\lim_{n \to \infty} S_n
		=		\sum_{k=0}^\infty a_k = a_0+ a_1 + a_2 + a_3 + a_4 + \cdots .
	\]
	Note if the sequence $a_k$ continues indefinitely,
	computing the sum requires an infinite number of addition operations.


		\subsubsection{Exact sums}
		
			Formulas exist for calculating the sum of certain series, even series with infinite number of terms.

			The formulas for the sum of the first $n$ positive integers is
			\[
			   \sum_{k=1}^n k = \frac{n(n+1)}{2}.
			\]
			The the sum of the squares of the first $n$ positive integers is
			\[
			   \sum_{k=1}^n k^2=\frac{n(n+1)(2n+1)}{6}.
			\]
			% MAYBE add k^3 formula too?
			% See problem \textbf{P\ref{problem:infinite_sum_formulas_derivation}} for the derivations of these formulas.

			There is another nice series for powers of $2$:
			\[
			   \sum_{k=0}^n 2^k = 1 + 2 + 4 + 8 + \cdots + 2^n = 2^{n+1} -1.
			\]
			
		
			\noindent
			The formula for the geometric sequence is $a_k = r^k$.
			The sum of the first $n$ terms in the geometric sequence is
			\[
			  S_n = \sum_{k=0}^n r^k
			   = 1 + r + r^2 + \cdots + r^n 
			   =\frac{1-r^{n+1}}{1-r}.
			\]
			If $|r|<1$, taking the limit $n\to \infty$ in the above expression leads to
			\[ 
			  S_\infty 
			   = \lim_{n \to \infty} S_n 
			   = \sum_{k=0}^\infty r^k
			   = 1 + r + r^2 +  r^3 + \cdots 
			   =\frac{1}{1-r}.
			\]

			
			\paragraph{Example}
				Consider the geometric series with $r=\frac{1}{2}$.
				Applying the above formula, we obtain 
				\[
				 S_\infty  
				 	=  \sum_{k=0}^\infty \left(\frac{1}{2}\right)^k
					= 1 + \frac{1}{2} + \frac{1}{4} + \frac{1}{8} + \frac{1}{16} + \frac{1}{32} + \cdots 
					=\frac{1}{1-\frac{1}{2}} = 2.
				\]
				You can visualize this infinite summation graphically in Figure~\ref{fig:geometric_progression_of_one_half}.
		
				\begin{figure}[htb]
				\centering
				\includegraphics[width=0.34\textwidth]{figures/calculus/geometric_progression_of_one_half.png}
				\caption{	A graphical representation of the infinite sum of the geometric series with $r=\frac{1}{2}$.
						The area of each region corresponds to one of the terms in the series.
						The total area is equal to $\sum_{k=0}^\infty (\frac{1}{2})^k=\frac{1}{1-\frac{1}{2}}=2$.}
				\label{fig:geometric_progression_of_one_half}
				\end{figure}


%TODO explain
The Binomial series
\[
	\sum_{k=0}^n {n \choose k} a^{n-k} b^k=(a+b)^n
\]
special case when one of the terms is 1:
\[
	\sum_{k=0}^n {n \choose k} x^k=(1+x)^n
\]


	\subsection{Taylor series}
	\label{series:taylor_series}

		The \emphindexdef{Taylor series} of a function $f(x)$ approximates the function by an infinitely long polynomial:
		\[
		    f(x)
			= \sum_{k=0}^\infty c_k x^k
			=  c_0 + c_1x + c_2x^2 + c_3x^3 + c_4x^4 + \cdots \,.
		\]
		Each term in the series is of the form $a_k=c_k x^k$, 
		where the coefficient $c_k$ depends on the properties of the function $f(x)$.
		Specifically,
		$c_k = \frac{f^{(k)}(0)}{k!}$,
		where $f^{(k)}(0)$ is the $k$\textsuperscript{th} derivative of $f(x)$ and $k!$ is the factorial function:
		\begin{align*}
		  f(x)
		 	& =f(0)+f'(0)x+\frac{f^{\prime\prime}(0)}{2!}x^2+\frac{f^{\prime\prime\prime}(0)}{3!}x^3 +\frac{f^{(4)}(0)}{4!}x^4 + \cdots \\
		 	& = \sum_{k=0}^\infty \frac{f^{(k)}(0)}{k!}x^k.
		\end{align*}
		Using this formula and your knowledge of derivatives,
		you can compute the Taylor series of any function $f(x)$.

		For example,
		let's find the Taylor series of the function $f(x)=e^x$ at $x=0$.
		The first derivative of $f(x)=e^x$ is $f'(x)=e^x$.
		The second derivative of $f(x)=e^x$ is $f''(x)=e^x$.
		In fact,
		all the derivatives of $f(x)$ will be $e^x$ because the $e^x$ is a special function that is equal to its derivative!
		The $k$\textsuperscript{th} coefficient in the power series of $f(x)=e^x$ at the point $x=0$ 
		is equal to the value of the $k$\textsuperscript{th} derivative of $f(x)$ evaluated at $x=0$.
		In the case of $f(x)=e^x$ we have $f^{(k)}(0)=e^0=1$,
		so the coefficient of the $k$\textsuperscript{th} term is $c_k = \tfrac{f^{(k)}(0)}{k!}  = \tfrac{1}{k!}$.
		The Taylor series of $f(x)=e^x$ is
		\[
		 e^x      	= \sum_{k=0}^\infty \frac{1}{k!}x^k
		 	 	= 1 + x + \frac{x^2}{2} + \frac{x^3}{3!} + \frac{x^4}{4!} + \frac{x^5}{5!} + \cdots 
		 \]
		Series are a powerful computational tool for approximating numbers and functions.
		As we compute more terms from the above series,
		our the polynomial approximation to the function $f(x)=e^x$ becomes more accurate.
		The exact value of the function at $x=1$ is $f(1) = e^1 = e$.
		The partial sum of the first six terms (as shown above) gives us an approximation of $e^1$ that is accurate to three decimals. 
		The partial sum of the first 12 terms gives us $e$ to an accuracy of nine decimals.
		% http://bit.ly/12DrCZY


\ifthenelse{\boolean{FORSTATSBOOK}}{
	TODO: any extra series formulas or concepts required to solve the exercises and problems in noBSstats.
}{}

