%!TEX root = ../calculus_tutorial.tex

\section{Integrals}
\label{sec:integrals}

	Integration is the process of computing the ``total'' of some function $f(x)$ accumulated over a range of its input values.
	The symbol $\int$ we use to denote integrals is an elongated letter $S$,
	which is short for \emph{summa}.
	This should give you a hint that integrals perform some kind of summation.
	There are actually two different integral operations:

	\begin{itemize}

		\item	The \emph{integral} of $f(x)$ from $x=a$ to $x=b$
			is denoted $\int_{x=a}^{x=b} f(x)\,dx$
			and corresponds to the area under the graph of $f(x)$ between $a$ and $b$,
			which we also denote $A_f(a,b)$.

		\item The \emph{integral function} $F_0(b) \eqdef A_f(0,b) = \int_{x=0}^{x=b} f(x)\,dx$
 			corresponds to the area-under-the-graph-of-$f(x)$ calculation
			as a function of the	% ALT. with a variable 
			upper limit of integration $b$.

	\end{itemize}
	
	\noindent
	The integral $\int_{x=a}^{x=b} f(x)\,dx$
	when both $a$ and $b$ are fixed is a number $A_f(a,b) \in \mathbb{R}$.
	In contrast,
	the integral $\int_{x=0}^{x=b} f(x)\,dx$ with a variable $b$
	is a function $F_0: \mathbb{R} \to \mathbb{R}$.
	The integral function $F_0(b) \eqdef \int_{x=0}^{x=b} f(x)\,dx$
	computes the area under the graph of $f(x)$
	as a function of the upper limit of integration $b$.
	% for any choice of upper limit of integration $b$.
	Both integral operations are important,
	and we'll discuss each of them in turn.


% This is called a \emph{definite integral} since the limits of integration $a$ and $b$ are defined.
%	Integrals where both limits of integration $a$ and $b$ are defined
%	are called \emph{definite integral}.	
%	The two integration operations are related.
%	The area under the curve $A_f(a,b)$ can be computed from the \emph{change}
%	integral function: $A_f(a,b)=F_0(b)-F_0(a)$.
