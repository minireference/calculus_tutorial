%!TEX root = ../calculus_tutorial.tex

\section{Sets and intervals}

	Sets are arbitrary collections of math objects.
	Many math ideas are expressed using the language of sets,
	so it's worth going over the basic definitions and notation conventions.

	\begin{itemize}

		\item $S,T$: the usual variable names for sets

		\item $s \in S$: this statement is read ``$s$ is an element of $S$'' or ``$s$ is in $S$''

		% seems not needed; put back if needed somewhere in the book (grep for it)
		%	\item $\emptyset$: the \emph{empty set} is a set that contains no elements.
		%		Mathematicians adopted the symbol $\emptyset$ because the notation $\{ \, \}$ is confusing.

		\item $\{ \textrm{  definition  } \}$: the curly brackets surround the definition of a set,
			and the expression inside the curly brackets describes what the set contains.
			% The math symbol ``$|$'' is shorthand for the phrase ``such that'' and it is often used in definitions.

		\item	$S^c$: the \emph{complement} of the set $S$,
			is defined as all elements that are not in the set $S$.
			% TODO: explain assumption of "universe" and complement as all elements of the universe that are not in $S$.

		\item $\mathbb{N}$: the set of natural numbers $\mathbb{N}\eqdef\{0,1,2,\ldots\}$

		\item $\mathbb{Z}$: the set of integers $\mathbb{Z}\eqdef\{\ldots,-2,-1,0,1,2,3,\ldots\}$

		\item $\mathbb{Q}$: the set of rational numbers,
			$\mathbb{Q} \eqdef \left\{  \frac{m}{n} \; \Big| \; m \in \mathbb{Z}, \; n \in \mathbb{N}, \; n \neq 0  \right\}$.
			The set $\mathbb{Q}$ consists of all numbers that can be expressed as \emph{fractions} of the form $\frac{m}{n}$,
			where $m$ is an integer, $n$ is a natural number, and $n \neq 0$.

		\item	$\mathbb{R}$: the set of real numbers

		\item	$\mathbb{R}_+$: the set of nonnegative real numbers.
			The definition of the nonnegative is written as
			$\mathbb{R}_+ \eqdef  \{ \text{all } x \text{ in } \mathbb{R} \text{ such that } x \geq 0 \}$,
			or it can be expressed more compactly as $\mathbb{R}_+ \eqdef \{ x \in \mathbb{R} \; | \; x \geq 0 \}$.

	\end{itemize}

	\noindent
	Note the multiple ways we use the  curly-brackets notation $\{ \}$ to denote sets.
	A \emph{finite set} is defined by simply listing all its elements.
	For example,
	the set of possible outcomes of a coin flip is $\{ \texttt{heads}, \texttt{tails} \}$.
	For an infinite set we can't write down all the elements,
	but we can show the pattern like $\mathbb{N} \eqdef \{0,1,2,3,4,\ldots\}$.
	The meaning of the three dots is ``and so on, continuing the same pattern.''
	Another way to define a set is to use the \emph{set-builder} notation $\{\, \cdot \; | \;  \cdot \, \}$.
	Inside the curly brackets we first describe the general kind of mathematical objects we are talking about,
	followed by the symbol ``$|$'' (read ``such that''),
	followed by the conditions that must be satisfied by all elements of the set.
	The definitions of the rational numbers $\mathbb{Q}$ and the nonnegative real numbers $\mathbb{R}_+$ above are examples of the set-builder notation.

	The \emph{number line} is a visual representation of the set of real numbers $\mathbb{R}$,
	as shown in Figure~\ref{fig:number_line_rationals_and_reals}.
	The real numbers correspond to all the points on the number line,
	from $-\infty$ to $\infty$.

	\begin{figure}[htb]
		\centering
		\includegraphics[width=0.4\textwidth]{figures/calculus/number_line_rationals_and_reals.pdf}
		\vspace{-2mm}
		\caption{The real numbers $\mathbb{R}$ cover the entire number line.}
		\label{fig:number_line_rationals_and_reals}
	\end{figure}

	\noindent
	The set of real numbers includes all the rational numbers like $-\frac{3}{2}$, $\frac{1}{2}$, and $\frac{9}{2}$,
	as well as irrational numbers like $\sqrt{2}$, $e$, and $\pi$.
	This means any number you are likely to run into when solving math problems
	can be visualized as a point on the number line.

	\subsection{Intervals}

		The number line can also be used to represent subsets of the real numbers,
		which we call \emph{intervals}.
		Figure~\ref{fig:interval_2closed_to_4closed} shows an illustration of the interval $[2,4] = \{ x \in \mathbb{R} \;|\; 2 \leq x \leq 4 \}$,
		which is a subset of the real numbers.

		Here are some more examples of various intervals:
		\begin{itemize}

			\item	$[a,b]$: the interval from $a$ to $b$.
				This corresponds to the set of real numbers between $a$ and~$b$,
				including the endpoints $a$ and~$b$.
				The interval $[a,b]$ corresponds to the set $\{ x\in \mathbb{R}\ | \ a \leq x \leq b \}$.

			\item	$[a,\infty)$: the interval from $a$ until infinity,
				which corresponds to the set $\{ x\in \mathbb{R}\ | \ a \leq x \}$.

			\item	$(-\infty,b]$: the interval from negative infinity until $b$,
				which corresponds to the set $\{ x\in \mathbb{R}\ | \ x \leq b \}$.

		\end{itemize}

		\noindent
		The notation $[a,b]$ describes the \emph{closed} interval from $a$ to $b$,
		which means the endpoints $a$ and $b$ are included in the interval.
		The notation $(a,b)$ describes the \emph{open} interval from $a$ to $b$,
		defined as the set $\{ x\in \mathbb{R}\ | \ a < x < b \}$,
		which doesn't include the endpoints $a$ and $b$.
		In other words,
		intervals defined using square brackets ``$[$'' include the endpoints (defined using less-than-or-equal conditions)
		while intervals defined with round brackets ``$($'' do not include their endpoints (defined using strictly-less-than conditions).
		The distinction between open and closed intervals is important in general,
		but makes no difference in the context of probability theory,
		so you don't need to worry about the difference between $[a,b]$ and $(a,b)$ in this book.

		% \SIDENOTE{ @Ivan: If not relevant to the book, why are we explaining it? }

		\begin{figure}[htb]
			\centering	
			\includegraphics[width=0.4\textwidth]{figures/calculus/interval_2closed_to_4closed.pdf}
			\vspace{-3mm}
			\caption{The interval $[2,4] \protect\eqdef \{ x \in \mathbb{R} \; | \; 2 \leq x \leq 4 \}$. }
			\label{fig:interval_2closed_to_4closed}
		\end{figure}




	\subsection{Set operations}

		We use set operations like union $\cup$, intersection $\cap$, and set difference~$\setminus$ to define composite sets.

		\begin{itemize}

			\item $S\cup T$: the \emph{union} of two sets.						\index{set!union|textit}
				The union of $S$ and $T$ corresponds to the elements in either $S$ or $T$.

			\item $S \cap T$: the \emph{intersection} of two sets.				\index{set!intersection|textit}
				The intersection of $S$ and $T$ corresponds to the elements that are in both $S$ and $T$.

			\item $S \, \setminus \, T$: \emph{set difference} or \emph{set minus}.		\index{set!difference|textit}
				The set difference $S \setminus T$ corresponds to the elements of $S$ that are not in $T$.

		\end{itemize}

		\noindent
		Consider the overlapping intervals $A = [a,b]$ and $B = [c,d]$ illustrated in Figure~\ref{fig:sets_A_B_and_set_operations}.
		The union of these two intervals is the set of numbers that are \emph{either} between $a$ and $b$ \emph{or} between $c$ and $d$,
		which corresponds to the interval $[a,d]$.
		The intersection of $A$ and $B$ is the set of numbers that are in \emph{both} $A$ and $B$,
		and corresponds to the interval $[c,b]$.
		The figure also illustrates the two set differences,
		$A\;\setminus\;B$ and $B\;\setminus\;A$ which correspond to numbers that are in one set,
		but not in the other.

		\begin{figure}[htb]
			\centering
			\includegraphics[width=0.3\textwidth]{figures/calculus/sets_A_B_and_set_operations.pdf}
			\caption{	Various intervals that can be obtained using set operations of the intervals $A$ and $B$.}
			\label{fig:sets_A_B_and_set_operations}
		\end{figure}

	\noindent
	I hope these definitions and examples made you feel more comfortable with sets,
	and the weird-looking curly bracket notation that mathematicians use to define sets.
	It might look a little complicated at first,
	but you'll get used to it in the rest of the book.
	% Robyn said: I love these reassuring paragraphs, though it's a bit repetative in this chapter. Just something to be aware of.

	In probability theory,
	we use finite sets and countably infinite sets like the natural numbers to represent the sample spaces of discrete random variables.
	We also use intervals to describe outcomes in the sample space of continuous random variables.
