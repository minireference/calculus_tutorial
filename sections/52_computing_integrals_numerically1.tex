%!TEX root = ../calculus_tutorial.tex


	\subsection{Computing integrals numerically}

		Computing the integral $\int_{a}^b f(x)dx$ \emph{numerically} means using a computer
		to find the approximation to $A_f(a,b)$
% TODO: simply to say     \emph{Riemann sum} approximation using a computer
		by splitting the region of integration into many (think thousands or millions) of strips,
		computing the areas of each strip,
		then adding up the areas to obtain the total area under the graph of $f(x)$.
		The key step is to come up with a general mathematical expression
		that describes the approximate area calculation with $n$ rectangular strips,
		then evaluate this expression for very large values of $n$.
		
		Let's start by looking
		at the math required to calculate the approximation to $\int_{0}^2 h(x)dx$ using $n=10$ rectangles,
		which is illustrated in Figure~\ref{fig:riemann_sum_n10_n20}~(a).
		The width of each rectangle is $\Delta x = \frac{b-a}{n} = \frac{2-0}{10} = 0.2$.
		% The $x$-coordinates of the left endpoints of the $10$ rectangles are $[0, 0.2, 0.4, 0.6, \ldots, 1.8]$.
		The $x$-coordinates of the right endpoints of the $10$ rectangles are $[0.2, 0.4, 0.6, \ldots, 1.8, 2.0]$.
		To find the area of the rectangles,
		we need to know the height of the function $h$ at these $x$-coordinates:
		$[h(0.2), h(0.4), h(0.6), \ldots, h(1.8), h(2.0)]$.
		The area of each rectangle is given by the height-times-width formula,
		and we sum together all of them to compute the total area:
		\[ 
			A_h(0.2)
			\approx
			h(0.2) \cdot 0.2
			+ h(0.4) \cdot 0.2
			+ \cdots
			+ h(2.0) \cdot 0.2
			= 4.92.
		\]

		\noindent
		Looking at figure Figure~\ref{fig:riemann_sum_n10_n20}~(a),
		we can clearly see that the approximation computed in this way is an underestimate
		to the true area under the curve.
		% TODO: revisit this sentence
		Let's ignore this fact for now,
		and ``trust the process'' because the ``quality'' of the approximations improve
		when we split up the region into finer and finer strips.

		The procedure we used for $n=10$ works more generally for any $n$.
		In the general case,
		the rectangles will have width $\Delta x = \frac{b-a}{n} = \frac{2}{n}$,
		which will get smaller and smaller as $n$ grows.
		% The $x$-coordinates of the left endpoints of the $10$ rectangles are $[0, 0.2, 0.4, 0.6, \ldots, 1.8]$.
		The $x$-coordinates of the right endpoints of the $n$ rectangles
		are $[\Delta x ,2\Delta x, 3\Delta x, \ldots, (n-1)\Delta x, n\Delta x]$.
		The height of the rectangles will be
		$[h(\Delta x), h(2\Delta x), h(3\Delta x), \ldots, h((n-1)\Delta x), h(n\Delta x)]$.
		To find the area under the graph of $h(x)$,
		we can sum together the individual height-times-width
		contributions of the $n$ rectangular strips: 
		\[ 
			A_h(0,2)
				\approx	h(\Delta x) \Delta x
						+ h(2\Delta x) \Delta x
						+ h(3\Delta x) \Delta x
						+ \cdots
						+ h(n\Delta x) \Delta x.
		\]

		\noindent
		Observe that all the terms in this summation follow the same pattern:
		the $k$\textsuperscript{th} term in this summation is $h(k\Delta x)\,\Delta x$,
		as $k$ varies from $1$ to $n$.
		When working with long summations as in the above expressions,
		mathematicians use the symbol $\sum$ (the capital Greek letter \emph{sigma}),
		which stands for sum.
		The approximation to the area under $h(x)$ between $x=a$ and $x=b$
		using $n$ rectangular strips corresponds to the following sum:
		$A_h(0,2) \approx \sum_{k=1}^{k=n} h(k\Delta x)\,\Delta x$.
		The labels above and below the summation symbol $\sum$
		play the same role as the superscript and subscript in integral notation.
		The label $k=1$ tells us where to start the summation,
		and label $k=n$ tells us where to stop the summation.


		We can take what we learned from the particular example above
		to write a general formula for approximating the area under the graph of
		any function $f(x)$ between $x=a$ and $x=b$ using $n$ rectangular strips:
		\[
			A_f(a,b) \approx \sum_{k=1}^{k=n} f(a + k\Delta x)\,\Delta x,
			\quad\text{where}\quad
			\Delta x = \tfrac{b-a}{n}.
		\]
		This is the famous \emph{Riemann sum} formula for computing areas.
		
		\noindent
		We'll now convert this math formula into a Python procedure
		that performs the $n$-rectangle area approximation calculation. %  for us:

		\begin{codeblock}[]
		>>> def integrate(f, a, b, n):
		        dx = (b - a) / n
		        xs = [a + k*dx for k in range(1,n+1)]
		        fxs = [f(x) for x in xs]
		        area = sum([fx*dx for fx in fxs])
		        return area
		\end{codeblock}


		\noindent
		The code implements the same operations
		as described by the summation $A_f(a,b) \approx \sum_{k=1}^{k=n} f(a + k\Delta x)\,\Delta x$.
		We first compute the width of the rectangles $\code{dx} = \Delta x = \frac{b-a}{n}$.
		We then create the list \tt{xs}
		that contains the $x$-coordinates of the right endpoints of the rectangles,
		$\tt{xs} = [a + \Delta x, a + 2k\Delta x, a + 3k\Delta x, \ldots, n\Delta x]$,
		and evaluate the function \tt{f} at these $x$-values
		to obtain $\tt{fxs} = [f(a + \Delta x), f(a + 2k\Delta x), f(a + 3k\Delta x), \ldots, f(n\Delta x)]$.
		We calculate the areas of the rectangles by multiplying the heights \tt{fxs} by the width \tt{dx},
		and summing everything together
		to obtain the total \tt{area},
		which we ``return'' as the output of the procedure.
		%	The definition of the \tt{integrate} procedure specifies a default value $n = 10\,000$,


		\subsubsection{Example 3 continued}
	
			Let's use the \tt{integrate} procedure
			to compute the integral of the function $h(x) = 4- x^2$.
			Recall we previously defined the Python function \tt{h}
			that implements the same operation as the math function $h$:

			\begin{codeblock}[]
			>>> def h(x):
			        return 4 - x**2
			\end{codeblock}
			
			To calculate the $n=10$ approximation
			for area under the graph of $h(x)$ between $x=0$ and $x=2$,
			we call the \tt{integrate} procedure with the desired arguments.
			
			\begin{codeblock}[]
			integrate(h, a=0, b=2, n=10)
			4.92
			\end{codeblock}

			\noindent
			Then we can compute the approximation with $n=20$ rectangles just as easily:
			
			\begin{codeblock}[]
			>>> integrate(h, a=0, b=2, n=20)
			5.13
			\end{codeblock}

			%	\noindent
			%	This approximation with $n=20$ rectangles is accurate to the first decimal.
				
			\begin{figure}[htb]
				\centering
				\includegraphics[width=0.98\columnwidth]{figures/calculus/riemann_sum_n10_n20.pdf}
				\vspace{-2mm}
				\caption{	Approximations to the area under the graph of the function $h(x)=4-x^2$ 
						computed using $n=10$ and $n=20$ rectangles.}
				\label{fig:riemann_sum_n10_n20}
			\end{figure}
			

			Let's keep going to see what happens with $n=50$ and $n=100$:
			
			\begin{codeblock}[]
			>>> integrate(h, a=0, b=2, n=50)
			5.2528
			>>> integrate(h, a=0, b=2, n=100)
			5.2932
			\end{codeblock}
			
 
			 \begin{figure}[htb]
				\centering
				\includegraphics[width=0.98\columnwidth]{figures/calculus/riemann_sum_n50_n100.pdf}
				\vspace{-2mm}
				\caption{	Approximations to the area under the graph of $h(x)=4-x^2$ 
						computed using $n=50$ and $n=100$ rectangles.}
				\label{fig:riemann_sum_n50_n100}
			\end{figure}

			\noindent
			The approximations get better and better
			as we increase the number of rectangles $n$.
			%	For $n=1000$, the area approximation is $5.329332$,
			%	for $n=10K$, the area approximation is $5.33293332$,
			%	and for $n=1M$, the area approximation is $5.333329333332$.

			\begin{codeblock}[]
			>>> integrate(h, a=0, b=2, n=1000)
			5.329332
			>>> integrate(h, a=0, b=2, n=10000)
			5.33293332
			>>> integrate(h, a=0, b=2, n=1_000_000)
			5.333329333332
			\end{codeblock}

			\noindent
			The approximation % to the area $A_h(0,2)$
			computed with $n=1M$ rectangles is accurate to $4$ decimals.
			The exact value of the area $A_h(0,2)$ is
			$\frac{16}{3} = 5\frac{1}{3} = 3.\overline{3} = 5.3333333333\ldots$.
			To obtain the exact value,
			we have to \textbf{split up the region into infinitely many rectangular strips},
			as we'll learn next.

%	% get closer and closer to the exact value $10\frac{2}{3}$
%	We can compute the approximation value of the integral to any desired precision
%	by splitting the region into enough rectangles.
%
%	We'll learn how to compute the exact values of integrals in the next section,
%	where we'll learn about symbolic integration techniques.
%	You can think of symbolic integration as a bunch of math ``shortcuts''
%	that allows us compute exact integral for certain functions
%	% MAYBE: == CALC II
%	For now,
%	we content ourselves with numerical approximations,
%	which are pretty good already!



			% CUTTABLE or notebook only
			%	\subsubsection{Examples 1 and 2 revisited}
			%	
			%		We can also use \tt{integrate} to compute the integral of a constant function $f(x)=3$
			%		and the linear function $g(x)=x$ that computed geometrically earlier.
			%
			%		\begin{codeblock}[]
			%		>>> def f(x):
			%		        return 3
			%		>>> integrate(f, a=0, b=5, n=100_000)
			%		15.000000000000002
			%		>>> def g(x):
			%		        return x
			%		>>> integrate(g, a=0, b=5, n=100_000)
			%		12.500125
			%		\end{codeblock}
			%
			%		\noindent
			%		The numerical approximations we obtain are very close
			%		to the exact answers $\int_0^5 f(x)\,dx = 3\cdot 5 = 15$
			%		and $\int_0^5 g(x) \, dx = \tfrac{1}{2} 5 \cdot 5 = 12.5$.
			% /CUTTABLE


%We can approximate the total area under the function $f(x)$ between $x=a$ and $x=b$ by splitting the region into $n$ tiny vertical strips of width $\Delta x$,
%then adding up the areas of the rectangular strips.
%This is known as a \emph{Riemann sum} approximation for an area.
%Figure~\ref{fig:riemannsum-25-50} shows the Riemann sum approximations for the area under the function
%$f(x)=x^3-5x^2+x+10$ between $x=-2$ and $x=2$,
%obtained by using $n=25$ and $n=50$ vertical rectangular strips.
%		it would take you forever.
%		Computing an approximation with $n=1000$ rectangles requires computing $1000$ rectangle areas
%		and the sum of $1000$ terms!






	\subsection{Formal definition of the integral as the limit of a Riemann sum}

		In the limit as the number of rectangles $n$ approaches $\infty$, 							\index{infinity}
		the approximation to the area under the curve becomes \emph{arbitrarily close} to the true area.

		\begin{shadebox}
		\vspace{1mm}
		The integral between $x=a$ and $x=b$ is \emph{defined}
		as the limit as $n$ goes to infinity:
		\[
			\int_{a}^{b}\!f(x)\:dx 
				\eqdef \lim_{n\to\infty} \sum_{k=1}^{n} f(a + k\Delta x)\Delta x.
		\]
		\end{shadebox}
		
		\noindent
		In words,
		the integral is defined as the limit of a Riemann sum
		that consists of infinitely thin rectangular strips.
		We previously defined the $\int_{a}^{b}\!f(x)\:dx$ geometrically
		as the area under the graph of $f(x)$, % $A_f(a,b)$,
		but now you know the formal math definition for integral that mathematicians use.

		Note the structural similarity between the summation formula on the right
		and the integral notation on the left:
		% expressions on the two sides of this equation:
		in both cases we evaluate $f$ at different values $x$ values,
		multiply by a width,
		and add all these contributions together to get the total.
		Perhaps now the weird notation we use for integrals will start to make more sense to you.
		In the limit as $n\to \infty$,
		the summation sign $\sum$ becomes an integral sign $\int$,
		and the step size $\Delta x$ becomes an infinitely small step $dx$.

		The integral $\int_{a}^{b}\!f(x)\:dx$ is defined as a \emph{procedure}
		with infinitely many steps ($\lim_{n\to\infty}$)
		that we perform on the function $f$.
		% expression $\lim_{n\to\infty} \sum_{k=1}^{n} f(a + k\Delta x)\Delta x$
		Recall that the formal definition of the derivative is also a procedure,
		specifically $f'(x) \eqdef \lim_{\delta \to 0} \frac{f(x+\delta)\ - \ f(x)}{\delta}$,
		which corresponds to rise-over-run calculation for computing the slope of $f$ at $x$,
		with an infinitely short step-length $\delta$.
		% ALT. defined as the rate of change of the function $f$ at $x$,
These two procedures are the foundations of calculus.
The limits $\lim_{n\to\infty}$ and $\lim_{\delta \to 0}$
allow us to perform math operations on functions.
% like competing their slopes and 
% TODO: sell applications
		
		
		





% where $\Delta x = \frac{b-a}{n}$ is the width of the rectangular strips.
%	The right endpoint of the $k$\textsuperscript{th}
%	is located at $x_k = a + k\Delta x$,
%	so the height of the rectangular strips $f(x_k)$
%	varies as $k$ goes from between $k=1$ (first strip)
%	and $k=n$ (last strip).


%		The total  of each rectangle is given by the  formula,
%		and we sum together all of them to compute the total area:


		% If we use the right endpoints of the rectangles to calculate their height,
		
%	The area of the first rectangle is $f(x_1)\Delta x = f(a + \Delta x)\Delta x$,
%	which is the height of the function $f$ at $a+\Delta x$ times the width $\Delta x$.
%	The second rectangle has height $f(a+2\Delta x)$,
%	the third rectangle has height $f(a+3\Delta x)$,
%	and so on until the last one which has height $f(a+n\Delta x) = f(a+n\,\frac{b-a}{n} ) =  f(b)$.
%	% we take a step of width $\Delta x$ to the right: $x_2 = x_1 + \Delta x = a + 2 \Delta x$.


%	We can calculate the area of the $k$\textsuperscript{th}
%	strip using the ``base times height'' formula for the area of a rectangle $\Delta x h(x_k)$,
%	where $x_k$ is the left endpoint of the $k$\textsuperscript{th} strip.


%	Let's check that the formula $f(a + k\Delta x)\Delta x$
%	correctly describes the area of the $k$\textsuperscript{th} rectangular strip.
%	We will calculate the height of the rectangles according to their right endpoints.
%	The area of the first rectangle is $f(x_1)\Delta x = f(a + \Delta x)\Delta x$,
%	which is the height of the function $f$ at $a+\Delta x$ times the width $\Delta x$.
%	The second rectangle has height $f(a+2\Delta x)$,
%	the third rectangle has height $f(a+3\Delta x)$,
%	and so on until the last one which has height $f(a+n\Delta x) = f(a+n\,\frac{b-a}{n} ) =  f(b)$.
%	% we take a step of width $\Delta x$ to the right: $x_2 = x_1 + \Delta x = a + 2 \Delta x$.
