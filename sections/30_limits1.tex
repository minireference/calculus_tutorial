%!TEX root = ../calculus_tutorial.tex

\section{Limits}
\label{sec:limits}

	% LIMITS
	Limit expressions are a precise mathematical language
	for talking about infinitely large numbers, infinitely small lengths,
	and mathematical procedures with an infinite number of steps.
	% Many important math and science quantities are defined as limit expressions.
	% For example
	The shorthand ``$\lim$'' is common to all limit expressions,
	with the specifics of the limiting behaviour described below.
	Here are some examples:

% We use limits to describe, with mathematical precision, infinitely large quantities,
%infinitely small quantities, and procedures with infinitely many steps.

	% of each different situations where limit expressions:


	\begin{itemize}

		% ASYMPTOTICS
		\item	$\displaystyle \lim_{x \to \infty} f(x)$: limit expression that describes what happens to $f(x)$
			when the input to the function $x$ tends to infinity (gets larger and larger).
			In words,
			this limit expression is read as ``limit of $f(x)$ as $x$ goes to infinity.''


		% INFINITELY SMALL
		\item	$\displaystyle  \lim_{\delta \to 0} f(\delta)$:
			limit expression that describes the value $f(\delta)$
			as the input $\delta$ tends to zero.
			The number $\delta$ (the Greek letter delta) usually describes a small distance,
			and the limit as delta goes to zero ($\delta \to 0$)
			describes the behaviour of the function $f(\delta)$ for infinitely short distance $\delta$.

		% INFINITELY MANY STEPS
		\item	$\displaystyle  \lim_{n \to \infty} \textrm{proc}(n)$:
			limit expression that describes the value of $\textrm{proc}(n)$
			as the integer $n$ tends to infinity.
			The integer $n$ usually describes the number of steps in a given procedure,
			and $\textrm{proc}(n)$ describes the output of this procedure when $n$ steps are used.

	\end{itemize}

	%	Using limits allows us to obtain answers computed by mathematical procedures with an infinite number of steps!


	\noindent
	Let's look at an example of an example of a math procedure with infinite number of steps
	that was invented by Archimedes,
	one of the OGs of calculus.
	

	\subsection{Example: area of a circle}

		Suppose we want to prove that area of a circle of radius $r$
		is described by the formula $A = \pi r^2$.
		We can approximate the circle as a regular polygon with $n$ sides
		inscribed in the circle.
		Figure~\ref{fig:inscribed-hexagon-octagon-dodecagon}
		shows the hexagonal (6-sides),
		octagonal (8-sides),
		and dodecagonal (12-sides) approximations to the circle.
		% side by side with triangle + labels on octagon 

		\begin{figure}[htb]
			\centering
			\includegraphics[width=0.99\columnwidth]{figures/calculus/inscribed-polygons.pdf}%
			\vspace{-3mm}
			\caption{	Approximations to the area of circle using a hexagon,
					an octagon, and a dodecagon inscribed inside a circle with radius $r$.}
			\label{fig:inscribed-hexagon-octagon-dodecagon}
		\end{figure}

		We can compute the area of the $n$-sided regular polygons
		by selling them up into triangular slices,
		and commuting the area of these slices using the formula for the area of a triangle $\frac{1}{2}bh$
		and trigonometric functions $\sin$ and $\cos$.
		Figure~\ref{fig:inscribed-hexagon-octagon-dodecagon}~(b)
		highlights one of the $16$ identical triangular slices in the case when $n=8$.
		The hypotenuse of this triangle has length $r$,
		the angle $\theta$ is $\frac{360\circ}{16} = \frac{2\pi}{16} = \frac{\pi}{8}$[rad],
		and its sides have length $a=r\cos(\frac{\pi}{8})$ and $b=r\sin(\frac{\pi}{8})$.
		\[
			 2n \times \tfrac{1}{2}ab 
			 = n \times  r\cos(\tfrac{\pi}{n})  \, r\sin(\tfrac{\pi}{n})
			 = n \times r^2  \cos(\tfrac{\pi}{n})\sin(\tfrac{\pi}{n}).
		\]
		In the limit as $n\to \infty$, 
		the $n$-sided-polygon approximation to the area of the circle 
		becomes more and more accurate.
		
		Here is the code that computes the approximations to the area
		of a circle of radius $r=1$ with polygons with higher and higher number of sides.

		\begin{codeblock}[]
		>>> import math
		>>> def calc_area(n, r=1):
		        theta = 2 * math.pi / (2 * n)
		        a = r * math.cos(theta)
		        b = r * math.sin(theta)
		        area = 2 * n * a * b / 2
		        return area
		>>> for n in [6, 8, 10, 50, 100, 1000, 10000]:
		        area_n = calc_area(n)
		        error = area_n - math.pi
		        print(f"{n=}, {area_n=}, {error=}")
		n=6, area_n=2.5980762113533156, error=-0.5435
		n=8, area_n=2.8284271247461903, error=-0.3132
		n=10, area_n=2.938926261462365, error=-0.2027
		n=50, area_n=3.133330839107606, error=-0.00826
		n=100, area_n=3.1395259764656687, error=-0.002067
		n=1000, area_n=3.141571982779476, error=-2.067e-05
		n=10000, area_n=3.1415924468812855, error=-2.067e-07
		\end{codeblock}

		As $n$ goes to infinity we get $\pi r^2$,
		which is the formula for the area of a circle.

		
		\begin{codeblock}[]
		>>> n, r = sp.symbols("n r")
		>>> A_n = n * r**2 * sp.cos(sp.pi/n) * sp.sin(sp.pi/n)
		>>> sp.limit(A_n, n, sp.oo)
		pi * r**2
		\end{codeblock}

%	An interesting example to consider is the number $\pi$, 
%	which is defined as the area of a circle of radius $1$.
%	We can approximate the area of the unit circle by drawing a many-sided regular polygon around the circle.


This example shows practically why considering the limiting behaviour 
can lead us to computing quantities.







%		A circle of radius $1$ is inscribed inside a
%		\emph{regular octagon} (a polygon with eight sides of length $b$).
%		Calculate the octagon's perimeter and its area.
%		\begin{center}
%		\includegraphics[width=0.63212\columnwidth]{figures/calculus/octagon_inscribed_in_circle.pdf}
%		\end{center}		
% cf. \input{../../../MATHPHYSbook/problems/figures/geometry/octagon_inscribed_in_circle.tex}
		% Split the octagon into eight isosceles triangles.

%				$A_{\scriptscriptstyle\scalebox{0.5}{\octagon}}=8\tan(22.5^\circ)$.
%
%			Split the octagon into eight isosceles triangles. 
%			The height of each triangle will be $1$,
%			and its angle measure at the centre will be $\frac{360^\circ}{8}=45^\circ$.
%			Split each of these triangles into two halves down the middle.
%			The octagon is now split into $16$ similar right-angle triangles
%			with angle measure $22.5^\circ$ at the centre.
%			%
%			In a right-angle triangle with angle $22.5^\circ$ and adjacent side $1$,
%			what is the length of the opposite side?
%			The opposite side of each of the $16$ triangles is $\frac{b}{2}=\tan(22.5^\circ)$,
%%			so the perimeter of the octagon is $P_{\scriptscriptstyle\scalebox{0.5}{\octagon}}=16\tan(22.5^\circ)$.
%%			In general,
%%			if a unit circle is inscribed inside an $n$-sided regular polygon,
%%			the perimeter of the polygon is $P_n = 2n\tan\!\left(\frac{360^\circ}{2n} \right)$.
%%			%
%			To find the area of the octagon,
%			we use the formula $A_{\scriptscriptstyle \triangle}=\frac{1}{2}bh$,
%			with $b=2\tan(22.5^\circ)$ and $h=1$ to find the area of each isosceles triangle.
%			The area of the octagon is 
%			$A_{\scriptscriptstyle\scalebox{0.5}{\octagon}}=8\cdot\frac{1}{2}(2\tan(22.5^\circ))(1)=8\tan(22.5^\circ)$.
%			For an $n$-sided regular polygon the area formula is $A_n = n\tan\!\left(\frac{360^\circ}{2n} \right)$.
%			Bonus points if you can tell me what happens to the formulas 
%			for $P_n$ and $A_n$ as $n$ goes to infinity
%			(see \href{http://bit.ly/1jGU1Kz}{\texttt{bit.ly/1jGU1Kz}}).

		



	%	\subsubsection{Example}
	%
	%		Let's begin with a simple example.
	%		Say you have a string of length $\ell$ and you want to divide it into infinitely many, infinitely short segments.
	%		There are infinitely many segments,
	%		and they are infinitely short, so together the segments add to the string's total length $\ell$.
	%
	%		It's easy enough to describe this process in words.
	%		Now let's describe the same process using the notion of a limit.
	%		If we divide the length of the string $\ell$ into $N$ equal pieces then each piece will have a length of
	%		\[
	%		   \delta = \frac{\ell}{N}  \,.
	%		\]
	%		Let's make sure that $N$ pieces of length $\delta$ added together equal the string's total length: 
	%		\[
	%		 N \delta = N \frac{\ell}{N} = \ell.
	%		\]
	%		
	%		\noindent
	%		Now imagine what happens when the variable $N$ becomes larger and larger.
	%		The larger $N$ becomes, the shorter the pieces of string will become.
	%		In fact, if $N$ goes to infinity (written $N \to \infty$),
	%		then the pieces of string will have zero length:
	%		\[
	%		 \lim_{N\to \infty}  \delta = \lim_{N\to \infty} \frac{\ell}{N} = 0.
	%		\]
	%		In the limit as $N \to \infty$, the pieces of string are \emph{infinitely small}.
	%		
	%		Note we can still add the pieces of string together to obtain the whole length:
	%		\[
	%		 \lim_{N\to \infty}  \left( N \delta \right) 
	%		 = 
	%		 \lim_{N\to \infty}  \left( N \frac{\ell}{N} \right)
	%		 = \ell.
	%		\]
	%		Even if the pieces of string are \emph{infinitely small},
	%		because there are \emph{infinitely many} of them,
	%		they still add to $\ell$.
	%		
	%		The take-home message is that as long as you clearly define your limits,
	%		you can use infinitely small numbers in your calculations.
	%		The notion of a limit is one of the central ideas in this course.





	\subsection{Limits at infinity}
	\label{limits:limits_to_infinity}

		We're often interested in describing what happens to a certain function
		when its input variable tends to infinity.
		Does $f(x)$ approach a finite number,
		or does it keep on growing to $\infty$?
		For example,
		consider the limit of the function $f(x)=\tfrac{1}{x}$ as $x$ goes to infinity:
		\[
			\lim_{x \to \infty} f(x)
				= 	\lim_{x \to \infty} \tfrac{1}{x}
				= 	0.
		\]
		The function $\tfrac{1}{x}$ never \emph{actually} reaches zero,
		so it would be wrong to write $f(x)=0$.
		However,
		the the expression $\frac{1}{x}$ closer and closer to $0$ as $x$ goes to infinity.
		Limits are a useful concept because we can write $\displaystyle \lim_{x\to \infty} f(x)=0$,
		even though $f(x)\neq 0$ for any number $x$.
		% MAYBE: limit = tendency

		The function $f(x)$ is said to \emph{converge} to the number $L$
		if the function approaches the value $L$ for large values of $x$:
		\[
			\lim_{x \to \infty} f(x)  =  L.
		\]
		We say ``The limit of $f(x)$ as $x$ goes to infinity is the number $L$.''
		See Figure~\ref{fig:limit-at-infinity-graph} for an illustration.
		% ALT.
		%	The limit equation $\displaystyle \lim_{x \to \infty} f(x) = L$
		%	states that the  ``limit at infinity'' of the function $f(x)$ is equal to the number $L$.

		\begin{figure}[htb]
			\centering
			\includegraphics[width=0.63212\columnwidth]{figures/calculus/limit-at-infinity-graph.png}
			\vspace{-3mm}
			\caption{	A function $f(x)$ that oscillates up and down initially,
					but after a while ``settles down'' close to the value $L$.}
			\label{fig:limit-at-infinity-graph}
		\end{figure}

		%	% CUTTABLE
		%	The limit expression is a concise way of saying the following precise mathematical statement:
		%	\begin{multline*}
		%	  \textrm{For all } \epsilon>0,
		%	  	 \textrm{ there exists a number } S \textrm{ such that } \\ 
		%		 	\left|f(x) - L\right| <\epsilon
		%			\textrm{ for all } x \textrm{ greater than or equal to } S.
		%	\end{multline*}
		%
		%	\noindent
		%
		%	You can think of this fancy statement
		%	as a formal way that a mathematician can prove $\displaystyle \lim_{x \to \infty} f(x) = L$ is true.
		%	You tell the mathematician a level of $\epsilon$ that would convince you,
		%	and the mathematician must find a starting point~$S$
		%	after which $f(x)$ becomes (and stays) $\epsilon$-close to the limit $L$.
		%	If the mathematician can succeed for all levels of precision $\epsilon$,
		%	not matte show small,
		%	then we have to believe that $\displaystyle \lim_{x \to \infty} f(x) = L$ is true.
		%	% /CUTTABLE

		\paragraph{Example}

			Calculate $\lim\limits_{x\to \infty} \frac{2x+1}{x}\,$.
			You are given the function $f(x)=\frac{2x+1}{x}$
			and must determine what the function looks like for very large values of $x$.
			% We can rewrite the function as $\frac{2x+1}{x}=2+\frac{1}{x}$ to more easily see what is going on:
			\[
				\lim_{x\to \infty}\!\! \frac{2x+1}{x} 
					= \lim_{x\to \infty}\!\left( 2 + \frac{1}{x} \right)
					= 2 + \!\lim_{x\to \infty}\!\!\left( \frac{1}{x} \right)
					= 2 + 0 = 2.
			\]
			As the denominator $x$ becomes larger and larger,
			the fraction $\frac{1}{x}$ becomes smaller and smaller,
			so $\displaystyle \lim_{x \to \infty} \tfrac{1}{x} = 0$.





	\subsection{Limits to a number}
	\label{limits:limits_to_a_number}

		The limit of $f(x)$ approaching $x=a$ \emph{from the right} is defined as
		\[
		 \lim_{x\to a^+} f(x) = \lim_{\delta \to 0} f(a + \delta). 
		\]
		To find the limit from the right at $a$, we let $x$ take on values like 
		$a+0.1$, $a+0.01$, $a+0.001$, $a+0.0001$, etc.

		The limit of $f(x)$ when $x$ approaches \emph{from the left} is defined analogously,
		\[
 		   \lim_{x\to a^-} f(x)  = \lim_{\delta \to 0} f(a - \delta).
		\]
		%		describes what happens to $f(x)$ as $x$ approaches $a$ from below
		%		(from the left) with values like $x=a-$, 
		%		with $\delta>0, \delta \to 0$.

		If both limits from the left and from the right of some number exist and 
		are equal to each other, we can talk about the limit as $x\to a$ 
		without specifying the direction of approach: 
		\[
		 \lim_{x\to a} f(x) =  \lim_{x\to a^+} f(x) =  \lim_{x\to a^-} f(x).
		\]
		For the two-sided limit of a function to exist at a point,
		both the limit from the left and the limit from the right must converge to the same number.
		If the function $f(x)$ obeys, $f(a) = L$ and $\displaystyle\lim_{x\to a} f(x) = L$,
		we say the function $f(x)$ is continuous at $x=a$.


	\subsection{Continuity}
	\label{limits:continuity}

% SHORTEN + COMPRESS
		A function is said to be \emph{continuous}
		if its graph looks like a smooth curve that doesn't make any sudden jumps and contains no gaps.
		If you can draw the graph of the function on a piece of paper without lifting your pen,
		the function is continuous.

		A more mathematically precise way to define continuity is to say the function is equal to its limit for all $x$.
		We say a function $f(x)$ is \emph{continuous} at $a$ if the limit of $f$ as $x\to a$ converges to $f(a)$:
		\[
		 \lim_{x \to a}  f(x) =  f(a).
		\]
		Remember,
		the two-sided limit $\lim_{x\to a}$ requires both the left and the right limit to exist and to be equal.
		Thus, the definition of continuity implies the equality $\lim_{x \to a^-}  f(x) =  f(a)$
		and $f(a) = \lim_{x \to a^+}  f(x)$,
		which correspond to the idea of ``not lifting the pen'' when drawing the graph at $x=a$.
%		In words,
%		this means that a function $f(x)$ is continuous at $x=a$
%		if the limit from the left $\lim_{x \to a^-}  f(x)$
%		and the limit from the right $\lim_{x \to a^+}  f(x)$
%		are both equal to the value of the function at $x=a$.
		% Take a moment to think about the mathematical definition of continuity at a point.
		% Can you connect the math definition to the intuitive idea that functions are continuous if they can be drawn without lifting the pen?
%		Most functions we'll study in calculus are continuous,
%		but not all functions are.
% /SHORTEN


		%	Functions that are not defined for some value, as well as functions that make sudden jumps, are not continuous.
		%	%		Another examples is the function $f(x)=\frac{2x+1}{x}$ which is discontinuous at $x=0$
		%	%		(because the limit $\lim_{x \to 0}  f(x)$ doesn't exist and $f(0)$ is not defined).
		%	For example,
		%	consider the function $f : \mathbb{R} \setminus \{0\} \to \mathbb{R}$ defined by
		%	\[
		%		f(x)
		%		=\frac{ | x-3| }{x-3} 
		%		= \left\{ 	\begin{array}{rl}
		%				1	\quad	&  \mathrm{  if } \;  x > 3, \\
		%				-1	\quad	&  \mathrm{  if } \;  x < 3.
		%		                        \end{array}                    \right.
		%	\]
		%	The function $f$ is continuous everywhere on the real line except at $x=3$.
		%	Since this function $f$ is ``missing'' only at a single point,
		%	we can try to ``patch it'' by filling in the missing value.
		%	Consider the function $g : \mathbb{R} \to \mathbb{R}$ defined as
		%	\[
		%		g(x)
		%		= \left\{ 	\begin{array}{rl}
		%				1	\quad	&  \mathrm{  if }\;  x > 3, \\
		%				1	\quad	&  \mathrm{  if }\;  x = 3, \\
		%				-1	\quad	&  \mathrm{  if } \;  x < 3.
		%		                        \end{array}                    \right.		
		%	\]
		%	The function $g$ is \emph{continuous from the right} at the point $x=3$,
		%	since $\lim_{x \to 3^+} g(x)=1=g(3)$.
		%	However,
		%	taking the limit from the left,
		%	we find $\lim_{x \to 3^-} g(x)=-1 \neq g(3)$,
		%	which tells us $g$ is not continuous from the left.
		%	We say the function $g$ has a \emph{jump discontinuity} at $x=3$.

		% Khan Academy
		% https://www.youtube.com/watch?v=Y7sqB1e4RBI

		%	\paragraph{Example 3}	
		%		We can calculate the limit $\displaystyle\lim_{x\to 5} \frac{2x+1}{x}$ as follows:
		%		\[
		%		 \lim_{x\to 5} \frac{2x+1}{x}
		%		  = \frac{2(5)+1}{5}
		%		  = \frac{11}{5}.
		%		\]
		%		There is nothing tricky going on here---we plug the number $5$ into the equation, and voila. 
		%		The function $f(x)=\frac{2x+1}{x}$ is continuous at the value $x=5$, so the limit of the function 
		%		as $x\to 5$ is equal to the value of the function $\displaystyle\lim_{x\to 5} f(x) = f(5)$.
		%		% This is true in general for any continuous function.
				


	\subsection{Limit formulas}
	
		The calculation of the limit of the sum, difference, product, and quotient of two functions
		is computed as follows, respectively:
		\begin{align*}	
			\lim_{x \to a} (f(x) + g(x)) 	& =  \lim_{x \to a} f(x) + \lim_{x \to a} g(x), 	\\
			\lim_{x \to a} (f(x) - g(x)) 	& =  \lim_{x \to a} f(x) - \lim_{x \to a} g(x), 	\\
			\lim_{x \to a} f(x)g(x) 		& =   \lim_{x \to a} f(x) \cdot \lim_{x \to a} g(x), \\
			\lim_{x \to a} \frac{f(x)}{g(x)} & =  \frac{ \displaystyle \lim_{x \to a} f(x) } { \displaystyle \lim_{x \to a} g(x)} \, .
		\end{align*}
		The above formulas indicate we are allowed to \emph{take the limit inside} of the basic arithmetic operations.

		Euler's number $e$ is defined as the limit
		$e \eqdef \lim_{n\rightarrow\infty}\left(1+\tfrac{1}{n}\right)^{\!n}$,
		which is the compound interest calculation for an annual interest rate of $100\%$
		with compounding is performed infinitely often.

	

	\subsection{Computing limits using SymPy}

		In SymPy,
		we use the symbol \texttt{oo} (two lowercase \texttt{o}s) to denote $\infty$.
		Infinity is not a number but a process: the process of counting forever.
		Thus, $\infty + 1 = \infty$, 
		$\infty$ is greater than any finite number,
		and $1/\infty = 0$.

		\begin{codeblock}[]
		>>> from sympy import oo
		>>> oo+1
		oo
		>>> 5000 < oo 
		True
		>>> 1/oo
		0
		\end{codeblock}

		\noindent
		The SymPy function \tt{limit} allows us to compute limit expressions.
		For example,
		here is the code for computing the limit $\lim_{x \to \infty} \tfrac{1}{x}$:

		\begin{codeblock}[]
		>>> import sympy as sp
		>>> x = sp.symbols("x")
		>>> sp.limit(1/x, x, oo)
		0
		\end{codeblock}
		
		\noindent
		The first line imports the \tt{sympy} module under the alias \tt{sp}.
		The second line defines the symbol \tt{x},
		which we can use to write math expressions.
		% The syntax for the function \tt{sp.limit(expr, var, val)}
		We provide the expression \tt{1/x} as the first first argument to the function \tt{limit},
		then specify the variable \tt{x} and destination \tt{oo} as the second and third arguments.
		
		%	Consider the function $f(x)=\frac{1}{x}$.
		%	The \texttt{limit} command shows us what happens to $f(x)$ near $x=0$ and as $x$ goes to infinity:
		Here is another example,
		that computes the limit of the fraction $\frac{2x+1}{x}$ as $x$ goes to infinity:
	
		\begin{codeblock}[sympy-limit-exp-over-x-100]
		>>> sp.limit( (2*x+1)/x , x, oo)
		2
		\end{codeblock}
	
		\noindent
		Recall the definition of Euler's number $e \eqdef \lim_{n\to \infty} \left( 1 + \frac{1}{n}\right)^{n}$
		that we showed in the introduction.
		We can check this definition by making SymPy compute the limit.

		\begin{codeblock}[sympy-e-from-limit]
		>>> n = sp.symbols("n")
		>>> limit((1+1/n)**n, n, oo)
		E
		\end{codeblock}

		\noindent
		Note SymPy produced the exact value $\tt{E} = 2.718281828\ldots$.
		% and not an approximation.

		% TODO: ADD MORE LIMIT EXAMPLES
		%	\noindent
		%	Here are some other examples of limits:
		%
		%	\begin{codeblock}[]
		%	>>> limit(sin(x)/x, x, 0)
		%	1
		%	>>> limit(sin(x)**2/x, x, 0)
		%	0
		%	>>> limit(exp(x)/x**100,x,oo) # which is bigger e^x or x^100 ?
		%	oo                            # exp f >> all poly f for big x  
		%	\end{codeblock}

		% BONUS SYMPY EXAMPLES
		%	\noindent
		%	The SymPy function \tt{limit} allows us to compute limit expressions.
		%	For example,
		%	if we want to see if the exponential function $e^x$ or the polynomial function $x^{100}$ grows faster
		%	in the limit as $x$ goes to infinity,
		%	The code for computing the limit of the ratio between these two expressions is
		%
		%	\begin{codeblock}[sympy-limit-exp-over-x-100]
		%	>>> from sympy import limit, exp, oo
		%	>>> limit(exp(x)/x**100, x, oo) 
		%	oo
		%	\end{codeblock}
		%
		%	\noindent
		%	The answer $\infty$,
		%	written as \tt{oo} (two lowercase letters ``o''),
		%	tells us exponential functions grow faster than polynomial functions.
		%	%	This result has implications in computer science,
		%	%	where algorithms whose running time grows exponentially with the size of their input are considered bad
		%
		%	% EXAMPLE 2: splitting up an interval into $n$ segments, then making $n$ go to infinity
		%	%	splitting with an infinite number of segments
		%	%	\begin{codeblock}[sympy-limit-sement-zero-length]
		%	%	>>> from sympy import limit, oo, summation
		%	%	>>> delta = (b - a)/n
		%	%	>>> limit(delta, n, oo)
		%	%	0
		%	%	\end{codeblock}
		%	%
		%	%	\begin{codeblock}[sympy-limit-sements-add-to-interval]
		%	%	>>> summation(delta, (i, 0, n-1))
		%	%	b - a				
		%	%	\end{codeblock}


	\subsection{Applications of limits}

		Limits are important in calculus
		because they are used in the formal definitions of derivatives, integrals, and series.

		\subsubsection*{Limits for derivatives}
		\label{limits:limits_for_derivatives}
	
			The formal definition of a function's derivative is expressed in terms
			of the rise-over-run formula for an infinitely short run:
			\[
				f'(x) 
					\; = \; \lim_{\textrm{run} \to 0} \frac{\text{rise}}{\text{run}} 
					\; = \;  \lim_{\delta \to 0} \frac{f(x+\delta)\; - \; f(x)}{x+\delta \; - \;  x}.
			\]
			We'll continue the discussion of this formula in Section~\ref{sec:derivatives}.


		\subsubsection*{Limit for integrals}
		\label{limits:limits_for_integrals}

			One way to approximate the area under the curve $f(x)$ between $x=a$ and $x=b$
			is to split the area into $n$ little rectangles of width $\Delta x = \frac{b-a}{n}$ and height $f(x)$,
			and then calculate the sum of the areas of the rectangles:
			\[ 
				A_f(a,b) \approx \underbrace{
								\Delta x f(a)
								\!+\! \Delta x f(a+\Delta x) 
		  						% \!+\! \Delta x f(a+2\Delta x) + \cdots 
								\!+\! \Delta x f(b-\Delta x)
							}_{n \textrm{ terms}}.
			\]
			We obtain the exact value of the area in the limit where we split the area into an infinite
			number of rectangles with infinitely small width:
			{ \small
			\[ 
			  \!\!\int_a^b\!\!\!f(x) \, dx
			  = \!A_f(a,b)
			   = \!\!\lim_{n \to \infty}\!\! \Delta x \left[ 
			   	 f(a) \!+\! f(a+\Delta x) \!+\! \cdots \!+\!  f(b-\Delta x) 
				\right].
			\]}
			
			\noindent 
			Computing the area under a function by splitting the area into infinitely 
			many rectangles is an approach known as a \emph{Riemann sum},
			which we'll discuss in Section~\ref{sec:integrals}.
				
			%	\subsection{Limits for sequences}	TODOv7
			%	\label{limits:limits_for_sequences}
	
	
		
		\subsubsection*{Limits for series}
		\label{limits:limits_for_series}
		
			We use limits to describe the convergence properties of series.
			For example, the partial sum of the first $n$ terms of the geometric series
			$a_k= r^k$ corresponds to the following expression:
			\[
			  S_n
			   = \sum_{k=0}^n r^k 
			   = 1 + r + r^2 + r^3 + \cdots + r^n.
			\]
			The \emph{series} $\sum a_k$ is defined as the limit $n\to \infty$ of the above expression.
			For values of $r$ that obey $|r|<1$,
			the series \emph{converges} to the a finite value:
			\[ 
			  S_\infty 
			   = \lim_{k \to \infty} S_k 
			   = \sum_{k=0}^\infty r^k
			   = 1 + r + r^2 +  r^3 + \cdots 
			   =\frac{1}{1-r}.
			\]
			%	% PROOF
			%	To convince yourself the above formula is correct,
			%	observe how the infinite sum $S_\infty$ is similar to a shifted version of itself: $S_\infty=1+rS_\infty$.
			%	Now solve for $S_\infty$ in the equation $S_\infty=1+rS_\infty$.			
			We'll learn more about series in Section~\ref{sec:sequences_and_series}.
	

	% LEAD OUT	
	In the remainder of this tutorial we'll use limits to evaluate derivatives, integrals, and series expressions.
%	In each of these domains,
%	limit expressions will help us make precise statements
%	that describe calculus procedures with infinite small lengths and infinite number of steps.


