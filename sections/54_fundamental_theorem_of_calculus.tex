%!TEX root = ../calculus_tutorial.tex

	\subsection{Act 3: Fundamental theorem of calculus}

		Note the pattern in the formulas for the integral functions $F_0(b)$ and $G_0(b)$.
		The integral function of the constant function $f(x)=3$ was a linear function $F_0(b) = 3b$.
		The integral of the linear function $g(x) = x$ is a quadratic function $G_0(x) = \frac{1}{2}b^2$.
		%	The integral of the quadratic polynomial $h(x) = 4 - x^2$
		%	is a third degree polynomial $H_0(b) = 4b - \frac{1}{3}b^3$.
		In each case,
		the integral function seems to increase the degree of the function.
		What is up with that?
		Is this a coincidence,
		or is there some fundamental math pattern
		we could follow to ``guess'' integral functions?

		The fundamental theorem of calculus (FTC)
		describes the inverse relation that exists between
		the integration operation $\int \tt{<f>}\,dx$
		and the differentiation operation $\frac{d}{dx}[\tt{<f>}]$.
		%
		A priori,
		there is no reason to suspect that integrals would be related to derivatives.
		The integral corresponds to the computation of an area,
		whereas the derivative operation computes the slope of a function.
		Yet behold:

		\begin{shadetheorem}[fundamental theorem of calculus]
			Let $f(x)$  be a continuous function,
			and let $a \in \mathbb{R}$ be a constant.
			Define the function $F_a(x)$ as follows:
			\[
				F_a(x)  \; \eqdef \; A_f(a, x) \; = \; \int_a^x f(u) \, du.
			\]
			Then,
			the derivative of $F_a(x)$ with respect to $x$ is equal to $f(x)$:
			\[
				\frac{d}{dx}\!\big[F_a(x)\big] \; = \; f(x).
			\]
		\end{shadetheorem}

		\noindent
		Note we use the new variable $u$ inside the integral
		since $x$ is already used to denote the upper limit of integration.

		% MAYBE: 	In words, the FTC tells us that ...

		
		To understand the inverse relationship between integration and differentiation,
		we can draw an analogy with the relationship between a function $f$
		and its inverse function $f^{-1}$,
		which \emph{undoes} the effects of $f$.
		See Figure~\ref{fig:functions-inverse} on page~\pageref{fig:functions-inverse}.
		Given some initial value $x$,
		if we apply the function $f$ to obtain the number $f(x)$,
		then apply the inverse function $f^{-1}$ on the number $f(x)$,
		we get back to the initial value $x$ we started from:
		\[
			f^{-1}\!\left( f(x) \right)	=	x.
		\]
		Similarly,
		\textbf{the derivative operation is the inverse of the integral operation}.
		If you perform the integral operation $\int \tt{<f>}\,dx$
		followed by the derivative operation $\frac{d}{dx}[\tt{<f>}]$ on any function $\tt{<f>}$,
		you'll get back to original function:
		\[
			\frac{d}{dx} \int_c^x f(u)\,du = f(x).
		\]

		\noindent
		Let's use SymPy to verify that the fundamental theorem of calculus is true.
		We'll start with the function $\tt{f} = f(x) = x^2$,
		compute its integral function $\tt{F}$ using \tt{sp.integrate},
		then take the derivative of $\tt{F}$ using \tt{sp.diff}.

		\begin{codeblock}[]
		>>> f = x**2
		>>> F = sp.integrate(f, x)
		>>> F
		x**3/3           # + C
		>>> sp.diff(F, x)
		x**2
		\end{codeblock}

		\noindent
		We see that sequence $\tt{diff(integrate(}f(x)\tt{))} = \frac{d}{dx} \int_0^x f(u)\,du$
		bring us back to the original $f(x)$ we started from.


		% DIRECTION 2
		For ordinary math functions,
		we know that if the function $f^{-1}$ is the undo action for the function $f$,
		then $f$ is also the undo action for $f^{-1}$:
		$f\!\left( f^{-1}(y) \right) = y$.
		Similarly,
		the inverse relationship between integrals and derivative holds in the other direction too.
		\textbf{The integral operation is the inverse operator of the derivative operation}.
		If we start with some function $G(x)$,
		calculate its derivative function $G^{\prime\!}(x)$,
		then compute the integral of the derivative function $G^{\prime\!}(x)$,
		we arrive back at the original function $G(x)$
		(up to an additive constant):
		\[
			\int_c^x G^{\prime\!}(u)\,du = G(x) + C.
		\]
		% The constant $C$ 
		%  choice of the starting point for integration

		\noindent
		Let's use SymPy to verify this formula.
		We'll start with the function $\tt{G} = G(x) = x^3$,
		commute its derivative $\tt{dGdx} = G^{\prime\!}(x)$ using \tt{sp.diff},
		then use \tt{sp.integrate} to compute the integral function of $G^{\prime\!}(x)$.

		\begin{codeblock}[]
		>>> G = x**3
		>>> dGdx = sp.diff(F, x)
		>>> dGdx
		3*x**2
		>>> sp.integrate(dGdx, x)
		x**3    # + C
		\end{codeblock}

		\noindent
		We see the operations $\tt{integrate(diff(}G(x)\tt{))} = \int_0^x G^{\prime\!}(u)\,du$
		bring us back to the original $G(x)$ we started from.


		\subsubsection{Using antiderivatives to compute integrals}

			The fundamental theorem of calculus
			gives us a way to compute integrals and integral functions
			by ``reverse engineering'' derivatives.
			In order to explain this idea,
			we'll introducing a new concept.
			
			\begin{shadebox}
				\vspace{1mm}
				Given some function $f(x)$,
				any function $F(x)$ that satisfies the equation $F^{\prime\!}(x) = f(x)$
				is called an \emph{antiderivative} of $f(x)$.
				\vspace{1mm}
			\end{shadebox}
			
			\noindent
			In words,
			an antiderivative of $f(x)$ is a function whose derivative is $f(x)$.
			There is no single antiderivative function,
			since adding an constant to an antiderivative function
			still satisfies the definition $\frac{d}{dx}\big[ F(x) + C \big] = f(x)$,
			because the derivative of the constant $C$ is zero.
			%	Instead of saying \emph{the} antiderivative of $f(x)$,
			%	it is more appropriate to talk about the \emph{family of antiderivative} functions,
			%	which we denote as $F(x) + C$.

			The fundamental theorem of calculus
			tells us that antiderivative functions $F(x)+C$
			are closely related to integral functions $F_c(x) = \int_c^x f(u)\,du$.
			The integral function $F_0(x)$ is equal to an antiderivative function $F(x) + C$,
			for some additive constant $+C$.
			%	Equivalently,
			%	every antiderivative function $F(x) + C$
			%	is equal to an integral function $F_c(x)$ for some starting point $c$.

			This equivalence gives us a analytical shortcut
			to find the integrals functions $F_0(b) = \int_0^b f(x)\,dx$
			by finding the antiderivative function of $f(x)$.
			% for many simple functions $f(x)$.
			% If we can find an  such that $F^{\prime\!}(x) = f(x)$,
			% then we know $F_0(b) = F(b) + C$.
			To find an antiderivative of $f(x)$,
			we can look for a function $F(x)$ whose derivative is $f(x)$.
			We can use the table of derivative formulas
			(Table~\ref{table:derivatives} on page~\pageref{table:derivatives})
			in the reverse direction to find antiderivatives.
			For example,
			to find the antiderivative of the function $f(x) = m$,
			we look for a row where this function appears on the right side of the table,
			and then look at the corresponding function on the left side of the table,
			which is the function $F(x) = mx + b$ in this case.
			We can verify that $F^{\prime\!}(x) = f(x)$,
			so indeed $F(x) = mx + b$ is an antiderivative of $f(x) = m$.
			Furthermore,
			the equivalence between antiderivatives and integral functions
			tells us that the integral function of $f(x)$ is $F_a(x) = \int_a^x m\,du = mx + b$,
			for some constant $b$.

			%	\subsubsection{Computing integral functions using antiderivatives}
			%	Recall that all integral functions differ by a constant $+C$,
			%	which is related to the starting point of the integral function:
			%	\[
			%		F_a(b) = \int_a^b f(x)\,dx = F_0 + C.
			%	\]
			%	This means $F(x) + C = F_0(b)$
			%	for some $+C$.

			%	\subsubsection{Computing integrals using antiderivatives}
			%	% Using integrals by reverse engineering derivatives
			%	% ALT. by antidifferentiation
			%	
			%	If we can find a function $F(x)$ such that $F^{\prime\!}(x) = f(x)$,
			%	then we know $F(x)$ is an integral function of $f(x)$
			%	and we can compute the integral using:
			%	\[
			%		\int_a^b  f(x)\,dx = F(b) - F(a).
			%	\]

			%	Thus, to find an integral function of the function $f(x)$,
			%	we must find a function $F(x)$ such that $F^{\prime\!}(x)=f(x)$. 

			%I could go on but I think you get the point:
			%all the derivative formulas you learned (see page \pageref{sec:derivative_formulas})
			%can be used in the opposite direction as integral formulas.
			
			%Remember to always add a constant term $+C$ to your answer.
			%The answer to the indefinite integral question $\int\!f(x)\:dx$ is not a single function $F(x)$,
			%but a whole family of functions $F(x)+C$ that differ by an additive constant $C$.
			

			Let's use the derivatives ``reverse engineering'' procedure to
			find the integral function $H_0(b)$ in Example~3.
			

			\paragraph{Example 3 continued}

				We're given the function $h(x) = 4 - x^2$
				and we want to find its integral function $H_0(b) = \int_0^b  h(x)\, dx$.
				This fundamental theorem of calculus tells us this problem
				is equivalent to finding a function $H(x)$ whose derivative is $h(x)$.
				The function $h(x) = 4 - x^2$ has two terms.
				The first term is a constant $4$.
				We can guess that the corresponding term in the antiderivative function $H(x)$
				will be $4x$,
				since $\frac{d}{dx}[ 4x ] = 4$.
				Now for the quadratic term $-x^2$.
				Remembering the derivative formulas for polynomials,
				we can guess that antiderivative $H(x)$ must contain a $x^3$ term,
				because taking the derivative of a cubic term results in a quadratic term.
				Therefore,
				the antiderivative function we're looking for has the form $H(x)=3x - kx^3$,
				for some multiplicative constant $k$.
				Pick the constant $k$ that makes this equation true:
				$H^{\prime\!}(x) = 4 - 3kx^2 = 4 - x^2$.
				Solving $3k=1$,
				we find $k=\frac{1}{3}$
				and so the antiderivative function
				we're looking for is $H(x) = 4x - \frac{1}{3}x^3 + C$.
				The equivalence between antiderivatives and integral functions
				tells us that the integral function we're looking for
				has the form $H_0(b) =  4b - \frac{1}{3}b^3 + C$
				for some constant $C$.
				We know from the geometric definition of the integral that
				when $b=0$ the integral function must have value zero,
				so $C=0$ in this case.
				The integral function we're looking for is therefore $H_0(b) = 4b - \frac{1}{3}b^3$.



		\subsubsection{Using derivative formulas in reverse}

			Computing integral functions by finding antiderivatives is very powerful.
			We can use it to find the internal functions
			for all the function listed in the table of derivative formulas
			(see page~\pageref{table:derivatives}).
			% based on using the derivative formulas in reverse to guess the value of $F(x)$ 
			For example,
			the table tells us that the derivative of the linear function $f(x) = mx+b$
			is the constant function $f^{\prime\!}(x) = m$.
			This means the integral of a constant function
			is a linear function $\int m\,dx = mx + C$.
			The integral function of an exponential
			is also an exponential $\int e^x \,dx = e^x + C$,
			since $\frac{d}{dx}[e^x] = e^x$.
			The derivative of $\log_e(x)$ is $\frac{1}{x}$,
			therefore the integral of $\frac{1}{x}$ is $\log(x)$.
			Similarly for the trigonometric functions $\int \cos(x) \,dx = \sin(x)$
			and $\int - \sin(x) \,dx = \cos(x)$.
			For economy of space,
			we'll verify all these integral formulas by computing
			the integral of the function $f(x) = m + e^x + \frac{1}{x} + \cos(x) - \sin(x)$
			that contain the mix of several functions on the right side of Table~\ref{table:derivatives}.

			\begin{codeblock}[]
			>>> fx = m + sp.exp(x) + 1/x + sp.cos(x) - sp.sin(x)
			>>> sp.integrate(fx, x)
			m*x + exp(x) + log(x) + sin(x) + cos(x)
			\end{codeblock}

			\noindent
			SymPy tells us the integral function
			$F_0$ is $F_0(x) = mx + e^x + \log(x) + \sin(x) + \cos(x)$,
			which are all the corresponding terms on the left side of the table of derivative formulas.

			\ifthenelse{\boolean{FORSTATSBOOK}}{
				In probability theory,
				the FTC tells us that the probability density
				can be obtained from the cumulative distribution using differentiation
				\[
					f_X(x) = \frac{d}{dx}\!\left[ F_X(x) \right] = \frac{dF_X}{dx}(x) = F'_X(x).
				\]
				The fact that we can obtain $f_X$ from $F_X$ and vice versa,
				means we only need to define one of the two functions,
				and obtain the other function using differentiation or integration.
				In this book,
				we define the random variable $X$ through its probability distribution function $f_X$,
				then define $F_X$ as the integral of $f_X$.
				In other books,
				you might see the random variable $X$ being defined through its cumulative distribution function $F_X$,
				with its probability density function $f_X$ defined as the derivative of $F_X$.
			}{}


			% LEAD OUT: what do we do when there is no simple formula?
			\medskip
			\noindent
			Okay,
			but what do we do if the function we want to integrate
			doesn't appear in Table~\ref{table:derivatives}?


% TODO: explain what it means for integral not to have closed form
%	MAYBE: 	warn there is no general F for any f
%			only for certain special cases have exact symbolic formula
%			for all other cases we're forced to do the split-into-vertical-strips,
%			i.e. there is no analytical shortcut.





% FTC CUT MATERIAL

%	The integral function $F_a(x)$ is obtained from the original function $f(x)$ using integration,
%	$F_a(x) = \int_a^x f(u) du$.
%	Another way to describe this is to say we \emph{applied} the integration operator $\int \cdot dx$
%	on the function $f(x)$ to obtain the integral function $F_a(x)$.
%	The derivative function $f^{\prime\!}(x)$ is defined by the formula
%	$f^{\prime\!}(x) = \lim_{\delta \to 0} \frac{f(x+\delta)\ - \ f(x)}{\delta}$.
%	We can also say we \emph{applied} the derivative operator $\frac{d}{dx}[\cdot]$
%	to the function $f(x)$ to obtain the derivative function $f^{\prime\!}(x)$.
%	I use the word ``operator'' here to refer to an operation that acts on functions.


% The fundamental theorem of calculus describes the relationship between
% derivatives and integrals.
%	The fundamental theorem of calculus establishes an equivalence between the set 
%	of integral functions and the set of antiderivative functions:
%	\[
%		A_a(x)=F(x)+C.
%	\]
%	All integral functions $F_a(x)$ are antiderivatives of $f(x)$.


%	There is no reason \emph{a priori} to think that integration and differentiation might be related:
%	the former is a calculation about areas,
%	while the latter is a calculation about slopes.
%	The fundamental theorem of calculus reveals that they are in fact inverse operations:
%	we can obtain the original function $f(x)$ from the integral function $F_a(x)$ by computing it's derivative:
%	\[
%		\frac{d}{dx}\big[F_a(x)\big] 	= 	\frac{d}{dx}\left[\int_a^x f(u) \, du \right]  	= 	f(x).
%	\]
%	Note we used a temporary variable $u$ as the integration variable,
%	since $x$ is already used to denote the upper limit of integration.


%		if you compute the derivative of a function followed by the integral,
%		you will obtain the original function $f(x)$ (up to a constant):
%		\[
%		  \left( \int dx \circ \frac{d}{dx}\right) f(x) = \int_c^x f^{\prime\!}(u)\,du = f(x) + C.
%		\]


% That's kind of cool, no?
