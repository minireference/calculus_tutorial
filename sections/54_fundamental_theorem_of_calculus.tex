%!TEX root = ../calculus_tutorial.tex

	\subsection{Act 3: Fundamental theorem of calculus}

Note the pattern in the formulas for the integral functions
$F_0(b)$, $G_0(b)$, and $H_0(b)$.
The integral function of the constant function $f(x)=3$ was a linear function $F_0(b) = 3b$.
The integral of the linear function $g(x) = x$ is a quadratic function $G_0(x) = \frac{1}{2}b^2$.
%	The integral of the quadratic polynomial $h(x) = 4 - x^2$
%	is a third degree polynomial $H_0(b) = 4b - \frac{1}{3}b^3$.
In each case,
the integral function seems to increase the degree of the function.
What is up with that?
Is this a coincidence,
or some fundamental math pattern
we could use to ``guess'' the integral function of any polynomial?

	
		The fundamental theorem of calculus (FTC)
		describes the inverse relation that exists between
		the integration operation $\int \tt{<f>}\,dx$
		and the differentiation operation $\frac{d}{dx}[\tt{<f>}]$.
		%
		A priori,
		there is no reason to suspect that integrals would be related to derivatives.
		The integral corresponds to the computation of an area,
		whereas the derivative operation computes the slope of a function.
		Yet behold:

		\begin{shadetheorem}[fundamental theorem of calculus]
			Let $f(x)$  be a continuous function,
			and let $a \in \mathbb{R}$ be a constant.
			Define the function $F_a(x)$ as follows:
			\[
				F_a(x)  \; \eqdef \; A_f(a, x) \; = \; \int_a^x f(u) \: du.
			\]
			Then,
			the derivative of $F_a(x)$ with respect to $x$ is equal to $f(x)$:
			\[
				\frac{d}{dx}\!\big[F_a(x)\big] \; = \; f(x).
			\]
		\end{shadetheorem}

\noindent
In words,
the FTC tells us that ...
	
		In order to understand the inverse relationship between integration and differentiation,
		we can draw an analogy with the inverse relationship between a function $f$ and its inverse function $f^{-1}$,
		which \emph{undoes} the effects of $f$.
		See Figure~\ref{fig:functions-inverse} on page~\pageref{fig:functions-inverse}.
		Given some initial value $x$,
		if we apply the function $f$ to obtain the number $f(x)$,
		and apply the inverse function $f^{-1}$ on the number $f(x)$,
		then the result will be the initial value $x$ we started from:
		\[
			f^{-1}\!\left( f(x) \right)	=	x.
		\]
		Similarly,
		\textbf{the derivative operation is the inverse of the integral operation}.
		If you perform the integral operation $\int \tt{<f>}\,dx$
		followed by the derivative operation $\frac{d}{dx}[\tt{<f>}]$ on any function $\tt{<f>}$,
		you'll get back to original function:
		\[
			\frac{d}{dx} \int_c^x f(u)\:du = f(x).
		\]

		\noindent
		We can verify this using SymPy
		by starting with some function $\tt{f} = f(x) = x^2$,
		computing its integral function $\tt{F}$,
		then using \tt{sp.integrate} then take the derivative of $\tt{F}$:

		\begin{codeblock}[]
		>>> f = x**2
		>>> F = integrate(f, x)
		>>> F
		x**3/3           # + C
		>>> diff(F, x)
		x**2
		\end{codeblock}



		% DIRECTION 2
		For ordinary math functions,
		we know that if the function $f^{-1}$ is the undo action for the function $f$,
		then $f$ is also the undo action for $f^{-1}$:
		$f\!\left( f^{-1}(y) \right) = y$.
		Similarly,
		the inverse relationship between integrals and derivative holds in the other direction too.
		%	Going back to the analogy with ordinary math functions,
		% if we apply $f^{-1}$ followed by $f$,
		% we get back the original number we started from:
		\textbf{The integral operation is the inverse operator of the derivative operation}.
		If we start with some function $G(x)$,
		calculate the derivative function $G^{\prime\!}(x)$,
		then compute the integral of the derivative function $G^{\prime\!}(x)$,
		we arrive back at the original function $G(x)$
		(up to an additive constant):
		\[
			G(x) = \int_c^x G^{\prime\!}(u)\;du = G(x) + C.
		\]
		% The constant $C$ 
		%  choice of the starting point for integration

		\noindent
		Let's use SymPy to verify the fundamental theorem of calculus.
		Recall that the function \tt{sp.diff} computes derivative functions
		and \tt{sp.integrate} computes integral functions.
		To verify $\int_c^x G^{\prime\!}(u)\;du = G(x) + C$,
		we start with some expression \tt{G} for the function $G(x) = x^3$.

		\begin{codeblock}[]
		>>> G = x**3
		>>> dGdx = diff(F, x)
		>>> dGdx
		3*x**2
		>>> integrate(dGdx, x)
		x**3    # + C
		\end{codeblock}





%(A)
Define the anti-derivative function $F(x)$,
which is a function whose derivative equals $f(x)$:
$F^{\prime\!}(x) = f(x)$.

Note the anti-derivative function is not unique;
it is only defined up to an additive constant $F(x) + C$.

For example...




		\subsubsection{Computing integrals using anti-derivatives}
		% Using integrals by reverse engineering derivatives}
  
	  		% ALT. by anti-differentiation
			The fundamental theorem of calculus
			gives us a way for computing integrals functions
			using ``reverse engineering'' thinking
			and the table of derivative formulas
			(see page~\pageref{mathematical_preliminiaries:derivative_formulas}).
			To find an integral function of $f(x)$,
			we can look for a function $F(x)$
			whose derivative is the the function $f(x)$.
			% MAYBE: introduce noton of anti-derivatives
			If we can find a function $F(x)$ such that $F^{\prime\!}(x) = f(x)$,
			then we know the integral function of $f(x)$ is $F_a(x)$ 
			\[
				\int_a^b  f(x)\,dx = F_a(b) + C.
			\]
% MERGE
%The fundamental theorem of calculus gives us a symbolic trick for computing
%the integrals functions $\int f(x)\,dx$ for many simple functions.
%If you can find a function $F(x)$ such that $F^{\prime\!}(x) = f(x)$,
%then 
	
			\paragraph{Example 3 continued}
	
				Suppose you're given a function $h(x) = 4 - x^2$
				and asked to find its integral function $H_0(b) = \int_0^b  h(x)\, dx$.
				This fundamental theorem of calculus tells us this problem
				is equivalent to finding a function $H(x)$ whose derivative is $h(x)$.
				% $H^{\prime\!}(x) = h(x)$.
				%	For example,
				%	suppose you want to find the indefinite integral $\int x^2\,dx$.
				%	We can rephrase this problem as the search for some function $H(x)$ such that $H^{\prime\!}(x) = x^2$.
				The function $h(x) = 4 - x^2$ has two terms.
				The first term is a constant $4$.
				We can guess that the corresponding term in the anti-derivative function $H(x)$
				will be $4x$,
				since $\frac{d}{dx}[ 4x ] = 4$.
				Now for the quadratic term $-x^2$.
				Remembering the derivative formulas for polynomials,
				we can guess that anti-derivative $H(x)$ must contain a $x^3$ term,
				because taking the derivative of a cubic term results in a quadratic term.
				Therefore,
				the anti-derivative function we're looking for has the form $H(x)=3x - cx^3$,
				for some constant $c$.
				Pick the constant $c$ that makes this equation true: $H^{\prime\!}(x) = 4 - 3cx^2 = 4 - x^2$.
				Solving $3c=1$,
				we find $c=\frac{1}{3}$ and so the anti-derivative function
				we're looking for is $H(x) = 4x - \frac{1}{3}x^3 + C$.
				% We can now compute the integral $\int x^2 \:dx = 
%	In other words,
%	the area under the graph of $f(x)=x^2$ is described by the family of functions $F(x) = \frac{1}{3}x^3 + C$.
	% The constant $C$ varies depending on the choice of lower limit of integration $a$ for the area calculation $A_f(a, x)$.
	% You can verify that $\frac{d}{dx}\left[\frac{1}{3}x^3 + C\right] = x^2$.
	


		%	The integral is the ``inverse operation'' of the derivative.
		%	If you perform the integral operation followed by the derivative operation on some function, 
		%	you'll obtain the same function:
		%	Note we need a new variable $u$ inside the integral since $x$ is already 
		%	used to denote the upper limit of integration.

		% That's kind of cool, no?
		% TODO: mention application integration by finding anti-derivative functions
		% TODO: explain what it means for integral not to have closed form


		\subsubsection{Using derivative formulas in reverse}
		
			This procedure based on using the derivative formulas
			in reverse to guess the value of $F(x)$
			is very useful.
			We can use it for all the function listed in the table of derivative formulas.
			For example,
			the table tells us that the derivative of the linear function $f(x) = mx+b$
			is the constant function $f^{\prime\!}(x) = m$,
			which means the integral of a constant function is a linear function
			$\int m\,dx = mx + C$.
			The integral function of an exponential is also an exponential $\int e^x \,dx = e^x + C$,
			since $\frac{d}{dx}[e^x] = e^x$.
			The derivative of $\log_e(x)$ is $\frac{1}{x}$,
			therefore the integral of $\frac{1}{x}$ is $\log(x)$.
			Similarly for the trigonometric functions $\int \cos(x) \,dx = \sin(x)$
			and $\int - \sin(x) \,dx = \cos(x)$.
			For economy of space,
			we'll verify all these integral formulas by computing
			the integral of the function $f(x) = m + e^x + \frac{1}{x} + \cos(x) - \sin(x)$
			that contain the mix of several functions on the right side of the table
			of the derivative formulas table.

			\begin{codeblock}[]
			>>> fx = m + sp.exp(x) + 1/x + sp.cos(x) - sp.sin(x)
			>>> sp.integrate(fx, x)
			m*x + exp(x) + log(x) + sin(x) + cos(x)
			\end{codeblock}
			
			\noindent
			SymPy tells us the integral function
			$F_0$ is $F_0(x) = mx + e^x + \log(x) + \sin(x) + \cos(x)$,
			% $F_0(b) = mb + e^b + log(b) + sin(b) + cos(b)$,
			which are all the corresponding terms on the left side of the table of derivative fromulas.



%LEAD OUT: what do we do when there is no simple formula?
\bigskip
Okay,
but what do we do if the function we want to integrate
doesn't appear on the right side of the table?




		\ifthenelse{\boolean{FORSTATSBOOK}}{

			\noindent
			In probability theory,
			the FTC tells us that the probability density can be obtained from the cumulative distribution using differentiation
			\[
				f_X(x) = \frac{d}{dx}\!\left[ F_X(x) \right] = \frac{dF_X}{dx}(x) = F'_X(x).
			\]
			The fact that we can obtain $f_X$ from $F_X$ and vice versa,
			means we only need to define one of the two functions,
			and obtain the other function using differentiation or integration.
			In this book,
			we define the random variable $X$ through its probability distribution function $f_X$,
			then define $F_X$ as the integral of $f_X$.
			In other books,
			you might see the random variable $X$ being defined through its cumulative distribution function $F_X$,
			with its probability density function $f_X$ defined as the derivative of $F_X$.

		}{}



% FTC CUT MATERIAL

%	The integral function $F_a(x)$ is obtained from the original function $f(x)$ using integration,
%	$F_a(x) = \int_a^x f(u) du$.
%	Another way to describe this is to say we \emph{applied} the integration operator $\int \cdot dx$
%	on the function $f(x)$ to obtain the integral function $F_a(x)$.
%	The derivative function $f^{\prime\!}(x)$ is defined by the formula
%	$f^{\prime\!}(x) = \lim_{\delta \to 0} \frac{f(x+\delta)\ - \ f(x)}{\delta}$.
%	We can also say we \emph{applied} the derivative operator $\frac{d}{dx}[\cdot]$
%	to the function $f(x)$ to obtain the derivative function $f^{\prime\!}(x)$.
%	I use the word ``operator'' here to refer to an operation that acts on functions.


		% The fundamental theorem of calculus describes the relationship between
		% derivatives and integrals.
		%	The fundamental theorem of calculus establishes an equivalence between the set 
		%	of integral functions and the set of antiderivative functions:
		%	\[
		%		A_a(x)=F(x)+C.
		%	\]
		%	All integral functions $F_a(x)$ are antiderivatives of $f(x)$.


		%	There is no reason \emph{a priori} to think that integration and differentiation might be related:
		%	the former is a calculation about areas,
		%	while the latter is a calculation about slopes.
		%	The fundamental theorem of calculus reveals that they are in fact inverse operations:
		%	we can obtain the original function $f(x)$ from the integral function $F_a(x)$ by computing it's derivative:
		%	\[
		%		\frac{d}{dx}\big[F_a(x)\big] 	= 	\frac{d}{dx}\left[\int_a^x f(u) \; du \right]  	= 	f(x).
		%	\]
		%	Note we used a temporary variable $u$ as the integration variable,
		%	since $x$ is already used to denote the upper limit of integration.


%		if you compute the derivative of a function followed by the integral,
%		you will obtain the original function $f(x)$ (up to a constant):
%		\[
%		  \left( \int dx \circ \frac{d}{dx}\right) f(x) = \int_c^x f^{\prime\!}(u)\;du = f(x) + C.
%		\]
