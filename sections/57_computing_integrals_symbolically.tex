%!TEX root = ../calculus_tutorial.tex


	\subsection{Computing integrals functions symbolically using SymPy}

		We can use Python to do \emph{symbolic} integration using variables (symbols) instead of numbers.
		The SymPy function \tt{sp.integrate} allows us to obtain the formulas for integrals and integral functions.
		We'll now revisit the integral calculations from the three examples
		using symbolic math calculations.
		Before we can begin,
		we must define symbolic symbols variables \tt{x}, \tt{a}, and \tt{b},
		which we'll use to express the function $f$, $g$, and $h$
		and the limits of integration.

		\begin{codeblock}[define-symbols-xabcm]
		>>> import sympy as sp
		>>> x, a, b = sp.symbols("x a b")
		\end{codeblock}

		
		\subsubsection{Example 1S: Constant function}
	
			Consider the constant function $f(x) = 3$,
			which we can define as follows:
	
			\begin{codeblock}[sympy-define-fx]
			>>> fx = 3
			>>> fx
			3
			\end{codeblock}
	
			\noindent
			To compute the integral $\int_a^b f(x) dx$,
			we call the SymPy function \tt{sp.integrate},
			passing in the function as the first argument,
			and the triple $\tt{(x,a,b)}$ as the second argument,
			where $x$ specifies the variable of integration,
			and $a$ and $b$ are the limits of integration:

			\begin{codeblock}[sympy-integrate-fx-a-b]
			>>> sp.integrate(fx, (x,a,b))  # = A_f(a,b)
			3*(b-a)
			\end{codeblock}

			\noindent
			Since $a$ and $b$ are arbitrary constants,
			the answer we obtain for $A_f(a,b) = \int_a^b f(x) dx$
			is a general-purpose formula
			that works for all possible intervals of integration $[a,b]$.
			Geometrically,
			we recognize the result as the height-times-width formula
			for the area of a rectangle,
			which we have seen several times already.
	
			To compute the definite integral between $a=0$ and $b=5$,
			we specify the numerical limits of integration instead of the symbols \tt{a} and \tt{b}.
	
			\begin{codeblock}[integrate-fx-subs-vals]
			>>> sp.integrate(fx, (x,0,5))
			15
			\end{codeblock}
	
			\noindent
			This result matches the value we obtained using
			geometrical calculation in Figure~\ref{fig:simple_integral_fx_eq_3},
			and the approximation we obtained using numerical integration \tt{quad(f,0,5)}.
	
			We can also compute the integral function $F_0(b)$,
			which is defined as $F_0(b) \eqdef \int_0^b f(x) dx$,
			for the function $f(x) = \tt{fx}$.
	
			\begin{codeblock}[sympy-integral-function-F]
			>>> F0b = sp.integrate(fx, (x,0,b))  # = F_0(b)
			>>> F0b
			3*b
			\end{codeblock}

			\noindent	
			Recall that the integral function $F_0$ is the area-under-the-graph calculation
			with a variable upper limit of integration $b$.
			See Figure~\ref{fig:simple_integral_function_fx_eq_3} for an illustration of the integral function $F_0(b)$.
			
			Given $F_0(b)$, %  = \tt{F0b}$,
			we can compute the definite integral between $a=0$ and $b=5$
			using the formula $\int_0^5 f(x) dx = F_0(5) - F_0(0)$.
			We'll need to used the method \tt{subs} (short for substitute)
			on the expression \tt{F0b} to ``plug in'' the values $b=5$ and $b=0$.

			\begin{codeblock}[]
			>>> F0b.subs({b:5}) - F0b.subs({b:0})
			15
			\end{codeblock}

			\noindent
			The \tt{subs} method expects as inputs a Python dictionary whose keys are symbols,
			and whose values represent the numbers we want to plug into the expression.













		\subsubsection{Example 2S: Linear function}
	
			Let's now compute the integral function of the linear function $g(x) = x$,
			which corresponds to the following SymPy expression:
	
			\begin{codeblock}[sympy-define-gx]
			>>> gx = 1*x
			>>> gx
			x
			\end{codeblock}
	
			\noindent
			To compute the integral function $G_0(b) \eqdef \int_0^b g(x) dx$,
			we call \tt{sp.integrate} using the symbol \tt{b} for the upper limit of integration:
	
			\begin{codeblock}[sympy-integral-function-G]
			>>> G0b = sp.integrate(gx, (x,0,b))  # = G_0(b)
			>>> G0b
			b**2 / 2
			\end{codeblock}
	
			\noindent
			The expression $G_0(b) = \frac{1}{2}b^2$ we obtain
			is identical to the formula we obtained
			from the geometric calculation in Figure~\ref{fig:simple_integral_function_gx_eq_x}.
			% = the area of a triangle with base $b$ and height $b$

			\noindent
			Given $G_0(b)= \tt{G0b}$,
			we can compute the definite integral $\int_0^5 g(x)\,dx$
			using the formula $\int_0^5 g(x)\,dx = G_0(5) - G_0(0)$.
			We plug in $b=5$ and $b=0$ using the \tt{subs} method:
	
			\begin{codeblock}[integrate-gx-subs-vals]
			>>> G0b.subs({b:5}) - G0b.subs({b:0})
			25/2
			\end{codeblock}

			\noindent
			SymPy computed the exact answer for us as a fraction $\frac{25}{2}$.
			This answer matches the value we obtained earlier
			using numerical integration, $\tt{quad(g,0,5)[0]} = \tt{12.5}$.

			%	We sometimes want to force the answer to be computed as a floating-point number (a Python \tt{float}),
			%	which we can do by calling the \tt{.evalf()} method on the SymPy expression.				
			%	\begin{codeblock}[integrate-gx-subs-vals-evalf]
			%	>>> sp.integrate(gx, (x,a,b)).subs({a:0, b:5}).evalf()
			%	12.5
			%	\end{codeblock}
	












		\subsubsection{Example 3S: Polynomial function}

			Define $h(x) = 4 -x^2$ 

			\begin{codeblock}[sympy-define-hx]
			>>> hx = 4 - x**2
			>>> hx
			4 - x**2
			\end{codeblock}
		
			\noindent
			The 
			
			\begin{codeblock}[]
			>>> H0 = sp.integrate(hx, (x,0,b))
			>>> H0
			4*b - b**3/3
			\end{codeblock}
			
			\noindent
			The integral function $H_{0}(b) = 4b - \frac{1}{3}b^3$
			corresponds to the area calculation under $h(x)$
			starting at $x=0$,
			which only covers the right side of the region.




	
% TODO: sync with first warning in previous section earlier  + reinforce idea
%		Unfortunately,
%		it's not always possible to use symbolic manipulations to find integrals.
%		We can only use \tt{sympy.integrate} for certain simple examples
%		where it is possible to obtain exact expressions for integral functions.
%		For most practical calculations in probability and statistics,
%		we'll need to rely on the \tt{scipy.integrate} function \tt{quad(f,a,b)},
%		which computes the integral $\int_a^b f(x)dx$ for \emph{any} function $f(x)$ expressed as a Python function \tt{f}.











% SYMPY INTEGRALS CUT MATERIAL 

%$F(x) = \int_0^x f(u)\,du$.

%\begin{codeblock}[]
%>>> integrate(x**3, x)
%x**4/4
%>>> integrate(sin(x), x)
%-cos(x)
%>>> integrate(ln(x), x)
%x*log(x) - x
%\end{codeblock}

%	In contrast, 
%	a \emph{definite integral} computes the area under $f(x)$ between $x=a$ and $x=b$.
%	Use \texttt{integrate(f, (x,a,b))} to compute the definite integrals of the form $A_f(a,b)=\int_a^b f(x) \, dx$:
%	
%	\begin{codeblock}[]
%	>>> integrate(x**3, (x,0,1))    
%	1/4              # the area under x^3 from x=0 to x=1
%	\end{codeblock}
%	
%	\noindent
%	We can obtain the same area by first calculating the indefinite integral $F(c)=\int_0^c \!f(x)\,dx$,
%	then using $A_f(a,b) = F(x)\big\vert_a^b = F(b) - F(a)$:
%	
%	\begin{codeblock}[]
%	>>> F = integrate(x**3, x)
%	>>> F.subs({x:1}) - F.subs({x:0})   
%	1/4
%	\end{codeblock}