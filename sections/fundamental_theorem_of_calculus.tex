%!TEX root = ../calculus_tutorial.tex

\subsection{Fundamental theorem of calculus}

		A priori,
		there is no reason to suspect the integral function would be related to the derivative operation.
		The integral corresponds to the computation of an area,												\index{integral}
		whereas the derivative operation computes the slope of a function.
		The fundamental theorem of calculus describes the relationship between 									\index{fundamental theorem of calculus|textit}
		derivatives and integrals.

			\begin{shadetheorem}[fundamental theorem of calculus]
			Let $f(x)$  be a continuous function on the interval $[a,b]$,										\index{continuous function}
			and let $\alpha \in \mathbb{R}$ be a constant.
			Define the function $A_\alpha(x)$ as follows:
			\[
			  A_\alpha(x)  \equiv A(\alpha, x) =  \int_\alpha^x f(u) \: du.
			\]
			Then, the derivative of $A_\alpha(x)$ with respect to $x$ is equal to $f(x)$:
			\[
			  \frac{d}{dx}\!\big[A_\alpha(x)\big] = f(x),
			\]
			for any $x \in (a,b)$.
			\end{shadetheorem}
	
		\noindent
		The fundamental theorem of calculus establishes an equivalence between the set 
		of integral functions and the set of antiderivative functions:											\index{antiderivative}
		\[
			A_\alpha(x)=F(x)+C.
		\]
		All integral functions $A_\alpha(x)$ are antiderivatives of $f(x)$.
	
		Differential calculus and integral calculus are two sides of the same coin.
		If you understand why the theorem is true, 
		you will understand something very deep about calculus. 
		Differentiation is the inverse operation of integration.

		\subsubsection{Integration and differentiation as inverse operations}
						
			You previously studied the inverse relationship for functions.
			Recall that for any \emph{bijective} function $f$
			(a one-to-one relationship) there exists an \emph{inverse function}
			$f^{-1}$ that \emph{undoes} the effects of $f$:
			\[
			  (f^{-1}\!\circ f) (x) \equiv f^{-1}(f(x))  = 1x
			\]
			and also
			\[
			 (f \circ f^{-1}) (y)  \equiv f(f^{-1}(y))  = 1y.
			\]
			The integral is the ``inverse operation'' of the derivative.
			If you perform the integral operation followed by the derivative operation on some function, 
			you'll obtain the same function:
			\[
				\left(\frac{d}{dx} \circ \int dx \right) f(x) \equiv \frac{d}{dx} \int_c^x f(u)\:du = f(x).
			\]
			Note we need a new variable $u$ inside the integral since $x$ is already 
			used to denote the upper limit of integration.
			
			Alternately, if you compute the derivative followed by the integral,
			you will obtain the original function $f(x)$ (up to a constant):
			\[
				\left( \int dx \circ \frac{d}{dx}\right) f(x) \equiv \int_c^x f'(u)\;du = f(x) + C.
			\]

			\bigskip
			\noindent
			TODO: power sentence here to summarize FTC
			\vspace{1cm}



		\subsubsection{Computing integrals using the fundamental theorem of calculus}
		\label{mathematical_preliminiaries:computing_integrals_2}

			The fundamental theorem of calculus gives us an alternative way for computing integrals.
			You can find integral functions using a table of derivative formulas (see page~\pageref{mathematical_preliminiaries:derivative_formulas})
			and some ``reverse engineering'' thinking.
			To find an integral function of the function $f(x)$,
			we can look for a function function $F(x)$ such that $F'(x)=f(x)$.

			\paragraph{Example}
				Suppose you're given a function $f(x)$ and asked to find its integral function $F(x) = \int  f(x)\: dx$.
				This fundamental theorem of calculus tells us this problem is equivalent to finding a function $F(x)$ whose derivative is $f(x)$: $F'(x) = f(x)$.
				For example, suppose you want to find the indefinite integral $\int x^2\:dx$.
				We can rephrase this problem as the search for some function $F(x)$ such that $F'(x) = x^2$.
				Remembering the derivative formulas we saw above, you guess that $F(x)$ must contain an $x^3$ term.
				Taking the derivative of a cubic term results in a quadratic term.
				Therefore, the function you are looking for has the form $F(x)=cx^3$,
				for some constant $c$.
				Pick the constant $c$ that makes this equation true: $F'(x) = 3cx^2 = x^2$.
				Solving $3c=1$, we find $c=\frac{1}{3}$ and so the integral function is $F(x) = \int x^2 \:dx = \frac{1}{3}x^3 + C$.
				In other words,
				the area under the graph of $f(x)=x^2$ is described by the family of functions $F(x) = \frac{1}{3}x^3 + C$.
				% The constant $C$ varies depending on the choice of lower limit of integration $\alpha$ for the area calculation $A(\alpha, x)$.
				% You can verify that $\frac{d}{dx}\left[\frac{1}{3}x^3 + C\right] = x^2$.


