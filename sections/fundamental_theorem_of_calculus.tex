%!TEX root = ../calculus_tutorial.tex



	\subsection{Fundamental theorem of calculus}
	
		The fundamental theorem of calculus (FTC) is a deep insight about the
		inverse relation that exists between the operations of integration $\int \cdot dx$
		and differentiation $\frac{d}{dx}[\cdot]$.

		%	The integral function $F_a(x)$ is obtained from the original function $f(x)$ using integration,
		%	$F_a(x) = \int_a^x f(u) du$.
		%	Another way to describe this is to say we \emph{applied} the integration operator $\int \cdot dx$
		%	on the function $f(x)$ to obtain the integral function $F_a(x)$.
		%	The derivative function $f'(x)$ is defined by the formula $f'(x) = \lim_{\delta \to 0} \frac{f(x+\delta)\ - \ f(x)}{\delta}$.
		%	We can also say we \emph{applied} the derivative operator $\frac{d}{dx}[\cdot]$
		%	to the function $f(x)$ to obtain the derivative function $f'(x)$.
		%	I use the word ``operator'' here to refer to an operation that acts on functions.

		A priori,
		there is no reason to suspect the integral function would be related to the derivative operation.
		The integral corresponds to the computation of an area,
		whereas the derivative operation computes the slope of a function.
		The fundamental theorem of calculus describes the relationship between
		derivatives and integrals.

		\begin{shadetheorem}[fundamental theorem of calculus]
			Let $f(x)$  be a continuous function on the interval $[a,b]$,										\index{continuous function}
			and let $\alpha \in \mathbb{R}$ be a constant.
			Define the function $A_\alpha(x)$ as follows:
			\[
			  A_\alpha(x)  \equiv A(\alpha, x) =  \int_\alpha^x f(u) \: du.
			\]
			Then, the derivative of $A_\alpha(x)$ with respect to $x$ is equal to $f(x)$:
			\[
			  \frac{d}{dx}\!\big[A_\alpha(x)\big] = f(x),
			\]
			for any $x \in (a,b)$.
		\end{shadetheorem}

		%	The fundamental theorem of calculus establishes an equivalence between the set 
		%	of integral functions and the set of antiderivative functions:
		%	\[
		%		A_\alpha(x)=F(x)+C.
		%	\]
		%	All integral functions $A_\alpha(x)$ are antiderivatives of $f(x)$.
	
		Differential calculus and integral calculus are two sides of the same coin.
		If you understand why the theorem is true, 
		you will understand something very deep about calculus. 
		Differentiation is the inverse operation of integration.


		%	There is no reason \emph{a priori} to think that integration and differentiation might be related:
		%	the former is a calculation about areas,
		%	while the latter is a calculation about slopes.
		%	The fundamental theorem of calculus reveals that they are in fact inverse operations:
		%	we can obtain the original function $f(x)$ from the integral function $F_a(x)$ by computing it's derivative:
		%	\[
		%		\frac{d}{dx}\big[F_a(x)\big] 	= 	\frac{d}{dx}\left[\int_a^x f(u) \; du \right]  	= 	f(x).
		%	\]
		%	Note we used a temporary variable $u$ as the integration variable,
		%	since $x$ is already used to denote the upper limit of integration.

		In order to understand the inverse relationship between integration and differentiation,
		we can draw an analogy with the inverse relationship between a function $f$ and its inverse function $f^{-1}$,
		which \emph{undoes} the effects of $f$.
		See Figure~\ref{fig:functions-inverse} on page~\pageref{fig:functions-inverse}.
		Given some initial value $x$,
		if we apply the function $f$ to obtain the number $f(x)$,
		and apply the inverse function $f^{-1}$ on the number $f(x)$,
		then the result will be the initial value $x$ we started from:
		\[
			f^{-1}\!\left( f(x) \right)	=	x.
		\]
		Similarly,
		the derivative operator is the ``inverse operator'' of the integral operator.
		If you perform the integral operation followed by the derivative operation on some function,
		you'll get back to original function:
		\[
			\frac{d}{dx} \int_c^x f(u)\:du = f(x).
		\]
		We can use SymPy to verify the fundamental theorem of calculus.
		First we construct a function \tt{f} and compute its integral function \tt{F} using \tt{integrate}:

		\begin{codeblock}[sympy-FTC-obtain-F]
		>>> from sympy import diff, integrate, log, exp, sin
		>>> f =  log(x) + exp(x) + sin(x)
		>>> F = integrate(f)
		>>> F
		x*log(x) - x + exp(x) - cos(x)
		\end{codeblock}

		If we now take the derivative of the function \tt{F},
		we get back the original function \tt{f}.

		\begin{codeblock}[sympy-FTC-get-back-f]
		>>> diff(F)
		log(x) + exp(x) + sin(x)
		>>> diff(integrate(f)) == f  # FTC part 1
		True
		\end{codeblock}




The integral is the ``inverse operation'' of the derivative.			\index{inverse!operation}
If you perform the integral operation followed by the derivative operation on some function,
you'll obtain the same function:
\[
  \left(\frac{d}{dx} \circ \int dx \right) f(x) = \frac{d}{dx} \int_c^x f(u)\:du = f(x).
\]



\begin{codeblock}[]
>>> f = x**2
>>> F = integrate(f, x)
>>> F
x**3/3           # + C
>>> diff(F, x)
x**2
\end{codeblock}

\noindent
Alternately, if you compute the derivative of a function followed by the integral,
you will obtain the original function $f(x)$ (up to a constant):
\[
  \left( \int dx \circ \frac{d}{dx}\right) f(x) = \int_c^x f'(u)\;du = f(x) + C.
\]



\begin{codeblock}[]
>>> f = x**2
>>> df = diff(f, x)
>>> df
2*x
>>> integrate(df, x)
x**2    # + C
\end{codeblock}




			


		\noindent
		The inverse relationship also holds for the opposite order of application:
		if we take the derivative of some function,
		then compute the integral of the derivative,
		then we arrive back at the original function (up to an additive constant factor).  % due to the arbitrary choice of the starting point for integration
		\[
			\int_c^x f'(u)\;du = f(x) + C.
		\]


		\begin{codeblock}[sympy-FTC-part-2]
		>>> integrate(diff(f)) == f  # FTC part 2
		True
		\end{codeblock}

		%	The integral is the ``inverse operation'' of the derivative.
		%	If you perform the integral operation followed by the derivative operation on some function, 
		%	you'll obtain the same function:
		%	Note we need a new variable $u$ inside the integral since $x$ is already 
		%	used to denote the upper limit of integration.

		% That's kind of cool, no?
		% TODO: mention application integration by finding anti-derivative functions
		% TODO: explain what it means for integral not to have closed form
		
		\ifthenelse{\boolean{FORSTATSBOOK}}{

			\noindent
			In probability theory,
			the FTC tells us that the probability density can be obtained from the cumulative distribution using differentiation
			\[
				f_X(x) = \frac{d}{dx}\!\left[ F_X(x) \right] = \frac{dF_X}{dx}(x) = F'_X(x).
			\]
			The fact that we can obtain $f_X$ from $F_X$ and vice versa,
			means we only need to define one of the two functions,
			and obtain the other function using differentiation or integration.
			In this book,
			we define the random variable $X$ through its probability distribution function $f_X$,
			then define $F_X$ as the integral of $f_X$.
			In other books,
			you might see the random variable $X$ being defined through its cumulative distribution function $F_X$,
			with its probability density function $f_X$ defined as the derivative of $F_X$.

		}{}


		% ALT. by anti-differentiation
		The fundamental theorem of calculus gives us an alternative way for computing integrals.
		You can find integral functions using a table of derivative formulas (see page~\pageref{mathematical_preliminiaries:derivative_formulas})
		and some ``reverse engineering'' thinking.
		To find an integral function of the function $f(x)$,
		we can look for a function function $F(x)$ such that $F'(x)=f(x)$.

		\paragraph{Example}

			Suppose you're given a function $f(x)$ and asked to find its integral function $F(x) = \int  f(x)\: dx$.
			This fundamental theorem of calculus tells us this problem is equivalent to finding a function $F(x)$ whose derivative is $f(x)$: $F'(x) = f(x)$.
			For example, suppose you want to find the indefinite integral $\int x^2\:dx$.
			We can rephrase this problem as the search for some function $F(x)$ such that $F'(x) = x^2$.
			Remembering the derivative formulas we saw above, you guess that $F(x)$ must contain an $x^3$ term.
			Taking the derivative of a cubic term results in a quadratic term.
			Therefore, the function you are looking for has the form $F(x)=cx^3$,
			for some constant $c$.
			Pick the constant $c$ that makes this equation true: $F'(x) = 3cx^2 = x^2$.
			Solving $3c=1$, we find $c=\frac{1}{3}$ and so the integral function is $F(x) = \int x^2 \:dx = \frac{1}{3}x^3 + C$.
			In other words,
			the area under the graph of $f(x)=x^2$ is described by the family of functions $F(x) = \frac{1}{3}x^3 + C$.
			% The constant $C$ varies depending on the choice of lower limit of integration $\alpha$ for the area calculation $A(\alpha, x)$.
			% You can verify that $\frac{d}{dx}\left[\frac{1}{3}x^3 + C\right] = x^2$.


LEAD OUT: what do we do when there is no simple formula?



