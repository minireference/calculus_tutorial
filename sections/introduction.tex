%!TEX root = ../calculus_tutorial.tex

\section{Inroduction}
\label{sec:introduction}

	In order to do calculations with continuous random variables,
	you need to know about the calculus procedure of \emph{integration},
	which is used to compute the ``total amount accumulated'' of some quantity described by a continuous function,
	over an interval of inputs.

	This section is your opportunity to learn (or relearn) the main concepts from calculus that you need to know
	to do calculations with continuous probability distributions.

% TUTORIAL
	We'll talk about concepts like sets, functions, and integrals,
	which are essential for understanding what's going on in the rest of the book.
	% These are the essential tools we'll need to do probability calculations.
	We'll start with a practical example in which integration is used.
	I want to show you that integration is not some fancy new idea,
	but an operation you're already familiar with from everyday life.

	\paragraph{Banking example}

		Consider the function $\textrm{ba}(t)$ that represents your bank account balance at time $t$.
		Also consider the function $\textrm{tr}(t)$,
		which corresponds to the transactions (deposits and withdrawals) on your account.

		Suppose you have a record of all the transactions on your account $\textrm{tr}(t)$,
		and you want to compute the final account balance at the end of the month $\textrm{ba}(30)$.
		You can use the integration procedure on the transactions $\textrm{tr}(t)$
		to calculate the total change in the account balance at the end of the month,
		relative to the account balance at the beginning of the month $\textrm{ba}(0)$.
		The end-of-the-month-balance calculation is described by the following equation:
		\[ 
			\textrm{ba}(30)		=	\textrm{ba}(0)	+	\int_0^{30} \textrm{tr}(t)\;dt.
		\]
		The integral $\int_0^{30} \textrm{tr}(t)dt$ describes the process of computing 
		the total of all the transactions that occurred between day $0$ and day $30$.
		The weird-looking integral sign ``$\int$'' comes from the Latin word \emph{summa} for sum.

		We use integrals every time we need to calculate the total of some quantity over a time period.
		The integral $\int_a^b q(t)dt$ is the calculation of the \emph{total}
		of some quantity $q(t)$ that accumulates during the time period from $t=a$ to $t=b$.

% Robyn said: 	Since this is in the intro, it makes it sound like integrals are only used to calculate a value over time. 
% 			We should make it clear that they can also be used to calculate probabilities
%			and other quantities over a range of values or outcomes.

		% TODO: mention accumulation can be done over any variable; from now on variables x, n, u, etc.

