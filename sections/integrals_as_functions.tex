%!TEX root = ../calculus_tutorial.tex


	\subsection{Integrals as functions}
	
		The \emph{integral function} $F_0(b)$ corresponds to the area calculation
		with a variable upper limit of integration $A_f(0,b)$.
		The variable $b$,
		which serves as the input for the integral function $F_0$,
		corresponds to the upper limit of integration in the following calculation:
		\[
			F_0(b) \;\; \eqdef \;\; A_f(0,b) = \int_{x=0}^{x=b} \! f(x)\:dx\,.
		\]
		There are two variables and one constant in this formula.
		The input variable $b$ describes the upper limit of integration.
		The \emph{integration variable} $x$ performs a sweep from $x=0$ until $x=b$.
		The constant $0$ describes the lower limit of integration.
		As a matter of convention,
		we'll always denote the integral function using the capital letter of the same letter as the original function.
	
		Note that choosing $x=0$ for the starting point of the integral function was an arbitrary choice,
		and we obtain another integral function if we use $x=a$ as the starting point,
		$F_a(b)=\int_a^b \! f(x)\,dx$.
		Two integral functions differ only by a constant term.
		For example,
		$F_0(b) = F_a(b) + C$,
		where $C = \int_{x=0}^{x=a} f(x)\,dx$.
	
		The integral function $F(b)$ contains the ``precomputed'' information about the area under the graph of $f(x)$.
		% The integral function $F(b)$ tells us the ``area under the graph'' property of the function $f(x)$ for \emph{all} possible limits of integration.
		The area under $f(x)$ between $x=a$ and $x=b$ can be obtained by calculating the \emph{change} in the integral function as follows:
		\[
			A_f(a,b) = \int_a^b \! f(x)\,dx	=  F(b)-F(a).
		\]
		Intuitively,
		this formula computes the area $A_f(a,b)$ as the difference between the areas of two regions:
		the area until $x=b$ minus the area until $x=a$,
		as illustrated in Figure~\ref{fig:integral_as_difference_off}.
	
		\begin{figure}[htb]
			\centering
			\includegraphics[width=0.4\textwidth]{figures/calculus/integral_as_difference_off.pdf}
			\caption{	The area under $f(x)$ between $x=a$ and $x=b$ is computed using the formula $A_f(a,b)=F_0(b)-F_0(a)$,
					which describes the change in the output of $F_0(x)$ between $x=a$ and $x=b$.}
			\label{fig:integral_as_difference_off}
		\end{figure}
	

TODO: warn there is no general F for any f
only for certain special cases
have exact symbolic formula
for all other cases
we're forced to do the split-into-vertical-strips --- i.e. there is no analytical shortcut.


	
		% TODO: explain intuition: area until b  minus area until a equals area between a and b
	
	
	
		\subsubsection{Example 1 revisited}
	
			We can easily find the integral function for the constant function $f(x)=~3$,
			because the region under the curve is rectangular.
			Choosing $x=0$ as the starting point,
			we obtain the integral function $F_0(b)$
			that corresponds to the area under $f(x)$ between $x=0$ and $x=b$
			as follows:
			\[ 
				F_0(b) = A_f(0,b) = \int_0^b \! f(x)\,dx	= 3 b.
			\]
			The area is equal to the rectangle's height times its width,
			as illustrated in Figure~\ref{fig:simple_integral_function_fx_eq_3}.
	
			\begin{figure}[htb]
				\centering
				\includegraphics[width=0.4\textwidth]{figures/calculus/simple_integral_function_fx_eq_3.pdf}
				\caption{The area of a rectangle of height $3$ and width $b$ is equal to $3b$.}
				\label{fig:simple_integral_function_fx_eq_3}
			\end{figure}
			
			Knowing the function $F_0(b)$ allows us to compute the area under the graph of $f(x)$
			between any two points $x=a$ and $x=b$ using the formula $A_f(a,b) = F_0(b) - F_0(a) = 3(b-a)$.
	
	
		\subsubsection{Example 2 revisited}
		
			Consider now the area under the graph of the line $g(x)=x$, starting from $x=0$.
			Since the region under the curve is triangular,
			we can compute its area using the formula for the area of a triangle:
			base times height divided by two.
		
			The general formula for the area under $g(x)$ from $x=0$ until $x=b$
			is described by the following integral calculation:
			\[
				G_0(b) = A_g(0,b) = \int_0^b g(x) \, dx = \tfrac{1}{2} ( b\times b ) = \tfrac{1}{2}b^2.
			\]
		
			\begin{figure}[htb]
				\centering
				\includegraphics[width=0.4\textwidth]{figures/calculus/simple_integral_function_gx_eq_x.pdf}
				\caption{The area of a triangle with base $b$ and height $b$ is equal to $\frac{1}{2}b^2$.}
				\label{fig:simple_integral_function_gx_eq_x}
			\end{figure}
			
			Knowing the function $G_0(b)$ allows us to compute the area under the graph of $g(x)$
			between $x=a$ and $x=b$ as the difference $A_g(a,b) = G_0(b) - G_0(a) = \frac{1}{2}b^2 - \frac{1}{2}a^2$.
	
	
		%	But don't worry,
		%	you don't need to take an integral calculus to learn statistics.
		%	What is important right now is that you understand the concept of integration.
		%	The integral of a function is the area under the graph of the function,
		%	which is in some sense the total amount of the function accumulated during some interval of time.
	
		\subsubsection{Example 3 revisited}
		
			We 
			\[
				H_{-1}(b) = A_h(-1,b) = \int_{-1}^b h(x) \, dx = ?
			\]
			
			one of the special cases where there IS a shotctu
			
			
		
	
	
	
\bigskip
\noindent
We were able to compute the above integrals thanks to the simple geometries of the areas under the graphs.
%	Computing integrals of more complicated functions requires more advanced techniques.
%	There is an entire course called integral calculus which is dedicated to the task of finding integrals
%	using various tricks and techniques.


%	Taking a calculus course would be useful if you plan to study physics or engineering,
%	but for the purpose of learning probability and statistics,
%	you're not required to learn all these integration techniques.
%	Instead,
%	you can rely on computers to do integration for you.
%	Specifically,
%	you can use the Python modules SciPy and SymPy to compute all the integrals you need,
%	as we'll show in the next two sections.

% Robyn said: 	Confusing that before you said that integrals is a topic that can't be skipped,
%			and here you say that you can rely on computers.
% 			If they don't need to know the math, then perhaps this section could be made even shorter.
