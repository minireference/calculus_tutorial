%!TEX root = ../calculus_tutorial.tex


	\subsection{Computing integrals using SciPy}
	\label{integrals:scipy}

		% The symbolic math techniques for calculating integrals
		The Python function \tt{integrate}
		that we showed in Section~\ref{integrals:computing_integrals_numerically}
		is a useful teaching tool,
		but it would be much too slow to use for practical scientific computing tasks.
		% that require computing integrals thousands of times.
		The function \tt{quad} from the \tt{scipy.integrate} module
		is a much more powerful tool for computing numerical integrals.
		The name \tt{quad} is short for ``quadrature,''
		which a historical name for the find-the-area-under-the-graph-of-a-function calculations.
		% performed by splitting it into small sections % (squares)
		% and adding up the total area of the sections.

		Let's revisit the examples 1N, 2N, and 3N using the function \tt{quad}.
		To compute the integral $\int_0^5 f(x) dx$,
		we call the function \tt{quad} with inputs \tt{f} as the first argument,
		and the limits of integration $a=0$ and $b=5$ as the second and third arguments.
	
		\begin{codeblock}[quad-f-0-5-tuple]
		>>> from scipy.integrate import quad
		>>> quad(f, 0, 5)
		(15.0, 1.1102230246251565e-13)
		\end{codeblock}
	
		\noindent
		The function \tt{quad} returns two numbers as outputs: $(\tt{area},\epsilon)$.
		The first number is the value of the area we're interested in.
		The second number $\epsilon$ tells us the accuracy of the procedure used to calculate the area.
		In the above calculation,
		the output tells us the definite integral $\int_0^5 f(x) dx$ is equal to $\tt{15.0}$
		up to a precision on the order of $10^{-13}$.
		%
		Since we're usually only interested in the $\tt{area}$ value
		and not the precision $\epsilon$,
		we often select the first half of \tt{quad}'s output.
		% which we can do using the expression \tt{quad(...)[0]}.
		% This is why you'll often see the expression \tt{quad(...)[0]} in code examples.
	
		\begin{codeblock}[quad-f-0-5]
		>>> quad(f, 0, 5)[0]
		15.0
		\end{codeblock}

		\noindent
		We can similarly use \tt{quad} to calculate the integrals $\int_{0}^5 g(x)\,dx$ and $\int_{0}^2 h(x)\,dx$
		from the other two examples.

		\begin{codeblock}[]
		>>> quad(g, 0, 5)[0]
		12.5
		>>> quad(h, 0, 2)[0]
		5.33333333333333333
		\end{codeblock}
	
		\noindent
		The answers we obtain match the results we obtained earlier.
		% the general formula we obtained above, $A_g(0,5) = \frac{1}{2}b^2$,
		% when the upper limit of integration is $b=5$.
		The main takeaway message
		is that the \tt{quad} function is your friend whenever you need to compute integrals.
		All the scary-looking math equations that contain the $\int$ symbol
		can be computed using one or two lines of Python code.
		Specifically,
		whenever you see $\int_a^b \tt{<f>}\,dx$ in a math formula,
		you can replace that with \tt{quad(f,a,b)[0]}.
	


		\ifthenelse{\boolean{FORSTATSBOOK}}{	
			We'll use the function \tt{quad} hundreds of times in the remainder of the book to compute various integrals
			as part of probability and statistics calculations,
			so make sure you understand what is going on in the above code examples.
		}{}
