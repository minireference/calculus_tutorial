%\documentclass[11pt,onecolumn,draftclsnofoot]{IEEEtran}
\documentclass[10pt]{IEEEtran}

\usepackage[T1]{fontenc}
\usepackage{lmodern}
\usepackage{amssymb,amsmath}
\usepackage{ifxetex,ifluatex}





\usepackage{fixltx2e} % provides \textsubscript
% use upquote if available, for straight quotes in verbatim environments
\IfFileExists{upquote.sty}{\usepackage{upquote}}{}
\ifnum 0\ifxetex 1\fi\ifluatex 1\fi=0 % if pdftex
  \usepackage[utf8]{inputenc}
\else % if luatex or xelatex
  \ifxetex
    \usepackage{mathspec}
    \usepackage{xltxtra,xunicode}
  \else
    \usepackage{fontspec}
  \fi
  \defaultfontfeatures{Mapping=tex-text,Scale=MatchLowercase}
  \newcommand{\euro}{€}
\fi
% use microtype if available
\IfFileExists{microtype.sty}{\usepackage{microtype}}{}
\ifxetex
  \usepackage[setpagesize=false, % page size defined by xetex
              unicode=false, % unicode breaks when used with xetex
              xetex]{hyperref}
\else
  \usepackage[unicode=true]{hyperref}
\fi
\hypersetup{breaklinks=true,
            bookmarks=true,
            pdfauthor={},
            pdftitle={Calculus tutorial},
            colorlinks=true,
            citecolor=blue,
            urlcolor=black,
            linkcolor=magenta,
            pdfborder={0 0 0}}
\urlstyle{same}  % don't use monospace font for urls
\setlength{\parindent}{0pt}
\setlength{\parskip}{6pt plus 2pt minus 1pt}
\setlength{\emergencystretch}{3em}  % prevent overfull lines
\setcounter{secnumdepth}{0}

\usepackage{etoolbox}



% NEW CODEBLOCK MACRO -- auto-numbered with labels in the margin
\usepackage{listings}
\usepackage{lstautogobble}
\usepackage{upquote}
\lstset{
	basicstyle=\footnotesize\ttfamily,
	language=Python,
	escapechar=¡,
	showstringspaces=false,
	commentstyle=\color{lightgray},
	literate={~}{\midtilde}{1},
	autogobble
}

%\newcounter{codeblock}
\makeatletter
\lstnewenvironment{codeblock2}[1][]{}{}
\makeatother







\title{{\Huge Calculus tutorial}}
%	A tale of with equations and code}
%\author{Ivan Savov}
\author{{\normalsize Tutorial based on the \href{http://minireference.com}{{\sc No bullshit guide}} series of textbooks by \href{mailto:ivan@minireference.com}{Ivan Savov}}}
\author{Excerpt from the \href{http://minireference.com/}{\textbf{No bullshit guide to math and physics}} by Ivan Savov} 
\date{\today}



%\usepackage{listings}
\usepackage{moreverb}
\usepackage[letterpaper,bmargin=1.3cm,rmargin=0.95cm,lmargin=0.95cm,tmargin=1.1cm,headsep=0.2cm,footskip=0.5cm]{geometry}

\usepackage{bbm}
\usepackage{wrapfig}
\usepackage{graphicx}

\usepackage{ifthen}
\usepackage{subfigure}						% Riemann sum will be better side by side.



\newcommand{\printcp}{}
\newcommand{\printni}{}

\setcounter{secnumdepth}{2}
\setcounter{tocdepth}{3}
\usepackage{setspace}


\newcommand*{\eqdef}{\stackrel{\raisebox{-2pt}{\scalebox{0.48}{def}}}{=}}		% "defined to be equal" symbol (prev. used \equiv; other common is := )
\newcommand{\emphindexdef}[1]{\emph{#1}\index{#1|textit}}		% Used for term definition --- emph and index entry in italics

\usepackage{xcolor}
\definecolor{light-gray}{gray}{0.75} % used for annotation arrows
\definecolor{dark-gray}{gray}{0.50}  % used for annotation text
% NEW CODEBLOCK MACRO -- auto-numbered with labels in the margin

\usepackage{lstautogobble}
\usepackage{upquote}
\lstset{
	basicstyle=\footnotesize\ttfamily,
	language=Python,
	escapechar=¡,
	showstringspaces=false,
	commentstyle=\color{lightgray},
	literate={~}{\midtilde}{1},
	autogobble
}

\newcounter{codeblock}
\makeatletter
\@ifundefined{section}{}{\numberwithin{codeblock}{section}}
\lstnewenvironment{codeblock}[1][]{%
	\refstepcounter{codeblock}%
	\label{#1}%
	% \marginpar{\vspace{1em}\linespread{0.9}\footnotesize code \\ \thecodeblock}%
}{}
\makeatother

% TMP macro while writing (to be expanded manually in final version)
\def\tt{\texttt}


\def\calX{\mathcal{X}}
\def\calU{\mathcal{U}}
\def\EE{\mathbb{E}}


\def\tt{\texttt}
\def\code{\texttt}	% OLD MACRO


\usepackage{shadethm}
\newshadetheorem{shadetheorem}{Theorem}
\renewcommand{\theshadeshadetheorem}{}			% turn off theorem numbering
\setlength\shadedtextwidth{0.49\textwidth}





\begin{document}

\makeatletter
\preto{\@verbatim}{\topsep=0pt \partopsep=0pt \vspace{-1.2mm}}
\makeatother





        \maketitle


\begin{abstract}
aaa
\end{abstract}

\begin{spacing}{-1}
\tableofcontents
\end{spacing}


% ========================================================================
%   NOTES AND CONVENTIONS
% ========================================================================
%
% Geometrical intuition for calculus conventions:
%	interval = subset of the domain of integration are called  (one dimensional)        not region
%	region = two-dimensional area under the graph of f(x)

\vspace{2em}


%!TEX root = ../calculus_tutorial.tex


\section{Introduction}
\label{sec:introduction}


\vspace{3in}



	\paragraph{Banking example}

		Consider the function $\textrm{ba}(t)$ that represents your bank account balance at time $t$.
		Also consider the function $\textrm{tr}(t)$,
		which corresponds to the transactions (deposits and withdrawals) on your account.

		Suppose you have a record of all the transactions on your account $\textrm{tr}(t)$,
		and you want to compute the final account balance at the end of the month $\textrm{ba}(30)$.
		You can use the integration procedure on the transactions $\textrm{tr}(t)$
		to calculate the total change in the account balance at the end of the month,
		relative to the account balance at the beginning of the month $\textrm{ba}(0)$.
		The end-of-the-month-balance calculation is described by the following equation:
		\[ 
			\textrm{ba}(30)		=	\textrm{ba}(0)	+	\int_0^{30} \textrm{tr}(t)\;dt.
		\]
		The integral $\int_0^{30} \textrm{tr}(t)dt$ describes the process of computing 
		the total of all the transactions that occurred between day $0$ and day $30$.
		The weird-looking integral sign ``$\int$'' comes from the Latin word \emph{summa} for sum.

		We use integrals every time we need to calculate the total of some quantity over a time period.
		The integral $\int_a^b q(t)dt$ is the calculation of the \emph{total}
		of some quantity $q(t)$ that accumulates during the time period from $t=a$ to $t=b$.

% Robyn said: 	Since this is in the intro, it makes it sound like integrals are only used to calculate a value over time. 
% 			We should make it clear that they can also be used to calculate probabilities
%			and other quantities over a range of values or outcomes.

		% TODO: mention accumulation can be done over any variable; from now on variables x, n, u, etc.



%!TEX root = ../calculus_tutorial.tex


\section{Definitions}

	Let's start by defining all the concepts from university-level math you need to know about.
	Don't worry if you're seeing some of these concepts for the first time,
	you'll see plenty of examples using these concepts,
	so you'll get to know them very well by the end of this section.

	\begin{itemize}

		\item \emph{set}: a collection of math objects.
			Sets are denoted using curly brackets $\{ \ldots \}$.
% Patrick said: you've already covered sets
			A set can be defined as a finite list of elements like $\{ \texttt{heads}, \texttt{tails} \}$,
			by specifying a pattern $\{ 0, 1, 2, 3, \ldots \}$,
			or through some other math expression $\{ \textrm{<def'n>} \}$.

		\item $f(x)$: a function of the form $f: \mathbb{R} \to \mathbb{R}$,
			which means $f$ takes real numbers as inputs and produces real numbers as outputs.
			Functions are usually defined through an analytical formula like $f(x) = x^2$,
			which tells us how to compute the output $f(x)$ for a given input $x$.
			Functions can also be represented visually as a function graph
			\includegraphics[width=2em]{figures/calculus/graph_of_f_nonnegative.pdf},
			which is a curve that passes through all the coordinates pairs $(x,f(x))$ in the Cartesian plane.

		\item	$A_f(a,b)$: the value of the \emph{area} under the graph of the function $f(x)$ from $x=a$ until $x=b$.
			The area $A_f(a,b)$ corresponds to the following integral
			\[
				A_f(a,b) \eqdef \int_a^b f(x)\;dx.
			\] 
			The $\int$ sign stands for \emph{sum}.
			Indeed,
			the integral is the ``sum'' of all the values of $f(x)$ for inputs $x$ between $x=a$ and $x=b$.

		\item	$F_0(b) \eqdef A_f(0,b)$: the \emph{integral function} of $f(x)$.
			The integral function corresponds to the computation of the area under $f(x)$
			as a function of the upper limit of integration:
			\[
				F(b) \eqdef A_f(0,b) = \int_{0}^{b}\! f(x)\:dx.
			\]
			The choice of $x=0$ as the lower limit of integration is arbitrary.
			We could define any number of other integral functions $F_a(b)$ for different starting points $x=a$.

	\end{itemize}


	\noindent
	In the next few pages,
	we'll go into some details about each of these math concepts.
	Don't be intimidated by all the fancy-looking math notation---it's just a bunch of language mathematicians
	invented in order to describe concepts precisely and concisely.
	It looks weird to everyone who sees this specialized math notation for the first time (a.k.a. alien symbols),
	but you'll quickly get used to it.





%!TEX root = ../calculus_tutorial.tex

\section{Sets and intervals}

	Sets are arbitrary collections of math objects.
	Many math ideas are expressed using the language of sets,
	so it's worth going over the basic definitions and notation conventions.

	\begin{itemize}

		\item $S,T$: the usual variable names for sets

		\item $s \in S$: this statement is read ``$s$ is an element of $S$'' or ``$s$ is in $S$''

		% seems not needed; put back if needed somewhere in the book (grep for it)
		%	\item $\emptyset$: the \emph{empty set} is a set that contains no elements.
		%		Mathematicians adopted the symbol $\emptyset$ because the notation $\{ \, \}$ is confusing.

		\item $\{ \textrm{  definition  } \}$: the curly brackets surround the definition of a set,
			and the expression inside the curly brackets describes what the set contains.
			% The math symbol ``$|$'' is shorthand for the phrase ``such that'' and it is often used in definitions.

		\item	$S^c$: the \emph{complement} of the set $S$,
			is defined as all elements that are not in the set $S$.
			% TODO: explain assumption of "universe" and complement as all elements of the universe that are not in $S$.

		\item $\mathbb{N}$: the set of natural numbers $\mathbb{N}\eqdef\{0,1,2,\ldots\}$

		\item $\mathbb{Z}$: the set of integers $\mathbb{Z}\eqdef\{\ldots,-2,-1,0,1,2,3,\ldots\}$

		\item $\mathbb{Q}$: the set of rational numbers,
			$\mathbb{Q} \eqdef \left\{  \frac{m}{n} \; \Big| \; m \in \mathbb{Z}, \; n \in \mathbb{N}, \; n \neq 0  \right\}$.
			The set $\mathbb{Q}$ consists of all numbers that can be expressed as \emph{fractions} of the form $\frac{m}{n}$,
			where $m$ is an integer, $n$ is a natural number, and $n \neq 0$.

		\item	$\mathbb{R}$: the set of real numbers

		\item	$\mathbb{R}_+$: the set of nonnegative real numbers.
			The definition of the nonnegative is written as
			$\mathbb{R}_+ \eqdef  \{ \text{all } x \text{ in } \mathbb{R} \text{ such that } x \geq 0 \}$,
			or it can be expressed more compactly as $\mathbb{R}_+ \eqdef \{ x \in \mathbb{R} \; | \; x \geq 0 \}$.

	\end{itemize}

	\noindent
	Note the multiple ways we use the  curly-brackets notation $\{ \}$ to denote sets.
	A \emph{finite set} is defined by simply listing all its elements.
	For example,
	the set of possible outcomes of a coin flip is $\{ \texttt{heads}, \texttt{tails} \}$.
	For an infinite set we can't write down all the elements,
	but we can show the pattern like $\mathbb{N} \eqdef \{0,1,2,3,4,\ldots\}$.
	The meaning of the three dots is ``and so on, continuing the same pattern.''
	Another way to define a set is to use the \emph{set-builder} notation $\{\, \cdot \; | \;  \cdot \, \}$.
	Inside the curly brackets we first describe the general kind of mathematical objects we are talking about,
	followed by the symbol ``$|$'' (read ``such that''),
	followed by the conditions that must be satisfied by all elements of the set.
	The definitions of the rational numbers $\mathbb{Q}$ and the nonnegative real numbers $\mathbb{R}_+$ above are examples of the set-builder notation.

	The \emph{number line} is a visual representation of the set of real numbers $\mathbb{R}$,
	as shown in Figure~\ref{fig:number_line_rationals_and_reals}.
	The real numbers correspond to all the points on the number line,
	from $-\infty$ to $\infty$.

	\begin{figure}[htb]
		\centering
		\includegraphics[width=0.4\textwidth]{figures/calculus/number_line_rationals_and_reals.pdf}
		\vspace{-2mm}
		\caption{The real numbers $\mathbb{R}$ cover the entire number line.}
		\label{fig:number_line_rationals_and_reals}
	\end{figure}

	\noindent
	The set of real numbers includes all the rational numbers like $-\frac{3}{2}$, $\frac{1}{2}$, and $\frac{9}{2}$,
	as well as irrational numbers like $\sqrt{2}$, $e$, and $\pi$.
	This means any number you are likely to run into when solving math problems
	can be visualized as a point on the number line.

	\subsection{Intervals}

		The number line can also be used to represent subsets of the real numbers,
		which we call \emph{intervals}.
		Figure~\ref{fig:interval_2closed_to_4closed} shows an illustration of the interval $[2,4] = \{ x \in \mathbb{R} \;|\; 2 \leq x \leq 4 \}$,
		which is a subset of the real numbers.

		Here are some more examples of various intervals:
		\begin{itemize}

			\item	$[a,b]$: the interval from $a$ to $b$.
				This corresponds to the set of real numbers between $a$ and~$b$,
				including the endpoints $a$ and~$b$.
				The interval $[a,b]$ corresponds to the set $\{ x\in \mathbb{R}\ | \ a \leq x \leq b \}$.

			\item	$[a,\infty)$: the interval from $a$ until infinity,
				which corresponds to the set $\{ x\in \mathbb{R}\ | \ a \leq x \}$.

			\item	$(-\infty,b]$: the interval from negative infinity until $b$,
				which corresponds to the set $\{ x\in \mathbb{R}\ | \ x \leq b \}$.

		\end{itemize}

		\noindent
		The notation $[a,b]$ describes the \emph{closed} interval from $a$ to $b$,
		which means the endpoints $a$ and $b$ are included in the interval.
		The notation $(a,b)$ describes the \emph{open} interval from $a$ to $b$,
		defined as the set $\{ x\in \mathbb{R}\ | \ a < x < b \}$,
		which doesn't include the endpoints $a$ and $b$.
		In other words,
		intervals defined using square brackets ``$[$'' include the endpoints (defined using less-than-or-equal conditions)
		while intervals defined with round brackets ``$($'' do not include their endpoints (defined using strictly-less-than conditions).
		The distinction between open and closed intervals is important in general,
		but makes no difference in the context of probability theory,
		so you don't need to worry about the difference between $[a,b]$ and $(a,b)$ in this book.

		% \SIDENOTE{ @Ivan: If not relevant to the book, why are we explaining it? }

		\begin{figure}[htb]
			\centering	
			\includegraphics[width=0.4\textwidth]{figures/calculus/interval_2closed_to_4closed.pdf}
			\vspace{-3mm}
			\caption{The interval $[2,4] \protect\eqdef \{ x \in \mathbb{R} \; | \; 2 \leq x \leq 4 \}$. }
			\label{fig:interval_2closed_to_4closed}
		\end{figure}




	\subsection{Set operations}

		We use set operations like union $\cup$, intersection $\cap$, and set difference~$\setminus$ to define composite sets.

		\begin{itemize}

			\item $S\cup T$: the \emph{union} of two sets.						\index{set!union|textit}
				The union of $S$ and $T$ corresponds to the elements in either $S$ or $T$.

			\item $S \cap T$: the \emph{intersection} of two sets.				\index{set!intersection|textit}
				The intersection of $S$ and $T$ corresponds to the elements that are in both $S$ and $T$.

			\item $S \, \setminus \, T$: \emph{set difference} or \emph{set minus}.		\index{set!difference|textit}
				The set difference $S \setminus T$ corresponds to the elements of $S$ that are not in $T$.

		\end{itemize}

		\noindent
		Consider the overlapping intervals $A = [a,b]$ and $B = [c,d]$ illustrated in Figure~\ref{fig:sets_A_B_and_set_operations}.
		The union of these two intervals is the set of numbers that are \emph{either} between $a$ and $b$ \emph{or} between $c$ and $d$,
		which corresponds to the interval $[a,d]$.
		The intersection of $A$ and $B$ is the set of numbers that are in \emph{both} $A$ and $B$,
		and corresponds to the interval $[c,b]$.
		The figure also illustrates the two set differences,
		$A\;\setminus\;B$ and $B\;\setminus\;A$ which correspond to numbers that are in one set,
		but not in the other.

		\begin{figure}[htb]
			\centering
			\includegraphics[width=0.3\textwidth]{figures/calculus/sets_A_B_and_set_operations.pdf}
			\caption{	Various intervals that can be obtained using set operations of the intervals $A$ and $B$.}
			\label{fig:sets_A_B_and_set_operations}
		\end{figure}

	\noindent
	I hope these definitions and examples made you feel more comfortable with sets,
	and the weird-looking curly bracket notation that mathematicians use to define sets.
	It might look a little complicated at first,
	but you'll get used to it in the rest of the book.
	% Robyn said: I love these reassuring paragraphs, though it's a bit repetative in this chapter. Just something to be aware of.

	In probability theory,
	we use finite sets and countably infinite sets like the natural numbers to represent the sample spaces of discrete random variables.
	We also use intervals to describe outcomes in the sample space of continuous random variables.





%!TEX root = ../calculus_tutorial.tex


\section{Functions}

	A \emphindexdef{function} is a mathematical object that takes numbers as inputs and produces numbers as outputs.
	We use the notation
	\[
		f \colon A \to B
	\]
	to denote a function from the input set $A$ to the output set $B$.
	For every input $x$, the output value of $f$ for that input is denoted $f(x)$.

	\subsection{Function graph}

		The \emph{graph} of a function is a line that passes through all input-output pairs of a function.
		Imagine we take out a piece of paper and draw a coordinate system with a horizontal axis and a vertical axis.
		The horizontal axis describes the different input values $x$,
		while the vertical axis describes the output values $f(x)$.
		Each input-output pair of the function $f$ corresponds to the point $(x,f(x))$ in the coordinate system.
		We obtain the graph of the function by varying the input coordinate $x$ and plotting all the points $(x, f(x))$,
		as illustrated in Figure~\ref{fig:graph_of_f_nonnegative}.

		\begin{figure}[htb]
			\centering
			\includegraphics[width=0.3\textwidth]{figures/calculus/graph_of_f_nonnegative.pdf}
% REPLACE WITH original (x, 0.5 x^2)
			\caption{	The graph of the function $f$ consists of all the points with coordinates $(x,f(x))$ over some interval of $x$ values.}
			\label{fig:graph_of_f_nonnegative}
		\end{figure}

		The graph of the function $f$ allows us to see at a glance the behaviour of the function for all possible inputs,
		and forms an essential visualization tool.
		Indeed,
		many phenomena and calculations related to functions
		can be understood geometrically as operations based on the graph of the function.

		In probability theory,
		we use functions to describe the probability distributions of random variables.
		Discrete random variables are described by a probability mass function of the form $f\colon \mathcal{X} \to \mathbb{R}$,
		where the sample space $\mathcal{X}$ is either a finite set or a countably infinite set like the natural numbers $\mathbb{N}$.
		Continuous random variables are described by probability density functions of the form 
		$f\colon \mathcal{X} \to \mathbb{R}$,
		where the sample space $\mathcal{X}$ is some subset of the real numbers $\mathbb{R}$.


	\subsection{Plotting function graphs using NumPy and Seaborn}

		
		We can use a combination of the \tt{numpy} and \tt{seaborn} modules
		to plot the graph of the function $g(x) = \frac{1}{2}x^2$,
		as shown in the code example below.

		\begin{codeblock}[plot-gx-0-10]
		>>> def g(x):
		        return 0.5 * x**2
		>>> import numpy as np
		>>> import seaborn as sns
		>>> xs = np.linspace(0, 10, 1000)
		>>> gxs = g(xs)
		>>> sns.lineplot(x=xs, y=gxs, label="Graph of g(x)")
		See Figure ¡\ref{fig:graph_of_function_g_eq_halfx2}¡ for the output.
		\end{codeblock}

		\noindent
		We import the module \tt{numpy} under the alias \tt{np}.
		We use the function \tt{np.linspace} to create an array (a list of numbers) \tt{xs},
		which contains 1000 input values that range from $x=0$ until $x=10$.
		Next we apply the function $g$ to the array of inputs \tt{xs} and store the result in the array \tt{gxs},
		which contains all the output values of the function for the input values \tt{xs}.
		At this point,
		the arrays \tt{xs} and \tt{gxs} contain $1000$ input-output pairs of the form $(x, g(x))$,
		which is exactly what we need to plot the graph of the function.
		On the last line,
		we call the function \tt{lineplot} to create the graph of $g(x)$,
		which produces the plot shown in Figure~\ref{fig:graph_of_function_g_eq_halfx2}.

		\begin{figure}[htb]
			\centering
			\includegraphics[width=0.4\textwidth]{figures/graph_of_function_g_eq_halfx2.pdf}
			\vspace{-2mm}
			\caption{	Graph of the function $g(x)=x$ from $x=0$ until $x=10$.}
			\label{fig:graph_of_function_g_eq_halfx2}
\vspace{-4mm}
		\end{figure}

%			Note the steps we used to obtain the function graph in code~\ref{plot-gx-0-10}
%			correspond exactly to the mathematical procedure for drawing the graph of $f(x)$:
%			draw the line that passes through all $(x, f(x))$ input-output pairs.







	\subsection{Inverse functions}

		The inverse function $f^{-1} \colon B \to A$ performs the \emph{inverse operation} of the function $f \colon A \to B$.
		If you start from some $x$, apply $f$, and then apply $f^{-1}$,
		you'll arrive---full circle---back to the original input $x$:
		\[
			f^{-1}\!\big( \; f(x) \; \big) = x.
		\]
		In Figure~\ref{fig:functions-inverse} the function $f$ is represented as a forward arrow,
		and the inverse function $f^{-1}$ is represented as a backward arrow
		that puts the value $f(x)$ back to the $x$ it came from.

		\begin{figure}[htb]
			\centering
			\includegraphics[width=0.25\textwidth]{figures/calculus/functions-inverse.pdf}
			\caption{The inverse $f^{-1}$ undoes the operation of the function $f$.}
			\label{fig:functions-inverse}
		\end{figure}

		For example,
		if we compute the square root of a number, then square the result,
		we obtain the original number,
		since the quadratic function $x^2$ is the inverse of the square-root function $\sqrt{x}$.

		\begin{codeblock}[fun-inv-fun-combo-sqrt]
		>>> np.sqrt(4)**2
		4.0
		\end{codeblock}

		\noindent
		The exponential function $e^x$ is the inverse of the logarithmic function $\log_e(x)$,
		so if we compute the logarithm of a number then apply the exponential function,
		we get back the original input.

		\begin{codeblock}[fun-inv-fun-combo-log]
		>>> np.exp(np.log(4))
		4.0
		\end{codeblock}

		\noindent
		In probability theory,
		we often do calculations using the cumulative distribution function (CDF) $F_X \colon \calX \to [0,1]$,
		and also use the inverse of the cumulative distribution function $F_X^{-1}\colon [0,1] \to \calX$.
		Knowing about inverse functions (and the weird superscript $^{-1}$ notation used to describe them)
		is useful for your conceptual understanding of these concepts:
		instead of thinking about the inverse-CDF $F_X^{-1}$ as some new complicated concept you have to memorize,
		you can think of $F_X^{-1}$ as the ``undo operation'' for $F_X$.
		In other words,
		$F_X$ and $F_X^{-1}$ describe the same mapping,
		but used in opposite directions.

		%		$F_X(b) = q_b$
		%		vs. $F_X^{-1}(q) = b_q$  how far in the sample space do I have to go to encompass proportion $q$ of the total probability.




%!TEX root = ../calculus_tutorial.tex

\section{Limits}

	In high school math,
	we learn all kinds of math procedures for solving problems using a finite number of steps of math operations.
	Whether you're manipulating expressions using algebra,
	or applying the inverse function to simplify an equation,
	all problems in high school math can be solved by using less than five steps,
	or if your teacher really doesn't like you 10 steps.
	% INFINITY
	In calculus,
	we learn a broader class of problem-solving strategies
	that include procedures with an infinite number of steps.

	% LIMITS
	Limit expressions provide a precise mathematical language
	for talking about infinitely large numbers, infinitely small steps,
	and mathematical procedures with an infinite number of steps.
	Here are three representative examples of limit expressions:

	\begin{itemize}

		% ASYMPTOTICS
		\item	$\lim_{x \to \infty} f(x)$: limit expression that describes what happens to $f(x)$
			when the input to the function $x$ tends to infinity (gets larger and larger).
			In words,
			this limit expression is read as ``limit of $f(x)$ as $n$ goes to infinity.''

		% INFINITELY MANY
		\item	$\lim_{n \to \infty} \textrm{proc}(n)$: limit expression that describes the value of $\textrm{proc}(n)$ as the integer $n$ tends to infinity.
			The integer $n$ usually describes the number of steps in a given procedure,
			and $\textrm{proc}(n)$ describes the output of this procedure when $n$ steps are used.

		% INFINITELY SMALL
		\item	$\lim_{\delta \to 0} h(\delta)$: limit expression that describes what happens to
			$h(\delta)$ as the real number $\delta$ tends to zero.
			The number $\delta$ (the Greek letter delta) usually describes a small distance,
			and the limit as delta goes to zero ($\delta \to 0$) describes the behaviour of the expression $h(\delta)$
			for an infinitely short distance $\delta$.

	\end{itemize}

	%	Using limits allows us to obtain answers computed by mathematical procedures with an infinite number of steps!

	\noindent
	The SymPy function \tt{limit} allows us to compute limit expressions.
	For example,
	if we want to see if the exponential function $e^x$ or the polynomial function $x^{100}$ grows faster
	in the limit as $x$ goes to infinity,
	The code for computing the limit of the ratio between these two expressions is

	\begin{codeblock}[sympy-limit-exp-over-x-100]
	>>> from sympy import limit, exp, oo
	>>> limit(exp(x)/x**100, x, oo) 
	oo
	\end{codeblock}

	\noindent
	The answer $\infty$,
	written as \tt{oo} (two lowercase letters ``o''),
	tells us exponential functions grow faster than polynomial functions.
	%	This result has implications in computer science,
	%	where algorithms whose running time grows exponentially with the size of their input are considered bad

	% EXAMPLE 2: splitting up an interval into $n$ segments, then making $n$ go to infinity
	%	splitting with an infinite number of segments
	%	\begin{codeblock}[sympy-limit-sement-zero-length]
	%	>>> from sympy import limit, oo, summation
	%	>>> delta = (b - a)/n
	%	>>> limit(delta, n, oo)
	%	0
	%	\end{codeblock}
	%
	%	\begin{codeblock}[sympy-limit-sements-add-to-interval]
	%	>>> summation(delta, (i, 0, n-1))
	%	b - a				
	%	\end{codeblock}

	Limits are important in calculus because they are used in the formal definitions of the derivative and integral operations.
	The derivative is defined as a rise-over-run calculation for an infinitely short run.
	The integral is defined as a Riemann sum with infinitely narrow rectangles.
	We'll explain both of these in the next sections.






	\subsubsection{Example}

		Let's begin with a simple example.
		Say you have a string of length $\ell$ and you want to divide it into infinitely many, infinitely short segments.
		There are infinitely many segments,
		and they are infinitely short, so together the segments add to the string's total length $\ell$.

		It's easy enough to describe this process in words.
		Now let's describe the same process using the notion of a limit.
		If we divide the length of the string $\ell$ into $N$ equal pieces then each piece will have a length of
		\[
		   \delta = \frac{\ell}{N}  \,.
		\]
		Let's make sure that $N$ pieces of length $\delta$ added together equal the string's total length: 
		\[
		 N \delta = N \frac{\ell}{N} = \ell.
		\]
		
		\noindent
		Now imagine what happens when the variable $N$ becomes larger and larger.
		The larger $N$ becomes, the shorter the pieces of string will become.
		In fact, if $N$ goes to infinity (written $N \to \infty$),
		then the pieces of string will have zero length:
		\[
		 \lim_{N\to \infty}  \delta = \lim_{N\to \infty} \frac{\ell}{N} = 0.
		\]
		In the limit as $N \to \infty$, the pieces of string are \emph{infinitely small}.
		
		Note we can still add the pieces of string together to obtain the whole length:
		\[
		 \lim_{N\to \infty}  \left( N \delta \right) 
		 = 
		 \lim_{N\to \infty}  \left( N \frac{\ell}{N} \right)
		 = \ell.
		\]
		Even if the pieces of string are \emph{infinitely small},
		because there are \emph{infinitely many} of them,
		they still add to $\ell$.
		
		The take-home message is that as long as you clearly define your limits,
		you can use infinitely small numbers in your calculations.
		The notion of a limit is one of the central ideas in this course.


	\subsection{Limits at infinity}
	\label{limits:limits_to_infinity}

		In math,
		we're often interested in describing what happens to a certain function when its input variable tends to infinity.			\index{infinity}
		This information helps us draw the function's graph.
		Does $f(x)$ approach a finite number,
		or does it keep on growing to $\infty$?

		As an example of this type of calculation,
		consider the limit of the function $f(x)=\tfrac{1}{x}$ as $x$ goes to infinity:
		\[
		 \lim_{x \to \infty} f(x) = \lim_{x \to \infty} \tfrac{1}{x} = 0.
		\]
		This statement is true,
		even though the function $\tfrac{1}{x}$ never \emph{actually} reaches zero.
		The function gets closer and closer to the $x$-axis but never touches it.
		This is why the concept of a limit is useful:
		it allows us to write $\displaystyle \lim_{x\to \infty} f(x)=0$ even though $f(x)\neq 0$ for any $x \in \mathbb{R}$.

		The function $f(x)$ is said to \emph{converge} to the number $L$											\index{convergence}
		if the function approaches the value $L$ for large values of $x$:
		\[
		 \lim_{x \to \infty} f(x) = L.
		\]
		We say ``The limit of $f(x)$ as $x$ goes to infinity is the number $L$.''
		See Figure~\ref{fig:limit-at-infinity-graph} for an illustration.
		The limit expression is a concise way of saying the following precise mathematical statement:
		for \emph{any} precision $\epsilon>0$,
		there exists a starting point $S$,
		after which $f(x)$ equals $L$ within a precision $\epsilon$.

		\begin{figure}[htb]
			\centering
			\includegraphics[width=0.3\textwidth]{figures/calculus/limit-at-infinity-graph.png}
			\vspace{-2mm}
			\caption{	A function $f(x)$ whose oscillations around the value $L$ are smaller than $\epsilon$ for all $x\geq S$.
					The starting point $S$ depends on the choice of precision~$\epsilon$.}
			\label{fig:limit-at-infinity-graph}
		\end{figure}


		The precise mathematical meaning of $\displaystyle \lim_{x \to \infty} f(x) = L$ is
		\[ 
		  \forall \epsilon>0 \; \exists S \in \mathbb{R} \textrm{ such that }  \;  \forall x \geq S \;  \;  \left|f(x) - L\right| <\epsilon.
		\]
		I know what you are thinking. Whoa!
		What just happened here?
		Chill.
		I know we saw that upside-down-A and backward-E business all the way
		back in Chapter~\ref{chapter:math_fundamentals} (see page~\pageref{sec:set_notation}),
		so let me rewrite the expression for you in plain English:
		\begin{multline*}
		  \textrm{For all } \epsilon>0,  \\
		  	 \textrm{ there exists a number } S \textrm{ such that } \qquad \qquad \qquad \qquad  \\
			 	 \left|f(x) - L\right| <\epsilon
		   \textrm{ for all } x \textrm{ greater than or equal to } S.
		\end{multline*}

		 \vspace{1mm}

		\noindent
		The limit equation {\small$\displaystyle \lim_{x \to \infty} f(x) = L$} states that the 
		``limit at infinity'' of the function $f(x)$ is equal to the number $L$.
		This statement is true if and only if there exists a winning strategy for an $\epsilon,\!S$-game,
		similar to the $\epsilon,\!N$-game played by the computer scientist and the mathematician.
		In the new $\epsilon,\!S$-game,
		the mathematician specifies the precision $\epsilon$,
		and the computer scientists must find a starting point~$S$ after which $f(x)$ becomes (and stays) $\epsilon$-close to the limit $L$.
		If the computer scientist can succeed for all levels of precision $\epsilon$,
		then the mathematician will be convinced the equation $\displaystyle \lim_{x \to \infty} f(x) = L$ is true.


		\paragraph{Example 2}		
			Calculate $\lim\limits_{x\to \infty} \frac{2x+1}{x}\,$.

			You are given the function $f(x)=\frac{2x+1}{x}$
			and must determine what the function looks like for very large values of $x$.
			We can rewrite the function as $\frac{2x+1}{x}=2+\frac{1}{x}$ to more easily see what is going on:
			\[
			 \lim_{x\to \infty} \frac{2x+1}{x} 
			  = \lim_{x\to \infty}\left( 2 + \frac{1}{x} \right)
			  = 2 + \lim_{x\to \infty}\left( \frac{1}{x} \right)
			  = 2 + 0,
			\]
			since $\frac{1}{x}$ tends toward zero for large values of $x$.

			In an introductory calculus course, you will not be required to 
			give formal proofs for statements like $\lim_{x\to \infty}\frac{1}{x}=0$; 
			instead, you can assume the result is obvious and needs no proof.
			As the denominator $x$ becomes larger and larger,
			the fraction $\frac{1}{x}$ becomes smaller and smaller.



	\subsection{Limits to a number}
	\label{limits:limits_to_a_number}

		The limit of $f(x)$ approaching $x=a$ \emph{from the right} is defined as
		\[
		 \lim_{x\to a^+} f(x) = \lim_{\delta \to 0} f(a + \delta). 
		\]
		To find the limit from the right at $a$, we let $x$ take on values like 
		$a+0.1$, $a+0.01$, $a+0.001$, $a+0.0001$, etc.
		Figure~\ref{fig:limit_epsilon_delta_f_at_a_from_right} shows the graph of a function $f(x)$ near the point $(a,f(a))$.
		To prove the statement
		\[
			\lim_{x\to a^+} f(x) =  L, 
		\]
		you must show that
		\begin{multline*}
		  \qquad \qquad \forall \epsilon>0, \\
		  	\exists \delta>0 \textrm{ such that }  \qquad \qquad \qquad  \\
			 \forall x \in (a, a+\delta) \;  \;  \left|f(x) - L\right| <\epsilon. \qquad \qquad 
		\end{multline*}
		In other words, the limit from the right corresponds to an $\epsilon,\!\delta$-game
		in which the mathematician specifies the precision $\epsilon>0$,
		and the computer scientist must find a distance $\delta>0$,
		such that $\left|f(x) - L\right| <\epsilon$, for all $x$ in the range $(a,a+\delta)$.

		\begin{figure}[htb]
			\centering
			\includegraphics[width=0.3\textwidth]{figures/calculus/limit_epsilon_delta_f_at_a_from_right.png}
			\caption{	A function $f(x)$ whose variation around the value $L$ is smaller than $\epsilon$ for all $x$ in the interval $(a, a+\delta)$.
					The value $\delta$ depends the choice of $\epsilon$.}
			\label{fig:limit_epsilon_delta_f_at_a_from_right}
		\end{figure}

		The limit of $f(x)$ when $x$ approaches \emph{from the left} is defined analogously,
		\[
 		   \lim_{x\to a^-} f(x)  = \lim_{\delta \to 0} f(a - \delta).
		\]
		%		describes what happens to $f(x)$ as $x$ approaches $a$ from below
		%		(from the left) with values like $x=a-$, 
		%		with $\delta>0, \delta \to 0$.

		If both limits from the left and from the right of some number exist and 
		are equal to each other, we can talk about the limit as $x\to a$ 
		without specifying the direction of approach: 
		\[
		 \lim_{x\to a} f(x) =  \lim_{x\to a^+} f(x) =  \lim_{x\to a^-} f(x).
		\]
		For the two-sided limit of a function to exist at a point,
		both the limit from the left and the limit from the right must converge to the same number.
		If the function $f(x)$ obeys, $f(a) = L$ and $\displaystyle\lim_{x\to a} f(x) = L$,
		we say the function $f(x)$ is continuous at $x=a$.

		\begin{figure}[htb]
			\centering
			\includegraphics[width=0.3\textwidth]{figures/calculus/limit_epsilon_delta_f_at_a.png}
			\caption{The two-sided limit $\lim_{x\to a} f(x)=L$ exists if both the limit from the left and the limit from the right exist and are equal to $L$.}
			\label{fig:limit_epsilon_delta_f_at_a}
		\end{figure}
		
		




	\subsection{Continuity}
	\label{limits:continuity}
	
		
		A function is said to be \emph{continuous}														\index{continuous function|textbf}
		if its graph looks like a smooth curve that doesn't make any sudden jumps and contains no gaps.
		If you can draw the graph of the function on a piece of paper without lifting your pen,
		the function is continuous.

		A more mathematically precise way to define continuity is to say the function is equal to its limit for all $x$.
		We say a function $f(x)$ is \emph{continuous} at $a$ if the limit of $f$ as $x\to a$ converges to $f(a)$:
		\[
		 \lim_{x \to a}  f(x) =  f(a).
		\]
		Remember,
		the two-sided limit $\lim_{x\to a}$ requires both the left and the right limit to exist and to be equal.
		Thus, the definition of continuity implies the following equality:
		\[
		 \lim_{x \to a^-}  f(x) =  f(a) = \lim_{x \to a^+}  f(x).
		\]
		In words,
		this means that a function $f(x)$ is continuous at $x=a$
		if the limit from the left $\lim_{x \to a^-}  f(x)$
		and the limit from the right $\lim_{x \to a^+}  f(x)$
		are both equal to the value of the function at $x=a$.

		Take a moment to think about the mathematical definition of continuity at a point.
		Can you connect the math definition to the intuitive idea that functions are continuous if they can be drawn without lifting the pen?

		Most functions we'll study in calculus are continuous,
		but not all functions are.
		Functions that are not defined for some value, as well as functions that make sudden jumps, are not continuous.
		%		Another examples is the function $f(x)=\frac{2x+1}{x}$ which is discontinuous at $x=0$
		%		(because the limit $\lim_{x \to 0}  f(x)$ doesn't exist and $f(0)$ is not defined).

		For example,
		consider the function $f : \mathbb{R} \setminus \{0\} \to \mathbb{R}$ defined by
		\[
			f(x)
			=\frac{ | x-3| }{x-3} 
			= \left\{ 	\begin{array}{rl}
					1	\quad	&  \mathrm{  if } \;  x > 3, \\
					-1	\quad	&  \mathrm{  if } \;  x < 3.
			                        \end{array}                    \right.
		\]
		The function $f$ is continuous everywhere on the real line except at $x=3$.
		Since this function $f$ is ``missing'' only at a single point,
		we can try to ``patch it'' by filling in the missing value.
		Consider the function $g : \mathbb{R} \to \mathbb{R}$ defined as
		\[
			g(x)
			= \left\{ 	\begin{array}{rl}
					1	\quad	&  \mathrm{  if }\;  x > 3, \\
					1	\quad	&  \mathrm{  if }\;  x = 3, \\
					-1	\quad	&  \mathrm{  if } \;  x < 3.
			                        \end{array}                    \right.		
		\]
		The function $g$ is \emph{continuous from the right} at the point $x=3$,
		since $\lim_{x \to 3^+} g(x)=1=g(3)$.
		However,
		taking the limit from the left,
		we find $\lim_{x \to 3^-} g(x)=-1 \neq g(3)$,
		which tells us $g$ is not continuous from the left.
		We say the function $g$ has a \emph{jump discontinuity} at $x=3$.

		% Khan Academy
		% https://www.youtube.com/watch?v=Y7sqB1e4RBI


		\paragraph{Example 3}	
			We can calculate the limit $\displaystyle\lim_{x\to 5} \frac{2x+1}{x}$ as follows:
			\[
			 \lim_{x\to 5} \frac{2x+1}{x}
			  = \frac{2(5)+1}{5}
			  = \frac{11}{5}.
			\]
			There is nothing tricky going on here---we plug the number $5$ into the equation, and voila. 
			The function $f(x)=\frac{2x+1}{x}$ is continuous at the value $x=5$, so the limit of the function 
			as $x\to 5$ is equal to the value of the function $\displaystyle\lim_{x\to 5} f(x) = f(5)$.
			% This is true in general for any continuous function.
			





	\subsection{Applications of limits}

		Limits are fundamentally important for calculus.															\index{limit}
		Indeed, the three main calculus topics we'll discuss in the remainder of this chapter are
		derivatives, integrals, and series---all of which are defined using limits.
	
	
		\subsubsection{Limits for derivatives}
		\label{limits:limits_for_derivatives}
	
			The formal definition of a function's derivative is expressed in terms
			of the rise-over-run formula for an infinitely short run:
			\[
			 f'(x) 
			 \; = \; \lim_{\textrm{run} \to 0} \frac{\text{rise}}{\text{run}} 
			 \; = \;  \lim_{\delta \to 0} \frac{f(x+\delta)\; - \; f(x)}{x+\delta \; - \;  x}.
			\]
			We'll continue the discussion of this formula in Section~\ref{sec:derivatives}.
	
	
		\subsubsection{Limit for integrals}
		\label{limits:limits_for_integrals}
	
			One way to approximate the area under the curve $f(x)$ between $x=a$ and $x=b$
			is to split the area into $N$ little rectangles of width $\epsilon = \frac{b-a}{N}$ and height $f(x)$,
			and then calculate the sum of the areas of the rectangles:
			\[ 
			  A(a,b) \approx \underbrace{ \epsilon f(a) + \epsilon f(a+\epsilon) 
			  						+ \epsilon f(a+2\epsilon) + \cdots 
									+ \epsilon f(b-\epsilon)}_{N \textrm{ terms}}.
			\]
			We obtain the exact value of the area in the limit where we split the area into an infinite
			number of rectangles with infinitely small width:
			{ \small
			\[ 
			  \!\!\int_a^b\!\!\!f(x) \: dx
			  = \!A(a,b)
			   = \!\!\lim_{N \to \infty}\!\!\left[ 
			   	\epsilon f(a) + \epsilon f(a+\epsilon) + \epsilon f(a+2\epsilon) + \cdots + \epsilon f(b-\epsilon) 
				\right].
			\]}
			
			\noindent 
			Computing the area under a function by splitting the area into infinitely 
			many rectangles is an approach known as a \emph{Riemann sum},										\index{Riemann sum}
			which we'll discuss in Section~\ref{sec:riemann_sum}.
				
			%	\subsection{Limits for sequences}	TODOv7
			%	\label{limits:limits_for_sequences}
	
	
		
		\subsubsection{Limits for series}
		\label{limits:limits_for_series}
		
			We use limits to describe the convergence properties of series.											\index{convergence}
			For example, the partial sum of the first $N$ terms of the geometric series
			$a_n= r^n$ corresponds to the following expression:
			\[
			  S_N 
			   = \sum_{n=0}^N r^n 
			   = 1 + r + r^2 + r^3 + \cdots + r^N.
			\]
			The \emph{series} $a_n$ is defined as the limit $N\to \infty$ of the above expression.
			For values of $r$ that obey $|r|<1$,
			the series converges:
			\[ 
			  S_\infty 
			   = \lim_{N \to \infty} S_N 
			   = \sum_{n=0}^\infty r^n
			   = 1 + r + r^2 +  r^3 + \cdots 
			   =\frac{1}{1-r}.
			\]
			To convince yourself the above formula is correct,
			observe how the infinite sum $S_\infty$ is similar to a shifted version of itself: $S_\infty=1+rS_\infty$.
			Now solve for $S_\infty$ in the equation $S_\infty=1+rS_\infty$.
			
			You'll find more about series in Section~\ref{sec:series}.
	



%!TEX root = ../calculus_tutorial.tex

\section{Derivatives}

		The \emph{derivative} function, denoted $f'(x)$, $\frac{d}{dx}f(x)$, or $\frac{df}{dx}$, 				\index{derivative}
		describes the \emph{rate of change} of the function $f(x)$.
		For example,
		the constant function $f(x)=c$ has derivative $f'(x)=0$ since the function $f(x)$ does not change at all.
		The derivative function describes the \emph{slope} of the graph of the function $f(x)$.
		The derivative of a line $f(x)=mx+b$ is $f'(x)=m$,
		since the slope of this line is equal to $m$.
		In general,
		the slope of a function is different at different values of~$x$,
		so mathematicians invented a new notation for describing ``the slope (rate of change) of the function $f(x)$''
		and obtained formulas for finding the derivative of any function.

		The derivative function $f'(x)$ is defined as the rate of change of the function $f$ at $x$:
		\[
			f'(x) = \lim_{\delta \to 0} \frac{f(x+\delta)\ - \ f(x)}{\delta}\,.
		\]
		In words,
		this formula describes the general rise-over-run calculation for computing the slope of a line that connects
		the points $(x,f(x))$ and $(x+\delta, f(x+\delta))$,
		with the step-length $\delta$ becoming infinitely small.

		%	The definition of the derivative comes from the rise-over-run formula for calculating the slope of a line:
		%	\[
		%	  \frac{ \textrm{rise} } { \textrm{run} } = \frac{ \Delta y } { \Delta x } 
		%		=  \frac{y_f - y_i}{x_f - x_i} = 
		%		\frac{f(x+\delta)\ - \ f(x)}{x + \delta \  -\  x}.
		%	\]
		%	By making $\delta$ tend to zero in the above expression,
		%	we're performing a rise-over-run of the function $f(x)$ at a point.


		Geometrically,
		the derivative function computes the slope of the graph of the function $f(x)$ for all values of $x$.
		In general,
		the slope of a function is different for different values of $x$.
		Figure~\ref{fig:derivative_as_slope_small-0} shows the slope calculation for the function $f(x) = \frac{1}{2}x^2$
		for two different values of $x$: $x=-0.5$ and $x=1$.
		% the slope of the function is the same as the line passing through this point


		\begin{figure}[htb]
			\centering
			\includegraphics[width=0.5\textwidth]{figures/calculus/derivative_as_slope_small-0.pdf}
			\caption{	The derivative of the function at $x=a$ is denoted $f'(a)$ and describes the slope function at that point.}
			%	\caption{	The diagram illustrates how to compute the derivative of the function $f(x)=\tfrac{1}{2}x^2$
			%			at three different points on the graph of the function.
			%			To calculate the derivative of $f(x)$ at $x=1$, 
			%			we can ``zoom in'' near the point $(1,\tfrac{1}{2})$ and 
			%			draw a line that has the same slope as the function.
			%	 	 	We can then calculate the slope of the line using a rise-over-run calculation,
			%			aided by the mini coordinate system that is provided.
			%			The derivative calculations for $x=-\tfrac{1}{2}$ and  $x=2$ are also shown.
			%			Note that the slope of the function is different for each value of $x$. 
			%			What is the value of the derivative at $x=0$?
			%			Can you find the general pattern?}
			\label{fig:derivative_as_slope_small-0}
		\end{figure}


		%	Derivatives occur so often in math that people have devised many ways to denote them:
		%	\[
		%	    Df(x) \equiv f'(x) \equiv  \frac{d}{dx}f(x) \equiv \frac{df}{dx} \equiv \frac{dy}{dx} \equiv \nabla f.
		%	\]
		%	Don't be fooled by this multitude of notations---all of them refer to the same concept.

		% TODO: mention the derivative is a function of the form $f': \mathbb{R} \to \mathbb{R}$.
		The derivative function $f'(x)$ is a property of the function $f(x)$.
		Indeed, this is where the name \emph{derivative} comes from:
		$f'(x)$ is not an independent function---it is \emph{derived} from the original function $f(x)$.
		
		The \emph{derivative operator}, 
		denoted $\frac{d}{dx}$ or simply $D$, 
		takes as input a function $f(x)$ and produces as output the derivative function $f'(x)$,
		which is a function of the form $f': \mathbb{R} \to \mathbb{R}$.
		For each input $x_0$ the derivative function tells you the slope of $f(x)$ when $x=x_0$.
		Applying the derivative operator to a function is also called ``taking the derivative'' of a function.
		
		For example, 
		the derivative of the function $f(x)=\tfrac{1}{2}x^2$ is the function $f'(x)=x$.
		We can describe this relationship as $(\tfrac{1}{2}x^2)^{\prime} = x$
		or as $\tfrac{d}{dx}(\tfrac{1}{2}x^2)=x$.
		Look at Figure~\ref{fig:derivative_as_slope} and use the graph to prove to yourself
		that the slope of $f(x)=\tfrac{1}{2}x^2$ is described by $f'(x)=x$ everywhere on the graph.



LETS SEE SOME CODE


		Here is a simple computer program for computing the numerical approximations to the derivative of any function at any point:
		
		\begin{codeblock}[]
		def differentiate(func, x, delta):
		    """
		    Compute the slope of the function `func` at `x` using
		    the rise-over-run calculation with run of length `delta`.
		    """
		    rise = func(x+delta) - func(x)
		    run = delta
		    return rise/run
		\end{codeblock}


		
		\vspace{-3mm}
		\noindent
		You can then define the function $f=\frac{1}{2}x^2$ using the code

		\begin{codeblock}[]
		def f(x):
		    return 0.5*x**2
		\end{codeblock}
		
		\noindent
		and compute the derivative at any point $x$ by calling the function \texttt{differentiate(f,x,delta)},
		where \texttt{delta} is some small number.
		
		Here are some of the outputs of the differentiation procedure at $x=1$
		using different values of horizontal ``run'' parameter \texttt{delta}:
		\begin{itemize}
			\item	The output of \texttt{differentiate(f,x=1,delta=0.01)} is $1.005$,
				which is within $0.5\%$ of the exact answer $f'(1)=1$.
			\item	The output of \texttt{differentiate(f,x=1,delta=0.001)} is $1.0005$,
				which is within within $0.05\%$ accuracy of the exact answer.
			\item The output for \texttt{differentiate(f,1,0.0001)} is $1.00005$,
				which is an even more accurate approximation of the exact value.
			\item	The output of \texttt{differentiate(f,1,0.00001)} is $1.000005$.
		\end{itemize}
		Note the approximations get more and more accurate as the parameter \texttt{delta} decreases.
		For most practical applications,
		we can always choose a sufficient small \texttt{delta}
		so the fact that numeric approximations computed are a little bit ``off'' does not become a problem.

%		The mathematical notion of a limit describes logical continuation of the above thinking with where the parameter $\code{delta}$ becomes infinitely small.
%		The mathematical definition $f'(x) \equiv \lim_{ \delta \rightarrow 0}\frac{f(x+\delta)-f(x)}{\delta}$
%		allows us to find a closed form expression for the derivative function that applies for all values of $x$.
%		Finding the exact formula for the derivative requires a little bit more work upfront (to simplify an expressions that involves Greek symbols),
%		but once you find the exact formula for the derivative $f'(x)$,
%		you can compute the slope of $f(x)$ at the point $x=c$ by simply evaluating the the derivative function at that point:
%		$f'(c) = \{ \textrm{the slope of} \ f(x) \ \textrm{at} \ x=c\}$.




% SYMPY DERIVATIVES
%
%	The code below shows how to compute the derivative of the function $f(x) = mx +b$.
%
%	\begin{codeblock}[sympy-diff-line]
%	>>> from sympy import diff
%	>>> f = b + m*x
%	>>> diff(f, x)
%	m
%	\end{codeblock}
%
%	\noindent
%	In words,
%	this calculation tells us the derivative of the function $f(x) = mx +b$ is the constant function $f'(x)=m$.
%	The expression \tt{diff(f,x)} tells SymPy to compute the derivative of the expression \tt{f} with respect to the variable \tt{x}.
%
%	Let's now define the function $f(x) = \frac{c}{2}x^2$ and compute its derivative.
%
%	\begin{codeblock}[sympy-diff-quadratic]
%	>>> f = c/2 * x**2
%	>>> diff(f, x)
%	c*x
%	\end{codeblock}
%
%	\noindent
%	The derivative function is $f'(x)=cx$.
%	See the plot in Figure~\ref{fig:derivative_as_slope_small-0} for an illustration of the case when $c=1$.
%
%	Here is another example of a complicated-looking function $f$,
%	that includes an exponential, a trigonometric, and a logarithmic function:
%
%	\begin{codeblock}[sympy-diff-fancy-mix]
%	>>> from sympy import log, exp, sin
%	>>> f = exp(x) + sin(x) + log(x)
%	>>> diff(f, x)
%	exp(x) + cos(x) + 1/x
%	\end{codeblock}
%
%	\noindent
%	As you can see,
%	using the function SymPy function \tt{diff} allows you to compute the derivative function for any function $f(x)$.


		\subsubsection{Derivative formulas}
		\label{mathematical_preliminiaries:derivative_formulas}
		
			You don't need to apply the complicated derivative formula $f'(x) \equiv \lim_{ \delta \rightarrow 0}\frac{f(x+\delta)-f(x)}{\delta}$
			every time you need to find the derivative of a function.
			For each function $f(x)$,
			it's enough to use the complicated formula once and record the formula you obtain for $f'(x)$,
			then you can reuse that formula whenever it comes up again in a calculation.
			Indeed,
			that's what most scientists and engineers do---whenever they need to know the derivative of some function $f(x)$,
			the lookup the answer in a table of derivative formulas.

			The following table shows the derivative formulas for a number of commonly used functions.								\index{derivative} \index{integral}

			{\allowdisplaybreaks
			\begin{align*}
			f(x)			&  \ -\textrm{derivative}\to  \   		f'(x)		\\
			%	F(x)			&  \ - \textrm{ derivative } \to  \quad		F'(x)		\\
			%	\int f(x)\;dx   	& \ \ \leftarrow \textrm{ integral } -   \quad 	f(x)     	\\
			a			&\qquad  \raisebox{.52ex}{\rule{0.9em}{.4pt}}\;\tfrac{d}{dx} \rightarrow \qquad 				0		\\
			\alpha f(x)+ \beta g(x)		&\qquad  \raisebox{.52ex}{\rule{0.9em}{.4pt}}\;\tfrac{d}{dx} \rightarrow \qquad 	\alpha f'(x)+ \beta g'(x)	\\
			x			&\qquad  \raisebox{.52ex}{\rule{0.9em}{.4pt}}\;\tfrac{d}{dx} \rightarrow \qquad				1		\\
			%	af(x)			&\qquad  \raisebox{.52ex}{\rule{0.9em}{.4pt}}\;\tfrac{d}{dx} \rightarrow \qquad 				af'(x)		\\
			%	f(x)+g(x)		&\qquad  \raisebox{.52ex}{\rule{0.9em}{.4pt}}\;\tfrac{d}{dx} \rightarrow \qquad 				f'(x)+g'(x)	\\
			x^n			&\qquad  \raisebox{.52ex}{\rule{0.9em}{.4pt}}\;\tfrac{d}{dx} \rightarrow \qquad 				nx^{n-1}	\\
			\frac{1}{x}\equiv x^{-1}		&\qquad  \raisebox{.52ex}{\rule{0.9em}{.4pt}}\;\tfrac{d}{dx} \rightarrow \qquad 	\frac{-1}{x^2} \equiv -x^{-2}		\\
			\sqrt{x} \equiv x^{\frac{1}{2}}	&\qquad  \raisebox{.52ex}{\rule{0.9em}{.4pt}}\;\tfrac{d}{dx} \rightarrow \qquad 	\frac{1}{2\sqrt{x}} \equiv \frac{1}{2}x^{-\frac{1}{2}}	\\
			e^x			&\qquad  \raisebox{.52ex}{\rule{0.9em}{.4pt}}\;\tfrac{d}{dx} \rightarrow \qquad 				e^x	\\
			a^x			&\qquad  \raisebox{.52ex}{\rule{0.9em}{.4pt}}\;\tfrac{d}{dx} \rightarrow \qquad 				a^x\ln(a)	\\
			\ln(x)			&\qquad  \raisebox{.52ex}{\rule{0.9em}{.4pt}}\;\tfrac{d}{dx} \rightarrow \qquad 				\frac{1}{x}		\\
			\log_a(x)		&\qquad  \raisebox{.52ex}{\rule{0.9em}{.4pt}}\;\tfrac{d}{dx} \rightarrow \qquad 				(x\ln(a))^{-1}	\\
			\sin(x)			&\qquad  \raisebox{.52ex}{\rule{0.9em}{.4pt}}\;\tfrac{d}{dx} \rightarrow \qquad 				\cos(x)		\\
			\cos(x)			&\qquad  \raisebox{.52ex}{\rule{0.9em}{.4pt}}\;\tfrac{d}{dx} \rightarrow \qquad 				-\sin(x)	\\
			\tan(x)			&\qquad  \raisebox{.52ex}{\rule{0.9em}{.4pt}}\;\tfrac{d}{dx} \rightarrow \qquad 				\sec^2(x)\equiv\cos^{-2}(x)
			%	\csc(x) \equiv \frac{1}{\sin(x)}		&\qquad  \raisebox{.52ex}{\rule{0.9em}{.4pt}}\;\tfrac{d}{dx} \rightarrow \qquad 	-\sin^{-2}(x)\cos(x)	\\
			%	\sec(x) \equiv \frac{1}{\cos(x)}		&\qquad  \raisebox{.52ex}{\rule{0.9em}{.4pt}}\;\tfrac{d}{dx} \rightarrow \qquad 	\tan(x)\sec(x)	\\
			%	\cot(x) \equiv \frac{1}{\tan(x)}		&\qquad  \raisebox{.52ex}{\rule{0.9em}{.4pt}}\;\tfrac{d}{dx} \rightarrow \qquad 	-\csc^2(x)	\\
			%	\sin^{-1}(x)		&\qquad  \raisebox{.52ex}{\rule{0.9em}{.4pt}}\;\tfrac{d}{dx} \rightarrow \qquad 				\frac{1}{\sqrt{1-x^2}}	\\
			%	\cos^{-1}(x)		&\qquad  \raisebox{.52ex}{\rule{0.9em}{.4pt}}\;\tfrac{d}{dx} \rightarrow \qquad 				\frac{-1}{\sqrt{1-x^2}}	\\
			%	\tan^{-1}(x)		&\qquad  \raisebox{.52ex}{\rule{0.9em}{.4pt}}\;\tfrac{d}{dx} \rightarrow \qquad 				\frac{1}{1+x^2}	\\
			%	\sinh(x)		&\qquad  \raisebox{.52ex}{\rule{0.9em}{.4pt}}\;\tfrac{d}{dx} \rightarrow \qquad 				\cosh(x)	\\
			%	\cosh(x)		&\qquad  \raisebox{.52ex}{\rule{0.9em}{.4pt}}\;\tfrac{d}{dx} \rightarrow \qquad 				\sinh(x)	
			\end{align*}
			}
			
			\noindent
			We'll be using these derivative formulas a lot in Section~\ref{sec:continuous_prob_distr},
			so it's a good idea to mentally bookmark this page so you can come back to it when derivatives come up.



		\subsubsection{Derivative rules}

			In addition to the table of derivative formulas show above,
			there are some important derivatives rules that you should know about.
			These rules will allow you to find derivatives of \emph{composite} functions.

			\paragraph{Linearity}		
				The derivative is a \emph{linear} operation, which means:								\index{linearity}
				\vspace{1mm}
				\[
					\frac{d}{dx} \left[\alpha f(x) + \beta g(x)\right]
					=
					\alpha \frac{d}{dx}f(x) + \beta \frac{d}{dx}g(x).
				\]
				
				\noindent
				In other words,
				the derivative of a linear combination of functions $\alpha f(x) + \beta g(x)$ is equal 
				to the same linear combination of the derivatives $\alpha f'(x) + \beta g'(x)$.

			\paragraph{Product rule}
				The derivative of a product of two functions is obtained as follows:									\index{product rule|textit}
				\[
				 \left[ f(x)g(x) \right]^\prime 
				 = f^\prime(x)g(x)  + f(x)g^\prime(x).
				\]

				The product rule also applies to the product of three functions $f(x)g(x)h(x)$,
				for which the derivative is
				\[
				 \left[ f(x)g(x)h(x) \right]^\prime 
					 =
					 f'(x)g(x)h(x) + 
					 f(x)g'(x)h(x) + 
					 f(x)g(x)h'(x).
				\]
				For each term,
				take the derivative of one of the functions and multiply this derivative by the other two functions:


			\paragraph{Quotient rule}
				The \emphindexdef{quotient rule} tells us how to obtain the derivative of a fraction of two functions:
				
				\[
					\left[ \frac{f(x)}{g(x)}\right]^\prime=\frac{f'(x)g(x)-f(x)g'(x)}{g(x)^2}.
				\]
				
				
			\paragraph{Chain rule}
				If you encounter a situation that includes an inner function and an outer function,						\index{chain rule|textit}
				like $f(g(x))$, you can obtain the derivative by a two-step process:		
				\[
				 \left[ f(g(x)) \right]^\prime
				 =
				 f^\prime(g(x))g^\prime(x).
				\]%
				
				\noindent
				In the first step, leave the inner function $g(x)$ alone,
				and focus on taking the derivative of the outer function $f(x)$.
				This step gives us $f'(g(x))$,
				the value of $f'$ evaluated at $g(x)$.
				As the second step,
				we multiply the resulting expression by the derivative of the \emph{inner} function $g'(x)$.

				The chain rule so applies to functions of functions of functions $f(g(h(x)))$.
				To take the derivative,
				start from the outermost function and work your way toward $x$.
				\[
				 \left[ f(g(h(x))) \right]^\prime 
					 =
					 f'(g(h(x)))
					 g'(h(x))
					 h'(x).
				\]


		\subsubsection{Examples}
		
			The above rules define \emph{all} you need to know to take the derivative of any function, 
			no matter how complicated.
			Let's look at some examples.

			\paragraph{Example 1}
				To calculate the derivative of $f(x) = e^{x^2}$,
				we use the chain rule: $f'(x) = e^{x^2}[x^2]'  = e^{x^2}2x$.
	
			\paragraph{Example 2}
				To find the derivative of $f(x) = \sin(x)e^{x^2}$,
				we use the product rule and the chain rule: $f'(x) = \cos(x)e^{x^2} + \sin(x)2xe^{x^2}$.
	
			\paragraph{Example 3}
				To compute the derivative of $f(x) = \sin(x)e^{x^2}\ln(x)$,
				we apply the product rule for three terms: 
				$f'(x) = \cos(x)e^{x^2}\ln(x) + \sin(x)2xe^{x^2}\ln(x) + \sin(x)e^{x^2}\frac{1}{x}$.
	
			\paragraph{Example 4}
				The derivative of $\sin(x^2)$ requires using the chain rule:
				$\left[ \sin(x^2) \right]^\prime =  \cos(x^2)\left[x^2\right]' =  \cos(x^2)2x$.

			\paragraph{Example 5}
				The derivative of $\sin( \ln( x^3) )$ requires the triple chain rule:
				\begin{align*}
				   \!\left[ \sin( \ln( x^3) )  \right]^\prime
				   	&= \cos( \ln(x^3) ) \!\left[ \ln(x^3)  \right]^\prime 			\\
				   	&= \cos( \ln(x^3) )\tfrac{1}{x^3}\!\left[ x^3 \right]^\prime 	\\
					%	  = \cos( \ln(x^3) ) \frac{1}{x^3} 3x^2 
					&= \cos( \ln(x^3) ) \tfrac{3}{x}\,.
				\end{align*}
				Simple, right?

			%	\paragraph{Example 6}
			%		To find the derivative of $f(x) = \sin\!\left( \cos\!\left( \tan(x) \right) \right)$,
			%		we need the triple chain rule again:
			%		\begin{align*}
			%		 f'(x) 
			%		 & = \cos\!\left( \cos\!\left( \tan(x) \right) \right)
			%		  \left[ \cos\!\left( \tan(x) \right) \right]^\prime \\
			%		 & = -\cos\!\left( \cos\!\left( \tan(x) \right) \right)
			%		  \sin\!\left( \tan(x) \right)\left[ \tan(x) \right]^\prime \\
			%		 & = -\cos\!\left( \cos\!\left( \tan(x) \right) \right)
			%		  \sin\!\left( \tan(x) \right)\sec^2(x).
			%		\end{align*}


    
		\subsubsection{Applications of derivatives}

			We use derivatives to solve problems in physics, chemistry, computing, biology, business,
			and many other areas of science.
			The derivative operator comes up whenever we study the rate of change of a quantity.
			Derivatives are also useful for solving optimization problems,
			which consist of finding the maximum or minimum value of some function $f(x)$.
			
			Optimization techniques form a key building block for many machine learning algorithms,
			so it's good to know a derivative of two if you're going to learn about machine learning topics.
			In this book,
			we won't go too far so a ``superficial'' familiarity with the concept will be sufficient,
			but if you want to pursue machine learning topics in more depth you'll have to read up more on derivatives.
			
			The derivative operation is also important because it is the ``inverse operation'' of the integration operation,
			which is the subject we'll discuss in the next section.
			Skip ahead to page~\pageref{mathematical_preliminiaries:FTC} if you want spoilers about the inverse 
			relationship between differentiation and integration,
			or read on to watch the calculus movie in order.








	\subsection{Optimization algorithms}

		One of the most prominent applications of derivatives is \emph{optimization}:              \index{optimization}
		the process of finding a function's maximum and minimum values.                           \index{maximum} \index{minimum}
		
		Consider the shape of the function near a minimum value.
		The function is decreasing just before it reaches its minimum,
		and the function increases just after its minimum.
		This means we can start at any point $x_0$ near the minimum
		and take ``downhill'' steps following the descending direction of the function,
		we'll end up at the minimum value.
		This simple procedure that repeatedly takes steps in the direction where the function is decreasing
		turns out to be a very powerful tool that can find the minimum of any function.
		This procedure is called the \emph{gradient descent algorithm},
		where the name \emph{gradient} % (denoted $\nabla f(x,y)$)
		refers to the derivative operation for multivariable functions. % like $f(x,y)$.
		% TODO: say see notebook for implementation of derivative_descent algo. /TODO
		%	You'll encounter the gradient descent algorithm and its numerous variations
		%	if you choose to purse the topic of machine learning,
		%	since they are used in many machine learning applications.	% where multivariable functions are the norm

		In this book,
		we won't discuss the details behind optimization algorithms,
		and instead rely on the computational tools available in \tt{numpy}, \tt{scipy}, and \tt{sympy} to do optimization-type calculations for us.
		We'll encounter optimization ideas (maximization and minimization) in several concepts in statistics:
		\emph{maximum likelihood} and \emph{least squares},
		and rely on ``visual proofs'' for these optimization procedures.
		If you're interested in attaining a deeper understanding of optimization algorithms,
		you can follow the links provided at the end of this section,
		but note such ``under the hood'' understanding is not required to continue with the rest of the book.
		
		Here is a quick code example that shows how to use the function \tt{minimize} defined in the module \tt{scipy.optimize}
		to find the minimum value of the function $f(x)=(x-5)^2$.

		\begin{codeblock}[sympy-minimize-fx]
		>>> from scipy.optimize import minimize
		>>> def f(x):
		        return (x-5)**2
		>>> res = minimize(f, x0=0)
		>>> res["x"][0]  # = argmin f(x)
		4.99999997455944
		\end{codeblock}

		\noindent
		The \tt{minimize} function takes two arguments:
		the function to minimize,
		and a initial value $x_0$ where to start the minimization process.




	\subsection{Fundamental theorem of calculus}
	
		The fundamental theorem of calculus (FTC) is a deep insight about the
		inverse relation that exists between the operations of integration $\int \cdot dx$
		and differentiation $\frac{d}{dx}[\cdot]$.

		The integral function $F_a(x)$ is obtained from the original function $f(x)$ using integration,
		$F_a(x) = \int_a^x f(u) du$.
		Another way to describe this is to say we \emph{applied} the integration operator $\int \cdot dx$
		on the function $f(x)$ to obtain the integral function $F_a(x)$.
		The derivative function $f'(x)$ is defined by the formula $f'(x) = \lim_{\delta \to 0} \frac{f(x+\delta)\ - \ f(x)}{\delta}$.
		We can also say we \emph{applied} the derivative operator $\frac{d}{dx}[\cdot]$
		to the function $f(x)$ to obtain the derivative function $f'(x)$.
		I use the word ``operator'' here to refer to an operation that acts on functions.

		There is no reason \emph{a priori} to think that integration and differentiation might be related:
		the former is a calculation about areas,
		while the latter is a calculation about slopes.
		The fundamental theorem of calculus reveals that they are in fact inverse operations:
		we can obtain the original function $f(x)$ from the integral function $F_a(x)$ by computing it's derivative:
		\[
			\frac{d}{dx}\big[F_a(x)\big] 	= 	\frac{d}{dx}\left[\int_a^x f(u) \; du \right]  	= 	f(x).
		\]
		Note we used a temporary variable $u$ as the integration variable,
		since $x$ is already used to denote the upper limit of integration.

		In order to understand the inverse relationship between integration and differentiation,
		we can draw an analogy with the inverse relationship between a function $f$ and its inverse function $f^{-1}$,
		which \emph{undoes} the effects of $f$.
		See Figure~\ref{fig:functions-inverse} on page~\pageref{fig:functions-inverse}.
		Given some initial value $x$,
		if we apply the function $f$ to obtain the number $f(x)$,
		and apply the inverse function $f^{-1}$ on the number $f(x)$,
		then the result will be the initial value $x$ we started from:
		\[
			f^{-1}\!\left( f(x) \right)	=	x.
		\]
		Similarly,
		the derivative operator is the ``inverse operator'' of the integral operator.
		If you perform the integral operation followed by the derivative operation on some function,
		you'll obtain the same function:
		\[
			\tt{diff( }\tt{integrate(} f(x) \tt{) )} = f(x),
		\]
		where we've used the SymPy functions \tt{integrate} for computing the integrals,
		and \tt{diff} (short for ``differentiate'') for computing derivatives.

		The code example below shows how we can construct a complicated-looking function \tt{f}
		and compute its integral function \tt{F} using SymPy.

		\begin{codeblock}[sympy-FTC-obtain-F]
		>>> from sympy import diff, integrate, log, exp, sin
		>>> f =  log(x) + exp(x) + sin(x)
		>>> F = integrate(f)
		>>> F
		x*log(x) - x + exp(x) - cos(x)
		\end{codeblock}


		\noindent
		If we now take the derivative of the function \tt{F},
		we get back the original function \tt{f}.

		\begin{codeblock}[sympy-FTC-get-back-f]
		>>> diff(F)
		log(x) + exp(x) + sin(x)
		>>> diff(integrate(f)) == f  # FTC part 1
		True
		\end{codeblock}

		\noindent
		The inverse relationship also holds for the opposite order of application:
		if we take the derivative of some function,
		then compute the integral of the derivative,
		then we arrive back at the original function (up to an additive constant factor).  % due to the arbitrary choice of the starting point for integration

		\begin{codeblock}[sympy-FTC-part-2]
		>>> integrate(diff(f)) == f  # FTC part 2
		True
		\end{codeblock}

		\noindent
		That's kind of cool, no?
		% TODO: mention application integration by finding anti-derivative functions
		% TODO: explain what it means for integral not to have closed form

		\bigskip
		
		\noindent
		In probability theory,
		the FTC tells us that the probability density can be obtained from the cumulative distribution using differentiation
		\[
			f_X(x) = \frac{d}{dx}\!\left[ F_X(x) \right] = \frac{dF_X}{dx}(x) = F'_X(x).
		\]
		The fact that we can obtain $f_X$ from $F_X$ and vice versa,
		means we only need to define one of the two functions,
		and obtain the other function using differentiation or integration.
		In this book,
		we define the random variable $X$ through its probability distribution function $f_X$,
		then define $F_X$ as the integral of $f_X$.
		In other books,
		you might see the random variable $X$ being defined through its cumulative distribution function $F_X$,
		with its probability density function $f_X$ defined as the derivative of $F_X$.



%!TEX root = ../calculus_tutorial.tex


\section{Integrals}


%!TEX root = ../calculus_tutorial.tex


\subsection{Integrals as area calculations}

	An integral corresponds to the computation of the \emph{area} enclosed between
	the curve $f(x)$ and the $x$-axis over some interval of $x$ values:
	\[
		A_f(a,b) = \int_{x=a}^{x=b} f(x) \: dx.
	\]
	We refer to the numbers $a$ and $b$ as the \emph{limits of integration},
	and the notation $\int_a^b f(x)\:dx$ is shorthand for $\int_{x=a}^{x=b} f(x)\: dx$.

	\begin{figure}[htb]	% LAYOUT
		\centering
		\includegraphics[width=0.3\textwidth]{figures/calculus/integral_as_region_under_curve_Aab.pdf}
		\vspace{-2mm}
		\caption{	The integral of the function $f(x)$ between $x=a$ and $x=b$ corresponds to the shaded area.}
		\label{fig:integral_as_region_under_curve_Aab_repeat}
	\end{figure}

	\noindent
	The notion of an integral is foundational for understanding continuous random variables.
	Every time we compute the probability of some outcome of a continuous random variable,
	there is an integral calculation going on under the hood,
	so integrals is not a topic you can skip.	%, if you want to represent.

	If this is the first time you're learning about integrals,
	it's understandable if you feel intimidated by the complicated math notation,
	but you have to trust me on this one:
	except for the notation,
	there is nothing to worry about!
	In the next few pages,
	I'll do my best to introduce you to the topic of integrals,
	and you'll learn three different ways to do compute integrals.

	Let's start with some examples.


	\subsubsection{Example 1: integral of a constant function}

		Consider the constant function $f(x)=~3$.
		No matter what the input $x$ is,
		the output is always $3$.
		We can easily find the area under the graph of the function $f(x)$ between any two points,
		since the region under the graph has a rectangular shape.
		See Figure~\ref{fig:simple_integral_fx_eq_3} for an illustration.

		The area under $f(x)$ between $x=0$ and $x=5$ corresponds to the following calculation:
		\[
			A_f(0,5) = \int_0^5 f(x)\;dx = 3\cdot 5 = 15.
		\]
		The area under the graph of $f(x)$ is a rectangle with height $3$ and width $5$,
		so its area is $3 \cdot 5 = 15$.

		\begin{figure}[htb]
			\centering
			\includegraphics[width=0.3\textwidth]{figures/calculus/simple_integral_fx_eq_3.pdf}
			\caption{The area of a rectangle of height $3$ and width $5$ is equal to $15$.}
			\label{fig:simple_integral_fx_eq_3}
		\end{figure}


	\subsubsection{Example 2: integral of a linear function}
	
		Consider now the area under the graph of the line $g(x)=x$ between $x=0$ and $x=5$,
		as shown in Figure~\ref{fig:simple_integral_gx_eq_x}.
		Since the region under the curve is triangular,
		we can compute its area using the formula for the area of a triangle,
		which is ``base times height divided by 2.''
	
		The integral of $g(x)$ from $x=0$ until $x=5$ is described by the following calculation:
		\[
			A_g(0,5) = \int_0^5 g(x) \; dx = \tfrac{1}{2} 5 \cdot 5 = \tfrac{1}{2}5^2 = \frac{25}{2} = 12.5.
		\]
	
		\begin{figure}[htb]
			\centering
			\includegraphics[width=0.3\textwidth]{figures/calculus/simple_integral_gx_eq_x.pdf}
			\caption{The area of a triangle with base $5$ and height $5$ is equal to $\frac{1}{2}5^2=\frac{25}{2}=12.5$.}
			\label{fig:simple_integral_gx_eq_x}
		\end{figure}


	\bigskip
	\noindent
	I hope these examples helped you see that the scary-looking integral sign is not that complicated after all.
	It's just a fancy way to describe ``area under the graph of a function'' calculations.



\subsection{Properties of integrals}

	We'll now state some properties of integrals that follow from their interpretation as area calculations.

	\begin{itemize}
	
		\item \textbf{Additivity.}
			The integral from $a$ to $b$ plus the integral from $b$ to $c$ is equal to the integral from $a$ to $c$:
			\[
				\int_a^b f(x) \; dx + \int_b^c f(x) \; dx		=	\int_a^c f(x) \; dx.
			\]

		% TODO: add backward steps giving negative?

		\item \textbf{Constant multiple of a function.}
			The integral of the function $cf(x)$ is equal to $c$ times the integral of $f(x)$,
			for any constant $c$:
			\[
				\int cf(x)\; dx	=	c\int f(x)\; dx.
			\]

		\item \textbf{Sum of two functions.}
			The integral of a sum of two functions is equal to the sum of the integrals of the individual functions:
			\[
				\int [f(x) + g(x)]\; dx	=	\int f(x)\; dx +  \int g(x)\; dx.
			\]

		\item \textbf{Linearity.}
			The combination of the above two properties tells us that integration is a \emph{linear} operation,
			meaning it preserves linear combinations.
			The integral of the linear combination of two functions $\alpha f(x) + \beta g(x)$,
			is equal to the same linear combination of the integrals of the two functions:
			\[
				\int [\alpha f(x) + \beta g(x)]\; dx 
				= \alpha  \int f(x)\; dx  \; \; + \; \; \beta \int g(x)\; dx,
			\]
			where $\alpha$ and $\beta$ are two arbitrary constants.

		\item \textbf{Integral at a single point.}
			Integrals over intervals with zero length have zero value for any function $f(x)$:
			\[
				\int_a^a f(x)\; dx	=	0.
			\]
			Thinking geometrically,
			this integral defines a region with height $f(x)$ and width~$0$,
			so it corresponds to zero area.
			% see https://www.khanacademy.org/math/ap-calculus-ab/ab-integration-new/ab-6-6/v/same-integration-bounds

	\end{itemize}

	% exercise https://www.khanacademy.org/math/ap-calculus-ab/ab-integration-new/ab-6-6/a/definite-integrals-properties-review




LETS SEE SOME CODE



			Relax, we won't be doing the calculation by hand.
			We can write a computer program and make a computer performs the integration procedure for us.
			Here is a sample code that takes an arbitrary function \tt{func}
			and performs the $n$-rectangle area approximation calculation:

			\begin{codeblock}[]
			def integrate(func, a, b, n):
			    """
			    Compute the area under `func` between x=`a` and x=`b`
			    using an approximation with `n` rectangles.
			    """
			    dx = (b-a)/n               # width of each rectangle
			    total = 0.0                # accumulator variable for S_n
			    k = 1                      # counter variable
			    x = a + dx                 # start at first right endpoint
			    while k <= n:              # repeat n times:
			        total += func(x)*dx    #   s_k = height * width
			        x += dx                #   move one step to the right
			        k += 1                 #   increment counter
			    return total
			\end{codeblock}

			\noindent
			The logic of the of the sample code follows closely follows procedure we defined in the equations above.
			We variable \code{dx} holds the information about the width of the rectangles used in the approximation $\Delta x$,
			and we use the counter variable \code{k} to step through the interval $[a,b]$ using $n$ steps of width $\Delta x$.
			
			We can then use this code to compute the integral of any function.
			To do this,
			we must first define the function we want to integrate:

			\begin{codeblock} 
			def f(x):
			    return x**3 - 5*x**2 + x + 10
    			\end{codeblock}

			\noindent
			Then you can compute $S_{25}$ by calling \code{integrate(f, -1, 4, 25)},
			which returns $S_{25}=12.4$.
			Calling \code{integrate(f, -1, 4, 50)} you'll obtain $S_{50}=12.6625$.
			The approximations $S_n$ get better and better as the number of rectangles used in the approximations grows.
			For $n=100$,
			the sum of the rectangles' areas is $S_{100} =  12.7906$,
			for $n=1000$ the approximation gives us $S_{1000} = 12.9041562$,
			which is accurate to the first decimal.


%		We can approximate the total area under the function $f(x)$ between $x=a$ and $x=b$ by splitting the region into $n$ tiny vertical strips of width $\Delta x$,
%		then adding up the areas of the rectangular strips.
%		This is known as a \emph{Riemann sum} approximation for an area.
%		Figure~\ref{fig:riemannsum-25-50} shows the Riemann sum approximations for the area under the function
%		$f(x)=x^3-5x^2+x+10$ between $x=-1$ and $x=4$,
%		obtained by using $n=25$ and $n=50$ vertical rectangular strips.

		\begin{figure}[htb]
			\subfigure[$n=25$]{				
				\includegraphics[width=0.24\textwidth]{figures/calculus/riemannsum-25.png}
			}
			\subfigure[$n=50$]{				
				\includegraphics[width=0.24\textwidth]{figures/calculus/riemannsum-50.png}
			}
			\vspace{-2mm}
			\caption{An approximation to the area under the graph of the function $f(x)=x^3-5x^2+x+10$ 
					using $n=25$ and $n=50$ rectangles.}
			\label{fig:riemannsum-25-50}
		\end{figure}
		
%		As you can see,
%		the approximations get better and better as we increase the number of rectangles.
%		Let's come up with some math expression to describe the $n$-rectangle approximate area calculation.
%		The width of each rectangle is $\Delta x = \frac{4-(-1)}{n}=\frac{5}{n}$.
%		The left endpoint of the first rectangle is at $x=a$ and its right endpoint is at $x=x_1 \equiv a+\Delta x$.
%		Since we're using choosing the height of the rectangles according to their right endpoints,
%		the area of the first rectangle is
%		\[
%			s_1 = f(x_1)\Delta x = f(a + \Delta x)\Delta x,
%		\]
%		which corresponds to the height of the function $f$ at $a+\Delta x$ times the width $\Delta x$.
%		To find the $x$ coordinate of the right endpoint of the second rectangle,
%		we take a step of width $\Delta x$ to the right: $x_2 = x_1 + \Delta x = a + 2 \Delta x$.
%		The area of the second rectangle is $s_2 = f(a+2\Delta x)\Delta x$.
%		
%		We can iterate this one-step-to-the-right procedure to obtain all the right endpoints
%		\[
%			x_{k+1} = x_k + \Delta x, % , \quad \textrm{for all}  \ \ k < n.
%		\]
%		and compute the area of each rectangle using
%		\[
%			s_k = f(x_k)\Delta x =  (x_k^3-5x_k^2+x_k+10)\frac{5}{n}.
%		\]
%		The total area of the $n$-rectangle approximation is the sum of the rectangles' areas:
%		\begin{align*}
%		  S_{n}(a,b) \equiv \sum_{k=1}^n s_k
%			&= \sum_{k=1}^{n} f(a + k\Delta x)\Delta x 				\\
%		  	&= \sum_{k=1}^{n} (x_k^3-5x_k^2+x_k+10)\frac{5}{n}.
%		\end{align*}
%		Wow that looks like a mean math expression!
%		Indeed if you had to do all these calculations by hand,
%		it would take you forever.
%		Computing an approximation with $n=1000$ rectangles requires computing $1000$ rectangle areas
%		and the sum of $1000$ terms!


		In the limit as the number of rectangles $n$ approaches $\infty$, 							\index{infinity}
		the approximation to the area under the curve becomes \emph{arbitrarily close} to the true area.
		The notion of applying the a rectangular-strip approximation to the area of a function,
		where the number of rectangles grows to infinity is known as the \emph{Riemann sum}
		and is the basis for the definitions of the integral:

		The definite integral between $x=a$ and $x=b$ is \emph{defined} as the limit of a 			\index{integral}
		Riemann sum as $n$ goes to infinity:											\index{Riemann sum|textit}
		\[
			\int_{a}^{b}\!f(x)\:dx 
				\equiv \lim_{n\to\infty} \sum_{k=1}^{n} f(a + k\Delta x)\Delta x \equiv A(a,b).
		\]
	





%!TEX root = ../calculus_tutorial.tex


	\subsection{Integrals as functions}
	
		The \emph{integral function} $F_0(b)$ corresponds to the area calculation
		with a variable upper limit of integration $A_f(0,b)$.
		The variable $b$,
		which serves as the input for the integral function $F_0$,
		corresponds to the upper limit of integration in the following calculation:
		\[
			F_0(b) \;\; \eqdef \;\; A_f(0,b) = \int_{x=0}^{x=b} \! f(x)\:dx\,.
		\]
		There are two variables and one constant in this formula.
		The input variable $b$ describes the upper limit of integration.
		The \emph{integration variable} $x$ performs a sweep from $x=0$ until $x=b$.
		The constant $0$ describes the lower limit of integration.
		As a matter of convention,
		we'll always denote the integral function using the capital letter of the same letter as the original function.
	
		Note that choosing $x=0$ for the starting point of the integral function was an arbitrary choice,
		and we obtain another integral function if we use $x=a$ as the starting point,
		$F_a(b)=\int_a^b \! f(x)\,dx$.
		Two integral functions differ only by a constant term.
		For example,
		$F_0(b) = F_a(b) + C$,
		where $C = \int_{x=0}^{x=a} f(x)\,dx$.
	
		The integral function $F(b)$ contains the ``precomputed'' information about the area under the graph of $f(x)$.
		% The integral function $F(b)$ tells us the ``area under the graph'' property of the function $f(x)$ for \emph{all} possible limits of integration.
		The area under $f(x)$ between $x=a$ and $x=b$ can be obtained by calculating the \emph{change} in the integral function as follows:
		\[
			A_f(a,b) = \int_a^b \! f(x)\,dx	=  F(b)-F(a).
		\]
		Intuitively,
		this formula computes the area $A_f(a,b)$ as the difference between the areas of two regions:
		the area until $x=b$ minus the area until $x=a$,
		as illustrated in Figure~\ref{fig:integral_as_difference_off}.
	
		\begin{figure}[htb]
			\centering
			\includegraphics[width=0.4\textwidth]{figures/calculus/integral_as_difference_off.pdf}
			\caption{	The area under $f(x)$ between $x=a$ and $x=b$ is computed using the formula $A_f(a,b)=F_0(b)-F_0(a)$,
					which describes the change in the output of $F_0(x)$ between $x=a$ and $x=b$.}
			\label{fig:integral_as_difference_off}
		\end{figure}
	

TODO: warn there is no general F for any f
only for certain special cases
have exact symbolic formula
for all other cases
we're forced to do the split-into-vertical-strips --- i.e. there is no analytical shortcut.


	
		% TODO: explain intuition: area until b  minus area until a equals area between a and b
	
	
	
		\subsubsection{Example 1 revisited}
	
			We can easily find the integral function for the constant function $f(x)=~3$,
			because the region under the curve is rectangular.
			Choosing $x=0$ as the starting point,
			we obtain the integral function $F_0(b)$
			that corresponds to the area under $f(x)$ between $x=0$ and $x=b$
			as follows:
			\[ 
				F_0(b) = A_f(0,b) = \int_0^b \! f(x)\,dx	= 3 b.
			\]
			The area is equal to the rectangle's height times its width,
			as illustrated in Figure~\ref{fig:simple_integral_function_fx_eq_3}.
	
			\begin{figure}[htb]
				\centering
				\includegraphics[width=0.4\textwidth]{figures/calculus/simple_integral_function_fx_eq_3.pdf}
				\caption{The area of a rectangle of height $3$ and width $b$ is equal to $3b$.}
				\label{fig:simple_integral_function_fx_eq_3}
			\end{figure}
			
			Knowing the function $F_0(b)$ allows us to compute the area under the graph of $f(x)$
			between any two points $x=a$ and $x=b$ using the formula $A_f(a,b) = F_0(b) - F_0(a) = 3(b-a)$.
	
	
		\subsubsection{Example 2 revisited}
		
			Consider now the area under the graph of the line $g(x)=x$, starting from $x=0$.
			Since the region under the curve is triangular,
			we can compute its area using the formula for the area of a triangle:
			base times height divided by two.
		
			The general formula for the area under $g(x)$ from $x=0$ until $x=b$
			is described by the following integral calculation:
			\[
				G_0(b) = A_g(0,b) = \int_0^b g(x) \, dx = \tfrac{1}{2} ( b\times b ) = \tfrac{1}{2}b^2.
			\]
		
			\begin{figure}[htb]
				\centering
				\includegraphics[width=0.4\textwidth]{figures/calculus/simple_integral_function_gx_eq_x.pdf}
				\caption{The area of a triangle with base $b$ and height $b$ is equal to $\frac{1}{2}b^2$.}
				\label{fig:simple_integral_function_gx_eq_x}
			\end{figure}
			
			Knowing the function $G_0(b)$ allows us to compute the area under the graph of $g(x)$
			between $x=a$ and $x=b$ as the difference $A_g(a,b) = G_0(b) - G_0(a) = \frac{1}{2}b^2 - \frac{1}{2}a^2$.
	
	
		%	But don't worry,
		%	you don't need to take an integral calculus to learn statistics.
		%	What is important right now is that you understand the concept of integration.
		%	The integral of a function is the area under the graph of the function,
		%	which is in some sense the total amount of the function accumulated during some interval of time.
	
		\subsubsection{Example 3 revisited}
		
			We 
			\[
				H_{-1}(b) = A_h(-1,b) = \int_{-1}^b h(x) \, dx = ?
			\]
			
			one of the special cases where there IS a shotctu
			
			
		
	
	
	
\bigskip
\noindent
We were able to compute the above integrals thanks to the simple geometries of the areas under the graphs.
%	Computing integrals of more complicated functions requires more advanced techniques.
%	There is an entire course called integral calculus which is dedicated to the task of finding integrals
%	using various tricks and techniques.


%	Taking a calculus course would be useful if you plan to study physics or engineering,
%	but for the purpose of learning probability and statistics,
%	you're not required to learn all these integration techniques.
%	Instead,
%	you can rely on computers to do integration for you.
%	Specifically,
%	you can use the Python modules SciPy and SymPy to compute all the integrals you need,
%	as we'll show in the next two sections.

% Robyn said: 	Confusing that before you said that integrals is a topic that can't be skipped,
%			and here you say that you can rely on computers.
% 			If they don't need to know the math, then perhaps this section could be made even shorter.

%!TEX root = ../calculus_tutorial.tex


\subsection{Computing integrals using SymPy}

	We can also use Python to do \emph{symbolic} integration using variables (symbols) instead of numbers.
	Symbolic integration allows us to obtain exact formulas for integrals
	that are valid for \emph{any} limits of integration $x=a$ and $x=b$.
	The Python module \tt{sympy} provides the functionality for doing symbolic math calculations
	similar to the calculation you could do using pen and paper.

	The following code block imports the SymPy function \tt{symbols},
	which is used to define new symbolic variables,
	and the function \tt{integrate} that we'll use for computing integrals.

	\begin{codeblock}[import-sympy-symbols-and-integrate]
	>>> from sympy import symbols, integrate
	\end{codeblock}
	
	\noindent
	Next we define four symbols \tt{x}, \tt{a}, \tt{b}, and \tt{c},
	which we'll use to denote mathematical variables and constants in the following code examples.

	\begin{codeblock}[define-symbols-xabcm]
	>>> x, a, b, c = symbols('x a b c')
	\end{codeblock}
	% TODO: explain symbol variables like x, a, b, c is symbol can be used in expressions

	\subsubsection{Example 1 revisited again}

		Consider the constant function $f(x) = c$.
		The symbolic expression that represents the value of this function is simply the constant $c$,
		which we can define as follows:

		\begin{codeblock}[sympy-define-fx]
		>>> fx = c
		>>> fx
		c
		\end{codeblock}

		\noindent
		The variable \tt{fx} is defined as the constant \tt{c},
		one of the SymPy symbols we defined earlier,
		which we assume corresponds to some unspecified constant value.

		To compute the integral $\int_a^b f(x) dx$,
		we call the SymPy function \tt{integrate},
		passing in the expression we want to integrate as the first argument.
		The second argument is a triple $(x,a,b)$,
		which specifies the variable of integration $x$,
		the lower limit of integration $a$,
		and the upper limit of integration $b$.

		\begin{codeblock}[sympy-integrate-fx-a-b]
		>>> integrate(fx, (x,a,b))  # = A_f(a,b)
		c*(b-a)
		\end{codeblock}

		\noindent
		Since $a$, $b$, and $c$ are arbitrary constants,
		the expression we obtain for $A_f(a,b) = \int_a^b f(x) dx$ is a general purpose formula
		that works for all functions $f(x) = c$ and all possible integration intervals $[a,b]$.
		Geometrically speaking,
		this is just the height-times-width formula for the area of a rectangle.

		To compute the specific integral between $a=0$ and $b=5$ under the graph of $f(x)=3$,
		we can use the method \tt{subs} (short for substitute) on the SymPy expression we obtained as a result of the integration.
		The \tt{subs} method expects as inputs a Python dictionary whose keys are symbols,
		and whose values represent the numbers we want to ``plug'' into the expression.
		In our case,
		we want to make the substitutions $c=3$, $a=0$, and $b=5$.

		\begin{codeblock}[integrate-fx-subs-vals]
		>>> integrate(fx, (x,a,b)).subs({c:3, a:0, b:5})
		15
		\end{codeblock}

		\noindent
		This result matches the value we obtained using the intuitive geometrical calculation (see Figure~\ref{fig:simple_integral_fx_eq_3})
		and the value we obtained using numerical integration, \tt{quad(f,0,5) = 15}.

		We can also use SymPy to compute the integral function $F_0(b)$,
		which is defined as  $F_0(b) \eqdef \int_0^b f(x) dx$,
		for the function $f(x) = \tt{fx}$.

		\begin{codeblock}[sympy-integral-function-F]
		>>> integrate(fx, (x,0,b))  # = F_0(b)
		b*c
		\end{codeblock}

		\noindent
		Recall that the integral function $F_0$ is simply the area-under-the-graph calculation
		with a variable upper limit of integration $b$.
		See Figure~\ref{fig:simple_integral_function_fx_eq_3} for an illustration of the integral function $F_0(b)$.



	\subsubsection{Example 2 revisited again}

		Let's now compute some integrals of the function $g(x) = x$.
		First we'll define a SymPy expression that corresponds to the function.

		\begin{codeblock}[sympy-define-gx]
		>>> gx = 1*x
		>>> gx
		x
		\end{codeblock}

		\noindent
		We can now compute the integral $\int_a^b g(x) dx$
		by calling the function \tt{integrate} with arguments \tt{gx},
		followed by the triple specifying the variable of the integration and the limits of integration.

		\begin{codeblock}[sympy-integrate-gx-a-b]
		>>> integrate(gx, (x,a,b))  # = A_g(a,b)
		b**2/2 - a**2/2
		\end{codeblock}

		\noindent
		To obtain the numerical value for the integral  $\int_0^5 g(x) dx$,
		we call the method \tt{subs} on the result of the integration.

		\begin{codeblock}[integrate-gx-subs-vals]
		>>> integrate(gx, (x,a,b)).subs({a:0, b:5})
		25/2
		\end{codeblock}

		\noindent
		SymPy computed the exact answer for us as a fraction $\frac{25}{2}$,
		but we sometimes want to force the answer to be computed as a floating-point number (a Python \tt{float}),
		which we can do by calling the \tt{.evalf()} method on the SymPy expression.				

		\begin{codeblock}[integrate-gx-subs-vals-evalf]
		>>> integrate(gx, (x,a,b)).subs({a:0, b:5}).evalf()
		12.5
		\end{codeblock}

		\noindent
		This result matches the value we obtained earlier using numerical integration,
		\tt{quad(g,0,5) = 12.5}.

		If we use the symbol \tt{b} for the upper limit of integration,
		we can obtain an expression for the integral function $G_0(b) \eqdef \int_0^b g(x) dx$.

		\begin{codeblock}[sympy-integral-function-G]
		>>> integrate(gx, (x,0,b))  # = G_0(b)
		b**2 / 2
		\end{codeblock}

		\noindent
		Note the expression for $G_0(b)$ we obtain from SymPy is identical
		to the formula we obtained earlier using a geometrical calculation (the area of a triangle with base $b$ and height $b$).
		See Figure~\ref{fig:simple_integral_function_gx_eq_x}.


	\bigskip
	\noindent
	Unfortunately,
	it's not always possible to use symbolic manipulations to find integrals.
	We can only use \tt{sympy.integrate} for certain simple examples
	where it is possible to obtain exact expressions for integral functions.
	For most practical calculations in probability and statistics,
	we'll need to rely on the \tt{scipy.integrate} function \tt{quad(f,a,b)},
	which computes the integral $\int_a^b f(x)dx$ for \emph{any} function $f(x)$ expressed as a Python function \tt{f}.


%!TEX root = ../calculus_tutorial.tex



	\subsection{Fundamental theorem of calculus}
	
		The fundamental theorem of calculus (FTC) is a deep insight about the
		inverse relation that exists between the operations of integration $\int \cdot dx$
		and differentiation $\frac{d}{dx}[\cdot]$.

		%	The integral function $F_a(x)$ is obtained from the original function $f(x)$ using integration,
		%	$F_a(x) = \int_a^x f(u) du$.
		%	Another way to describe this is to say we \emph{applied} the integration operator $\int \cdot dx$
		%	on the function $f(x)$ to obtain the integral function $F_a(x)$.
		%	The derivative function $f'(x)$ is defined by the formula $f'(x) = \lim_{\delta \to 0} \frac{f(x+\delta)\ - \ f(x)}{\delta}$.
		%	We can also say we \emph{applied} the derivative operator $\frac{d}{dx}[\cdot]$
		%	to the function $f(x)$ to obtain the derivative function $f'(x)$.
		%	I use the word ``operator'' here to refer to an operation that acts on functions.

		A priori,
		there is no reason to suspect the integral function would be related to the derivative operation.
		The integral corresponds to the computation of an area,
		whereas the derivative operation computes the slope of a function.
		The fundamental theorem of calculus describes the relationship between
		derivatives and integrals.

		\begin{shadetheorem}[fundamental theorem of calculus]
			Let $f(x)$  be a continuous function on the interval $[a,b]$,										\index{continuous function}
			and let $\alpha \in \mathbb{R}$ be a constant.
			Define the function $A_\alpha(x)$ as follows:
			\[
			  A_\alpha(x)  \equiv A(\alpha, x) =  \int_\alpha^x f(u) \: du.
			\]
			Then, the derivative of $A_\alpha(x)$ with respect to $x$ is equal to $f(x)$:
			\[
			  \frac{d}{dx}\!\big[A_\alpha(x)\big] = f(x),
			\]
			for any $x \in (a,b)$.
		\end{shadetheorem}

		%	The fundamental theorem of calculus establishes an equivalence between the set 
		%	of integral functions and the set of antiderivative functions:
		%	\[
		%		A_\alpha(x)=F(x)+C.
		%	\]
		%	All integral functions $A_\alpha(x)$ are antiderivatives of $f(x)$.
	
		Differential calculus and integral calculus are two sides of the same coin.
		If you understand why the theorem is true, 
		you will understand something very deep about calculus. 
		Differentiation is the inverse operation of integration.


		%	There is no reason \emph{a priori} to think that integration and differentiation might be related:
		%	the former is a calculation about areas,
		%	while the latter is a calculation about slopes.
		%	The fundamental theorem of calculus reveals that they are in fact inverse operations:
		%	we can obtain the original function $f(x)$ from the integral function $F_a(x)$ by computing it's derivative:
		%	\[
		%		\frac{d}{dx}\big[F_a(x)\big] 	= 	\frac{d}{dx}\left[\int_a^x f(u) \; du \right]  	= 	f(x).
		%	\]
		%	Note we used a temporary variable $u$ as the integration variable,
		%	since $x$ is already used to denote the upper limit of integration.

		In order to understand the inverse relationship between integration and differentiation,
		we can draw an analogy with the inverse relationship between a function $f$ and its inverse function $f^{-1}$,
		which \emph{undoes} the effects of $f$.
		See Figure~\ref{fig:functions-inverse} on page~\pageref{fig:functions-inverse}.
		Given some initial value $x$,
		if we apply the function $f$ to obtain the number $f(x)$,
		and apply the inverse function $f^{-1}$ on the number $f(x)$,
		then the result will be the initial value $x$ we started from:
		\[
			f^{-1}\!\left( f(x) \right)	=	x.
		\]
		Similarly,
		the derivative operator is the ``inverse operator'' of the integral operator.
		If you perform the integral operation followed by the derivative operation on some function,
		you'll get back to original function:
		\[
			\frac{d}{dx} \int_c^x f(u)\:du = f(x).
		\]
		We can use SymPy to verify the fundamental theorem of calculus.
		First we construct a function \tt{f} and compute its integral function \tt{F} using \tt{integrate}:

		\begin{codeblock}[sympy-FTC-obtain-F]
		>>> from sympy import diff, integrate, log, exp, sin
		>>> f =  log(x) + exp(x) + sin(x)
		>>> F = integrate(f)
		>>> F
		x*log(x) - x + exp(x) - cos(x)
		\end{codeblock}

		If we now take the derivative of the function \tt{F},
		we get back the original function \tt{f}.

		\begin{codeblock}[sympy-FTC-get-back-f]
		>>> diff(F)
		log(x) + exp(x) + sin(x)
		>>> diff(integrate(f)) == f  # FTC part 1
		True
		\end{codeblock}




The integral is the ``inverse operation'' of the derivative.			\index{inverse!operation}
If you perform the integral operation followed by the derivative operation on some function,
you'll obtain the same function:
\[
  \left(\frac{d}{dx} \circ \int dx \right) f(x) = \frac{d}{dx} \int_c^x f(u)\:du = f(x).
\]



\begin{codeblock}[]
>>> f = x**2
>>> F = integrate(f, x)
>>> F
x**3/3           # + C
>>> diff(F, x)
x**2
\end{codeblock}

\noindent
Alternately, if you compute the derivative of a function followed by the integral,
you will obtain the original function $f(x)$ (up to a constant):
\[
  \left( \int dx \circ \frac{d}{dx}\right) f(x) = \int_c^x f'(u)\;du = f(x) + C.
\]



\begin{codeblock}[]
>>> f = x**2
>>> df = diff(f, x)
>>> df
2*x
>>> integrate(df, x)
x**2    # + C
\end{codeblock}




			


		\noindent
		The inverse relationship also holds for the opposite order of application:
		if we take the derivative of some function,
		then compute the integral of the derivative,
		then we arrive back at the original function (up to an additive constant factor).  % due to the arbitrary choice of the starting point for integration
		\[
			\int_c^x f'(u)\;du = f(x) + C.
		\]


		\begin{codeblock}[sympy-FTC-part-2]
		>>> integrate(diff(f)) == f  # FTC part 2
		True
		\end{codeblock}

		%	The integral is the ``inverse operation'' of the derivative.
		%	If you perform the integral operation followed by the derivative operation on some function, 
		%	you'll obtain the same function:
		%	Note we need a new variable $u$ inside the integral since $x$ is already 
		%	used to denote the upper limit of integration.

		% That's kind of cool, no?
		% TODO: mention application integration by finding anti-derivative functions
		% TODO: explain what it means for integral not to have closed form
		
		\ifthenelse{\boolean{FORSTATSBOOK}}{

			\noindent
			In probability theory,
			the FTC tells us that the probability density can be obtained from the cumulative distribution using differentiation
			\[
				f_X(x) = \frac{d}{dx}\!\left[ F_X(x) \right] = \frac{dF_X}{dx}(x) = F'_X(x).
			\]
			The fact that we can obtain $f_X$ from $F_X$ and vice versa,
			means we only need to define one of the two functions,
			and obtain the other function using differentiation or integration.
			In this book,
			we define the random variable $X$ through its probability distribution function $f_X$,
			then define $F_X$ as the integral of $f_X$.
			In other books,
			you might see the random variable $X$ being defined through its cumulative distribution function $F_X$,
			with its probability density function $f_X$ defined as the derivative of $F_X$.

		}{}


		% ALT. by anti-differentiation
		The fundamental theorem of calculus gives us an alternative way for computing integrals.
		You can find integral functions using a table of derivative formulas (see page~\pageref{mathematical_preliminiaries:derivative_formulas})
		and some ``reverse engineering'' thinking.
		To find an integral function of the function $f(x)$,
		we can look for a function function $F(x)$ such that $F'(x)=f(x)$.

		\paragraph{Example}

			Suppose you're given a function $f(x)$ and asked to find its integral function $F(x) = \int  f(x)\: dx$.
			This fundamental theorem of calculus tells us this problem is equivalent to finding a function $F(x)$ whose derivative is $f(x)$: $F'(x) = f(x)$.
			For example, suppose you want to find the indefinite integral $\int x^2\:dx$.
			We can rephrase this problem as the search for some function $F(x)$ such that $F'(x) = x^2$.
			Remembering the derivative formulas we saw above, you guess that $F(x)$ must contain an $x^3$ term.
			Taking the derivative of a cubic term results in a quadratic term.
			Therefore, the function you are looking for has the form $F(x)=cx^3$,
			for some constant $c$.
			Pick the constant $c$ that makes this equation true: $F'(x) = 3cx^2 = x^2$.
			Solving $3c=1$, we find $c=\frac{1}{3}$ and so the integral function is $F(x) = \int x^2 \:dx = \frac{1}{3}x^3 + C$.
			In other words,
			the area under the graph of $f(x)=x^2$ is described by the family of functions $F(x) = \frac{1}{3}x^3 + C$.
			% The constant $C$ varies depending on the choice of lower limit of integration $\alpha$ for the area calculation $A(\alpha, x)$.
			% You can verify that $\frac{d}{dx}\left[\frac{1}{3}x^3 + C\right] = x^2$.


LEAD OUT: what do we do when there is no simple formula?




\input{sections/tecniques_of_integration.tex}
%!TEX root = ../calculus_tutorial.tex


	\subsection{Computing integrals numerically}

		Relax, we won't be doing the calculation by hand.
		We can write a computer program and make a computer performs the integration procedure for us.
		Here is a sample code that takes an arbitrary function \tt{func}
		and performs the $n$-rectangle area approximation calculation:

		\begin{codeblock}[]
		def integrate(func, a, b, n):
		    """
		    Compute the area under `func` between x=`a` and x=`b`
		    using an approximation with `n` rectangles.
		    """
		    dx = (b-a)/n               # width of each rectangle
		    total = 0.0                # accumulator variable for S_n
		    k = 1                      # counter variable
		    x = a + dx                 # start at first right endpoint
		    while k <= n:              # repeat n times:
		        total += func(x)*dx    #   s_k = height * width
		        x += dx                #   move one step to the right
		        k += 1                 #   increment counter
		    return total
		\end{codeblock}

		\noindent
		The logic of the of the sample code follows closely follows procedure we defined in the equations above.
		We variable \code{dx} holds the information about the width of the rectangles used in the approximation $\Delta x$,
		and we use the counter variable \code{k} to step through the interval $[a,b]$ using $n$ steps of width $\Delta x$.
		
		We can then use this code to compute the integral of any function.
		To do this,
		we must first define the function we want to integrate:

		\begin{codeblock} 
		def h(x):
		    return x**3 - 5*x**2 + x + 10
    			\end{codeblock}

		\noindent
		Then you can compute $S_{25}$ by calling \code{integrate(f, -1, 4, 25)},
		which returns $S_{25}=12.4$.
		Calling \code{integrate(f, -1, 4, 50)} you'll obtain $S_{50}=12.6625$.
		The approximations $S_n$ get better and better as the number of rectangles used in the approximations grows.
		For $n=100$,
		the sum of the rectangles' areas is $S_{100} =  12.7906$,
		for $n=1000$ the approximation gives us $S_{1000} = 12.9041562$,
		which is accurate to the first decimal.


%		We can approximate the total area under the function $f(x)$ between $x=a$ and $x=b$ by splitting the region into $n$ tiny vertical strips of width $\Delta x$,
%		then adding up the areas of the rectangular strips.
%		This is known as a \emph{Riemann sum} approximation for an area.
%		Figure~\ref{fig:riemannsum-25-50} shows the Riemann sum approximations for the area under the function
%		$f(x)=x^3-5x^2+x+10$ between $x=-1$ and $x=4$,
%		obtained by using $n=25$ and $n=50$ vertical rectangular strips.

		\begin{figure}[htb]
			\subfigure[$n=25$]{				
				\includegraphics[width=0.24\textwidth]{figures/calculus/riemannsum-25.png}
			}
			\subfigure[$n=50$]{				
				\includegraphics[width=0.24\textwidth]{figures/calculus/riemannsum-50.png}
			}
			\vspace{-2mm}
			\caption{An approximation to the area under the graph of the function $f(x)=x^3-5x^2+x+10$ 
					using $n=25$ and $n=50$ rectangles.}
			\label{fig:riemannsum-25-50}
		\end{figure}
		
%		As you can see,
%		the approximations get better and better as we increase the number of rectangles.
%		Let's come up with some math expression to describe the $n$-rectangle approximate area calculation.
%		The width of each rectangle is $\Delta x = \frac{4-(-1)}{n}=\frac{5}{n}$.
%		The left endpoint of the first rectangle is at $x=a$ and its right endpoint is at $x=x_1 \equiv a+\Delta x$.
%		Since we're using choosing the height of the rectangles according to their right endpoints,
%		the area of the first rectangle is
%		\[
%			s_1 = f(x_1)\Delta x = f(a + \Delta x)\Delta x,
%		\]
%		which corresponds to the height of the function $f$ at $a+\Delta x$ times the width $\Delta x$.
%		To find the $x$ coordinate of the right endpoint of the second rectangle,
%		we take a step of width $\Delta x$ to the right: $x_2 = x_1 + \Delta x = a + 2 \Delta x$.
%		The area of the second rectangle is $s_2 = f(a+2\Delta x)\Delta x$.
%		
%		We can iterate this one-step-to-the-right procedure to obtain all the right endpoints
%		\[
%			x_{k+1} = x_k + \Delta x, % , \quad \textrm{for all}  \ \ k < n.
%		\]
%		and compute the area of each rectangle using
%		\[
%			s_k = f(x_k)\Delta x =  (x_k^3-5x_k^2+x_k+10)\frac{5}{n}.
%		\]
%		The total area of the $n$-rectangle approximation is the sum of the rectangles' areas:
%		\begin{align*}
%		  S_{n}(a,b) \equiv \sum_{k=1}^n s_k
%			&= \sum_{k=1}^{n} f(a + k\Delta x)\Delta x 				\\
%		  	&= \sum_{k=1}^{n} (x_k^3-5x_k^2+x_k+10)\frac{5}{n}.
%		\end{align*}
%		Wow that looks like a mean math expression!
%		Indeed if you had to do all these calculations by hand,
%		it would take you forever.
%		Computing an approximation with $n=1000$ rectangles requires computing $1000$ rectangle areas
%		and the sum of $1000$ terms!


		In the limit as the number of rectangles $n$ approaches $\infty$, 							\index{infinity}
		the approximation to the area under the curve becomes \emph{arbitrarily close} to the true area.
		The notion of applying the a rectangular-strip approximation to the area of a function,
		where the number of rectangles grows to infinity is known as the \emph{Riemann sum}
		and is the basis for the definitions of the integral:

		The definite integral between $x=a$ and $x=b$ is \emph{defined} as the limit of a 			\index{integral}
		Riemann sum as $n$ goes to infinity:											\index{Riemann sum|textit}
		\[
			\int_{a}^{b}\!f(x)\:dx 
				\equiv \lim_{n\to\infty} \sum_{k=1}^{n} f(a + k\Delta x)\Delta x \equiv A(a,b).
		\]
	






	\subsection{Computing integrals numerically using SciPy}
	
		There are numerous ways to compute integrals using Python.
		Computing integrals ``numerically'' means we're splitting the region of integration into thousands or millions of subregions,
		computing the areas of these subregions,
		then adding up the areas of the subregions to obtain the total area.
	
		The Python function \tt{quad} in the module \tt{scipy.integrate} allows us to compute the integral of any function.
		The name \tt{quad} is short for ``quadrature'' which is the historical math term used for find-the-area procedures.
		Let's start by importing the \tt{quad} function.
	
		\begin{codeblock}[import-quad-from-scipy]
		>>> from scipy.integrate import quad
		\end{codeblock}
	
		\noindent
		Now let's define a Python function \tt{f} that corresponds to the constant function $f(x) = 3$.
	
		\begin{codeblock}[deffun-f-eq-3-and-call]
		>>> def f(x):
		        return 3
		>>> f(333)
		3
		\end{codeblock}
		
		\noindent
		No matter what input $x$ we choose,
		the output will always be the same $f(x)=3$.
	
		To compute the integral $\int_0^5 f(x) dx$ we call the function \tt{quad}
		with inputs \tt{f} as the first argument,
		and the limits of integration $a=0$ and $b=5$ as the second and third arguments.
	
		\begin{codeblock}[quad-f-0-5-tuple]
		>>> quad(f, 0, 5)
		(15.0, 1.1102230246251565e-13)
		\end{codeblock}
	
		\noindent
		The function \tt{quad} returns a tuple (a pair of numbers) as output: $(A,\epsilon)$.
		The first number in the tuple is the value of the area calculation.
		The second number $\epsilon$ tells us the accuracy of the procedure used to calculate the area.
		In the above calculation,
		the output tells us the integral $\int_0^5 f(x) dx$ is equal to $15.0$ up to a precision on the order of $10^{-13}$.
	
		Since we're usually only interested in the value of the area $A$ and not the precision $\epsilon$,
		we often select the first number in the output of \tt{quad}.
		This is why you'll often see the expression \tt{quad(...)[0]} in code examples.
	
		\begin{codeblock}[quad-f-0-5]
		>>> quad(f, 0, 5)[0]  # extract A
		15.0
		\end{codeblock}
	
		\noindent
		As a second example,
		let's calculate the area under the graph of the function $g(x)=x$ between $x=0$ and $x=5$.
	
		\begin{codeblock}[deffun-g-eq-x-and-quad-g-0-5]
		>>> def g(x):
		        return x
		>>> quad(g, 0, 5)[0]
		12.5
		\end{codeblock}
	
		\noindent
		The answer we obtained matches the results of the general formula we obtained above,
		$A_g(0,5) = \frac{1}{2}b^2$,
		when the upper limit of integration is $b=5$.
	
		We'll use the function \tt{quad} hundreds of times in the remainder of the book to compute various integrals
		as part of probability and statistics calculations,
		so make sure you understand what is going on in the above code examples.
		The main takeaway message is that the \tt{quad} function is your friend whenever you need to compute integrals.
		All the scary-looking math equations that contain the $\int$ symbol can be computed using one or two lines of Python code.
	
	
	

%!TEX root = ../calculus_tutorial.tex


\subsection{Applications of integration}
\label{mathematical_preliminiaries:applications-of-integration}
	
			One of the key applications of integration to computing probabilities for continuous random variables.
			A continuous random variable $X$ is described by its probability density function $f_X$
			and the probability of the event $\{ a \leq X \leq b\}$ is defined as the following integral:
			\[
				\Pr( \{ a \leq X \leq b\} )
				\equiv
				\int_a^b f_X(x)\:dx.
			\]
			The probability density $f_X$ varies for different values of $x$,
			so if we want to compute the total probability of $X$ falling between $x=a$ and $x=b$,
			we must compute the integral of $f_X$ between $x=a$ and $x=b$.
			
			We also use integration to compute \emph{expectations} for quantities that depend on continuous random variables.
			The expected value of the quantity $G=g(X)$ under the randomness of a continuous random variable $X$
			is defined as the following integral calculation:
			\[
				\EE_X [G]
				\equiv
				\EE_X[g(X)]
				\equiv
				\int_{x \in \calX} g(x)f_X(x)\:dx.
			\]
			The expected value is computed by ``weighing'' each value of $g(x)$ by the corresponding probability density for the event $\{X=x\}$,
			summed over all possible values for the random variable $X$.
	
			The mean $\mu = \EE_X[X]$ and the variance $\sigma^2 = \EE_X[(X-\mu)^2]$
			are two central concepts in probability theory and statistics that are computed as expectation integrals.
			Every time we use the $\EE_X$ notation in Section~\ref{sec:continuous_prob_distr},
			there will be some integral calculation going on behind the scenes,
			so if you want know what's going on you need to know a thing or two about integrals.
	








%!TEX root = ../calculus_tutorial.tex

\section{Sequences and series}
\label{sec:sequences_and_series}

	A sequence is a function of the form $a: \mathbb{N} \to \mathbb{R}$.	
	The sequence's input variable is usually denoted $k$ or $n$,
	and it corresponds to the \emph{index} or number in the sequence.
	We describe sequences either by specifying the formula $a_k$ for the $k$\textsuperscript{th} term in the sequence
	or by listing all the values of the sequence:
	\[
		a_k, k \in \mathbb{N}  \ \ \Leftrightarrow \ \  \left(a_0, a_1, a_2, a_3, a_4, \ldots \, \right).
	\]
	Note the new notation for the input variable as a subscript.
	This is the standard notation for describing sequences,
	and is used instead of the standard function notation $a(k)$.

	We're often interested in computing the sum of all the values in this given a sequence $a_k$.
	To describe the sum of 3\textsuperscript{rd}, 4\textsuperscript{th}, and 5\textsuperscript{th} elements of the sequence $a_k$,
	we turn to summation notation:
    	\[
	      a_3 + a_4 + a_5 
	      \equiv \sum_{3 \leq k \leq 5}\!\! a_k 
	      \equiv \sum_{k=3}^{5} a_k \,.      
	\] 
	The capital Greek letter \emph{sigma} stands in for the word \emph{sum}, 
	and the range of index values included in this sum is denoted below and above the summation sign.

	The partial sum of the sequence values $a_k$ ranging from $k=0$ until $k=n$ is denoted as
	\[
		S_n = \sum_{k=0}^n a_k =  a_0 + a_1 + a_2 + \cdots + a_{n-1} + a_n.
	\]

	In calculus,
	the notion of a \emph{series} describes the sum of \emph{all} the values in the sequence $a_k$:
	\[
	   \sum a_k 
	    	\equiv 	S_\infty	
		= 		\lim_{n \to \infty} S_n
		=		\sum_{k=0}^\infty a_k = a_0+ a_1 + a_2 + a_3 + a_4 + \cdots .
	\]
	Note if the sequence $a_k$ continues indefinitely,
	computing the sum requires an infinite number of addition operations.


		\subsubsection{Exact sums}
		
			Formulas exist for calculating the sum of certain series, even series with infinite number of terms.

			The formulas for the sum of the first $n$ positive integers is
			\[
			   \sum_{k=1}^n k = \frac{n(n+1)}{2}.
			\]
			The the sum of the squares of the first $n$ positive integers is
			\[
			   \sum_{k=1}^n k^2=\frac{n(n+1)(2n+1)}{6}.
			\]
			% MAYBE add k^3 formula too?
			% See problem \textbf{P\ref{problem:infinite_sum_formulas_derivation}} for the derivations of these formulas.

			There is another nice series for powers of $2$:
			\[
			   \sum_{k=0}^n 2^k = 1 + 2 + 4 + 8 + \cdots + 2^n = 2^{n+1} -1.
			\]
			
		
			\noindent
			The formula for the geometric sequence is $a_k = r^k$.
			The sum of the first $n$ terms in the geometric sequence is
			\[
			  S_n = \sum_{k=0}^n r^k
			   = 1 + r + r^2 + \cdots + r^n 
			   =\frac{1-r^{n+1}}{1-r}.
			\]
			If $|r|<1$, taking the limit $n\to \infty$ in the above expression leads to
			\[ 
			  S_\infty 
			   = \lim_{n \to \infty} S_n 
			   = \sum_{k=0}^\infty r^k
			   = 1 + r + r^2 +  r^3 + \cdots 
			   =\frac{1}{1-r}.
			\]

			
			\paragraph{Example}
				Consider the geometric series with $r=\frac{1}{2}$.
				Applying the above formula, we obtain 
				\[
				 S_\infty  
				 	=  \sum_{k=0}^\infty \left(\frac{1}{2}\right)^k
					= 1 + \frac{1}{2} + \frac{1}{4} + \frac{1}{8} + \frac{1}{16} + \frac{1}{32} + \cdots 
					=\frac{1}{1-\frac{1}{2}} = 2.
				\]
				You can visualize this infinite summation graphically in Figure~\ref{fig:geometric_progression_of_one_half}.
		
				\begin{figure}[htb]
				\centering
				\includegraphics[width=0.34\textwidth]{figures/calculus/geometric_progression_of_one_half.png}
				\caption{	A graphical representation of the infinite sum of the geometric series with $r=\frac{1}{2}$.
						The area of each region corresponds to one of the terms in the series.
						The total area is equal to $\sum_{k=0}^\infty (\frac{1}{2})^k=\frac{1}{1-\frac{1}{2}}=2$.}
				\label{fig:geometric_progression_of_one_half}
				\end{figure}


%TODO explain
The Binomial series
\[
	\sum_{k=0}^n {n \choose k} a^{n-k} b^k=(a+b)^n
\]
special case when one of the terms is 1:
\[
	\sum_{k=0}^n {n \choose k} x^k=(1+x)^n
\]


	\subsection{Taylor series}
	\label{series:taylor_series}

		The \emphindexdef{Taylor series} of a function $f(x)$ approximates the function by an infinitely long polynomial:
		\[
		    f(x)
			= \sum_{k=0}^\infty c_k x^k
			=  c_0 + c_1x + c_2x^2 + c_3x^3 + c_4x^4 + \cdots \,.
		\]
		Each term in the series is of the form $a_k=c_k x^k$, 
		where the coefficient $c_k$ depends on the properties of the function $f(x)$.
		Specifically,
		$c_k = \frac{f^{(k)}(0)}{k!}$,
		where $f^{(k)}(0)$ is the $k$\textsuperscript{th} derivative of $f(x)$ and $k!$ is the factorial function:
		\begin{align*}
		  f(x)
		 	& =f(0)+f'(0)x+\frac{f^{\prime\prime}(0)}{2!}x^2+\frac{f^{\prime\prime\prime}(0)}{3!}x^3 +\frac{f^{(4)}(0)}{4!}x^4 + \cdots \\
		 	& = \sum_{k=0}^\infty \frac{f^{(k)}(0)}{k!}x^k.
		\end{align*}
		Using this formula and your knowledge of derivatives,
		you can compute the Taylor series of any function $f(x)$.

		For example,
		let's find the Taylor series of the function $f(x)=e^x$ at $x=0$.
		The first derivative of $f(x)=e^x$ is $f'(x)=e^x$.
		The second derivative of $f(x)=e^x$ is $f''(x)=e^x$.
		In fact,
		all the derivatives of $f(x)$ will be $e^x$ because the $e^x$ is a special function that is equal to its derivative!
		The $k$\textsuperscript{th} coefficient in the power series of $f(x)=e^x$ at the point $x=0$ 
		is equal to the value of the $k$\textsuperscript{th} derivative of $f(x)$ evaluated at $x=0$.
		In the case of $f(x)=e^x$ we have $f^{(k)}(0)=e^0=1$,
		so the coefficient of the $k$\textsuperscript{th} term is $c_k = \tfrac{f^{(k)}(0)}{k!}  = \tfrac{1}{k!}$.
		The Taylor series of $f(x)=e^x$ is
		\[
		 e^x      	= \sum_{k=0}^\infty \frac{1}{k!}x^k
		 	 	= 1 + x + \frac{x^2}{2} + \frac{x^3}{3!} + \frac{x^4}{4!} + \frac{x^5}{5!} + \cdots 
		 \]
		Series are a powerful computational tool for approximating numbers and functions.
		As we compute more terms from the above series,
		our the polynomial approximation to the function $f(x)=e^x$ becomes more accurate.
		The exact value of the function at $x=1$ is $f(1) = e^1 = e$.
		The partial sum of the first six terms (as shown above) gives us an approximation of $e^1$ that is accurate to three decimals. 
		The partial sum of the first 12 terms gives us $e$ to an accuracy of nine decimals.
		% http://bit.ly/12DrCZY


\ifthenelse{\boolean{FORSTATSBOOK}}{
	TODO: any extra series formulas or concepts required to solve the exercises and problems in noBSstats.
}{}





%!TEX root = ../calculus_tutorial.tex

\section{Multivariable calculus}
\label{mathematical_preliminiaries:multivariable_calculus}

		Multivariable calculus is the extension of the ideas of differential and integral calculus
		to functions like $f(x,y)$ that depend on multiple input variables.
		You can plot a function $f: \mathbb{R} \times \mathbb{R}  \to \mathbb{R}$ as a \emph{surface},
		where the height $z$ of the surface above the point $(x,y)$ is function output $z=f(x,y)$.

		If you know single-variable calculus (derivatives and integrals),
		then then you won't have much new math to learn in multivariable calculus:
		it's essentially the same concepts but with more variables.

			
		\subsection{Partial derivatives}
			
			For a function of two variables $f(x,y)$,
			there is an ``$x$-derivative'' operator $\frac{\partial}{\partial x}$
			and a ``$y$-derivative'' operator $\frac{\partial}{\partial y}$.
			The operation $\frac{\partial}{\partial x}f(x,y)$ describes taking the derivative of $f(x,y)$ with respect to the input variable $x$,
			while keeping the input  variable $y$ constant.
			Taking the derivative of a multivariable function with respect to one of its input variables is called a \emph{partial derivative}
			and denoted with the symbol $\partial$.

			The partial derivative of $f(x,y)$ with respect to $x$ is
			\[
				\frac{\partial}{\partial x}f(x,y)
				\equiv
				\frac{\partial f}{\partial x}
				\equiv
				\lim_{ \delta \rightarrow 0}	\frac{f(x+\delta, y)-f(x,y)}{\delta}.
			\]
			Similarly the partial derivative of with respect to $y$ is
			\[
				\frac{\partial}{\partial y}f(x,y)
				\equiv
				\frac{\partial f}{\partial y}
				\equiv
				\lim_{ \delta \rightarrow 0}	\frac{f(x, y+\delta)-f(x,y)}{\delta}.
			\]
			Note that both $\frac{\partial f}{\partial x}$  and $\frac{\partial f}{\partial y}$ are function of $x$ and $y$.
			Indeed, we can ask the questions ``what is the slope in the $x$-direction''
			and ``what is the slope in the $y$-direction'' at any point $(x,y)$ on the surface of the function.
			That's precisely the information returned by the functions $\frac{\partial f}{\partial x}(x,y)$ and $\frac{\partial f}{\partial y}(x,y)$.

			TODO: example
			
			
		\subsection{Gradient}
		\label{mathematical_preliminiaries:gradient}


			The operator $\nabla$ is a combination of both the $x$ and $y$ derivatives:
			\[
				\nabla f(x,y)
				\equiv
				\left(
					\frac{\partial f}{\partial x},
					\frac{\partial f}{\partial y}
				\right).
			\]
			Note that $\nabla$ acts on a function $f(x,y)$ to produce a vector output.
			This vector is called the \emph{gradient} vector and it tells you the combined $x$- and $y$-slopes of the surface.
			More specifically,
			the gradient vector tells you the direction of the function's maximum increase---the
			``uphill'' direction at the surface of graph of $f(x,y)$ at the point $(x,y)$.
			
			
			\paragraph{Mountain map}
				Suppose the height of a mountain is described by the function $h(x,y)$.
				The coordinates $(x,y)$ tell us the horizontal position point in the $xy$-plane
				and the value of the function $h(x,y)$ represents the height of the mountain at those coordinate.

				We identify the $z$ coordinate with the hight of the mountain $z=h(x,y)$
				and graph the function $h(x,y)$ is as a surface in 3D as illustrated in Figure~\ref{fig:multivariable_caclulus_3d_surface_plot}.

				\begin{figure}[htb]
				\centering
				\includegraphics[width=0.3\textwidth]{figures/calculus/multivariable_caclulus_3d_surface_plot.pdf}
				\vspace{-2mm}
				\caption{The 3D surface plot of the the function $h(x,y)$.}
				\label{fig:multivariable_caclulus_3d_surface_plot}
				\end{figure}

				\noindent			
				Three dimensional surface plots are very good for visualizing multivariable functions,
				but they can be difficult to draw by hand.

				Another approach for representing function of the form $h(x,y)$ is to use a two-dimensional plot that shows the ``view from above'' of the surface $h(x,y)$.
				We can use colour to represent the height of the function through different sharing:
				darker-shading to represent large values of $h(x,y)$ and lighter-shading to represent small values of $h(x,y)$.			% TODO: update plot to show shading
				We can also trace \emph{level curves} in the plot,
				which is the same approach used for topographic maps:
				each level curve show the points that are at a certain height.
				
				\begin{figure}[htb]
				\centering
				\includegraphics[width=0.3\textwidth]{figures/calculus/multivariable_caclulus_topographic_map.png}
				\vspace{-4mm}
				\caption{The topographic map that shows the height of the function $h(x,y)$ using sharing to represent height.
						The level curves at each 10m intervals are also shown.}
				\label{fig:multivariable_caclulus_topographic_map}
				\end{figure}
				
				\noindent
				The curve labeled 30m line you see in Figure~\ref{fig:multivariable_caclulus_topographic_map} represents
				the solution to the equation $30= h(x,y)$,
				where $h(x,y)$ is the height of this hill for all coordinates $(x,y)$ on the map.
				
				Recall that the gradient vector $\nabla f(x,y)$ at any point $(x,y)$ tells you which way is ``uphill'' on the surface,
				and by extension,
				the negative of the gradient vector points ``downhill.''
				The gradient vector is always perpendicular to the \emph{level curve} at that point.

				The notion of an uphill or downhill direction for the surface $h(x,y)$ turns out to be very useful for optimization.
				If you want to find the local maximum of a function,
				you can start at some point and keep moving uphill (in the direction of $\nabla f(x,y)$ and you'll arrive at a local peak of the mountain.
				Similarly,
				to find the lowest point on the surface (minimum value of $h(x,y)$),
				you can start at some point and keep moving in the opposite direction to the gradient $-\nabla f(x,y)$.

				Figure~\ref{fig:multivariable_caclulus_topographic_map_waterflow_to_bottom} illustrates this process.
				Consider the path of a water stream whose source in some arbitrary point on the surface of the mountain.
				The water stream will naturally move downhill
				and descend the slope of the mountain until it reaches the minimum at the bottom of a valley.
				This intuitive notion of ``keep moving downhill until you get to a local minimum'' is
				the general idea behind the \emph{gradient descent} optimization algorithm which is very important for machine learning applications.
							
				\begin{figure}[htb]
				\centering
				\includegraphics[width=0.3\textwidth]{figures/calculus/multivariable_caclulus_topographic_map_waterflow_to_bottom.png}
				\vspace{-4mm}
				\caption{This graph shows the path taken by a hypothetical water as it flows to the bottom of the valley $h(x,y)$.}
				\label{fig:multivariable_caclulus_topographic_map_waterflow_to_bottom}
				\end{figure}

				We know we've reached the bottom of the valley,
				since the the gradient vector will be zero at the minimum of the function $h(x,y)$,
				since surface is locally flat there.



		\subsection{Multivariable integrals}
		
			The multivariable generalization of the integral $\int_{x \in I} f(x) \, dx$
			that computes the ``total'' amount of $f(x)$ on some interval $I=[a,b]$
			is the multivariable integral of the form:
			\[
				\int \! \int_{(x,y) \in R} f(x,y) \, dxdy,
			\]
			where $R$ is called the \emph{region of integration} and corresponds to some subset of the cartesian plane $\mathbb{R} \times \mathbb{R}$.
			The idea behind multivariable integrals is the same as for single variable integrals---to compute the total amount of some function for some range of input values.
			For single-variable integrals,
			we split the region into thing rectangular strips of width $dx$.
			For double integrals we split the two-dimensional region of integration into small squares of area $dxdy$,
			and compute the total volume of a many vertical columns whose base area is $dxdy$
			and whose height is given by the function $f(x,y)$.
			
			TODO: insert graphic of 3D integral split into vertical columns
			
			TODO: explain "sweep along x then sweep along y" idea + hint at change-of-variables techniques
			\vspace{1in}






	

\clearpage

%!TEX root = ../calculus_tutorial.tex

\section{Calculus using SymPy}
% \label{}	


Calculus is the study of the properties of functions.
The operations of calculus are used to describe the limit behaviour of functions,
calculate their rates of change,
and calculate the areas under their graphs.
In this section we'll learn about the \texttt{SymPy} functions for calculating
limits, derivatives, integrals, and summations.

\subsection{Infinity}
\label{calculus:infinity}

\small
\begin{verbatimtab}
from sympy import oo
\end{verbatimtab}
\normalsize

\noindent
The infinity symbol is denoted \texttt{oo} (two lowercase \texttt{o}s) in \texttt{SymPy}.						\index{infinity}
Infinity is not a number but a process: the process of counting forever.
Thus, $\infty + 1 = \infty$, 
$\infty$ is greater than any finite number,
and $1/\infty$ is an infinitely small number.
Sympy knows how to correctly treat infinity in expressions:

\small
\begin{verbatimtab}
>>> oo+1
oo
>>> 5000 < oo 
True
>>> 1/oo
0
\end{verbatimtab}
\normalsize

\subsection{Limits}
\label{calculus:limits}
\small
\begin{verbatimtab}
from sympy import limit
\end{verbatimtab}
\normalsize

\noindent
We use limits to describe, with mathematical precision, infinitely large quantities,						\index{limit}
infinitely small quantities, and procedures with infinitely many steps.

The number $e$ is defined as the limit
$\displaystyle e \eqdef \lim_{n\to \infty} \left( 1 + \frac{1}{n}\right)^{n}$:
\small
\begin{verbatimtab}
>>> limit( (1+1/n)**n, n, oo)
E          # = 2.71828182845905
\end{verbatimtab}
\normalsize

\noindent
This limit expression describes the annual growth rate of a loan
with a nominal interest rate of $100\%$ and infinitely frequent compounding.							\index{interest rate}
Borrow $\$1000$ in such a scheme,
and you'll owe $\$2718.28$ after one~year.

Limits are also useful to describe the behaviour of functions.
Consider the function $f(x)=\frac{1}{x}$.
The \texttt{limit} command shows us what happens to $f(x)$ near $x=0$ and as $x$ goes to infinity:

\small
\begin{verbatimtab}
>>> limit( 1/x, x, 0, dir="+")
oo
>>> limit( 1/x, x, 0, dir="-")
-oo
>>> limit( 1/x, x, oo)
0
\end{verbatimtab}
\normalsize

\noindent
As $x$ becomes larger and larger, the fraction $\frac{1}{x}$ becomes smaller and smaller.
In the limit where $x$ goes to infinity, $\frac{1}{x}$ approaches zero: $\lim_{x\to \infty} \frac{1}{x} = 0$. 
On the other hand, when $x$ takes on smaller and smaller positive values,
the expression $\frac{1}{x}$ becomes infinite: $\lim_{x \to 0^+} \frac{1}{x} = \infty$.
When $x$ approaches $0$ from the left, we have $\lim_{x \to 0^-} \frac{1}{x} = -\infty$.
If these calculations are not clear to you,
study the graph of $f(x)=\frac{1}{x}$.


Here are some other examples of limits:

\small
\begin{verbatimtab}
>>> limit(sin(x)/x, x, 0)
1
>>> limit(sin(x)**2/x, x, 0)
0
>>> limit(exp(x)/x**100,x,oo) # which is bigger e^x or x^100 ?
oo                            # exp f >> all poly f for big x  
\end{verbatimtab}
\normalsize

Limits are used to define the derivative and the integral operations.

\subsection{Derivatives}
\label{calculus:derivatives}
The derivative function, denoted $f'(x)$, $\frac{d}{dx}f(x)$, $\frac{df}{dx}$, or $\frac{dy}{dx}$, 								\index{derivative}
describes the \emph{rate of change} of the function $f(x)$.
The \texttt{SymPy} function \texttt{diff} computes the derivative of any expression:



\small
\begin{verbatimtab}
>>> diff(x**3, x)
3*x**2
\end{verbatimtab}
\normalsize

\noindent
The differentiation operation knows the product rule $[f(x)g(x)]^\prime=f^\prime(x)g(x)+f(x)g^\prime(x)$, 
the chain rule $f(g(x))' = f'(g(x))g'(x)$, 
and the quotient rule $\left[\frac{f(x)}{g(x)}\right]^\prime = \frac{f'(x)g(x) - f(x)g'(x)}{g(x)^2}$:



\small
\begin{verbatimtab}
>>> diff( x**2*sin(x), x )
2*x*sin(x) + x**2*cos(x)
>>> diff( sin(x**2), x )
cos(x**2)*2*x
>>> diff( x**2/sin(x), x )
(2*x*sin(x) - x**2*cos(x))/sin(x)**2
\end{verbatimtab}
\normalsize

\noindent
The second derivative of a function \texttt{f} is \texttt{diff(f,x,2)}:



\small
\begin{verbatimtab}
>>> diff(x**3, x, 2)       # same as diff(diff(x**3, x), x)
6*x
\end{verbatimtab}
\normalsize


\subsection{Tangent lines}
\label{calculus:tangent_lines}

The \emph{tangent line} to the function $f(x)$ at $x=x_0$ is 													\index{tangent line}
the line that passes through the point $(x_0, f(x_0))$ and has 
the same slope as the function at that point.
The tangent line to the function $f(x)$ at the point $x=x_0$ is described by the equation
\[
   T_1(x) =  f(x_0) \; + \;  f'(x_0)(x-x_0).
\]
What is the equation of the tangent line to $f(x)=\frac{1}{2}x^2$ at $x_0=1$?



\small
\begin{verbatimtab}
>>> f = S('1/2')*x**2
>>> f
x**2/2
>>> df = diff(f, x)
>>> df
x
>>> T_1 = f.subs({x:1}) + df.subs({x:1})*(x - 1)
>>> T_1
x - 1/2           #  y = x - 1/2
\end{verbatimtab}
\normalsize

\noindent
The tangent line $T_1(x)$ has the same value and slope as the function $f(x)$ at $x=1$:
\small
\begin{verbatimtab}
>>> T_1.subs({x:1}) == f.subs({x:1})
True
>>> diff(T_1, x).subs({x:1}) == diff(f, x).subs({x:1})
True
\end{verbatimtab}
\normalsize



\subsection{Optimization}
\label{calculus:optimization}
																								\index{optimization}


	Optimization is about choosing an input for a function $f(x)$ that results in the best value for $f(x)$.					
	The best value usually means the \emph{maximum} value 
	(if the function represents something desirable like profits) 
	or the \emph{minimum} value 
	(if the function represents something undesirable like costs).

	The derivative $f'(x)$ encodes the information about the \emph{slope} of $f(x)$.
	Positive slope $f'(x)>0$ means $f(x)$ is increasing,
	negative slope $f'(x)<0$ means $f(x)$ is decreasing, 
	and zero slope $f'(x)=0$ means the graph of the function is horizontal.
	The \emph{critical points} of a function $f(x)$ are the solutions to the equation $f'(x)=0$.
	Each critical point is a candidate to be either a maximum or a minimum of the function.

	The second derivative $f^{\prime\prime}(x)$ encodes the information about the \emph{curvature} of $f(x)$.
	Positive curvature means the function looks like~$x^2$,
	negative curvature means the function looks like $-x^2$.


%	Recall the \emph{second derivative test} for finding the maxima and minima of a function,
%	which we learned on page~\pageref{optimization_algorithm:alternate_algorithm}.

Let's find the critical points of the function $f(x)=x^3-2x^2+x$ and use the information from its second derivative 			\index{critical point}
to find the maximum of the function on the interval $x \in [0,1]$.													\index{maximum}

\small
\begin{verbatimtab}
>>> x = Symbol('x')
>>> f = x**3-2*x**2+x
>>> diff(f, x)
3*x**2 - 4*x + 1
>>> sols = solve( diff(f,x),  x)
>>> sols
[1/3, 1]
>>> diff(diff(f,x), x).subs( {x:sols[0]} )
-2
>>> diff(diff(f,x), x).subs( {x:sols[1]} )
2
\end{verbatimtab}
\normalsize

\noindent
\href{https://www.google.ca/\#q=plot+x**3-2*x**2++\%2B+x&safe=off}{It will help to look at the graph of this function.}
The point $x=\frac{1}{3}$ is a local maximum because it is a critical point of $f(x)$
where the curvature is negative, meaning $f(x)$ looks like the peak of a mountain at $x=\frac{1}{3}$.
The maximum value of $f(x)$ on the interval $x\in [0,1]$ is $f\!\left(\frac{1}{3}\right)=\frac{4}{27}$.
The point $x=1$ is a local minimum because it is a critical point													\index{minimum}
with positive curvature, meaning $f(x)$ looks like the bottom of a valley at $x=1$.




\subsection{Integrals}
\label{calculus:integrals}


The \emph{integral} of $f(x)$ corresponds to the computation of the area under the graph of $f(x)$.						\index{area}
The area under $f(x)$ between the points $x=a$ and $x=b$ is denoted as follows:
\[
 A(a,b) = \int_a^b f(x) \: dx.
\]
The \emph{integral function} $F$ corresponds to the area calculation as a function of the upper limit of integration:
\[
  F(c) \eqdef \int_0^c \! f(x)\:dx\,.
\]
The area under $f(x)$ between $x=a$ and $x=b$ is obtained by calculating the \emph{change} in the integral function:
\[
   A(a,b) = \int_a^b \! f(x)\:dx  =  F(b)-F(a).
\]


In \texttt{SymPy} we use \texttt{integrate(f, x)} to obtain the integral function $F(x)$ of any function $f(x)$:					\index{integral}
$F(x) = \int_0^x f(u)\,du$.

\small
\begin{verbatimtab}
>>> integrate(x**3, x)
x**4/4
>>> integrate(sin(x), x)
-cos(x)
>>> integrate(ln(x), x)
x*log(x) - x
\end{verbatimtab}
\normalsize
This is known as an \emph{indefinite integral} since the limits of integration are not defined. 

In contrast, 
a \emph{definite integral} computes the area under $f(x)$ between $x=a$ and $x=b$.
Use \texttt{integrate(f, (x,a,b))} to compute the definite integrals of the form $A(a,b)=\int_a^b f(x) \, dx$:

\small
\begin{verbatimtab}
>>> integrate(x**3, (x,0,1))    
1/4              # the area under x^3 from x=0 to x=1
\end{verbatimtab}
\normalsize

\noindent
We can obtain the same area by first calculating the indefinite integral $F(c)=\int_0^c \!f(x)\,dx$,
then using $A(a,b) = F(x)\big\vert_a^b = F(b) - F(a)$:



\small
\begin{verbatimtab}
>>> F = integrate(x**3, x)
>>> F.subs({x:1}) - F.subs({x:0})   
1/4
\end{verbatimtab}
\normalsize
Integrals correspond to \emph{signed} area calculations:



\small
\begin{verbatimtab}
>>> integrate(sin(x), (x,0,pi))
2
>>> integrate(sin(x), (x,pi,2*pi))
-2
>>> integrate(sin(x), (x,0,2*pi))
0
\end{verbatimtab}
\normalsize

\noindent
During the first half of its $2\pi$-cycle,
the graph of $\sin(x)$ is above the $x$-axis, so it has a positive contribution to the area under the curve.
During the second half of its cycle (from $x=\pi$ to $x=2\pi$),
$\sin(x)$ is below the $x$-axis, so it contributes negative area.
Draw a graph of $\sin(x)$ to see what is going on. 

\subsection{Fundamental theorem of calculus}
\label{calculus:fundamental_theorem_of_calculus}
																								\index{fundamental theorem of calculus}
The integral is the ``inverse operation'' of the derivative.			\index{inverse!operation}
If you perform the integral operation followed by the derivative operation on some function,
you'll obtain the same function:
\[
  \left(\frac{d}{dx} \circ \int dx \right) f(x) = \frac{d}{dx} \int_c^x f(u)\:du = f(x).
\]



\small
\begin{verbatimtab}
>>> f = x**2
>>> F = integrate(f, x)
>>> F
x**3/3           # + C
>>> diff(F, x)
x**2
\end{verbatimtab}
\normalsize

\noindent
Alternately, if you compute the derivative of a function followed by the integral,
you will obtain the original function $f(x)$ (up to a constant):
\[
  \left( \int dx \circ \frac{d}{dx}\right) f(x) = \int_c^x f'(u)\;du = f(x) + C.
\]



\small
\begin{verbatimtab}
>>> f = x**2
>>> df = diff(f, x)
>>> df
2*x
>>> integrate(df, x)
x**2    # + C
\end{verbatimtab}
\normalsize

\noindent
The fundamental theorem of calculus is important because it tells us how to solve differential equations.
If we have to solve for $f(x)$ in the differential equation $\frac{d}{dx}f(x) = g(x)$,
we can take the integral on both sides of the equation to obtain the answer $f(x) = \int g(x)\,dx + C$.

\subsection{Sequences}
\label{calculus:sequences}

Sequences are functions that take whole numbers as inputs.												\index{sequence}
Instead of continuous inputs $x\in \mathbb{R}$,
sequences take natural numbers $n\in\mathbb{N}$ as inputs.
We denote sequences as $a_n$ instead of the usual function notation $a(n)$.

We define a sequence by specifying an expression for its $n$\textsuperscript{th} term:



\small
\begin{verbatimtab}
>>> a_n = 1/n
>>> b_n = 1/factorial(n)
\end{verbatimtab}
\normalsize

\noindent
Substitute the desired value of $n$ to see the value of the $n$\textsuperscript{th} term:

\small
\begin{verbatimtab}
>>> a_n.subs({n:5})
1/5
\end{verbatimtab}
\normalsize

\noindent
%We can use 
The Python list comprehension syntax \texttt{[item for item in list]}
can be used to print the sequence values for some range of indices:



\small
\begin{verbatimtab}
>>> [ a_n.subs({n:i}) for i in range(0,8) ]
[oo, 1, 1/2, 1/3, 1/4,  1/5,   1/6,   1/7]  
>>> [ b_n.subs({n:i}) for i in range(0,8) ]
[1,  1, 1/2, 1/6, 1/24, 1/120, 1/720, 1/5040]
\end{verbatimtab}
\normalsize

\noindent
Observe that $a_n$ is not properly defined for $n=0$ since $\frac{1}{0}$ is a division-by-zero error.
To be precise, we should say $a_n$'s domain is the positive naturals $a_n:\mathbb{N}^+ \to \mathbb{R}$.
Observe how quickly the \texttt{factorial} function $n!=1\cdot2\cdot3\cdots(n-1)\cdot n$ grows:
$7!= 5040$, $10!=3628800$, $20! > 10^{18}$.



We're often interested in calculating the limits of sequences as $n\to \infty$.
What happens to the terms in the sequence when $n$ becomes large?

\small
\begin{verbatimtab}
>>> limit(a_n, n, oo)
0
>>> limit(b_n, n, oo)
0
\end{verbatimtab}
\normalsize

\noindent
Both $a_n=\frac{1}{n}$ and $b_n = \frac{1}{n!}$ \emph{converge} to $0$ as $n\to\infty$.							\index{convergence}

\medskip

Many important math quantities are defined as limit expressions.
An interesting example to consider is the number $\pi$,
which is defined as the area of a circle of radius $1$.	
We can approximate the area of the unit circle by drawing a many-sided regular polygon around the circle.
Splitting the $n$-sided regular polygon into identical triangular splices,
we can obtain a formula for its area $A_n$.
In the limit as $n\to \infty$, 
the $n$-sided-polygon approximation to the area of the unit-circle becomes exact:		\index{approximation}

\small
\begin{verbatimtab}
>>> A_n = n*tan(2*pi/(2*n))
>>> limit(A_n, n, oo)
pi
\end{verbatimtab}
\normalsize


\subsection{Series}
\label{calculus:series}

Suppose we're given a sequence $a_n$ and we want to compute the sum of all the values in this sequence $\sum_{n}^\infty a_n$.
Series are sums of sequences.																		\index{series}
Summing the values of a sequence $a_n:\mathbb{N}\to \mathbb{R}$
is analogous to taking the integral of a function $f:\mathbb{R}\to \mathbb{R}$.

To work with series in \texttt{SymPy},
use the \texttt{summation} function whose syntax is analogous to the \texttt{integrate} function: 						\index{summation}
																							\index{sequence!harmonic}

\small
\begin{verbatimtab}
>>> a_n = 1/n
>>> b_n = 1/factorial(n)
>>> summation(a_n, [n, 1, oo])
oo
>>> summation(b_n, [n, 0, oo])
E
\end{verbatimtab}
\normalsize

\noindent
We say the series $\sum a_n$ \emph{diverges} to infinity (or \emph{is divergent})								\index{divergence}
while the series $\sum b_n$ converges (or \emph{is convergent}).											\index{convergence}
As we sum together more and more terms of the sequence $b_n$, the total becomes closer and closer to some finite number.
In this case, the infinite sum $\sum_{n=0}^\infty \frac{1}{n!}$ converges to the number $e=2.71828\ldots$.


The \texttt{summation} command is useful because it allows us to compute \emph{infinite} sums,
but for most practical applications we don't need to take an infinite number of terms in a series to obtain a good approximation. 
This is why series are so neat: they represent a great way to obtain approximations.

Using standard Python commands,  
we can obtain an approximation to $e$ that is accurate to six decimals by summing 10 terms in the series: 

\small
\begin{verbatimtab}
>>> import math
>>> def b_nf(n): 
        return 1.0/math.factorial(n)
>>> sum( [b_nf(n) for n in range(0,10)] )
2.718281 52557319
>>> E.evalf()
2.718281 82845905       # true value
\end{verbatimtab}
\normalsize
\subsection{Taylor series}
\label{calculus:taylor_series}

Wait, there's more! 
Not only can we use series to approximate numbers,
we can also use them to approximate functions.

A \emph{power series} is a series whose terms contain different powers of the variable $x$.						\index{Taylor series}
The $n$\textsuperscript{th} term in a power series is a function of both the sequence index $n$ and the input variable $x$.

For example, the power series of the function $\exp(x)=e^x$ is 
\[
 \exp(x)	=	1 + x + \frac{x^2}{2} + \frac{x^3}{3!} + \frac{x^4}{4!} + \frac{x^5}{5!} + \cdots         
		= 	\sum_{n=0}^\infty \frac{x^n}{n!}.
\]
This is, IMHO, one of the most important ideas in calculus:
you can compute the value of $\exp(5)$ by taking the infinite sum of the terms in the power series with $x=5$:



\small
\begin{verbatimtab}
>>> exp_xn = x**n/factorial(n)
>>> summation( exp_xn.subs({x:5}), [n, 0, oo] ).evalf()
148.413159102577
>>> exp(5).evalf()
148.413159102577        # the true value
\end{verbatimtab}
\normalsize

\noindent
Note that \texttt{SymPy} is actually smart enough to recognize that the infinite series
you're computing corresponds to the closed-form expression $e^5$:



\small
\begin{verbatimtab}
>>> summation( exp_xn.subs({x:5}), [n, 0, oo])
exp(5)
\end{verbatimtab}
\normalsize
Taking as few as 35 terms in the series is sufficient to obtain an approximation to $e$
that is accurate to $16$ decimals:
%so series are not some abstract thing for mathematicians but a practical trick you can when you code:

\small
\begin{verbatimtab}
>>> import math                    # redo using only python 
>>> def exp_xnf(x,n): 
        return x**n/math.factorial(n)
>>> sum( [exp_xnf(5.0,i) for i in range(0,35)] )
148.413159102577
\end{verbatimtab}
\normalsize

\noindent
The coefficients in the power series of a function (also known as the \emph{Taylor series})
depend on the value of the higher derivatives of the function. 
The formula for the $n$\textsuperscript{th} term in the Taylor series of $f(x)$ expanded at $x=c$ is $a_n(x) = \frac{f^{(n)}(c)}{n!}(x-c)^n$,
where $f^{(n)}(c)$ is the value of the $n$\textsuperscript{th} derivative of $f(x)$ evaluated at $x=c$.
%The term \emph{Taylor series} applies to all series expansions of functions.
The term \emph{Maclaurin series} refers to Taylor series expansions at $x=0$.										\index{Maclaurin series}

The \texttt{SymPy} function \texttt{series} is a convenient way to obtain the series of any function.
Calling \texttt{series(expr,var,at,nmax)} 
will show you the series expansion of \texttt{expr} 
near \texttt{var}=\texttt{at} 
up to power \texttt{nmax}:

\small
\begin{verbatimtab}
>>> series( sin(x), x, 0, 8)
x - x**3/6 + x**5/120 - x**7/5040 + O(x**8)
>>> series( cos(x), x, 0, 8)
1 - x**2/2 + x**4/24 - x**6/720 + O(x**8)
>>> series( sinh(x), x, 0, 8)
x + x**3/6 + x**5/120 + x**7/5040 + O(x**8)
>>> series( cosh(x), x, 0, 8)
1 + x**2/2 + x**4/24 + x**6/720 + O(x**8)
\end{verbatimtab}
\normalsize

%Note the power series of $\sin$ and $\sinh$ contain only odd powers of $x$
%while the power series of $\cos$ and $\cosh$ contain only even powers.

\noindent
Some functions are not defined at $x=0$, so we expand them at a different value of $x$.
For example, the power series of $\ln(x)$ expanded at $x=1$ is

\small
\begin{verbatimtab}
>>> series(ln(x), x, 1, 6)     # Taylor series of ln(x) at x=1
x - x**2/2 + x**3/3 - x**4/4 + x**5/5  + O(x**6)    
\end{verbatimtab}
\normalsize

\noindent
Here, the result \texttt{SymPy} returns is misleading.
The Taylor series of $\ln(x)$ expanded at $x=1$ has terms of the form $(x-1)^n$:
\[
  \ln(x) = (x-1) - \frac{(x-1)^2}{2} + \frac{(x-1)^3}{3} - \frac{(x-1)^4}{4} + \frac{(x-1)^5}{5} + \cdots.
\]
Verify this is the correct formula by substituting $x=1$.
\texttt{SymPy} returns an answer in terms of coordinates \emph{relative} to $x=1$.
%That's okay, 
%because when dealing with series in general we're mostly interested in the coefficients.

Instead of expanding $\ln(x)$ around $x=1$,
we can obtain an equivalent expression if we expand $\ln(x+1)$ around $x=0$:



\small
\begin{verbatimtab}
>>> series(ln(x+1), x, 0, 6)   # Maclaurin series of ln(x+1)
x - x**2/2 + x**3/3 - x**4/4 + x**5/5 + O(x**6)
\end{verbatimtab}
\normalsize







\section*{Exercises}
\label{calculus_prerequisites:exercises}	

	TODO: 10x exercises on sets, functions, and integrals (geometric, numeric, and symbolic)


%		\input{statsprobs/chapter21_integrals_fEQc_and_gEQmx.tex}

	
	TODO: 3 more



\section*{Links}
\label{calculus_prerequisites:links}



	\subsection{Recommended calculus learning resources}
	

		Above all,
		my advice is not to think of calculus as ``advanced math theory'' that might be difficult to understand,
		but instead as practical, useful math that allows you to do calculations---just look at the name of the thing!
		% ALT: Just look at the name of the thing: if it's called \emph{calculus} then you know it's all about calculations!
		This means learning calculus is all about getting practical experience
		calculating limits, derivatives, and integrals of functions,
		which is best achieved by solving lots of problems.
		The problems and exercises in the books \emph{Calculus made simple} and \emph{No bullshit guide to math and physics}
		are therefore your best route for learning calculus,
		if you choose to pursue this subject.




	[ \emph{Essence of calculus} series by 3Blue1Brown ] \\
	\href{https://www.youtube.com/playlist?list=PLZHQObOWTQDMsr9K-rj53DwVRMYO3t5Yr}
	{\texttt{https://tinyurl.com/CALCess}}

	\medskip
	\noindent
	[ \emph{Calculus made simple} by Silvanus P. Thompson ] \\
	\href{https://gutenberg.org/ebooks/33283}
	{\texttt{https://gutenberg.org/ebooks/33283}}
	%	There is an excellent free book called \emph{}\cite{thompson1914calculus} by Silvanus P. Thompson,
	%	which is a very friendly introduction to the subject.
	%	The subtitle of Thompson's book includes the phrase
	%	``Being a very-simplest introduction to those beautiful methods which are generally called by terrifying names,''
	%	which should give you some idea about the author's attitude and the tone of his writing.
	%	You can also check out Chapter~5 in the \emph{No bullshit guide to math and physics}\cite{savov2014noBSmathphys},
	%	which is a compact introduction to calculus,
	%	and includes lots of examples from physics.

	%		\medskip
	%		\noindent
	%		[ SymPy tutorial by Ivan Savov ] \\
	%		\href{https://minireference.com/static/tutorials/sympy_tutorial.pdf}
	%		{\small \texttt{https://minireference.com/static/tutorials/sympy_tutorial.pdf}}
	%	Remember that you have SymPy at your disposal to solve calculus problems,
	%	so you don't have to do all the calculations by hand using pen and paper.
	%	Indeed,
	%	you can solve any calculus problem using just a few lines of code using the SymPy functions \tt{limit}, \tt{diff}, and \tt{integrate}.
	%	Check out Section~III of the \emph{SymPy tutorial}\cite{savov2017sympy} to learn how to use these functions.


	%		\medskip
	%		\noindent
	%		[  ] \\
	%		\href{}
	%		{\texttt{}}

	% TODO add more links to intro-calculus material

\vspace{4in}


	% BOOK PLUG
	%%%%%%%%%%%%%%%%%%%%%%%%%%%%%%%%%%%%%%%%%%%%%%%%%%%%%%%%%%%%
	\noindent
	If you want to learn more about calculus,
	I invite you to check out my book,
	the \textbf{No bullshit guide to math and physics}.

	\begin{wrapfigure}[8]{r}{0pt}
	\includegraphics[width=120pt]{figures/cover_v40_noline_lighter.jpg}
	%\includegraphics[width=125pt]{/Library/WebServer/Documents/miniref/data/media/physics/mass_spring-highres.png}
	\end{wrapfigure}

	This book contains short lessons on mechanics,
	differential calculus, and integral calculus
	written in a style that is jargon-free and to the point.
	% The main focus of the book is to show the intricate connections between the concepts of mechanics and calculus.
	%	Often calculus and mechanics are taught as separate subjects.
	%	It shouldn't be like that.
	%	If you learn calculus without mechanics, it will be boring.
	%	If you learn mechanics without calculus, you won't truly understand.
	This textbook covers both subjects in an integrated manner and aims to highlight the
	connections between them. % two subjects.

	Contents:
	\begin{itemize}
		\item	{\sc high school math}%: (40pp) %Review of algebra, functions and trigonometry.
		\item	{\sc vectors}%: (20pp) 
		\item	{\sc mechanics} (just 70 pages!)
		\item	{\sc differential calculus}%: (30pp)
		\item	{\sc integral calculus}%: (20pp)
		\item	{\sc sequences and series}%: (20pp)
	\end{itemize}

	\vspace{1mm}

	\noindent
	\hfill {\small   5\textonehalf[in] $\times$ 8\textonehalf[in] $\times$ 528[pages]} 

	\vspace{3mm}
	
	\noindent
	For more information, see the book's website \  %and find more information on the following website 
	\href{http://minireference.com/}{\texttt{minireference.com}}
	or you can get in touch with me by email here \texttt{ivan@minireference.com}.
	%	You can also follow \href{https://twitter.com/minireference}{\texttt{@minireference}} on twitter 
	%	and check out the facebook page \href{http://fb.me/noBSguide}{\texttt{fb.me/noBSguide}}.


\end{document}
